%% Generated by Sphinx.
\def\sphinxdocclass{report}
\documentclass[letterpaper,10pt,openany,english]{sphinxmanual}
\ifdefined\pdfpxdimen
   \let\sphinxpxdimen\pdfpxdimen\else\newdimen\sphinxpxdimen
\fi \sphinxpxdimen=.75bp\relax
\ifdefined\pdfimageresolution
    \pdfimageresolution= \numexpr \dimexpr1in\relax/\sphinxpxdimen\relax
\fi
%% let collapsible pdf bookmarks panel have high depth per default
\PassOptionsToPackage{bookmarksdepth=5}{hyperref}
%% turn off hyperref patch of \index as sphinx.xdy xindy module takes care of
%% suitable \hyperpage mark-up, working around hyperref-xindy incompatibility
\PassOptionsToPackage{hyperindex=false}{hyperref}
%% memoir class requires extra handling
\makeatletter\@ifclassloaded{memoir}
{\ifdefined\memhyperindexfalse\memhyperindexfalse\fi}{}\makeatother

\PassOptionsToPackage{booktabs}{sphinx}
\PassOptionsToPackage{colorrows}{sphinx}

\PassOptionsToPackage{warn}{textcomp}

\catcode`^^^^00a0\active\protected\def^^^^00a0{\leavevmode\nobreak\ }
\usepackage{cmap}
\usepackage{fontspec}
\defaultfontfeatures[\rmfamily,\sffamily,\ttfamily]{}
\usepackage{amsmath,amssymb,amstext}
\usepackage{polyglossia}
\setmainlanguage{english}



\setmainfont{FreeSerif}[
  Extension      = .otf,
  UprightFont    = *,
  ItalicFont     = *Italic,
  BoldFont       = *Bold,
  BoldItalicFont = *BoldItalic
]
\setsansfont{FreeSans}[
  Extension      = .otf,
  UprightFont    = *,
  ItalicFont     = *Oblique,
  BoldFont       = *Bold,
  BoldItalicFont = *BoldOblique,
]
\setmonofont{FreeMono}[Scale=0.9,
  Extension      = .otf,
  UprightFont    = *,
  ItalicFont     = *Oblique,
  BoldFont       = *Bold,
  BoldItalicFont = *BoldOblique,
]



\usepackage[Bjarne]{fncychap}
\usepackage{sphinx}

\fvset{fontsize=auto}
\usepackage{geometry}


% Include hyperref last.
\usepackage{hyperref}
% Fix anchor placement for figures with captions.
\usepackage{hypcap}% it must be loaded after hyperref.
% Set up styles of URL: it should be placed after hyperref.
\urlstyle{same}


\usepackage{sphinxmessages}




    % 依赖包
    \usepackage{fontspec}   % 字体操作依赖包    
    \usepackage{xeCJK}      % 中文字体,特殊字体包
    \usepackage{graphicx}   % 画图包
    \usepackage{tikz}
    \usepackage{xcolor}
    \usepackage{geometry}  

    \setmainfont{Carlito}   % 设定主字体    
    \setCJKmainfont{AR PL UMing CN}
    \setCJKsansfont{AR PL UMing CN}
    \setCJKmonofont{AR PL UMing CN}
        
    \geometry{a4paper, left=2cm, right=2cm, top=2cm, bottom=2cm}
    
    \setcounter{secnumdepth}{0}  % 关闭自动章节编号
    \setcounter{tocdepth}{2}     % 目录层级
    \renewcommand{\thechapter}{}

    % 定义封面所需的命令(如果您的文档类需要)
    \makeatletter
    \newcommand{\customtitle}{\@title}
    \newcommand{\customauthor}{\@author}
    \newcommand{\customdate}{\@date}
    \makeatother
    

\title{FETMX8MP-SMARC OKMX8MP-SMARC}
\date{Aug 29, 2025}
\release{}
\author{User's Software Manual}
\newcommand{\sphinxlogo}{\vbox{}}
\renewcommand{\releasename}{}
\makeindex
\begin{document}

\pagestyle{empty}


    \begin{titlepage}
    \thispagestyle{empty}
    
    \definecolor{forlinxblue}{RGB}{0, 102, 178} % 定义封面蓝条颜色
    \begin{tikzpicture}[remember picture, overlay]

      % --- 左侧窄条 (上半部):用于竖排文字
      \fill[forlinxblue] 
        (current page.north west) rectangle ++(2cm, -0.32\paperheight);

      % --- 左侧宽条 (下半部):作为装饰块
      \fill[forlinxblue] 
        (current page.south west) 
        rectangle ++(2cm, 0.58\paperheight);

      % --- 竖排文字:垂直居中于上窄条
      \node[
        white,
        rotate=270,
        font=\bfseries\huge,
        anchor=center
      ] 
      at ([xshift=1cm,yshift=-0.15\paperheight]current page.north west) 
      {\customauthor};

       % --- LOGO ---
      \node[anchor=north east] 
        at ([xshift=-2cm, yshift=-2cm]current page.north east) 
        {\includegraphics[width=6cm]{../../_static/logo.png}};

      % --- 产品型号/文档标题 ---
      \node[anchor=north west, font=\huge\bfseries]
        at ([yshift=-10.5cm, xshift=2.1cm]current page.north west) 
        {\customtitle};

      % --- 子标题 ---
      \node[anchor=north west, font=\large]
        at ([yshift=-11.5cm, xshift=2.1cm]current page.north west)
        {Embedded Development Platform};

      % --- 文档类型 ---
      \node[anchor=north east, font=\LARGE\bfseries]
        at ([yshift=-15cm, xshift=-2cm]current page.north east)
        {\customauthor};

      % --- 版本号 ---
      \node[anchor=north east, font=\normalsize]
        at ([yshift=-16cm, xshift=-2cm]current page.north east)
        {Rev. 1.1};

      % --- 公司信息 左对齐 + 右对齐 ---
      \node[anchor=south west, font=\normalsize]
        at ([xshift=3cm, yshift=2cm]current page.south west)
        {Forlinx Embedded Technology Co. Ltd.};

      \node[anchor=south east, font=\normalsize]
        at ([xshift=-2cm, yshift=2cm]current page.south east)
        {www.forlinx.net};

    \end{tikzpicture}
    \end{titlepage}
    
\pagestyle{plain}
\sphinxtableofcontents
\pagestyle{normal}
\phantomsection\label{\detokenize{hardware::doc}}



\chapter{1. Overview2}
\label{\detokenize{hardware:overview2}}

\section{1.1 SoM Function Description}
\label{\detokenize{hardware:som-function-description}}
\sphinxAtStartPar
The FET\sphinxhyphen{}MX8MPQ\sphinxhyphen{}SMARC is a module that complies with the SMARC 2.1 standard and features the high\sphinxhyphen{}performance i.MX 8M Plus processor. It combines advanced multimedia capabilities with optimized low power consumption, making it ideal for machine learning applications.

\sphinxAtStartPar
This module pairs with the OK\sphinxhyphen{}MX8MPQ\sphinxhyphen{}SMARC carrier board and connects to the mainboard using a 314\sphinxhyphen{}pin MXM connector.

\sphinxAtStartPar
It includes a wide range of functional interfaces and is widely utilized in various fields such as smart cities, industrial IoT, smart healthcare, and intelligent transportation, effectively meeting the diverse needs of multiple applications.


\section{1.2 i.MX8MP Series Processors!!!}
\label{\detokenize{hardware:i-mx8mp-series-processors}}
\sphinxAtStartPar
The i.MX 8M Plus series processors focus on machine learning, vision, advanced multimedia, and industrial automation, offering high reliability. They are designed to meet the needs of smart home, building, city, and Industry 4.0 applications.
\begin{itemize}
\item {} 
\sphinxAtStartPar
Powerful quad\sphinxhyphen{}core or dual\sphinxhyphen{}core Arm® Cortex®\sphinxhyphen{}A53 processors and an integrated Neural Processing Unit (NPU) capable of up to 2.3 TOPS;

\item {} 
\sphinxAtStartPar
Dual Image Signal Processors (ISP) and support for two camera inputs enable efficient, advanced vision systems;

\item {} 
\sphinxAtStartPar
Multimedia capabilities include video encoding (including H.265) and decoding, 3D/2D graphics acceleration, and a wide range of audio and voice features;

\item {} 
\sphinxAtStartPar
Real\sphinxhyphen{}time control is handled by a Cortex\sphinxhyphen{}M7 core, supporting dual CAN FD, dual Gigabit Ethernet, and Time\sphinxhyphen{}Sensitive Networking (TSN) functionality;

\item {} 
\sphinxAtStartPar
Designed for high industrial reliability, supporting DRAM inline ECC.

\end{itemize}

\sphinxAtStartPar
\sphinxincludegraphics{{6e9c9552c17a4110894efeba569b442a}.png}

\sphinxAtStartPar
The FET\sphinxhyphen{}MX8MPQ\sphinxhyphen{}SMARC SoM is compatible with the CPU models listed in the table below, with the default model being MIMX8ML8CVNKZAB:

\sphinxAtStartPar
\sphinxincludegraphics{{93f7664592494313ab7b0f44df8085b2}.png}

\sphinxAtStartPar
For more details about the MX8M Plus series, please visit the official NXP website:

\sphinxAtStartPar
\sphinxurl{https://www.nxp.com.cn/products/processors-and-microcontrollers/arm-processors/i-mx-applications-processors/i-mx-8-processors/i-mx-8m-plus-arm-cortex-a53-machine-learning-vision-multimediaand-industrial-iot:IMX8MPLUS}


\chapter{2. FET\sphinxhyphen{}MX8MPQ\sphinxhyphen{}SMARC Description}
\label{\detokenize{hardware:fet-mx8mpq-smarc-description}}

\section{2.1 SoM Appearance}
\label{\detokenize{hardware:som-appearance}}
\sphinxAtStartPar
\sphinxincludegraphics{{56cb26ad7a75406fab91666674e3fa58}.png}

\sphinxAtStartPar
Front

\sphinxAtStartPar
\sphinxincludegraphics{{e015c43758954a63b31c820ad880c0d0}.png}

\sphinxAtStartPar
Back


\section{2.2 Technical Specifications}
\label{\detokenize{hardware:technical-specifications}}
\sphinxAtStartPar
Processors\\
i.MX 8M Plus Quad
\begin{itemize}
\item {} 
\sphinxAtStartPar
Supports NPU, ISP, VPU, HiFi 4, CAN\sphinxhyphen{}FD

\item {} 
\sphinxAtStartPar
4 x Cortex\sphinxhyphen{}A53 up to 1.6 GHz

\item {} 
\sphinxAtStartPar
Cortex\sphinxhyphen{}M7 up to 800 MHz

\end{itemize}

\sphinxAtStartPar
Graphic Processing Unit (GPU)
\begin{itemize}
\item {} 
\sphinxAtStartPar
GC7000UL supports OpenCL and Vulkan

\item {} 
\sphinxAtStartPar
2 shaders

\item {} 
\sphinxAtStartPar
166 million triangles/sec

\item {} 
\sphinxAtStartPar
1.0 giga pixel/sec

\item {} 
\sphinxAtStartPar
16 GFLOPs 32\sphinxhyphen{}bit

\item {} 
\sphinxAtStartPar
Supports OpenGL ES 1.1, 2.0, 3.0, OpenCL 1.2, Vulkan

\item {} 
\sphinxAtStartPar
Core clock frequency of 1000 MHz

\item {} 
\sphinxAtStartPar
Shader clock frequency of 1000 MHz

\item {} 
\sphinxAtStartPar
GC520L for 2D acceleration

\item {} 
\sphinxAtStartPar
Render target compatibility between 3D and 2D GPU (super tile status buffer)

\end{itemize}

\sphinxAtStartPar
Video Processing Unit (VPU)

\sphinxAtStartPar
Video Decoding
\begin{itemize}
\item {} 
\sphinxAtStartPar
1080p60 HEVC/H.265 Main, Main 10 (up to level 5.1)

\item {} 
\sphinxAtStartPar
1080p60 VP9 Profile 0, 2

\item {} 
\sphinxAtStartPar
1080p60 VP8

\item {} 
\sphinxAtStartPar
1080p60 AVC/H.264 Baseline, Main, High decoder  video encoding.

\item {} 
\sphinxAtStartPar
1080p60 AVC/H.264 encoder

\item {} 
\sphinxAtStartPar
1080p60 HEVC/H.265 encoder

\end{itemize}

\sphinxAtStartPar
Neutral Processing Unit (NPU)\\
2.3 TOP/s Neutral Processing Unit (NPU)
\begin{itemize}
\item {} 
\sphinxAtStartPar
Keyword detect, noise reduction, beamforming

\item {} 
\sphinxAtStartPar
Speech recognition (i.e. Deep Speech 2)

\item {} 
\sphinxAtStartPar
Image recognition (i.e. ResNet\sphinxhyphen{}50)

\end{itemize}

\sphinxAtStartPar
Image Sensor Processor (ISP)
\begin{itemize}
\item {} 
\sphinxAtStartPar
375 Mpixel/s HDR ISP, supporting 12Mp@30fps, 4kp45 or 2x 1080p80 etc.

\end{itemize}

\sphinxAtStartPar
Memory (Memory)
\begin{itemize}
\item {} 
\sphinxAtStartPar
Soldered LPDDR4\sphinxhyphen{}4266 memory, 32\sphinxhyphen{}bit interface, optional 2GB and 4GB

\end{itemize}

\sphinxAtStartPar
Storage
\begin{itemize}
\item {} 
\sphinxAtStartPar
Solder eMMC 5.1 with 16GB and 32GB options

\end{itemize}

\sphinxAtStartPar
Video output Interfaces(Video output connector)\\
1 x HDMI 2.0a Tx
\begin{itemize}
\item {} 
\sphinxAtStartPar
Resolution 720 x 480p60, 1280 x 720p60, 1920 x 1080p60, 1920 x 1080p120,3840 x  2160p30

\item {} 
\sphinxAtStartPar
Pixel clock up to 297 MHz 1 X LVDS 18/24\sphinxhyphen{}bit single\sphinxhyphen{}/dual\sphinxhyphen{}channel (factory\sphinxhyphen{}optional) 1 X MIPI DSI (multiplexed with one of the LVDS channels, factory\sphinxhyphen{}optional)

\item {} 
\sphinxAtStartPar
Maximum resolution limited by a 250 MHz pixel clock and an effective pixel rate of 200 Mpixel/s for 24\sphinxhyphen{}bit RGB.

\item {} 
\sphinxAtStartPar
Supported resolutions include:

\sphinxAtStartPar
• 1080 p60

\sphinxAtStartPar
• WUXGA (1920x1200) at 60 Hz

\sphinxAtStartPar
• 1920x1440 at 60 Hz

\sphinxAtStartPar
• UWHD (2560x1080) at 60 Hz

\sphinxAtStartPar
• MIPI DSI: WQHD (2560x1440), supported by reducing blanking intervals

\end{itemize}

\sphinxAtStartPar
Camera
\begin{itemize}
\item {} 
\sphinxAtStartPar
1 x 4\sphinxhyphen{}lanes CSI camera interfaces

\item {} 
\sphinxAtStartPar
1 x 2\sphinxhyphen{}lanes CSI camera interfaces

\end{itemize}

\sphinxAtStartPar
Audio Interfaces
\begin{itemize}
\item {} 
\sphinxAtStartPar
Cadence® Tensilica® HiFi 4 DSP, maximum support 800 MHz

\end{itemize}

\sphinxAtStartPar
2 x I2S Audio interface
\begin{itemize}
\item {} 
\sphinxAtStartPar
All ports support 49.152 MHz BCLK

\end{itemize}

\sphinxAtStartPar
Connectivity (Communication interface)
\begin{itemize}
\item {} 
\sphinxAtStartPar
1 x PCIe Express (PCIe) single chanel, supports PCIe Gen3

\end{itemize}

\sphinxAtStartPar
Networking
\begin{itemize}
\item {} 
\sphinxAtStartPar
2 x Gigabit Ethernet interface

\end{itemize}

\sphinxAtStartPar
On\sphinxhyphen{}board IEEE 802.11 2X2 WiFi 5 MIMO Wireless LAN + Bluetooth 5 3 Combo LGA Module (factory optional)

\sphinxAtStartPar
USB
\begin{itemize}
\item {} 
\sphinxAtStartPar
1 x USB2.0 OTG (directly to CPU)

\item {} 
\sphinxAtStartPar
3 x USB2.0 Host

\item {} 
\sphinxAtStartPar
2 x USB3.0 Host

\end{itemize}

\sphinxAtStartPar
Serial ports
\begin{itemize}
\item {} 
\sphinxAtStartPar
2 x UART Tx / Rx / RTS / CTS

\item {} 
\sphinxAtStartPar
2 x UART Tx / Rx

\item {} 
\sphinxAtStartPar
2 x CAN Bus

\item {} 
\sphinxAtStartPar
The communication controller supports the CAN FD protocol and the CAN 2.0B protocol specification.

\end{itemize}

\sphinxAtStartPar
Other Interfaces
\begin{itemize}
\item {} 
\sphinxAtStartPar
1 x SD 1\sphinxhyphen{}bit/4\sphinxhyphen{}bit SDIO 3.0 interface

\item {} 
\sphinxAtStartPar
5 x I2C Bus

\item {} 
\sphinxAtStartPar
1 x SPI interface

\item {} 
\sphinxAtStartPar
1 x QuadSPI interface

\item {} 
\sphinxAtStartPar
14 x GPIOs

\end{itemize}

\sphinxAtStartPar
Startup option configuration signal

\sphinxAtStartPar
Power management signal

\sphinxAtStartPar
Power Voltage: +5VDC\\
RTC Voltage: 3.3V\\
Operating Temperature: Industrial level \sphinxhyphen{}40 ° C \textasciitilde{} + 85 ° C

\sphinxAtStartPar
Size: 50 x82 mm\\
\sphinxstylestrong{Note: The actual temperature will largely depend on the application, enclosure, and/or environment. Please consider an application \sphinxhyphen{} specific cooling solution for the final system to maintain the radiator temperature within the specified range.}


\section{2.3 FET\sphinxhyphen{}MX8MPQ\sphinxhyphen{}SMARC Module Structure}
\label{\detokenize{hardware:fet-mx8mpq-smarc-module-structure}}
\sphinxAtStartPar
\sphinxincludegraphics{{d93d62070e6847f48e0bb8ee3ef0f68d}.png}Figure 2\sphinxhyphen{}3:FET\sphinxhyphen{}MX8MPQ\sphinxhyphen{}SMARC (Top)

\sphinxAtStartPar
\sphinxincludegraphics{{ee16e760f47d4028b9b19c68ae64d37d}.png}Figure 2\sphinxhyphen{}4:FET\sphinxhyphen{}MX8MPQ\sphinxhyphen{}SMARC

\sphinxAtStartPar
To prevent the board from warping, a solder pad with a diameter of 6 mm is reserved inside the bottom layer of the board and connected to the GND network.

\sphinxAtStartPar
When designing the carrier board, appropriate fixing posts can be added at the specified solder pads to support the SoM.
\begin{itemize}
\item {} 
\sphinxAtStartPar
SoM Dimension: 82mm x 50mm

\item {} 
\sphinxAtStartPar
Fixing hole spacing: 74mm x 34mm

\item {} 
\sphinxAtStartPar
Fixing hole diameter: 2.7mm

\item {} 
\sphinxAtStartPar
PCB Layers: 10 layer PCB

\item {} 
\sphinxAtStartPar
PCB thickness: 1.2mm

\end{itemize}

\sphinxAtStartPar
Connector 314p Gold Finger for detailed dimensions of the module structure, refer to “SMARC 2.1.1 Specification 2020\sphinxhyphen{}05\sphinxhyphen{}20” 5.3 Module Outline – 82x50mm Module.\sphinxincludegraphics{{a565f2b0773f46318987b18ca2c550fd}.png}

\sphinxAtStartPar
Figure 2\sphinxhyphen{}5: SMARC 2.1.1 82 x 50mm Module Outline

\sphinxAtStartPar
When using connectors of different heights, please consider that according to the SMARC specification, the maximum component height on the bottom side of the module is 1.3mm. When selecting the height of the MXM connector, please pay attention to the above point if you need to place components on the carrier board below the SMARC module.


\section{2.4 FET\sphinxhyphen{}MX8MPQ\sphinxhyphen{}SMARC Block Diagram}
\label{\detokenize{hardware:fet-mx8mpq-smarc-block-diagram}}
\sphinxAtStartPar
\sphinxincludegraphics{{b4f4c260ef204d758137ccba4c89c1dc}.png}

\sphinxAtStartPar
Figure 2\sphinxhyphen{}6: FET\sphinxhyphen{}MX8MPQ\sphinxhyphen{}SMARC Block Diagram


\chapter{3. FET\sphinxhyphen{}MX8MPQ\sphinxhyphen{}SMARC Interface Description}
\label{\detokenize{hardware:fet-mx8mpq-smarc-interface-description}}

\section{3.1 SoM Connector}
\label{\detokenize{hardware:som-connector}}

\subsection{3.1.1 Golden\sphinxhyphen{}finger}
\label{\detokenize{hardware:golden-finger}}
\sphinxAtStartPar
\sphinxincludegraphics{{14a63d1ed6f14b07868cf61865c400ad}.png}

\sphinxAtStartPar
Figure 3\sphinxhyphen{}1 SoM Connector


\subsection{3.1.2 MXM 3.0 Connector}
\label{\detokenize{hardware:mxm-3-0-connector}}
\sphinxAtStartPar
The carrier board connector is a 314\sphinxhyphen{}pin, 0.5mm pitch right\sphinxhyphen{}angle component, designed for use with a 1.2mm thick PCB and features an appropriate edge finger pattern. This connector is commonly used in MXM3 graphics cards. The SMARC module uses this connector differently than the MXM3 standard.

\sphinxAtStartPar
\sphinxincludegraphics{{a9a69d8d841149e6a91d54d0545bc1ef}.png}

\sphinxAtStartPar
Figure 3\sphinxhyphen{}2 MXM 3.0 Carrier Board Connector


\subsection{3.1.3 Wi\sphinxhyphen{}Fi \& BT Antenna Connector}
\label{\detokenize{hardware:wi-fi-bt-antenna-connector}}
\sphinxAtStartPar
I\sphinxhyphen{}PEX MHF4 Connector Socket (20449) Main: Wi\sphinxhyphen{}Fi –> TX/RX\\
Aux: Wi\sphinxhyphen{}Fi/Bluetooth –> TX/RX

\sphinxAtStartPar
\sphinxincludegraphics{{19ae03a975664170b3284b0b9a3731db}.png}

\sphinxAtStartPar
Figure 3\sphinxhyphen{}3 Module Antenna Configuration

\sphinxAtStartPar
\sphinxincludegraphics{{ece919dd84234b1dbeed70bb0edef4a3}.png}

\sphinxAtStartPar
Figure 3\sphinxhyphen{}4 Module Antenna Configuration


\subsection{3.1.4 JTAG  Connector}
\label{\detokenize{hardware:jtag-connector}}
\sphinxAtStartPar
The processor’s JTAG interface is connected via a 7Pin, 1mm pitch connector.\\
The JTAG IO voltage level is 1.8V.

\sphinxAtStartPar
\sphinxincludegraphics{{e982c42b5099418c89925ef2bfe2e03e}.png}

\sphinxAtStartPar
Figure 3\sphinxhyphen{}5 JTAG Connector Line Sequence


\section{3.2 FET\sphinxhyphen{}MX8MPx\sphinxhyphen{}SMARC Connector Pin Out}
\label{\detokenize{hardware:fet-mx8mpx-smarc-connector-pin-out}}

\subsection{3.2.1 SMARC P\sphinxhyphen{}PIN Connector Output List}
\label{\detokenize{hardware:smarc-p-pin-connector-output-list}}
\sphinxAtStartPar
Table 3\sphinxhyphen{}1 SMARC P\sphinxhyphen{}PIN Connector Signal Output


\begin{savenotes}
\sphinxatlongtablestart
\sphinxthistablewithglobalstyle
\makeatletter
  \LTleft \@totalleftmargin plus1fill
  \LTright\dimexpr\columnwidth-\@totalleftmargin-\linewidth\relax plus1fill
\makeatother
\begin{longtable}{lllll}
\sphinxtoprule
\sphinxstyletheadfamily 
\sphinxAtStartPar
\sphinxstylestrong{PIN}
&\sphinxstyletheadfamily 
\sphinxAtStartPar
\sphinxstylestrong{Primary (Top) Side}
&\sphinxstyletheadfamily 
\sphinxAtStartPar
\sphinxstylestrong{I/O Type}
&\sphinxstyletheadfamily 
\sphinxAtStartPar
\sphinxstylestrong{I/O Level}
&\sphinxstyletheadfamily 
\sphinxAtStartPar
\sphinxstylestrong{PU / PD}
\\
\sphinxmidrule
\endfirsthead

\multicolumn{5}{c}{\sphinxnorowcolor
    \makebox[0pt]{\sphinxtablecontinued{\tablename\ \thetable{} \textendash{} continued from previous page}}%
}\\
\sphinxtoprule
\sphinxstyletheadfamily 
\sphinxAtStartPar
\sphinxstylestrong{PIN}
&\sphinxstyletheadfamily 
\sphinxAtStartPar
\sphinxstylestrong{Primary (Top) Side}
&\sphinxstyletheadfamily 
\sphinxAtStartPar
\sphinxstylestrong{I/O Type}
&\sphinxstyletheadfamily 
\sphinxAtStartPar
\sphinxstylestrong{I/O Level}
&\sphinxstyletheadfamily 
\sphinxAtStartPar
\sphinxstylestrong{PU / PD}
\\
\sphinxmidrule
\endhead

\sphinxbottomrule
\multicolumn{5}{r}{\sphinxnorowcolor
    \makebox[0pt][r]{\sphinxtablecontinued{continues on next page}}%
}\\
\endfoot

\endlastfoot
\sphinxtableatstartofbodyhook

\sphinxAtStartPar
P1
&
\sphinxAtStartPar
SMB\_ALERT\#
&
\sphinxAtStartPar
I CMOS
&
\sphinxAtStartPar
1.8V
&
\sphinxAtStartPar
PU 2k2
\\
\sphinxhline
\sphinxAtStartPar
P2
&
\sphinxAtStartPar
GND
&
\sphinxAtStartPar
\sphinxhyphen{}
&
\sphinxAtStartPar
\sphinxhyphen{}
&
\sphinxAtStartPar
\sphinxhyphen{}
\\
\sphinxhline
\sphinxAtStartPar
P3
&
\sphinxAtStartPar
CSI1\_CK+
&
\sphinxAtStartPar
ID\sphinxhyphen{}PHY
&
\sphinxAtStartPar
\sphinxhyphen{}
&
\sphinxAtStartPar
\sphinxhyphen{}
\\
\sphinxhline
\sphinxAtStartPar
P4
&
\sphinxAtStartPar
CSI1\_CK\sphinxhyphen{}
&
\sphinxAtStartPar
ID\sphinxhyphen{}PHY
&
\sphinxAtStartPar
\sphinxhyphen{}
&
\sphinxAtStartPar
\sphinxhyphen{}
\\
\sphinxhline
\sphinxAtStartPar
P5
&
\sphinxAtStartPar
GBE1\_SDP
&
\sphinxAtStartPar
I/O OD CMOS
&
\sphinxAtStartPar
3.3V *\sphinxstyleemphasis{1}
&
\sphinxAtStartPar
\sphinxhyphen{}
\\
\sphinxhline
\sphinxAtStartPar
P6
&
\sphinxAtStartPar
GBE0\_SDP
&
\sphinxAtStartPar
I/O OD CMOS
&
\sphinxAtStartPar
3.3V *\sphinxstyleemphasis{1}
&
\sphinxAtStartPar
\sphinxhyphen{}
\\
\sphinxhline
\sphinxAtStartPar
P7
&
\sphinxAtStartPar
CSI1\_RX0+
&
\sphinxAtStartPar
ID\sphinxhyphen{}PHY
&
\sphinxAtStartPar
\sphinxhyphen{}
&
\sphinxAtStartPar
\sphinxhyphen{}
\\
\sphinxhline
\sphinxAtStartPar
P8
&
\sphinxAtStartPar
CSI1\_RX0\sphinxhyphen{}
&
\sphinxAtStartPar
ID\sphinxhyphen{}PHY
&
\sphinxAtStartPar
\sphinxhyphen{}
&
\sphinxAtStartPar
\sphinxhyphen{}
\\
\sphinxhline
\sphinxAtStartPar
P9
&
\sphinxAtStartPar
GND
&
\sphinxAtStartPar
\sphinxhyphen{}
&
\sphinxAtStartPar
\sphinxhyphen{}
&
\sphinxAtStartPar
\sphinxhyphen{}
\\
\sphinxhline
\sphinxAtStartPar
P10
&
\sphinxAtStartPar
CSI1\_RX1+
&
\sphinxAtStartPar
ID\sphinxhyphen{}PHY
&
\sphinxAtStartPar
\sphinxhyphen{}
&
\sphinxAtStartPar
\sphinxhyphen{}
\\
\sphinxhline
\sphinxAtStartPar
P11
&
\sphinxAtStartPar
CSI1\_RX1\sphinxhyphen{}
&
\sphinxAtStartPar
ID\sphinxhyphen{}PHY
&
\sphinxAtStartPar
\sphinxhyphen{}
&
\sphinxAtStartPar
\sphinxhyphen{}
\\
\sphinxhline
\sphinxAtStartPar
P12
&
\sphinxAtStartPar
GND
&
\sphinxAtStartPar
\sphinxhyphen{}
&
\sphinxAtStartPar
\sphinxhyphen{}
&
\sphinxAtStartPar
\sphinxhyphen{}
\\
\sphinxhline
\sphinxAtStartPar
P13
&
\sphinxAtStartPar
CSI1\_RX2+
&
\sphinxAtStartPar
ID\sphinxhyphen{}PHY
&
\sphinxAtStartPar
\sphinxhyphen{}
&
\sphinxAtStartPar
\sphinxhyphen{}
\\
\sphinxhline
\sphinxAtStartPar
P14
&
\sphinxAtStartPar
CSI1\_RX2\sphinxhyphen{}
&
\sphinxAtStartPar
ID\sphinxhyphen{}PHY
&
\sphinxAtStartPar
\sphinxhyphen{}
&
\sphinxAtStartPar
\sphinxhyphen{}
\\
\sphinxhline
\sphinxAtStartPar
P15
&
\sphinxAtStartPar
GND
&
\sphinxAtStartPar
\sphinxhyphen{}
&
\sphinxAtStartPar
\sphinxhyphen{}
&
\sphinxAtStartPar
\sphinxhyphen{}
\\
\sphinxhline
\sphinxAtStartPar
P16
&
\sphinxAtStartPar
CSI1\_RX3+
&
\sphinxAtStartPar
ID\sphinxhyphen{}PHY
&
\sphinxAtStartPar
\sphinxhyphen{}
&
\sphinxAtStartPar
\sphinxhyphen{}
\\
\sphinxhline
\sphinxAtStartPar
P17
&
\sphinxAtStartPar
CSI1\_RX3\sphinxhyphen{}
&
\sphinxAtStartPar
ID\sphinxhyphen{}PHY
&
\sphinxAtStartPar
\sphinxhyphen{}
&
\sphinxAtStartPar
\sphinxhyphen{}
\\
\sphinxhline
\sphinxAtStartPar
P18
&
\sphinxAtStartPar
GND
&
\sphinxAtStartPar
\sphinxhyphen{}
&
\sphinxAtStartPar
\sphinxhyphen{}
&
\sphinxAtStartPar
\sphinxhyphen{}
\\
\sphinxhline
\sphinxAtStartPar
P19
&
\sphinxAtStartPar
GBE0\_MDI3\sphinxhyphen{}
&
\sphinxAtStartPar
I/O GBE MDI
&
\sphinxAtStartPar
\sphinxhyphen{}
&
\sphinxAtStartPar
\sphinxhyphen{}
\\
\sphinxhline
\sphinxAtStartPar
P20
&
\sphinxAtStartPar
GBE0\_MDI3+
&
\sphinxAtStartPar
I/O GBE MDI
&
\sphinxAtStartPar
\sphinxhyphen{}
&
\sphinxAtStartPar
\sphinxhyphen{}
\\
\sphinxhline
\sphinxAtStartPar
P21
&
\sphinxAtStartPar
GBE0\_LINK100\#
&
\sphinxAtStartPar
O OD CMOS
&
\sphinxAtStartPar
3.3V
&
\sphinxAtStartPar
\sphinxhyphen{}
\\
\sphinxhline
\sphinxAtStartPar
P22
&
\sphinxAtStartPar
GBE0\_LINK1000\#
&
\sphinxAtStartPar
O OD CMOS
&
\sphinxAtStartPar
3.3V
&
\sphinxAtStartPar
\sphinxhyphen{}
\\
\sphinxhline
\sphinxAtStartPar
P23
&
\sphinxAtStartPar
GBE0\_MDI2\sphinxhyphen{}
&
\sphinxAtStartPar
I/O GBE MDI
&
\sphinxAtStartPar
\sphinxhyphen{}
&
\sphinxAtStartPar
\sphinxhyphen{}
\\
\sphinxhline
\sphinxAtStartPar
P24
&
\sphinxAtStartPar
GBE0\_MDI2+
&
\sphinxAtStartPar
I/O GBEMDI
&
\sphinxAtStartPar
\sphinxhyphen{}
&
\sphinxAtStartPar
\sphinxhyphen{}
\\
\sphinxhline
\sphinxAtStartPar
P25
&
\sphinxAtStartPar
GBE0\_LINK\_ACT\#
&
\sphinxAtStartPar
O OD CMOS
&
\sphinxAtStartPar
3.3V
&
\sphinxAtStartPar
\sphinxhyphen{}
\\
\sphinxhline
\sphinxAtStartPar
P26
&
\sphinxAtStartPar
GBE0\_MDI1\sphinxhyphen{}
&
\sphinxAtStartPar
I/O GBE MDI
&
\sphinxAtStartPar
\sphinxhyphen{}
&
\sphinxAtStartPar
\sphinxhyphen{}
\\
\sphinxhline
\sphinxAtStartPar
P27
&
\sphinxAtStartPar
GBE0\_MDI1+
&
\sphinxAtStartPar
I/O GBE MDI
&
\sphinxAtStartPar
\sphinxhyphen{}
&
\sphinxAtStartPar
\sphinxhyphen{}
\\
\sphinxhline
\sphinxAtStartPar
P28
&
\sphinxAtStartPar
NC
&
\sphinxAtStartPar
\sphinxhyphen{}
&
\sphinxAtStartPar
\sphinxhyphen{}
&
\sphinxAtStartPar
\sphinxhyphen{}
\\
\sphinxhline
\sphinxAtStartPar
P29
&
\sphinxAtStartPar
GBE0\_MDI0\sphinxhyphen{}
&
\sphinxAtStartPar
I/O GBE MDI
&
\sphinxAtStartPar
\sphinxhyphen{}
&
\sphinxAtStartPar
\sphinxhyphen{}
\\
\sphinxhline
\sphinxAtStartPar
P30
&
\sphinxAtStartPar
GBE0\_MDI0+
&
\sphinxAtStartPar
I/O GBE MDI
&
\sphinxAtStartPar
\sphinxhyphen{}
&
\sphinxAtStartPar
\sphinxhyphen{}
\\
\sphinxhline
\sphinxAtStartPar
P31
&
\sphinxAtStartPar
SPI0\_CS1\#
&
\sphinxAtStartPar
O CMOS
&
\sphinxAtStartPar
1.8V
&
\sphinxAtStartPar
\sphinxhyphen{}
\\
\sphinxhline
\sphinxAtStartPar
P32
&
\sphinxAtStartPar
GND
&
\sphinxAtStartPar
\sphinxhyphen{}
&
\sphinxAtStartPar
\sphinxhyphen{}
&
\sphinxAtStartPar
\sphinxhyphen{}
\\
\sphinxhline
\sphinxAtStartPar
P33
&
\sphinxAtStartPar
SDIO\_WP
&
\sphinxAtStartPar
I OD CMOS
&
\sphinxAtStartPar
1.8V / 3.3V
&
\sphinxAtStartPar
PU 22k *\sphinxstyleemphasis{2}
\\
\sphinxhline
\sphinxAtStartPar
P34
&
\sphinxAtStartPar
SDIO\_CMD
&
\sphinxAtStartPar
I/O CMOS
&
\sphinxAtStartPar
1.8V / 3.3V
&
\sphinxAtStartPar
\sphinxhyphen{}
\\
\sphinxhline
\sphinxAtStartPar
P35
&
\sphinxAtStartPar
SDIO\_CD\#
&
\sphinxAtStartPar
I OD CMOS
&
\sphinxAtStartPar
1.8V / 3.3V
&
\sphinxAtStartPar
PU 22k *\sphinxstyleemphasis{2}
\\
\sphinxhline
\sphinxAtStartPar
P36
&
\sphinxAtStartPar
SDIO\_CK
&
\sphinxAtStartPar
O CMOS
&
\sphinxAtStartPar
1.8V / 3.3V
&
\sphinxAtStartPar
\sphinxhyphen{}
\\
\sphinxhline
\sphinxAtStartPar
P37
&
\sphinxAtStartPar
SDIO\_PWR\_EN
&
\sphinxAtStartPar
O CMOS
&
\sphinxAtStartPar
3.3V
&
\sphinxAtStartPar
PU 4k7
\\
\sphinxhline
\sphinxAtStartPar
P38
&
\sphinxAtStartPar
GND
&
\sphinxAtStartPar
\sphinxhyphen{}
&
\sphinxAtStartPar
\sphinxhyphen{}
&
\sphinxAtStartPar
\sphinxhyphen{}
\\
\sphinxhline
\sphinxAtStartPar
P39
&
\sphinxAtStartPar
SDIO\_D0
&
\sphinxAtStartPar
I/O CMOS
&
\sphinxAtStartPar
1.8V / 3.3V
&
\sphinxAtStartPar
\sphinxhyphen{}
\\
\sphinxhline
\sphinxAtStartPar
P40
&
\sphinxAtStartPar
SDIO\_D1
&
\sphinxAtStartPar
I/O CMOS
&
\sphinxAtStartPar
1.8V / 3.3V
&
\sphinxAtStartPar
\sphinxhyphen{}
\\
\sphinxhline
\sphinxAtStartPar
P41
&
\sphinxAtStartPar
SDIO\_D2
&
\sphinxAtStartPar
I/O CMOS
&
\sphinxAtStartPar
1.8V / 3.3V
&
\sphinxAtStartPar
\sphinxhyphen{}
\\
\sphinxhline
\sphinxAtStartPar
P42
&
\sphinxAtStartPar
SDIO\_D3
&
\sphinxAtStartPar
I/O CMOS
&
\sphinxAtStartPar
1.8V / 3.3V
&
\sphinxAtStartPar
\sphinxhyphen{}
\\
\sphinxhline
\sphinxAtStartPar
P43
&
\sphinxAtStartPar
SPI0\_CS0\#
&
\sphinxAtStartPar
O CMOS
&
\sphinxAtStartPar
1.8V
&
\sphinxAtStartPar
\sphinxhyphen{}
\\
\sphinxhline
\sphinxAtStartPar
P44
&
\sphinxAtStartPar
SPI0\_CK
&
\sphinxAtStartPar
O CMOS
&
\sphinxAtStartPar
1.8V
&
\sphinxAtStartPar
\sphinxhyphen{}
\\
\sphinxhline
\sphinxAtStartPar
P45
&
\sphinxAtStartPar
SPI0\_DIN
&
\sphinxAtStartPar
I CMOS
&
\sphinxAtStartPar
1.8V
&
\sphinxAtStartPar
\sphinxhyphen{}
\\
\sphinxhline
\sphinxAtStartPar
P46
&
\sphinxAtStartPar
SPI0\_DO
&
\sphinxAtStartPar
O CMOS
&
\sphinxAtStartPar
1.8V
&
\sphinxAtStartPar
\sphinxhyphen{}
\\
\sphinxhline
\sphinxAtStartPar
P47
&
\sphinxAtStartPar
GND
&
\sphinxAtStartPar
\sphinxhyphen{}
&
\sphinxAtStartPar
\sphinxhyphen{}
&
\sphinxAtStartPar
\sphinxhyphen{}
\\
\sphinxhline
\sphinxAtStartPar
P48
&
\sphinxAtStartPar
NC
&
\sphinxAtStartPar
\sphinxhyphen{}
&
\sphinxAtStartPar
\sphinxhyphen{}
&
\sphinxAtStartPar
\sphinxhyphen{}
\\
\sphinxhline
\sphinxAtStartPar
P49
&
\sphinxAtStartPar
NC
&
\sphinxAtStartPar
\sphinxhyphen{}
&
\sphinxAtStartPar
\sphinxhyphen{}
&
\sphinxAtStartPar
\sphinxhyphen{}
\\
\sphinxhline
\sphinxAtStartPar
P50
&
\sphinxAtStartPar
GND
&
\sphinxAtStartPar
\sphinxhyphen{}
&
\sphinxAtStartPar
\sphinxhyphen{}
&
\sphinxAtStartPar
\sphinxhyphen{}
\\
\sphinxhline
\sphinxAtStartPar
P51
&
\sphinxAtStartPar
NC
&
\sphinxAtStartPar
\sphinxhyphen{}
&
\sphinxAtStartPar
\sphinxhyphen{}
&
\sphinxAtStartPar
\sphinxhyphen{}
\\
\sphinxhline
\sphinxAtStartPar
P52
&
\sphinxAtStartPar
NC
&
\sphinxAtStartPar
\sphinxhyphen{}
&
\sphinxAtStartPar
\sphinxhyphen{}
&
\sphinxAtStartPar
\sphinxhyphen{}
\\
\sphinxhline
\sphinxAtStartPar
P53
&
\sphinxAtStartPar
GND
&
\sphinxAtStartPar
\sphinxhyphen{}
&
\sphinxAtStartPar
\sphinxhyphen{}
&
\sphinxAtStartPar
\sphinxhyphen{}
\\
\sphinxhline
\sphinxAtStartPar
P54
&
\sphinxAtStartPar
QSPI\_CS0\#
&
\sphinxAtStartPar
O CMOS
&
\sphinxAtStartPar
1.8V
&
\sphinxAtStartPar
\sphinxhyphen{}
\\
\sphinxhline
\sphinxAtStartPar
P55
&
\sphinxAtStartPar
QSPI\_CS1\#
&
\sphinxAtStartPar
O CMOS
&
\sphinxAtStartPar
1.8V
&
\sphinxAtStartPar
\sphinxhyphen{}
\\
\sphinxhline
\sphinxAtStartPar
P56
&
\sphinxAtStartPar
QSPI\_CK
&
\sphinxAtStartPar
O CMOS
&
\sphinxAtStartPar
1.8V
&
\sphinxAtStartPar
\sphinxhyphen{}
\\
\sphinxhline
\sphinxAtStartPar
P57
&
\sphinxAtStartPar
QSPI\_IO\_1
&
\sphinxAtStartPar
I/O CMOS
&
\sphinxAtStartPar
1.8V
&
\sphinxAtStartPar
\sphinxhyphen{}
\\
\sphinxhline
\sphinxAtStartPar
P58
&
\sphinxAtStartPar
QSPI\_IO\_0
&
\sphinxAtStartPar
O CMOS
&
\sphinxAtStartPar
1.8V
&
\sphinxAtStartPar
\sphinxhyphen{}
\\
\sphinxhline
\sphinxAtStartPar
P59
&
\sphinxAtStartPar
GND
&
\sphinxAtStartPar
\sphinxhyphen{}
&
\sphinxAtStartPar
\sphinxhyphen{}
&
\sphinxAtStartPar
\sphinxhyphen{}
\\
\sphinxhline
\sphinxAtStartPar
P60
&
\sphinxAtStartPar
USB0+
&
\sphinxAtStartPar
I/O USB
&
\sphinxAtStartPar
USB
&
\sphinxAtStartPar
\sphinxhyphen{}
\\
\sphinxhline
\sphinxAtStartPar
P61
&
\sphinxAtStartPar
USB0\sphinxhyphen{}
&
\sphinxAtStartPar
I/O USB
&
\sphinxAtStartPar
USB
&
\sphinxAtStartPar
\sphinxhyphen{}
\\
\sphinxhline
\sphinxAtStartPar
P62
&
\sphinxAtStartPar
USB0\_EN\_OC\#
&
\sphinxAtStartPar
I/O OD CMOS
&
\sphinxAtStartPar
3.3V
&
\sphinxAtStartPar
PU 10k
\\
\sphinxhline
\sphinxAtStartPar
P63
&
\sphinxAtStartPar
USB0\_VBUS\_DET
&
\sphinxAtStartPar
I USB VBUS 5V
&
\sphinxAtStartPar
USB VBUS 5V
&
\sphinxAtStartPar
\sphinxhyphen{}
\\
\sphinxhline
\sphinxAtStartPar
P64
&
\sphinxAtStartPar
USB0\_OTG\_ID
&
\sphinxAtStartPar
I OD CMOS
&
\sphinxAtStartPar
3.3
&
\sphinxAtStartPar
PU 10k
\\
\sphinxhline
\sphinxAtStartPar
P65
&
\sphinxAtStartPar
USB1+
&
\sphinxAtStartPar
I/O USB
&
\sphinxAtStartPar
USB
&
\sphinxAtStartPar
\sphinxhyphen{}
\\
\sphinxhline
\sphinxAtStartPar
P66
&
\sphinxAtStartPar
USB1\sphinxhyphen{}
&
\sphinxAtStartPar
I/O USB
&
\sphinxAtStartPar
USB
&
\sphinxAtStartPar
\sphinxhyphen{}
\\
\sphinxhline
\sphinxAtStartPar
P67
&
\sphinxAtStartPar
USB1\_EN\_OC\#
&
\sphinxAtStartPar
I/O OD CMOS
&
\sphinxAtStartPar
3.3V
&
\sphinxAtStartPar
PU 1k
\\
\sphinxhline
\sphinxAtStartPar
P68
&
\sphinxAtStartPar
GND
&
\sphinxAtStartPar
\sphinxhyphen{}
&
\sphinxAtStartPar
\sphinxhyphen{}
&
\sphinxAtStartPar
\sphinxhyphen{}
\\
\sphinxhline
\sphinxAtStartPar
P69
&
\sphinxAtStartPar
USB2+
&
\sphinxAtStartPar
I/O USB
&
\sphinxAtStartPar
USB
&
\sphinxAtStartPar
\sphinxhyphen{}
\\
\sphinxhline
\sphinxAtStartPar
P70
&
\sphinxAtStartPar
USB2\sphinxhyphen{}
&
\sphinxAtStartPar
I/O USB
&
\sphinxAtStartPar
USB
&
\sphinxAtStartPar
\sphinxhyphen{}
\\
\sphinxhline
\sphinxAtStartPar
P71
&
\sphinxAtStartPar
USB2\_EN\_OC\#
&
\sphinxAtStartPar
I/O OD CMOS
&
\sphinxAtStartPar
3.3V
&
\sphinxAtStartPar
PU 1k
\\
\sphinxhline
\sphinxAtStartPar
P72
&
\sphinxAtStartPar
NC
&
\sphinxAtStartPar
\sphinxhyphen{}
&
\sphinxAtStartPar
\sphinxhyphen{}
&
\sphinxAtStartPar
\sphinxhyphen{}
\\
\sphinxhline
\sphinxAtStartPar
P73
&
\sphinxAtStartPar
NC
&
\sphinxAtStartPar
\sphinxhyphen{}
&
\sphinxAtStartPar
\sphinxhyphen{}
&
\sphinxAtStartPar
\sphinxhyphen{}
\\
\sphinxhline
\sphinxAtStartPar
P74
&
\sphinxAtStartPar
USB3\_EN\_OC\#
&
\sphinxAtStartPar
I/O OD CMOS
&
\sphinxAtStartPar
3.3V
&
\sphinxAtStartPar
PU 1k
\\
\sphinxhline
\sphinxAtStartPar
**\sphinxstyleemphasis{Key}
&
\sphinxAtStartPar

&
\sphinxAtStartPar

&
\sphinxAtStartPar

&
\sphinxAtStartPar

\\
\sphinxhline
\sphinxAtStartPar
**\sphinxstyleemphasis{Key}
&
\sphinxAtStartPar

&
\sphinxAtStartPar

&
\sphinxAtStartPar

&
\sphinxAtStartPar

\\
\sphinxhline
\sphinxAtStartPar
**\sphinxstyleemphasis{Key}
&
\sphinxAtStartPar

&
\sphinxAtStartPar

&
\sphinxAtStartPar

&
\sphinxAtStartPar

\\
\sphinxhline
\sphinxAtStartPar
P75
&
\sphinxAtStartPar
PCIE\_A\_RST\#
&
\sphinxAtStartPar
O CMOS
&
\sphinxAtStartPar
3.3V
&
\sphinxAtStartPar
\sphinxhyphen{}
\\
\sphinxhline
\sphinxAtStartPar
P76
&
\sphinxAtStartPar
USB4\_EN\_OC\#
&
\sphinxAtStartPar
I/O OD CMOS
&
\sphinxAtStartPar
3.3V
&
\sphinxAtStartPar
PU 1k
\\
\sphinxhline
\sphinxAtStartPar
P77
&
\sphinxAtStartPar
NC
&
\sphinxAtStartPar
\sphinxhyphen{}
&
\sphinxAtStartPar

&
\sphinxAtStartPar
\sphinxhyphen{}
\\
\sphinxhline
\sphinxAtStartPar
P78
&
\sphinxAtStartPar
PCIE\_A\_CKREQ\#
&
\sphinxAtStartPar
I/O OD CMOS
&
\sphinxAtStartPar
3.3V
&
\sphinxAtStartPar
PU 10k
\\
\sphinxhline
\sphinxAtStartPar
P79
&
\sphinxAtStartPar
GND
&
\sphinxAtStartPar
\sphinxhyphen{}
&
\sphinxAtStartPar
\sphinxhyphen{}
&
\sphinxAtStartPar
\sphinxhyphen{}
\\
\sphinxhline
\sphinxAtStartPar
P80
&
\sphinxAtStartPar
NC
&
\sphinxAtStartPar
\sphinxhyphen{}
&
\sphinxAtStartPar
\sphinxhyphen{}
&
\sphinxAtStartPar
\sphinxhyphen{}
\\
\sphinxhline
\sphinxAtStartPar
P81
&
\sphinxAtStartPar
NC
&
\sphinxAtStartPar
\sphinxhyphen{}
&
\sphinxAtStartPar
\sphinxhyphen{}
&
\sphinxAtStartPar
\sphinxhyphen{}
\\
\sphinxhline
\sphinxAtStartPar
P82
&
\sphinxAtStartPar
GND
&
\sphinxAtStartPar
\sphinxhyphen{}
&
\sphinxAtStartPar
\sphinxhyphen{}
&
\sphinxAtStartPar
\sphinxhyphen{}
\\
\sphinxhline
\sphinxAtStartPar
P83
&
\sphinxAtStartPar
PCIE\_A\_REFCK+
&
\sphinxAtStartPar
O PCIE
&
\sphinxAtStartPar
\sphinxhyphen{}
&
\sphinxAtStartPar
\sphinxhyphen{}
\\
\sphinxhline
\sphinxAtStartPar
P84
&
\sphinxAtStartPar
PCIE\_A\_REFCK\sphinxhyphen{}
&
\sphinxAtStartPar
O PCIE
&
\sphinxAtStartPar
\sphinxhyphen{}
&
\sphinxAtStartPar
\sphinxhyphen{}
\\
\sphinxhline
\sphinxAtStartPar
P85
&
\sphinxAtStartPar
GND
&
\sphinxAtStartPar
\sphinxhyphen{}
&
\sphinxAtStartPar
\sphinxhyphen{}
&
\sphinxAtStartPar
\sphinxhyphen{}
\\
\sphinxhline
\sphinxAtStartPar
P86
&
\sphinxAtStartPar
PCIE\_A\_RX+
&
\sphinxAtStartPar
I PCIE
&
\sphinxAtStartPar
\sphinxhyphen{}
&
\sphinxAtStartPar
\sphinxhyphen{}
\\
\sphinxhline
\sphinxAtStartPar
P87
&
\sphinxAtStartPar
PCIE\_A\_RX\sphinxhyphen{}
&
\sphinxAtStartPar
I PCIE
&
\sphinxAtStartPar
\sphinxhyphen{}
&
\sphinxAtStartPar
\sphinxhyphen{}
\\
\sphinxhline
\sphinxAtStartPar
P88
&
\sphinxAtStartPar
GND
&
\sphinxAtStartPar
\sphinxhyphen{}
&
\sphinxAtStartPar
\sphinxhyphen{}
&
\sphinxAtStartPar
\sphinxhyphen{}
\\
\sphinxhline
\sphinxAtStartPar
P89
&
\sphinxAtStartPar
PCIE\_A\_TX+
&
\sphinxAtStartPar
O PCIE
&
\sphinxAtStartPar
\sphinxhyphen{}
&
\sphinxAtStartPar
\sphinxhyphen{}
\\
\sphinxhline
\sphinxAtStartPar
P90
&
\sphinxAtStartPar
PCIE\_A\_TX\sphinxhyphen{}
&
\sphinxAtStartPar
O PCIE
&
\sphinxAtStartPar
\sphinxhyphen{}
&
\sphinxAtStartPar
\sphinxhyphen{}
\\
\sphinxhline
\sphinxAtStartPar
P91
&
\sphinxAtStartPar
GND
&
\sphinxAtStartPar
\sphinxhyphen{}
&
\sphinxAtStartPar
\sphinxhyphen{}
&
\sphinxAtStartPar
\sphinxhyphen{}
\\
\sphinxhline
\sphinxAtStartPar
P92
&
\sphinxAtStartPar
HDMI\_D2+
&
\sphinxAtStartPar
O TMDS HDMI
&
\sphinxAtStartPar
\sphinxhyphen{}
&
\sphinxAtStartPar
\sphinxhyphen{}
\\
\sphinxhline
\sphinxAtStartPar
P93
&
\sphinxAtStartPar
HDMI\_D2\sphinxhyphen{}
&
\sphinxAtStartPar
O TMDS HDMI
&
\sphinxAtStartPar
\sphinxhyphen{}
&
\sphinxAtStartPar
\sphinxhyphen{}
\\
\sphinxhline
\sphinxAtStartPar
P94
&
\sphinxAtStartPar
GND
&
\sphinxAtStartPar
\sphinxhyphen{}
&
\sphinxAtStartPar
\sphinxhyphen{}
&
\sphinxAtStartPar
\sphinxhyphen{}
\\
\sphinxhline
\sphinxAtStartPar
P95
&
\sphinxAtStartPar
HDMI\_D1+
&
\sphinxAtStartPar
O TMDS HDMI
&
\sphinxAtStartPar
\sphinxhyphen{}
&
\sphinxAtStartPar
\sphinxhyphen{}
\\
\sphinxhline
\sphinxAtStartPar
P96
&
\sphinxAtStartPar
HDMI\_D1\sphinxhyphen{}
&
\sphinxAtStartPar
O TMDS HDMI
&
\sphinxAtStartPar
\sphinxhyphen{}
&
\sphinxAtStartPar
\sphinxhyphen{}
\\
\sphinxhline
\sphinxAtStartPar
P97
&
\sphinxAtStartPar
GND
&
\sphinxAtStartPar

&
\sphinxAtStartPar
\sphinxhyphen{}
&
\sphinxAtStartPar
\sphinxhyphen{}
\\
\sphinxhline
\sphinxAtStartPar
P98
&
\sphinxAtStartPar
HDMI\_D0+
&
\sphinxAtStartPar
O TMDS HDMI
&
\sphinxAtStartPar
\sphinxhyphen{}
&
\sphinxAtStartPar
\sphinxhyphen{}
\\
\sphinxhline
\sphinxAtStartPar
P99
&
\sphinxAtStartPar
HDMI\_D0\sphinxhyphen{}
&
\sphinxAtStartPar
O TMDS HDMI
&
\sphinxAtStartPar
\sphinxhyphen{}
&
\sphinxAtStartPar
\sphinxhyphen{}
\\
\sphinxhline
\sphinxAtStartPar
P100
&
\sphinxAtStartPar
GND
&
\sphinxAtStartPar

&
\sphinxAtStartPar
\sphinxhyphen{}
&
\sphinxAtStartPar
\sphinxhyphen{}
\\
\sphinxhline
\sphinxAtStartPar
P101
&
\sphinxAtStartPar
HDMI\_CK+
&
\sphinxAtStartPar
O TMDS HDMI
&
\sphinxAtStartPar
\sphinxhyphen{}
&
\sphinxAtStartPar
\sphinxhyphen{}
\\
\sphinxhline
\sphinxAtStartPar
P102
&
\sphinxAtStartPar
HDMI\_CK\sphinxhyphen{}
&
\sphinxAtStartPar
O TMDS HDMI
&
\sphinxAtStartPar
\sphinxhyphen{}
&
\sphinxAtStartPar
\sphinxhyphen{}
\\
\sphinxhline
\sphinxAtStartPar
P103
&
\sphinxAtStartPar
GND
&
\sphinxAtStartPar

&
\sphinxAtStartPar
\sphinxhyphen{}
&
\sphinxAtStartPar
\sphinxhyphen{}
\\
\sphinxhline
\sphinxAtStartPar
P104
&
\sphinxAtStartPar
HDMI\_HPD
&
\sphinxAtStartPar
I CMOS
&
\sphinxAtStartPar
1.8V
&
\sphinxAtStartPar
\sphinxhyphen{}
\\
\sphinxhline
\sphinxAtStartPar
P105
&
\sphinxAtStartPar
HDMI\_CTRL\_CK
&
\sphinxAtStartPar
I/O OD CMOS
&
\sphinxAtStartPar
1.8V
&
\sphinxAtStartPar
PU 22k *\sphinxstyleemphasis{2}
\\
\sphinxhline
\sphinxAtStartPar
P106
&
\sphinxAtStartPar
HDMI\_CTRL\_DAT
&
\sphinxAtStartPar
I/O OD CMOS
&
\sphinxAtStartPar
1.8V
&
\sphinxAtStartPar
PU 22k *\sphinxstyleemphasis{2}
\\
\sphinxhline
\sphinxAtStartPar
P107
&
\sphinxAtStartPar
DP1\_AUX\_SEL
&
\sphinxAtStartPar
I/O CMOS
&
\sphinxAtStartPar
1.8V
&
\sphinxAtStartPar
\sphinxhyphen{} *\sphinxstyleemphasis{3}
\\
\sphinxhline
\sphinxAtStartPar
P108
&
\sphinxAtStartPar
GPIO0 / CAM0\_PWR\#
&
\sphinxAtStartPar
O CMOS
&
\sphinxAtStartPar
1.8V
&
\sphinxAtStartPar
\sphinxhyphen{} *\sphinxstyleemphasis{4}
\\
\sphinxhline
\sphinxAtStartPar
P109
&
\sphinxAtStartPar
GPIO1 / CAM1\_PWR\#
&
\sphinxAtStartPar
O CMOS
&
\sphinxAtStartPar
1.8V
&
\sphinxAtStartPar
\sphinxhyphen{} *\sphinxstyleemphasis{4}
\\
\sphinxhline
\sphinxAtStartPar
P110
&
\sphinxAtStartPar
GPIO2 / CAM0\_RST\#
&
\sphinxAtStartPar
O CMOS
&
\sphinxAtStartPar
1.8V
&
\sphinxAtStartPar
\sphinxhyphen{} *\sphinxstyleemphasis{4}
\\
\sphinxhline
\sphinxAtStartPar
P111
&
\sphinxAtStartPar
GPIO3 / CAM1\_RST\#
&
\sphinxAtStartPar
O CMOS
&
\sphinxAtStartPar
1.8V
&
\sphinxAtStartPar
\sphinxhyphen{} *\sphinxstyleemphasis{4}
\\
\sphinxhline
\sphinxAtStartPar
P112
&
\sphinxAtStartPar
GPIO4 / HDA\_RST\#
&
\sphinxAtStartPar
O CMOS
&
\sphinxAtStartPar
1.8V
&
\sphinxAtStartPar
\sphinxhyphen{} *\sphinxstyleemphasis{4}
\\
\sphinxhline
\sphinxAtStartPar
P113
&
\sphinxAtStartPar
GPIO5 / PWM\_OUT
&
\sphinxAtStartPar
O CMOS
&
\sphinxAtStartPar
1.8V
&
\sphinxAtStartPar
\sphinxhyphen{} *\sphinxstyleemphasis{4}
\\
\sphinxhline
\sphinxAtStartPar
P114
&
\sphinxAtStartPar
GPIO6 / TACHIN
&
\sphinxAtStartPar
I CMOS
&
\sphinxAtStartPar
1.8V
&
\sphinxAtStartPar
\sphinxhyphen{} *\sphinxstyleemphasis{4}
\\
\sphinxhline
\sphinxAtStartPar
P115
&
\sphinxAtStartPar
GPIO7
&
\sphinxAtStartPar
I/O CMOS
&
\sphinxAtStartPar
1.8V
&
\sphinxAtStartPar
\sphinxhyphen{} *\sphinxstyleemphasis{4}
\\
\sphinxhline
\sphinxAtStartPar
P116
&
\sphinxAtStartPar
GPIO8
&
\sphinxAtStartPar
I/O CMOS
&
\sphinxAtStartPar
1.8V
&
\sphinxAtStartPar
\sphinxhyphen{} *\sphinxstyleemphasis{4}
\\
\sphinxhline
\sphinxAtStartPar
P117
&
\sphinxAtStartPar
GPIO9
&
\sphinxAtStartPar
I/O CMOS
&
\sphinxAtStartPar
1.8V
&
\sphinxAtStartPar
\sphinxhyphen{} *\sphinxstyleemphasis{4}
\\
\sphinxhline
\sphinxAtStartPar
P118
&
\sphinxAtStartPar
GPIO10
&
\sphinxAtStartPar
I/O CMOS
&
\sphinxAtStartPar
1.8V
&
\sphinxAtStartPar
\sphinxhyphen{} *\sphinxstyleemphasis{4}
\\
\sphinxhline
\sphinxAtStartPar
P119
&
\sphinxAtStartPar
GPIO11
&
\sphinxAtStartPar
I/O CMOS
&
\sphinxAtStartPar
1.8V
&
\sphinxAtStartPar
\sphinxhyphen{} *\sphinxstyleemphasis{4}
\\
\sphinxhline
\sphinxAtStartPar
P120
&
\sphinxAtStartPar
GND
&
\sphinxAtStartPar
\sphinxhyphen{}
&
\sphinxAtStartPar
\sphinxhyphen{}
&
\sphinxAtStartPar
\sphinxhyphen{}
\\
\sphinxhline
\sphinxAtStartPar
P121
&
\sphinxAtStartPar
I2C\_PM\_CK
&
\sphinxAtStartPar
O OD CMOS
&
\sphinxAtStartPar
1.8V
&
\sphinxAtStartPar
PU 2k2
\\
\sphinxhline
\sphinxAtStartPar
P122
&
\sphinxAtStartPar
I2C\_PM\_DAT
&
\sphinxAtStartPar
I/O OD CMOS
&
\sphinxAtStartPar
1.8V
&
\sphinxAtStartPar
PU 2k2
\\
\sphinxhline
\sphinxAtStartPar
P123
&
\sphinxAtStartPar
BOOT\_SEL0\#
&
\sphinxAtStartPar
I OD CMOS
&
\sphinxAtStartPar
1.8V
&
\sphinxAtStartPar
PU 10k
\\
\sphinxhline
\sphinxAtStartPar
P124
&
\sphinxAtStartPar
BOOT\_SEL1\#
&
\sphinxAtStartPar
I OD CMOS
&
\sphinxAtStartPar
1.8V
&
\sphinxAtStartPar
PU 10k
\\
\sphinxhline
\sphinxAtStartPar
P125
&
\sphinxAtStartPar
BOOT\_SEL2\#
&
\sphinxAtStartPar
I OD CMOS
&
\sphinxAtStartPar
1.8V
&
\sphinxAtStartPar
PU 10k
\\
\sphinxhline
\sphinxAtStartPar
P126
&
\sphinxAtStartPar
RESET\_OUT\#
&
\sphinxAtStartPar
O CMOS
&
\sphinxAtStartPar
1.8V
&
\sphinxAtStartPar
\sphinxhyphen{}
\\
\sphinxhline
\sphinxAtStartPar
P127
&
\sphinxAtStartPar
RESET\_IN\#
&
\sphinxAtStartPar
I OD CMOS
&
\sphinxAtStartPar
1.8V
&
\sphinxAtStartPar
PU 100k
\\
\sphinxhline
\sphinxAtStartPar
P128
&
\sphinxAtStartPar
POWER\_BTN\#
&
\sphinxAtStartPar
I OD CMOS
&
\sphinxAtStartPar
1.8V
&
\sphinxAtStartPar
PU 100k
\\
\sphinxhline
\sphinxAtStartPar
P129
&
\sphinxAtStartPar
SER0\_TX
&
\sphinxAtStartPar
O CMOS
&
\sphinxAtStartPar
1.8V
&
\sphinxAtStartPar
\sphinxhyphen{}
\\
\sphinxhline
\sphinxAtStartPar
P130
&
\sphinxAtStartPar
SER0\_RX
&
\sphinxAtStartPar
I CMOS
&
\sphinxAtStartPar
1.8V
&
\sphinxAtStartPar
\sphinxhyphen{} *\sphinxstyleemphasis{4}
\\
\sphinxhline
\sphinxAtStartPar
P131
&
\sphinxAtStartPar
SER0\_RTS\#
&
\sphinxAtStartPar
O CMOS
&
\sphinxAtStartPar
1.8V
&
\sphinxAtStartPar
\sphinxhyphen{}
\\
\sphinxhline
\sphinxAtStartPar
P132
&
\sphinxAtStartPar
SER0\_CTS\#
&
\sphinxAtStartPar
I CMOS
&
\sphinxAtStartPar
1.8V
&
\sphinxAtStartPar
\sphinxhyphen{} *\sphinxstyleemphasis{4}
\\
\sphinxhline
\sphinxAtStartPar
P133
&
\sphinxAtStartPar
GND
&
\sphinxAtStartPar
\sphinxhyphen{}
&
\sphinxAtStartPar
\sphinxhyphen{}
&
\sphinxAtStartPar
\sphinxhyphen{}
\\
\sphinxhline
\sphinxAtStartPar
P134
&
\sphinxAtStartPar
SER1\_TX
&
\sphinxAtStartPar
O CMOS
&
\sphinxAtStartPar
1.8V
&
\sphinxAtStartPar
\sphinxhyphen{}
\\
\sphinxhline
\sphinxAtStartPar
P135
&
\sphinxAtStartPar
SER1\_RX
&
\sphinxAtStartPar
I CMOS
&
\sphinxAtStartPar
1.8V
&
\sphinxAtStartPar
\sphinxhyphen{} *\sphinxstyleemphasis{4}
\\
\sphinxhline
\sphinxAtStartPar
P136
&
\sphinxAtStartPar
SER2\_TX
&
\sphinxAtStartPar
O CMOS
&
\sphinxAtStartPar
1.8V
&
\sphinxAtStartPar
\sphinxhyphen{}
\\
\sphinxhline
\sphinxAtStartPar
P137
&
\sphinxAtStartPar
SER2\_RX
&
\sphinxAtStartPar
I CMOS
&
\sphinxAtStartPar
1.8V
&
\sphinxAtStartPar
\sphinxhyphen{} *\sphinxstyleemphasis{4}
\\
\sphinxhline
\sphinxAtStartPar
P138
&
\sphinxAtStartPar
SER2\_RTS\#
&
\sphinxAtStartPar
O CMOS
&
\sphinxAtStartPar
1.8V
&
\sphinxAtStartPar
\sphinxhyphen{}
\\
\sphinxhline
\sphinxAtStartPar
P139
&
\sphinxAtStartPar
SER2\_CTS\#
&
\sphinxAtStartPar
I CMOS
&
\sphinxAtStartPar
1.8V
&
\sphinxAtStartPar
\sphinxhyphen{} *\sphinxstyleemphasis{4}
\\
\sphinxhline
\sphinxAtStartPar
P140
&
\sphinxAtStartPar
SER3\_TX
&
\sphinxAtStartPar
O CMOS
&
\sphinxAtStartPar
1.8V
&
\sphinxAtStartPar
\sphinxhyphen{}
\\
\sphinxhline
\sphinxAtStartPar
P141
&
\sphinxAtStartPar
SER3\_RX
&
\sphinxAtStartPar
I CMOS
&
\sphinxAtStartPar
1.8V
&
\sphinxAtStartPar
\sphinxhyphen{} *\sphinxstyleemphasis{4}
\\
\sphinxhline
\sphinxAtStartPar
P142
&
\sphinxAtStartPar
GND
&
\sphinxAtStartPar
\sphinxhyphen{}
&
\sphinxAtStartPar
\sphinxhyphen{}
&
\sphinxAtStartPar
\sphinxhyphen{}
\\
\sphinxhline
\sphinxAtStartPar
P143
&
\sphinxAtStartPar
CAN0\_TX
&
\sphinxAtStartPar
O CMOS
&
\sphinxAtStartPar
1.8V
&
\sphinxAtStartPar
\sphinxhyphen{}
\\
\sphinxhline
\sphinxAtStartPar
P144
&
\sphinxAtStartPar
CAN0\_RX
&
\sphinxAtStartPar
I CMOS
&
\sphinxAtStartPar
1.8V
&
\sphinxAtStartPar
\sphinxhyphen{}
\\
\sphinxhline
\sphinxAtStartPar
P145
&
\sphinxAtStartPar
CAN1\_TX
&
\sphinxAtStartPar
O CMOS
&
\sphinxAtStartPar
1.8V
&
\sphinxAtStartPar
\sphinxhyphen{}
\\
\sphinxhline
\sphinxAtStartPar
P146
&
\sphinxAtStartPar
CAN1\_RX
&
\sphinxAtStartPar
I CMOS
&
\sphinxAtStartPar
1.8V
&
\sphinxAtStartPar
\sphinxhyphen{}
\\
\sphinxhline
\sphinxAtStartPar
P147
&
\sphinxAtStartPar
VDD\_IN
&
\sphinxAtStartPar
Analog
&
\sphinxAtStartPar
5V
&
\sphinxAtStartPar
\sphinxhyphen{}
\\
\sphinxhline
\sphinxAtStartPar
P148
&
\sphinxAtStartPar
VDD\_IN
&
\sphinxAtStartPar
Analog
&
\sphinxAtStartPar
5V
&
\sphinxAtStartPar
\sphinxhyphen{}
\\
\sphinxhline
\sphinxAtStartPar
P149
&
\sphinxAtStartPar
VDD\_IN
&
\sphinxAtStartPar
Analog
&
\sphinxAtStartPar
5V
&
\sphinxAtStartPar
\sphinxhyphen{}
\\
\sphinxhline
\sphinxAtStartPar
P150
&
\sphinxAtStartPar
VDD\_IN
&
\sphinxAtStartPar
Analog
&
\sphinxAtStartPar
5V
&
\sphinxAtStartPar
\sphinxhyphen{}
\\
\sphinxhline
\sphinxAtStartPar
P151
&
\sphinxAtStartPar
VDD\_IN
&
\sphinxAtStartPar
Analog
&
\sphinxAtStartPar
5V
&
\sphinxAtStartPar
\sphinxhyphen{}
\\
\sphinxhline
\sphinxAtStartPar
P152
&
\sphinxAtStartPar
VDD\_IN
&
\sphinxAtStartPar
Analog
&
\sphinxAtStartPar
5V
&
\sphinxAtStartPar
\sphinxhyphen{}
\\
\sphinxhline
\sphinxAtStartPar
P153
&
\sphinxAtStartPar
VDD\_IN
&
\sphinxAtStartPar
Analog
&
\sphinxAtStartPar
5V
&
\sphinxAtStartPar
\sphinxhyphen{}
\\
\sphinxhline
\sphinxAtStartPar
P154
&
\sphinxAtStartPar
VDD\_IN
&
\sphinxAtStartPar
Analog
&
\sphinxAtStartPar
5V
&
\sphinxAtStartPar
\sphinxhyphen{}
\\
\sphinxhline
\sphinxAtStartPar
P155
&
\sphinxAtStartPar
VDD\_IN
&
\sphinxAtStartPar
Analog
&
\sphinxAtStartPar
5V
&
\sphinxAtStartPar
\sphinxhyphen{}
\\
\sphinxhline
\sphinxAtStartPar
P156
&
\sphinxAtStartPar
VDD\_IN
&
\sphinxAtStartPar
Analog
&
\sphinxAtStartPar
5V
&
\sphinxAtStartPar
\sphinxhyphen{}
\\
\sphinxbottomrule
\end{longtable}
\sphinxtableafterendhook
\sphinxatlongtableend
\end{savenotes}


\subsection{3.2.2 SMARC S\sphinxhyphen{}PIN Connector Pin Out List}
\label{\detokenize{hardware:smarc-s-pin-connector-pin-out-list}}
\sphinxAtStartPar
Table 3\sphinxhyphen{}2 SMARC S\sphinxhyphen{}PIN Connector Pin Output


\begin{savenotes}
\sphinxatlongtablestart
\sphinxthistablewithglobalstyle
\makeatletter
  \LTleft \@totalleftmargin plus1fill
  \LTright\dimexpr\columnwidth-\@totalleftmargin-\linewidth\relax plus1fill
\makeatother
\begin{longtable}{lllll}
\sphinxtoprule
\sphinxstyletheadfamily 
\sphinxAtStartPar
\sphinxstylestrong{PIN}
&\sphinxstyletheadfamily 
\sphinxAtStartPar
\sphinxstylestrong{Secondary (BOTTOM) Side}
&\sphinxstyletheadfamily 
\sphinxAtStartPar
\sphinxstylestrong{I/O Type}
&\sphinxstyletheadfamily 
\sphinxAtStartPar
\sphinxstylestrong{I/O Level}
&\sphinxstyletheadfamily 
\sphinxAtStartPar
\sphinxstylestrong{PD / PU}
\\
\sphinxmidrule
\endfirsthead

\multicolumn{5}{c}{\sphinxnorowcolor
    \makebox[0pt]{\sphinxtablecontinued{\tablename\ \thetable{} \textendash{} continued from previous page}}%
}\\
\sphinxtoprule
\sphinxstyletheadfamily 
\sphinxAtStartPar
\sphinxstylestrong{PIN}
&\sphinxstyletheadfamily 
\sphinxAtStartPar
\sphinxstylestrong{Secondary (BOTTOM) Side}
&\sphinxstyletheadfamily 
\sphinxAtStartPar
\sphinxstylestrong{I/O Type}
&\sphinxstyletheadfamily 
\sphinxAtStartPar
\sphinxstylestrong{I/O Level}
&\sphinxstyletheadfamily 
\sphinxAtStartPar
\sphinxstylestrong{PD / PU}
\\
\sphinxmidrule
\endhead

\sphinxbottomrule
\multicolumn{5}{r}{\sphinxnorowcolor
    \makebox[0pt][r]{\sphinxtablecontinued{continues on next page}}%
}\\
\endfoot

\endlastfoot
\sphinxtableatstartofbodyhook

\sphinxAtStartPar
S1
&
\sphinxAtStartPar
I2C\_CAM1\_CK
&
\sphinxAtStartPar
IO OD CMOS
&
\sphinxAtStartPar
1.8V
&
\sphinxAtStartPar
PU 2k2
\\
\sphinxhline
\sphinxAtStartPar
S2
&
\sphinxAtStartPar
I2C\_CAM1\_DAT
&
\sphinxAtStartPar
IO OD CMOS
&
\sphinxAtStartPar
1.8V
&
\sphinxAtStartPar
PU 2k2
\\
\sphinxhline
\sphinxAtStartPar
S3
&
\sphinxAtStartPar
GND
&
\sphinxAtStartPar
\sphinxhyphen{}
&
\sphinxAtStartPar
\sphinxhyphen{}
&
\sphinxAtStartPar
\sphinxhyphen{}
\\
\sphinxhline
\sphinxAtStartPar
S4
&
\sphinxAtStartPar
NC
&
\sphinxAtStartPar
\sphinxhyphen{}
&
\sphinxAtStartPar
\sphinxhyphen{}
&
\sphinxAtStartPar
\sphinxhyphen{}
\\
\sphinxhline
\sphinxAtStartPar
S5
&
\sphinxAtStartPar
I2C\_CAM0\_CK
&
\sphinxAtStartPar
IO OD CMOS
&
\sphinxAtStartPar
1.8V
&
\sphinxAtStartPar
PU 2k2
\\
\sphinxhline
\sphinxAtStartPar
S6
&
\sphinxAtStartPar
CAM\_MCK
&
\sphinxAtStartPar
O CMOS
&
\sphinxAtStartPar
1.8V
&
\sphinxAtStartPar
\sphinxhyphen{}
\\
\sphinxhline
\sphinxAtStartPar
S7
&
\sphinxAtStartPar
I2C\_CAM0\_DAT
&
\sphinxAtStartPar
IO OD CMOS
&
\sphinxAtStartPar
1.8V
&
\sphinxAtStartPar
PU 2k2
\\
\sphinxhline
\sphinxAtStartPar
S8
&
\sphinxAtStartPar
CSI0\_CK+
&
\sphinxAtStartPar
I D\sphinxhyphen{}PHY
&
\sphinxAtStartPar
\sphinxhyphen{}
&
\sphinxAtStartPar
\sphinxhyphen{}
\\
\sphinxhline
\sphinxAtStartPar
S9
&
\sphinxAtStartPar
CSI0\_CK\sphinxhyphen{}
&
\sphinxAtStartPar
I D\sphinxhyphen{}PHY
&
\sphinxAtStartPar
\sphinxhyphen{}
&
\sphinxAtStartPar
\sphinxhyphen{}
\\
\sphinxhline
\sphinxAtStartPar
S10
&
\sphinxAtStartPar
GND
&
\sphinxAtStartPar
\sphinxhyphen{}
&
\sphinxAtStartPar
\sphinxhyphen{}
&
\sphinxAtStartPar
\sphinxhyphen{}
\\
\sphinxhline
\sphinxAtStartPar
S11
&
\sphinxAtStartPar
CSI0\_RX0+
&
\sphinxAtStartPar
I D\sphinxhyphen{}PHY
&
\sphinxAtStartPar
\sphinxhyphen{}
&
\sphinxAtStartPar
\sphinxhyphen{}
\\
\sphinxhline
\sphinxAtStartPar
S12
&
\sphinxAtStartPar
CSI0\_RX0\sphinxhyphen{}
&
\sphinxAtStartPar
I D\sphinxhyphen{}PHY
&
\sphinxAtStartPar
\sphinxhyphen{}
&
\sphinxAtStartPar
\sphinxhyphen{}
\\
\sphinxhline
\sphinxAtStartPar
S13
&
\sphinxAtStartPar
GND
&
\sphinxAtStartPar
\sphinxhyphen{}
&
\sphinxAtStartPar
\sphinxhyphen{}
&
\sphinxAtStartPar
\sphinxhyphen{}
\\
\sphinxhline
\sphinxAtStartPar
S14
&
\sphinxAtStartPar
CSI0\_RX1+
&
\sphinxAtStartPar
I D\sphinxhyphen{}PHY
&
\sphinxAtStartPar
\sphinxhyphen{}
&
\sphinxAtStartPar
\sphinxhyphen{}
\\
\sphinxhline
\sphinxAtStartPar
S15
&
\sphinxAtStartPar
CSI0\_RX1\sphinxhyphen{}
&
\sphinxAtStartPar
I D\sphinxhyphen{}PHY
&
\sphinxAtStartPar
\sphinxhyphen{}
&
\sphinxAtStartPar
\sphinxhyphen{}
\\
\sphinxhline
\sphinxAtStartPar
S16
&
\sphinxAtStartPar
GND
&
\sphinxAtStartPar
\sphinxhyphen{}
&
\sphinxAtStartPar
\sphinxhyphen{}
&
\sphinxAtStartPar
\sphinxhyphen{}
\\
\sphinxhline
\sphinxAtStartPar
S17
&
\sphinxAtStartPar
GBE1\_MDI0+
&
\sphinxAtStartPar
I/O GBE MDI
&
\sphinxAtStartPar
\sphinxhyphen{}
&
\sphinxAtStartPar
\sphinxhyphen{}
\\
\sphinxhline
\sphinxAtStartPar
S18
&
\sphinxAtStartPar
GBE1\_MDI0\sphinxhyphen{}
&
\sphinxAtStartPar
I/O GBE MDI
&
\sphinxAtStartPar
\sphinxhyphen{}
&
\sphinxAtStartPar
\sphinxhyphen{}
\\
\sphinxhline
\sphinxAtStartPar
S19
&
\sphinxAtStartPar
GBE1\_LINK100\#
&
\sphinxAtStartPar
O OD CMOS
&
\sphinxAtStartPar
3.3V
&
\sphinxAtStartPar
\sphinxhyphen{}
\\
\sphinxhline
\sphinxAtStartPar
S20
&
\sphinxAtStartPar
GBE1\_MDI1+
&
\sphinxAtStartPar
I/O GBE MDI
&
\sphinxAtStartPar
\sphinxhyphen{}
&
\sphinxAtStartPar
\sphinxhyphen{}
\\
\sphinxhline
\sphinxAtStartPar
S21
&
\sphinxAtStartPar
GBE1\_MDI1\sphinxhyphen{}
&
\sphinxAtStartPar
I/O GBE MDI
&
\sphinxAtStartPar
\sphinxhyphen{}
&
\sphinxAtStartPar
\sphinxhyphen{}
\\
\sphinxhline
\sphinxAtStartPar
S22
&
\sphinxAtStartPar
GBE1\_LINK1000\#
&
\sphinxAtStartPar
O OD CMOS
&
\sphinxAtStartPar
3.3V
&
\sphinxAtStartPar
\sphinxhyphen{}
\\
\sphinxhline
\sphinxAtStartPar
S23
&
\sphinxAtStartPar
GBE1\_MDI2+
&
\sphinxAtStartPar
I/O GBE MDI
&
\sphinxAtStartPar
\sphinxhyphen{}
&
\sphinxAtStartPar
\sphinxhyphen{}
\\
\sphinxhline
\sphinxAtStartPar
S24
&
\sphinxAtStartPar
GBE1\_MDI2\sphinxhyphen{}
&
\sphinxAtStartPar
I/O GBE MDI
&
\sphinxAtStartPar
\sphinxhyphen{}
&
\sphinxAtStartPar
\sphinxhyphen{}
\\
\sphinxhline
\sphinxAtStartPar
S25
&
\sphinxAtStartPar
GND
&
\sphinxAtStartPar
\sphinxhyphen{}
&
\sphinxAtStartPar
\sphinxhyphen{}
&
\sphinxAtStartPar
\sphinxhyphen{}
\\
\sphinxhline
\sphinxAtStartPar
S26
&
\sphinxAtStartPar
GBE1\_MDI3+
&
\sphinxAtStartPar
I/O GBE MDI
&
\sphinxAtStartPar
\sphinxhyphen{}
&
\sphinxAtStartPar
\sphinxhyphen{}
\\
\sphinxhline
\sphinxAtStartPar
S27
&
\sphinxAtStartPar
GBE1\_MDI3\sphinxhyphen{}
&
\sphinxAtStartPar
I/O GBE MDI
&
\sphinxAtStartPar
\sphinxhyphen{}
&
\sphinxAtStartPar
\sphinxhyphen{}
\\
\sphinxhline
\sphinxAtStartPar
S28
&
\sphinxAtStartPar
NC
&
\sphinxAtStartPar
\sphinxhyphen{}
&
\sphinxAtStartPar
\sphinxhyphen{}
&
\sphinxAtStartPar
\sphinxhyphen{}
\\
\sphinxhline
\sphinxAtStartPar
S29
&
\sphinxAtStartPar
NC
&
\sphinxAtStartPar
\sphinxhyphen{}
&
\sphinxAtStartPar
\sphinxhyphen{}
&
\sphinxAtStartPar
\sphinxhyphen{}
\\
\sphinxhline
\sphinxAtStartPar
S30
&
\sphinxAtStartPar
NC
&
\sphinxAtStartPar
\sphinxhyphen{}
&
\sphinxAtStartPar
\sphinxhyphen{}
&
\sphinxAtStartPar
\sphinxhyphen{}
\\
\sphinxhline
\sphinxAtStartPar
S31
&
\sphinxAtStartPar
GBE1\_LINK\_ACT\#
&
\sphinxAtStartPar
O OD CMOS
&
\sphinxAtStartPar
3.3V
&
\sphinxAtStartPar
\sphinxhyphen{}
\\
\sphinxhline
\sphinxAtStartPar
S32
&
\sphinxAtStartPar
NC
&
\sphinxAtStartPar
\sphinxhyphen{}
&
\sphinxAtStartPar
\sphinxhyphen{}
&
\sphinxAtStartPar
\sphinxhyphen{}
\\
\sphinxhline
\sphinxAtStartPar
S33
&
\sphinxAtStartPar
NC
&
\sphinxAtStartPar
\sphinxhyphen{}
&
\sphinxAtStartPar
\sphinxhyphen{}
&
\sphinxAtStartPar
\sphinxhyphen{}
\\
\sphinxhline
\sphinxAtStartPar
S34
&
\sphinxAtStartPar
GND
&
\sphinxAtStartPar
\sphinxhyphen{}
&
\sphinxAtStartPar
\sphinxhyphen{}
&
\sphinxAtStartPar
\sphinxhyphen{}
\\
\sphinxhline
\sphinxAtStartPar
S35
&
\sphinxAtStartPar
USB4+
&
\sphinxAtStartPar
I/O USB
&
\sphinxAtStartPar
USB
&
\sphinxAtStartPar
\sphinxhyphen{}
\\
\sphinxhline
\sphinxAtStartPar
S36
&
\sphinxAtStartPar
USB4\sphinxhyphen{}
&
\sphinxAtStartPar
I/O USB
&
\sphinxAtStartPar
USB
&
\sphinxAtStartPar
\sphinxhyphen{}
\\
\sphinxhline
\sphinxAtStartPar
S37
&
\sphinxAtStartPar
NC
&
\sphinxAtStartPar
\sphinxhyphen{}
&
\sphinxAtStartPar
\sphinxhyphen{}
&
\sphinxAtStartPar
\sphinxhyphen{}
\\
\sphinxhline
\sphinxAtStartPar
S38
&
\sphinxAtStartPar
AUDIO\_MCK
&
\sphinxAtStartPar
O CMOS
&
\sphinxAtStartPar
1.8V
&
\sphinxAtStartPar
\sphinxhyphen{}
\\
\sphinxhline
\sphinxAtStartPar
S39
&
\sphinxAtStartPar
I2S0\_LRCK
&
\sphinxAtStartPar
I/O CMOS
&
\sphinxAtStartPar
1.8V
&
\sphinxAtStartPar
\sphinxhyphen{}
\\
\sphinxhline
\sphinxAtStartPar
S40
&
\sphinxAtStartPar
I2S0\_SDOUT
&
\sphinxAtStartPar
O CMOS
&
\sphinxAtStartPar
1.8V
&
\sphinxAtStartPar
\sphinxhyphen{}
\\
\sphinxhline
\sphinxAtStartPar
S41
&
\sphinxAtStartPar
I2S0\_SDIN
&
\sphinxAtStartPar
I CMOS
&
\sphinxAtStartPar
1.8V
&
\sphinxAtStartPar
\sphinxhyphen{}
\\
\sphinxhline
\sphinxAtStartPar
S42
&
\sphinxAtStartPar
I2S0\_CK
&
\sphinxAtStartPar
I/O CMOS
&
\sphinxAtStartPar
1.8V
&
\sphinxAtStartPar
\sphinxhyphen{}
\\
\sphinxhline
\sphinxAtStartPar
S43
&
\sphinxAtStartPar
NC
&
\sphinxAtStartPar
\sphinxhyphen{}
&
\sphinxAtStartPar
\sphinxhyphen{}
&
\sphinxAtStartPar
\sphinxhyphen{}
\\
\sphinxhline
\sphinxAtStartPar
S44
&
\sphinxAtStartPar
NC
&
\sphinxAtStartPar
\sphinxhyphen{}
&
\sphinxAtStartPar
\sphinxhyphen{}
&
\sphinxAtStartPar
\sphinxhyphen{}
\\
\sphinxhline
\sphinxAtStartPar
S45
&
\sphinxAtStartPar
NC
&
\sphinxAtStartPar
\sphinxhyphen{}
&
\sphinxAtStartPar
\sphinxhyphen{}
&
\sphinxAtStartPar
\sphinxhyphen{}
\\
\sphinxhline
\sphinxAtStartPar
S46
&
\sphinxAtStartPar
NC
&
\sphinxAtStartPar
\sphinxhyphen{}
&
\sphinxAtStartPar
\sphinxhyphen{}
&
\sphinxAtStartPar
\sphinxhyphen{}
\\
\sphinxhline
\sphinxAtStartPar
S47
&
\sphinxAtStartPar
GND
&
\sphinxAtStartPar
\sphinxhyphen{}
&
\sphinxAtStartPar
\sphinxhyphen{}
&
\sphinxAtStartPar
\sphinxhyphen{}
\\
\sphinxhline
\sphinxAtStartPar
S48
&
\sphinxAtStartPar
I2C\_GP\_CK
&
\sphinxAtStartPar
I/O OD CMOS
&
\sphinxAtStartPar
1.8V
&
\sphinxAtStartPar
PU 2k2
\\
\sphinxhline
\sphinxAtStartPar
S49
&
\sphinxAtStartPar
I2C\_GP\_DAT
&
\sphinxAtStartPar
I/O OD CMOS
&
\sphinxAtStartPar
1.8V
&
\sphinxAtStartPar
PU 2k2
\\
\sphinxhline
\sphinxAtStartPar
S50
&
\sphinxAtStartPar
I2S2\_LRCK *\sphinxstyleemphasis{5}
&
\sphinxAtStartPar
I/OCMOS
&
\sphinxAtStartPar
1.8V/1.5V
&
\sphinxAtStartPar
\sphinxhyphen{}
\\
\sphinxhline
\sphinxAtStartPar
S51
&
\sphinxAtStartPar
I2S2\_SDOUT *\sphinxstyleemphasis{5}
&
\sphinxAtStartPar
O CMOS
&
\sphinxAtStartPar
1.8V/1.5V
&
\sphinxAtStartPar
\sphinxhyphen{}
\\
\sphinxhline
\sphinxAtStartPar
S52
&
\sphinxAtStartPar
I2S2\_SDIN *\sphinxstyleemphasis{5}
&
\sphinxAtStartPar
I CMOS
&
\sphinxAtStartPar
1.8V/1.5V
&
\sphinxAtStartPar
\sphinxhyphen{}
\\
\sphinxhline
\sphinxAtStartPar
S53
&
\sphinxAtStartPar
I2S2\_CK *\sphinxstyleemphasis{5}
&
\sphinxAtStartPar
O CMOS
&
\sphinxAtStartPar
1.8V/1.5V
&
\sphinxAtStartPar
\sphinxhyphen{}
\\
\sphinxhline
\sphinxAtStartPar
S54
&
\sphinxAtStartPar
NC
&
\sphinxAtStartPar
\sphinxhyphen{}
&
\sphinxAtStartPar
\sphinxhyphen{}
&
\sphinxAtStartPar
\sphinxhyphen{}
\\
\sphinxhline
\sphinxAtStartPar
S55
&
\sphinxAtStartPar
USB5\_EN\_OC\#
&
\sphinxAtStartPar
I/OODCMOS
&
\sphinxAtStartPar
3.3V
&
\sphinxAtStartPar
PU 1k
\\
\sphinxhline
\sphinxAtStartPar
S56
&
\sphinxAtStartPar
QSPI\_IO\_2
&
\sphinxAtStartPar
I/OCMOS
&
\sphinxAtStartPar
1.8V
&
\sphinxAtStartPar
\sphinxhyphen{}
\\
\sphinxhline
\sphinxAtStartPar
S57
&
\sphinxAtStartPar
QSPI\_IO\_3
&
\sphinxAtStartPar
I/OCMOS
&
\sphinxAtStartPar
1.8V
&
\sphinxAtStartPar
\sphinxhyphen{}
\\
\sphinxhline
\sphinxAtStartPar
S58
&
\sphinxAtStartPar
NC
&
\sphinxAtStartPar
\sphinxhyphen{}
&
\sphinxAtStartPar
\sphinxhyphen{}
&
\sphinxAtStartPar
\sphinxhyphen{}
\\
\sphinxhline
\sphinxAtStartPar
S59
&
\sphinxAtStartPar
USB5+ *\sphinxstyleemphasis{6}
&
\sphinxAtStartPar
I/O USB
&
\sphinxAtStartPar
USB
&
\sphinxAtStartPar
\sphinxhyphen{}
\\
\sphinxhline
\sphinxAtStartPar
S60
&
\sphinxAtStartPar
USB5\sphinxhyphen{} *\sphinxstyleemphasis{6}
&
\sphinxAtStartPar
I/O USB
&
\sphinxAtStartPar
USB
&
\sphinxAtStartPar
\sphinxhyphen{}
\\
\sphinxhline
\sphinxAtStartPar
S61
&
\sphinxAtStartPar
GND
&
\sphinxAtStartPar
\sphinxhyphen{}
&
\sphinxAtStartPar
\sphinxhyphen{}
&
\sphinxAtStartPar
\sphinxhyphen{}
\\
\sphinxhline
\sphinxAtStartPar
S62
&
\sphinxAtStartPar
USB3\_SSTX+ *\sphinxstyleemphasis{6}
&
\sphinxAtStartPar
O USB SS
&
\sphinxAtStartPar
USB SS
&
\sphinxAtStartPar
\sphinxhyphen{}
\\
\sphinxhline
\sphinxAtStartPar
S63
&
\sphinxAtStartPar
USB3\_SSTX\sphinxhyphen{} *\sphinxstyleemphasis{6}
&
\sphinxAtStartPar
O USB SS
&
\sphinxAtStartPar
USB SS
&
\sphinxAtStartPar
\sphinxhyphen{}
\\
\sphinxhline
\sphinxAtStartPar
S64
&
\sphinxAtStartPar
GND
&
\sphinxAtStartPar
\sphinxhyphen{}
&
\sphinxAtStartPar
\sphinxhyphen{}
&
\sphinxAtStartPar
\sphinxhyphen{}
\\
\sphinxhline
\sphinxAtStartPar
S65
&
\sphinxAtStartPar
USB3\_SSRX+ *\sphinxstyleemphasis{6}
&
\sphinxAtStartPar
I USB SS
&
\sphinxAtStartPar
USB SS
&
\sphinxAtStartPar
\sphinxhyphen{}
\\
\sphinxhline
\sphinxAtStartPar
S66
&
\sphinxAtStartPar
USB3\_SSRX\sphinxhyphen{} *\sphinxstyleemphasis{6}
&
\sphinxAtStartPar
I USB SS
&
\sphinxAtStartPar
USB SS
&
\sphinxAtStartPar
\sphinxhyphen{}
\\
\sphinxhline
\sphinxAtStartPar
S67
&
\sphinxAtStartPar
GND
&
\sphinxAtStartPar
\sphinxhyphen{}
&
\sphinxAtStartPar
\sphinxhyphen{}
&
\sphinxAtStartPar
\sphinxhyphen{}
\\
\sphinxhline
\sphinxAtStartPar
S68
&
\sphinxAtStartPar
USB3+ *\sphinxstyleemphasis{6}
&
\sphinxAtStartPar
I/O USB
&
\sphinxAtStartPar
USB
&
\sphinxAtStartPar
\sphinxhyphen{}
\\
\sphinxhline
\sphinxAtStartPar
S69
&
\sphinxAtStartPar
USB3\sphinxhyphen{} *\sphinxstyleemphasis{6}
&
\sphinxAtStartPar
I/O USB
&
\sphinxAtStartPar
USB
&
\sphinxAtStartPar
\sphinxhyphen{}
\\
\sphinxhline
\sphinxAtStartPar
S70
&
\sphinxAtStartPar
GND
&
\sphinxAtStartPar
\sphinxhyphen{}
&
\sphinxAtStartPar
\sphinxhyphen{}
&
\sphinxAtStartPar
\sphinxhyphen{}
\\
\sphinxhline
\sphinxAtStartPar
S71
&
\sphinxAtStartPar
USB2\_SSTX+ *\sphinxstyleemphasis{6}
&
\sphinxAtStartPar
O USB SS
&
\sphinxAtStartPar
USB SS
&
\sphinxAtStartPar
\sphinxhyphen{}
\\
\sphinxhline
\sphinxAtStartPar
S72
&
\sphinxAtStartPar
USB2\_SSTX\sphinxhyphen{} *\sphinxstyleemphasis{6}
&
\sphinxAtStartPar
O USB SS
&
\sphinxAtStartPar
USB SS
&
\sphinxAtStartPar
\sphinxhyphen{}
\\
\sphinxhline
\sphinxAtStartPar
S73
&
\sphinxAtStartPar
GND
&
\sphinxAtStartPar
\sphinxhyphen{}
&
\sphinxAtStartPar
\sphinxhyphen{}
&
\sphinxAtStartPar
\sphinxhyphen{}
\\
\sphinxhline
\sphinxAtStartPar
S74
&
\sphinxAtStartPar
USB2\_SSRX+ *\sphinxstyleemphasis{6}
&
\sphinxAtStartPar
I USB SS
&
\sphinxAtStartPar
USB SS
&
\sphinxAtStartPar
\sphinxhyphen{}
\\
\sphinxhline
\sphinxAtStartPar
S75
&
\sphinxAtStartPar
USB2\_SSRX\sphinxhyphen{} *\sphinxstyleemphasis{6}
&
\sphinxAtStartPar
I USB SS
&
\sphinxAtStartPar
USB SS
&
\sphinxAtStartPar
\sphinxhyphen{}
\\
\sphinxhline
\sphinxAtStartPar
**\sphinxstyleemphasis{Key}
&
\sphinxAtStartPar

&
\sphinxAtStartPar

&
\sphinxAtStartPar

&
\sphinxAtStartPar

\\
\sphinxhline
\sphinxAtStartPar
**\sphinxstyleemphasis{Key}
&
\sphinxAtStartPar

&
\sphinxAtStartPar

&
\sphinxAtStartPar

&
\sphinxAtStartPar

\\
\sphinxhline
\sphinxAtStartPar
**\sphinxstyleemphasis{Key}
&
\sphinxAtStartPar

&
\sphinxAtStartPar

&
\sphinxAtStartPar

&
\sphinxAtStartPar

\\
\sphinxhline
\sphinxAtStartPar
S76
&
\sphinxAtStartPar
NC
&
\sphinxAtStartPar
\sphinxhyphen{}
&
\sphinxAtStartPar
\sphinxhyphen{}
&
\sphinxAtStartPar
\sphinxhyphen{}
\\
\sphinxhline
\sphinxAtStartPar
S77
&
\sphinxAtStartPar
NC
&
\sphinxAtStartPar
\sphinxhyphen{}
&
\sphinxAtStartPar
\sphinxhyphen{}
&
\sphinxAtStartPar
\sphinxhyphen{}
\\
\sphinxhline
\sphinxAtStartPar
S78
&
\sphinxAtStartPar
NC
&
\sphinxAtStartPar
\sphinxhyphen{}
&
\sphinxAtStartPar
\sphinxhyphen{}
&
\sphinxAtStartPar
\sphinxhyphen{}
\\
\sphinxhline
\sphinxAtStartPar
S79
&
\sphinxAtStartPar
NC
&
\sphinxAtStartPar
\sphinxhyphen{}
&
\sphinxAtStartPar
\sphinxhyphen{}
&
\sphinxAtStartPar
\sphinxhyphen{}
\\
\sphinxhline
\sphinxAtStartPar
S80
&
\sphinxAtStartPar
GND
&
\sphinxAtStartPar
\sphinxhyphen{}
&
\sphinxAtStartPar
\sphinxhyphen{}
&
\sphinxAtStartPar
\sphinxhyphen{}
\\
\sphinxhline
\sphinxAtStartPar
S81
&
\sphinxAtStartPar
NC
&
\sphinxAtStartPar
\sphinxhyphen{}
&
\sphinxAtStartPar
\sphinxhyphen{}
&
\sphinxAtStartPar
\sphinxhyphen{}
\\
\sphinxhline
\sphinxAtStartPar
S82
&
\sphinxAtStartPar
NC
&
\sphinxAtStartPar
\sphinxhyphen{}
&
\sphinxAtStartPar
\sphinxhyphen{}
&
\sphinxAtStartPar
\sphinxhyphen{}
\\
\sphinxhline
\sphinxAtStartPar
S83
&
\sphinxAtStartPar
GND
&
\sphinxAtStartPar
\sphinxhyphen{}
&
\sphinxAtStartPar
\sphinxhyphen{}
&
\sphinxAtStartPar
\sphinxhyphen{}
\\
\sphinxhline
\sphinxAtStartPar
S84
&
\sphinxAtStartPar
NC
&
\sphinxAtStartPar
\sphinxhyphen{}
&
\sphinxAtStartPar
\sphinxhyphen{}
&
\sphinxAtStartPar
\sphinxhyphen{}
\\
\sphinxhline
\sphinxAtStartPar
S85
&
\sphinxAtStartPar
NC
&
\sphinxAtStartPar
\sphinxhyphen{}
&
\sphinxAtStartPar
\sphinxhyphen{}
&
\sphinxAtStartPar
\sphinxhyphen{}
\\
\sphinxhline
\sphinxAtStartPar
S86
&
\sphinxAtStartPar
GND
&
\sphinxAtStartPar
\sphinxhyphen{}
&
\sphinxAtStartPar
\sphinxhyphen{}
&
\sphinxAtStartPar
\sphinxhyphen{}
\\
\sphinxhline
\sphinxAtStartPar
S87
&
\sphinxAtStartPar
NC
&
\sphinxAtStartPar
\sphinxhyphen{}
&
\sphinxAtStartPar
\sphinxhyphen{}
&
\sphinxAtStartPar
\sphinxhyphen{}
\\
\sphinxhline
\sphinxAtStartPar
S88
&
\sphinxAtStartPar
NC
&
\sphinxAtStartPar
\sphinxhyphen{}
&
\sphinxAtStartPar
\sphinxhyphen{}
&
\sphinxAtStartPar
\sphinxhyphen{}
\\
\sphinxhline
\sphinxAtStartPar
S89
&
\sphinxAtStartPar
GND
&
\sphinxAtStartPar
\sphinxhyphen{}
&
\sphinxAtStartPar
\sphinxhyphen{}
&
\sphinxAtStartPar
\sphinxhyphen{}
\\
\sphinxhline
\sphinxAtStartPar
S90
&
\sphinxAtStartPar
NC
&
\sphinxAtStartPar
\sphinxhyphen{}
&
\sphinxAtStartPar
\sphinxhyphen{}
&
\sphinxAtStartPar
\sphinxhyphen{}
\\
\sphinxhline
\sphinxAtStartPar
S91
&
\sphinxAtStartPar
NC
&
\sphinxAtStartPar
\sphinxhyphen{}
&
\sphinxAtStartPar
\sphinxhyphen{}
&
\sphinxAtStartPar
\sphinxhyphen{}
\\
\sphinxhline
\sphinxAtStartPar
S92
&
\sphinxAtStartPar
GND
&
\sphinxAtStartPar
\sphinxhyphen{}
&
\sphinxAtStartPar
\sphinxhyphen{}
&
\sphinxAtStartPar
\sphinxhyphen{}
\\
\sphinxhline
\sphinxAtStartPar
S93
&
\sphinxAtStartPar
NC
&
\sphinxAtStartPar
\sphinxhyphen{}
&
\sphinxAtStartPar
\sphinxhyphen{}
&
\sphinxAtStartPar
\sphinxhyphen{}
\\
\sphinxhline
\sphinxAtStartPar
S94
&
\sphinxAtStartPar
NC
&
\sphinxAtStartPar
\sphinxhyphen{}
&
\sphinxAtStartPar
\sphinxhyphen{}
&
\sphinxAtStartPar
\sphinxhyphen{}
\\
\sphinxhline
\sphinxAtStartPar
S95
&
\sphinxAtStartPar
NC
&
\sphinxAtStartPar
\sphinxhyphen{}
&
\sphinxAtStartPar
\sphinxhyphen{}
&
\sphinxAtStartPar
\sphinxhyphen{}
\\
\sphinxhline
\sphinxAtStartPar
S96
&
\sphinxAtStartPar
NC
&
\sphinxAtStartPar
\sphinxhyphen{}
&
\sphinxAtStartPar
\sphinxhyphen{}
&
\sphinxAtStartPar
\sphinxhyphen{}
\\
\sphinxhline
\sphinxAtStartPar
S97
&
\sphinxAtStartPar
NC
&
\sphinxAtStartPar
\sphinxhyphen{}
&
\sphinxAtStartPar
\sphinxhyphen{}
&
\sphinxAtStartPar
\sphinxhyphen{}
\\
\sphinxhline
\sphinxAtStartPar
S98
&
\sphinxAtStartPar
NC
&
\sphinxAtStartPar
\sphinxhyphen{}
&
\sphinxAtStartPar
\sphinxhyphen{}
&
\sphinxAtStartPar
\sphinxhyphen{}
\\
\sphinxhline
\sphinxAtStartPar
S99
&
\sphinxAtStartPar
NC
&
\sphinxAtStartPar
\sphinxhyphen{}
&
\sphinxAtStartPar
\sphinxhyphen{}
&
\sphinxAtStartPar
\sphinxhyphen{}
\\
\sphinxhline
\sphinxAtStartPar
S100
&
\sphinxAtStartPar
NC
&
\sphinxAtStartPar
\sphinxhyphen{}
&
\sphinxAtStartPar
\sphinxhyphen{}
&
\sphinxAtStartPar
\sphinxhyphen{}
\\
\sphinxhline
\sphinxAtStartPar
S101
&
\sphinxAtStartPar
GND
&
\sphinxAtStartPar
\sphinxhyphen{}
&
\sphinxAtStartPar
\sphinxhyphen{}
&
\sphinxAtStartPar
\sphinxhyphen{}
\\
\sphinxhline
\sphinxAtStartPar
S102
&
\sphinxAtStartPar
NC
&
\sphinxAtStartPar
\sphinxhyphen{}
&
\sphinxAtStartPar
\sphinxhyphen{}
&
\sphinxAtStartPar
\sphinxhyphen{}
\\
\sphinxhline
\sphinxAtStartPar
S103
&
\sphinxAtStartPar
NC
&
\sphinxAtStartPar
\sphinxhyphen{}
&
\sphinxAtStartPar
\sphinxhyphen{}
&
\sphinxAtStartPar
\sphinxhyphen{}
\\
\sphinxhline
\sphinxAtStartPar
S104
&
\sphinxAtStartPar
NC
&
\sphinxAtStartPar
\sphinxhyphen{}
&
\sphinxAtStartPar
\sphinxhyphen{}
&
\sphinxAtStartPar
\sphinxhyphen{}
\\
\sphinxhline
\sphinxAtStartPar
S105
&
\sphinxAtStartPar
NC
&
\sphinxAtStartPar
\sphinxhyphen{}
&
\sphinxAtStartPar
\sphinxhyphen{}
&
\sphinxAtStartPar
\sphinxhyphen{}
\\
\sphinxhline
\sphinxAtStartPar
S106
&
\sphinxAtStartPar
NC
&
\sphinxAtStartPar
\sphinxhyphen{}
&
\sphinxAtStartPar
\sphinxhyphen{}
&
\sphinxAtStartPar
\sphinxhyphen{}
\\
\sphinxhline
\sphinxAtStartPar
S107
&
\sphinxAtStartPar
LCD1\_BKLT\_EN
&
\sphinxAtStartPar
O CMOS
&
\sphinxAtStartPar
1.8V
&
\sphinxAtStartPar
\sphinxhyphen{}
\\
\sphinxhline
\sphinxAtStartPar
S108
&
\sphinxAtStartPar
LVDS1\_CK+
&
\sphinxAtStartPar
O LVDS
&
\sphinxAtStartPar
\sphinxhyphen{}
&
\sphinxAtStartPar
\sphinxhyphen{}
\\
\sphinxhline
\sphinxAtStartPar
S109
&
\sphinxAtStartPar
LVDS1\_CK\sphinxhyphen{}
&
\sphinxAtStartPar
O LVDS
&
\sphinxAtStartPar
\sphinxhyphen{}
&
\sphinxAtStartPar
\sphinxhyphen{}
\\
\sphinxhline
\sphinxAtStartPar
S110
&
\sphinxAtStartPar
GND
&
\sphinxAtStartPar
\sphinxhyphen{}
&
\sphinxAtStartPar
\sphinxhyphen{}
&
\sphinxAtStartPar
\sphinxhyphen{}
\\
\sphinxhline
\sphinxAtStartPar
S111
&
\sphinxAtStartPar
LVDS1\_0+
&
\sphinxAtStartPar
O LVDS
&
\sphinxAtStartPar
\sphinxhyphen{}
&
\sphinxAtStartPar
\sphinxhyphen{}
\\
\sphinxhline
\sphinxAtStartPar
S112
&
\sphinxAtStartPar
LVDS1\_0\sphinxhyphen{}
&
\sphinxAtStartPar
O LVDS
&
\sphinxAtStartPar
\sphinxhyphen{}
&
\sphinxAtStartPar
\sphinxhyphen{}
\\
\sphinxhline
\sphinxAtStartPar
S113
&
\sphinxAtStartPar
NC
&
\sphinxAtStartPar
\sphinxhyphen{}
&
\sphinxAtStartPar
\sphinxhyphen{}
&
\sphinxAtStartPar
\sphinxhyphen{}
\\
\sphinxhline
\sphinxAtStartPar
S114
&
\sphinxAtStartPar
LVDS1\_1+
&
\sphinxAtStartPar
O LVDS
&
\sphinxAtStartPar
\sphinxhyphen{}
&
\sphinxAtStartPar
\sphinxhyphen{}
\\
\sphinxhline
\sphinxAtStartPar
S115
&
\sphinxAtStartPar
LVDS1\_1\sphinxhyphen{}
&
\sphinxAtStartPar
O LVDS
&
\sphinxAtStartPar
\sphinxhyphen{}
&
\sphinxAtStartPar
\sphinxhyphen{}
\\
\sphinxhline
\sphinxAtStartPar
S116
&
\sphinxAtStartPar
LCD1\_VDD\_EN
&
\sphinxAtStartPar
O CMOS
&
\sphinxAtStartPar
1.8V
&
\sphinxAtStartPar
\sphinxhyphen{}
\\
\sphinxhline
\sphinxAtStartPar
S117
&
\sphinxAtStartPar
LVDS1\_2+
&
\sphinxAtStartPar
O LVDS
&
\sphinxAtStartPar
\sphinxhyphen{}
&
\sphinxAtStartPar
\sphinxhyphen{}
\\
\sphinxhline
\sphinxAtStartPar
S118
&
\sphinxAtStartPar
LVDS1\_2\sphinxhyphen{}
&
\sphinxAtStartPar
O LVDS
&
\sphinxAtStartPar
\sphinxhyphen{}
&
\sphinxAtStartPar
\sphinxhyphen{}
\\
\sphinxhline
\sphinxAtStartPar
S119
&
\sphinxAtStartPar
GND
&
\sphinxAtStartPar
\sphinxhyphen{}
&
\sphinxAtStartPar
\sphinxhyphen{}
&
\sphinxAtStartPar
\sphinxhyphen{}
\\
\sphinxhline
\sphinxAtStartPar
S120
&
\sphinxAtStartPar
LVDS1\_3+
&
\sphinxAtStartPar
O LVDS
&
\sphinxAtStartPar
\sphinxhyphen{}
&
\sphinxAtStartPar
\sphinxhyphen{}
\\
\sphinxhline
\sphinxAtStartPar
S121
&
\sphinxAtStartPar
LVDS1\_3\sphinxhyphen{}
&
\sphinxAtStartPar
O LVDS
&
\sphinxAtStartPar
\sphinxhyphen{}
&
\sphinxAtStartPar
\sphinxhyphen{}
\\
\sphinxhline
\sphinxAtStartPar
S122
&
\sphinxAtStartPar
LCD1\_BKLT\_PWM
&
\sphinxAtStartPar
O CMOS
&
\sphinxAtStartPar
1.8V
&
\sphinxAtStartPar
\sphinxhyphen{}
\\
\sphinxhline
\sphinxAtStartPar
S123
&
\sphinxAtStartPar
GPIO13
&
\sphinxAtStartPar
I/O CMOS
&
\sphinxAtStartPar
1.8V
&
\sphinxAtStartPar
\sphinxhyphen{} *\sphinxstyleemphasis{4}
\\
\sphinxhline
\sphinxAtStartPar
S124
&
\sphinxAtStartPar
GND
&
\sphinxAtStartPar
\sphinxhyphen{}
&
\sphinxAtStartPar
\sphinxhyphen{}
&
\sphinxAtStartPar
\sphinxhyphen{}
\\
\sphinxhline
\sphinxAtStartPar
S125
&
\sphinxAtStartPar
LVDS0\_0+ / DSI0\_D0+
&
\sphinxAtStartPar
O LVDS / O D\sphinxhyphen{}PHY
&
\sphinxAtStartPar
\sphinxhyphen{}
&
\sphinxAtStartPar
\sphinxhyphen{}
\\
\sphinxhline
\sphinxAtStartPar
S126
&
\sphinxAtStartPar
LVDS0\_0\sphinxhyphen{} / DSI0\_D0\sphinxhyphen{}
&
\sphinxAtStartPar
O LVDS / O D\sphinxhyphen{}PHY
&
\sphinxAtStartPar
\sphinxhyphen{}
&
\sphinxAtStartPar
\sphinxhyphen{}
\\
\sphinxhline
\sphinxAtStartPar
S127
&
\sphinxAtStartPar
LCD0\_BKLT\_EN
&
\sphinxAtStartPar
O CMOS
&
\sphinxAtStartPar
1.8V
&
\sphinxAtStartPar
\sphinxhyphen{}
\\
\sphinxhline
\sphinxAtStartPar
S128
&
\sphinxAtStartPar
LVDS0\_1+ / DSI0\_D1+
&
\sphinxAtStartPar
O LVDS / O D\sphinxhyphen{}PHY
&
\sphinxAtStartPar
\sphinxhyphen{}
&
\sphinxAtStartPar
\sphinxhyphen{}
\\
\sphinxhline
\sphinxAtStartPar
S129
&
\sphinxAtStartPar
LVDS0\_1\sphinxhyphen{} / DSI0\_D1\sphinxhyphen{}
&
\sphinxAtStartPar
O LVDS / O D\sphinxhyphen{}PHY
&
\sphinxAtStartPar
\sphinxhyphen{}
&
\sphinxAtStartPar
\sphinxhyphen{}
\\
\sphinxhline
\sphinxAtStartPar
S130
&
\sphinxAtStartPar
GND
&
\sphinxAtStartPar
\sphinxhyphen{}
&
\sphinxAtStartPar
\sphinxhyphen{}
&
\sphinxAtStartPar
\sphinxhyphen{}
\\
\sphinxhline
\sphinxAtStartPar
S131
&
\sphinxAtStartPar
LVDS0\_2+ / DSI0\_D2+
&
\sphinxAtStartPar
O LVDS / O D\sphinxhyphen{}PHY
&
\sphinxAtStartPar
\sphinxhyphen{}
&
\sphinxAtStartPar
\sphinxhyphen{}
\\
\sphinxhline
\sphinxAtStartPar
S132
&
\sphinxAtStartPar
LVDS0\_2\sphinxhyphen{} / DSI0\_D2\sphinxhyphen{}
&
\sphinxAtStartPar
O LVDS / O D\sphinxhyphen{}PHY
&
\sphinxAtStartPar
\sphinxhyphen{}
&
\sphinxAtStartPar
\sphinxhyphen{}
\\
\sphinxhline
\sphinxAtStartPar
S133
&
\sphinxAtStartPar
LCD0\_VDD\_EN
&
\sphinxAtStartPar
O CMOS
&
\sphinxAtStartPar
1.8V
&
\sphinxAtStartPar
\sphinxhyphen{}
\\
\sphinxhline
\sphinxAtStartPar
S134
&
\sphinxAtStartPar
LVDS0\_CK+ / DSI0\_CLK+
&
\sphinxAtStartPar
O LVDS / O D\sphinxhyphen{}PHY
&
\sphinxAtStartPar
\sphinxhyphen{}
&
\sphinxAtStartPar
\sphinxhyphen{}
\\
\sphinxhline
\sphinxAtStartPar
S135
&
\sphinxAtStartPar
LVDS0\_CK\sphinxhyphen{} / DSI0\_CLK
&
\sphinxAtStartPar
O LVDS / O D\sphinxhyphen{}PHY
&
\sphinxAtStartPar
\sphinxhyphen{}
&
\sphinxAtStartPar
\sphinxhyphen{}
\\
\sphinxhline
\sphinxAtStartPar
S136
&
\sphinxAtStartPar
GND
&
\sphinxAtStartPar
\sphinxhyphen{}
&
\sphinxAtStartPar
\sphinxhyphen{}
&
\sphinxAtStartPar
\sphinxhyphen{}
\\
\sphinxhline
\sphinxAtStartPar
S137
&
\sphinxAtStartPar
LVDS0\_3+ / DSI0\_D3+
&
\sphinxAtStartPar
O LVDS / O D\sphinxhyphen{}PHY
&
\sphinxAtStartPar
\sphinxhyphen{}
&
\sphinxAtStartPar
\sphinxhyphen{}
\\
\sphinxhline
\sphinxAtStartPar
S138
&
\sphinxAtStartPar
LVDS0\_3\sphinxhyphen{} / DSI0\_D3\sphinxhyphen{}
&
\sphinxAtStartPar
O LVDS / O D\sphinxhyphen{}PHY
&
\sphinxAtStartPar
\sphinxhyphen{}
&
\sphinxAtStartPar
\sphinxhyphen{}
\\
\sphinxhline
\sphinxAtStartPar
S139
&
\sphinxAtStartPar
I2C\_LCD\_CK
&
\sphinxAtStartPar
I/O OD CMOS
&
\sphinxAtStartPar
1.8V
&
\sphinxAtStartPar
PU 2k2
\\
\sphinxhline
\sphinxAtStartPar
S140
&
\sphinxAtStartPar
I2C\_LCD\_DAT
&
\sphinxAtStartPar
I/O OD CMOS
&
\sphinxAtStartPar
1.8V
&
\sphinxAtStartPar
PU 2k2
\\
\sphinxhline
\sphinxAtStartPar
S141
&
\sphinxAtStartPar
LCD0\_BKLT\_PWM
&
\sphinxAtStartPar
O CMOS
&
\sphinxAtStartPar
1.8V
&
\sphinxAtStartPar
\sphinxhyphen{}
\\
\sphinxhline
\sphinxAtStartPar
S142
&
\sphinxAtStartPar
GPIO12
&
\sphinxAtStartPar
I/O CMOS
&
\sphinxAtStartPar
1.8V
&
\sphinxAtStartPar
\sphinxhyphen{} *\sphinxstyleemphasis{4}
\\
\sphinxhline
\sphinxAtStartPar
S143
&
\sphinxAtStartPar
GND
&
\sphinxAtStartPar
\sphinxhyphen{}
&
\sphinxAtStartPar
\sphinxhyphen{}
&
\sphinxAtStartPar
\sphinxhyphen{}
\\
\sphinxhline
\sphinxAtStartPar
S144
&
\sphinxAtStartPar
DSI0\_TE
&
\sphinxAtStartPar
I CMOS
&
\sphinxAtStartPar
1.8V
&
\sphinxAtStartPar
PD 10k
\\
\sphinxhline
\sphinxAtStartPar
S145
&
\sphinxAtStartPar
WDT\_TIME\_OUT\#
&
\sphinxAtStartPar
O CMOS
&
\sphinxAtStartPar
1.8V
&
\sphinxAtStartPar
\sphinxhyphen{}
\\
\sphinxhline
\sphinxAtStartPar
S146
&
\sphinxAtStartPar
PCIE\_WAKE\#
&
\sphinxAtStartPar
I OD CMOS
&
\sphinxAtStartPar
3.3V
&
\sphinxAtStartPar
\sphinxhyphen{}
\\
\sphinxhline
\sphinxAtStartPar
S147
&
\sphinxAtStartPar
VDD\_RTC
&
\sphinxAtStartPar
Analog
&
\sphinxAtStartPar
2.0V to 3.25V
&
\sphinxAtStartPar
\sphinxhyphen{}
\\
\sphinxhline
\sphinxAtStartPar
S148
&
\sphinxAtStartPar
LID\#
&
\sphinxAtStartPar
I OD CMOS
&
\sphinxAtStartPar
1.8V
&
\sphinxAtStartPar
PU 10k
\\
\sphinxhline
\sphinxAtStartPar
S149
&
\sphinxAtStartPar
SLEEP\#
&
\sphinxAtStartPar
I OD CMOS
&
\sphinxAtStartPar
1.8 to 5V
&
\sphinxAtStartPar
PU 10k
\\
\sphinxhline
\sphinxAtStartPar
S150
&
\sphinxAtStartPar
VIN\_PWR\_BAD\#
&
\sphinxAtStartPar
I OD CMOS
&
\sphinxAtStartPar
VDD\_IN
&
\sphinxAtStartPar
PU 100k
\\
\sphinxhline
\sphinxAtStartPar
S151
&
\sphinxAtStartPar
CHARGING\#
&
\sphinxAtStartPar
I OD CMOS
&
\sphinxAtStartPar
1.8to5V
&
\sphinxAtStartPar
PU 10k
\\
\sphinxhline
\sphinxAtStartPar
S152
&
\sphinxAtStartPar
CHARGER\_PRSNT\#
&
\sphinxAtStartPar
I OD CMOS
&
\sphinxAtStartPar
1.8to5V
&
\sphinxAtStartPar
PU 10k
\\
\sphinxhline
\sphinxAtStartPar
S153
&
\sphinxAtStartPar
CARRIER\_STBY\#
&
\sphinxAtStartPar
O CMOS
&
\sphinxAtStartPar
1.8V
&
\sphinxAtStartPar
PU 10k
\\
\sphinxhline
\sphinxAtStartPar
S154
&
\sphinxAtStartPar
CARRIER\_PWR\_ON
&
\sphinxAtStartPar
O CMOS
&
\sphinxAtStartPar
1.8V
&
\sphinxAtStartPar
\sphinxhyphen{}
\\
\sphinxhline
\sphinxAtStartPar
S155
&
\sphinxAtStartPar
FORCE\_RECOV\#
&
\sphinxAtStartPar
I OD CMOS
&
\sphinxAtStartPar
1.8V
&
\sphinxAtStartPar
PU 10k
\\
\sphinxhline
\sphinxAtStartPar
S156
&
\sphinxAtStartPar
BATLOW\#
&
\sphinxAtStartPar
I OD CMOS
&
\sphinxAtStartPar
1.8to5V
&
\sphinxAtStartPar
PU 10k
\\
\sphinxhline
\sphinxAtStartPar
S157
&
\sphinxAtStartPar
TEST\#
&
\sphinxAtStartPar
I OD CMOS
&
\sphinxAtStartPar
1.8to5V
&
\sphinxAtStartPar
PU 10k
\\
\sphinxhline
\sphinxAtStartPar
S158
&
\sphinxAtStartPar
GND
&
\sphinxAtStartPar
\sphinxhyphen{}
&
\sphinxAtStartPar
\sphinxhyphen{}
&
\sphinxAtStartPar
\sphinxhyphen{}
\\
\sphinxbottomrule
\end{longtable}
\sphinxtableafterendhook
\sphinxatlongtableend
\end{savenotes}

\sphinxAtStartPar
1: Configuring this pin for the output function requires adding an additional pull\sphinxhyphen{}up resistor to the 3.3 V supply. The resistance value depends on the IO drive capability required by the receiver. When the pin is configured as an input function, the CPU needs to be internally configured as a pull\sphinxhyphen{}up;

\sphinxAtStartPar
2 : Depending on the internal pull\sphinxhyphen{}up of the i.MX8 MP SOC, the module itself has no pull\sphinxhyphen{}up or pull\sphinxhyphen{}down resistor.</font;

\sphinxAtStartPar
3 :On the module, this pin is connected to the HDMI \sphinxhyphen{} CEC of the MX8MP SOC through a 0Ω resistor. By default, the 0Ω resistor is left un \sphinxhyphen{} soldered;

\sphinxAtStartPar
4 : Depending on the internal pull\sphinxhyphen{}up of the i.MX8 MP SOC, the module itself has no pull\sphinxhyphen{}up or pull\sphinxhyphen{}down resistor.</font;

\sphinxAtStartPar
5 :This set of I2S signals can be configured as PCM signals to communicate with the WIFI \& BT module on the SoM, and by default, they are routed to the gold fingers of the FET\sphinxhyphen{}MX8MP\sphinxhyphen{}SMARC.

\sphinxAtStartPar
These two sets of USB 2.0 and USB 3.0 TX/RX interfaces can be combined separately to form a fully functional USB 3.0 connection.


\chapter{4. Hardware Interface}
\label{\detokenize{hardware:hardware-interface}}

\section{4.1 HDMI}
\label{\detokenize{hardware:hdmi}}

\subsection{4.1.1 HDMI TX Controller}
\label{\detokenize{hardware:hdmi-tx-controller}}
\sphinxAtStartPar
4.1.1.1 Overview

\sphinxAtStartPar
High\sphinxhyphen{}Definition Multimedia Interface (HDMI) TX is a wired digital interconnection that replaces analog TV output or VGA output. HDMI allows the transmission of uncompressed video, audio, and data through a single cable and is compatible with the HDMI v2.0a specification.

\sphinxAtStartPar
4.1.1.2 Features

\sphinxAtStartPar
Compatible with HDMI v2.0 a specification.

\sphinxAtStartPar
Refer to i.MX 8M Plus Applications Processor Reference Manual.

\sphinxAtStartPar
\sphinxstylestrong{Note:}\\
\sphinxstylestrong{The HDCP function is not supported.}


\subsection{4.1.2 HDMI TX PHY}
\label{\detokenize{hardware:hdmi-tx-phy}}
\sphinxAtStartPar
4.1.2.1 Overview

\sphinxAtStartPar
The HDMI (High\sphinxhyphen{}Definition Multimedia Interface) TX PHY is compatible with the HDMI v1.4/v2.0 specification and supports 4Kp30 video resolution. It accepts TransitionMinimized Differential Signaling (TMDS) encoded parallel data from the HDMI link layer and transmits it serially into the HDMI cable.

\sphinxAtStartPar
4.1.2.2 Features
\begin{itemize}
\item {} 
\sphinxAtStartPar
Supports 25 MHz to 594 MHz TMDS clock;

\item {} 
\sphinxAtStartPar
20\sphinxhyphen{}bit parallel data interface with transfer frequencies up to 297 MHz;

\item {} 
\sphinxAtStartPar
All DTV video formats for PC up to 1080p/12\sphinxhyphen{}bit, 3D, 4K X 2K/60 Hz and VGA/XGA/SXGA/UXGA formats).

\end{itemize}

\sphinxAtStartPar
Refer to i.MX 8M Plus Applications Processor Reference Manual for more details.


\subsection{4.1.3 HDMI External Signal}
\label{\detokenize{hardware:hdmi-external-signal}}
\sphinxAtStartPar
Table 4\sphinxhyphen{}1 HDMI Interface Signal


\begin{savenotes}\sphinxattablestart
\sphinxthistablewithglobalstyle
\centering
\begin{tabulary}{\linewidth}[t]{TTT}
\sphinxtoprule
\sphinxstyletheadfamily 
\sphinxAtStartPar
\sphinxstylestrong{Number}
&\sphinxstyletheadfamily 
\sphinxAtStartPar
\sphinxstylestrong{Name}
&\sphinxstyletheadfamily 
\sphinxAtStartPar
\sphinxstylestrong{Description}
\\
\sphinxmidrule
\sphinxtableatstartofbodyhook
\sphinxAtStartPar
\sphinxstylestrong{P92}
&
\sphinxAtStartPar
HDMI\_D2+
&
\sphinxAtStartPar
Positive  HDMI Tx Differential Data2
\\
\sphinxhline
\sphinxAtStartPar
\sphinxstylestrong{P93}
&
\sphinxAtStartPar
HDMI\_D2\sphinxhyphen{}
&
\sphinxAtStartPar
Negative  HDMI Tx Differential Data2
\\
\sphinxhline
\sphinxAtStartPar
\sphinxstylestrong{P95}
&
\sphinxAtStartPar
HDMI\_D1+
&
\sphinxAtStartPar
Positive  HDMI Tx Differential Data1
\\
\sphinxhline
\sphinxAtStartPar
\sphinxstylestrong{P96}
&
\sphinxAtStartPar
HDMI\_D1\sphinxhyphen{}
&
\sphinxAtStartPar
Negative  HDMI Tx Differential Data1
\\
\sphinxhline
\sphinxAtStartPar
\sphinxstylestrong{P98}
&
\sphinxAtStartPar
HDMI\_D0+
&
\sphinxAtStartPar
Positive  HDMI Tx Differential Data0
\\
\sphinxhline
\sphinxAtStartPar
\sphinxstylestrong{P99}
&
\sphinxAtStartPar
HDMI\_D0\sphinxhyphen{}
&
\sphinxAtStartPar
Negative  HDMI Tx Differential Data0
\\
\sphinxhline
\sphinxAtStartPar
\sphinxstylestrong{P101}
&
\sphinxAtStartPar
HDMI\_CK+
&
\sphinxAtStartPar
Positive  HDMI Tx Differential Clock
\\
\sphinxhline
\sphinxAtStartPar
\sphinxstylestrong{P102}
&
\sphinxAtStartPar
HDMI\_CK\sphinxhyphen{}
&
\sphinxAtStartPar
Negative  HDMI Tx Differential Clock
\\
\sphinxhline
\sphinxAtStartPar
\sphinxstylestrong{P104}
&
\sphinxAtStartPar
HDMI\_HPD
&
\sphinxAtStartPar
Hot  Plug Detect Input signal
\\
\sphinxhline
\sphinxAtStartPar
\sphinxstylestrong{P105}
&
\sphinxAtStartPar
HDMI\_CTRL\_CK
&
\sphinxAtStartPar
DDC  Clock line for HDMI panel
\\
\sphinxhline
\sphinxAtStartPar
\sphinxstylestrong{P106}
&
\sphinxAtStartPar
HDMI\_CTRL\_DAT
&
\sphinxAtStartPar
DDC  Data line for HDMI panel
\\
\sphinxhline
\sphinxAtStartPar
\sphinxstylestrong{P107}
&
\sphinxAtStartPar
HDMI\_CEC  *\sphinxstylestrong{1}
&
\sphinxAtStartPar
Consumer Electronics Control for HDMI panel
\\
\sphinxbottomrule
\end{tabulary}
\sphinxtableafterendhook\par
\sphinxattableend\end{savenotes}

\sphinxAtStartPar
* 1 P107 is NC (not connected) by default. If you want to configure it for HDMI\sphinxhyphen{}CEC function, please contact Forlinx and select HDMI\sphinxhyphen{}CEC version.


\section{4.2 LVDS Interface}
\label{\detokenize{hardware:lvds-interface}}
\sphinxAtStartPar
The FET\sphinxhyphen{}MX8MP\sphinxhyphen{}SMARC SoM supports LVDS0 and LVDS1. LVDS0 and DSI0 are optional and cannot be used together. The default configuration is LVDS0. If DSI0 is required, please contact Forlinx and select the DSI0 version.


\subsection{4.2.1 Overview}
\label{\detokenize{hardware:overview}}
\sphinxAtStartPar
The LVDS display bridge (LDB) is connected to an external LVDS display interface. The function of LDB is to support the synchronous transmission of RGB data streams to external display devices via the LVDS interface.


\subsection{4.2.2 Features}
\label{\detokenize{hardware:features}}\begin{itemize}
\item {} 
\sphinxAtStartPar
Connect to display\sphinxhyphen{}related devices with an LVDS receiver;

\item {} 
\sphinxAtStartPar
Arrange the data according to the requirements of the external display receiver and LVDS display standard;

\item {} 
\sphinxAtStartPar
Synchronization and control functions.

\end{itemize}


\subsection{4.2.3 SoM Function Description}
\label{\detokenize{hardware:id1}}\begin{itemize}
\item {} 
\sphinxAtStartPar
Single\sphinxhyphen{}channel (4\sphinxhyphen{}channel) output, with pixel clock and LVDS clock up to 80 MHz. It supports resolutions up to 1366×768p60;

\item {} 
\sphinxAtStartPar
Either Channel 0 or Channel 1 can be used for 4\sphinxhyphen{}channel LVDS;

\item {} 
\sphinxAtStartPar
Dual asynchronous channels (8 data, 2 clock). This is used for a single panel with two interfaces, transmitting data through two channels (even pixels / odd pixels). It is supported with a pixel clock up to 160 MHz (with LVDS clock up to 80 MHz, as each LVDS clock transmits 2 pixels), enabling resolutions higher than 1366×768p60, up to 1080p60.

\end{itemize}

\sphinxAtStartPar
The pixel mapper divides and reorders pixels from a single LCDIF display output to form odd and even pixel streams. This division and reordering are designed to match the speed and channel requirements of LVDS displays. Both VESA and JEIDA pixel mapping are supported.

\sphinxAtStartPar
The pixel mapper supports the following modes:

\sphinxAtStartPar
Table 4\sphinxhyphen{}2 Pixel Mapping Modes.


\begin{savenotes}\sphinxattablestart
\sphinxthistablewithglobalstyle
\centering
\begin{tabulary}{\linewidth}[t]{TTT}
\sphinxtoprule
\sphinxstyletheadfamily 
\sphinxAtStartPar
\sphinxstylestrong{Use Case}
&\sphinxstyletheadfamily 
\sphinxAtStartPar
\sphinxstylestrong{LVDS Channel 0}
&\sphinxstyletheadfamily 
\sphinxAtStartPar
\sphinxstylestrong{LVDS Channel 1}
\\
\sphinxmidrule
\sphinxtableatstartofbodyhook
\sphinxAtStartPar
Singles0
&
\sphinxAtStartPar
Display Interface(DI) of LCDIF
&
\sphinxAtStartPar
Disabled
\\
\sphinxhline
\sphinxAtStartPar
Single1
&
\sphinxAtStartPar
Disabled
&
\sphinxAtStartPar
Display Interface(DI) of LCDIF
\\
\sphinxhline
\sphinxAtStartPar
Dual
&
\sphinxAtStartPar
Display Interface(DI) of LCDIF
&
\sphinxAtStartPar
Display Interface(DI) of LCDIF
\\
\sphinxhline
\sphinxAtStartPar
Split
&
\sphinxAtStartPar
Display Interface(DI) of LCDIF
&
\sphinxAtStartPar
Display Interface(DI) of LCDIF
\\
\sphinxbottomrule
\end{tabulary}
\sphinxtableafterendhook\par
\sphinxattableend\end{savenotes}


\subsection{4.2.4 LVDS External Signal}
\label{\detokenize{hardware:lvds-external-signal}}
\sphinxAtStartPar
Table 4\sphinxhyphen{}3 LVDS0 Interface Signals


\begin{savenotes}\sphinxattablestart
\sphinxthistablewithglobalstyle
\centering
\begin{tabulary}{\linewidth}[t]{TTT}
\sphinxtoprule
\sphinxstyletheadfamily 
\sphinxAtStartPar
\sphinxstylestrong{Number}
&\sphinxstyletheadfamily 
\sphinxAtStartPar
\sphinxstylestrong{Name}
&\sphinxstyletheadfamily 
\sphinxAtStartPar
\sphinxstylestrong{Description}
\\
\sphinxmidrule
\sphinxtableatstartofbodyhook
\sphinxAtStartPar
\sphinxstylestrong{S125}
&
\sphinxAtStartPar
LVDS0\_0+
&
\sphinxAtStartPar
LVDS0  Positive Data0 Signal
\\
\sphinxhline
\sphinxAtStartPar
\sphinxstylestrong{S126}
&
\sphinxAtStartPar
LVDS0\_0\sphinxhyphen{}
&
\sphinxAtStartPar
LVDS0  Negative Data0 Signal
\\
\sphinxhline
\sphinxAtStartPar
\sphinxstylestrong{S128}
&
\sphinxAtStartPar
LVDS0\_1+
&
\sphinxAtStartPar
LVDS0  Positive Data1 Signal
\\
\sphinxhline
\sphinxAtStartPar
\sphinxstylestrong{S129}
&
\sphinxAtStartPar
LVDS0\_1\sphinxhyphen{}
&
\sphinxAtStartPar
LVDS0  Negative Data1 Signal
\\
\sphinxhline
\sphinxAtStartPar
\sphinxstylestrong{S131}
&
\sphinxAtStartPar
LVDS0\_2+
&
\sphinxAtStartPar
LVDS0  Positive Data2 Signal
\\
\sphinxhline
\sphinxAtStartPar
\sphinxstylestrong{S132}
&
\sphinxAtStartPar
LVDS0\_2\sphinxhyphen{}
&
\sphinxAtStartPar
LVDS0  Negative Data2 Signal
\\
\sphinxhline
\sphinxAtStartPar
\sphinxstylestrong{S134}
&
\sphinxAtStartPar
LVDS0\_CK+
&
\sphinxAtStartPar
LVDS0  Positive Clock Signal
\\
\sphinxhline
\sphinxAtStartPar
\sphinxstylestrong{S135}
&
\sphinxAtStartPar
LVDS0\_CK\sphinxhyphen{}
&
\sphinxAtStartPar
LVDS0  Negative Clock Signal
\\
\sphinxhline
\sphinxAtStartPar
\sphinxstylestrong{S137}
&
\sphinxAtStartPar
LVDS0\_3+
&
\sphinxAtStartPar
LVDS0  Positive Data3 Signal
\\
\sphinxhline
\sphinxAtStartPar
\sphinxstylestrong{S138}
&
\sphinxAtStartPar
LVDS0\_3\sphinxhyphen{}
&
\sphinxAtStartPar
LVDS0  Negative Data3 Signal
\\
\sphinxhline
\sphinxAtStartPar
\sphinxstylestrong{S127}
&
\sphinxAtStartPar
LCD0\_BKLT\_EN
&
\sphinxAtStartPar
Primary  LVDS Channel Backlight Enable
\\
\sphinxhline
\sphinxAtStartPar
\sphinxstylestrong{S133}
&
\sphinxAtStartPar
LCD0\_VDD\_EN
&
\sphinxAtStartPar
Primary  LVDS Channel Power Enable
\\
\sphinxhline
\sphinxAtStartPar
\sphinxstylestrong{S141}
&
\sphinxAtStartPar
LCD0\_BKLT\_PWM
&
\sphinxAtStartPar
Primary  LVDS Channel Brightness Control
\\
\sphinxbottomrule
\end{tabulary}
\sphinxtableafterendhook\par
\sphinxattableend\end{savenotes}

\sphinxAtStartPar
Table 4\sphinxhyphen{}4 LVDS1 Interface Signals


\begin{savenotes}\sphinxattablestart
\sphinxthistablewithglobalstyle
\centering
\begin{tabulary}{\linewidth}[t]{TTT}
\sphinxtoprule
\sphinxstyletheadfamily 
\sphinxAtStartPar
\sphinxstylestrong{Number}
&\sphinxstyletheadfamily 
\sphinxAtStartPar
\sphinxstylestrong{Name}
&\sphinxstyletheadfamily 
\sphinxAtStartPar
\sphinxstylestrong{Description}
\\
\sphinxmidrule
\sphinxtableatstartofbodyhook
\sphinxAtStartPar
\sphinxstylestrong{S108}
&
\sphinxAtStartPar
LVDS1\_CK+
&
\sphinxAtStartPar
LVDS1  Positive Clock Signal
\\
\sphinxhline
\sphinxAtStartPar
\sphinxstylestrong{S109}
&
\sphinxAtStartPar
LVDS1\_CK\sphinxhyphen{}
&
\sphinxAtStartPar
LVDS1  Negative Clock Signal
\\
\sphinxhline
\sphinxAtStartPar
\sphinxstylestrong{S111}
&
\sphinxAtStartPar
LVDS1\_0+
&
\sphinxAtStartPar
LVDS1  Positive Data0 Signal
\\
\sphinxhline
\sphinxAtStartPar
\sphinxstylestrong{S112}
&
\sphinxAtStartPar
LVDS1\_0\sphinxhyphen{}
&
\sphinxAtStartPar
LVDS1  Negative Data0 Signal
\\
\sphinxhline
\sphinxAtStartPar
\sphinxstylestrong{S114}
&
\sphinxAtStartPar
LVDS1\_1+
&
\sphinxAtStartPar
LVDS1  Positive Data1 Signal
\\
\sphinxhline
\sphinxAtStartPar
\sphinxstylestrong{S115}
&
\sphinxAtStartPar
LVDS1\_1\sphinxhyphen{}
&
\sphinxAtStartPar
LVDS1  Negative Data1 Signal
\\
\sphinxhline
\sphinxAtStartPar
\sphinxstylestrong{S117}
&
\sphinxAtStartPar
LVDS1\_2+
&
\sphinxAtStartPar
LVDS1  Positive Data2 Signal
\\
\sphinxhline
\sphinxAtStartPar
\sphinxstylestrong{S118}
&
\sphinxAtStartPar
LVDS1\_2\sphinxhyphen{}
&
\sphinxAtStartPar
LVDS1  Negative Data2 Signal
\\
\sphinxhline
\sphinxAtStartPar
\sphinxstylestrong{S120}
&
\sphinxAtStartPar
LVDS1\_3+
&
\sphinxAtStartPar
LVDS1  Positive Data3 Signal
\\
\sphinxhline
\sphinxAtStartPar
\sphinxstylestrong{S121}
&
\sphinxAtStartPar
LVDS1\_3\sphinxhyphen{}
&
\sphinxAtStartPar
LVDS1  Negative Data3 Signal
\\
\sphinxhline
\sphinxAtStartPar
\sphinxstylestrong{S107}
&
\sphinxAtStartPar
LCD1\_BKLT\_EN
&
\sphinxAtStartPar
Secondary  LVDS Channel Backlight Enable
\\
\sphinxhline
\sphinxAtStartPar
\sphinxstylestrong{S116}
&
\sphinxAtStartPar
LCD1\_VDD\_EN
&
\sphinxAtStartPar
Secondary  LVDS Channel Power Enable
\\
\sphinxhline
\sphinxAtStartPar
\sphinxstylestrong{S122}
&
\sphinxAtStartPar
LCD1\_BKLT\_PWM
&
\sphinxAtStartPar
Secondary  LVDS Channel Brightness Control
\\
\sphinxhline
\sphinxAtStartPar
\sphinxstylestrong{S139}
&
\sphinxAtStartPar
I2C\_LCD\_CK
&
\sphinxAtStartPar
DDC  Clock Line Used for Flat Panel Detection and Control
\\
\sphinxhline
\sphinxAtStartPar
\sphinxstylestrong{S140}
&
\sphinxAtStartPar
I2C\_LCD\_DAT
&
\sphinxAtStartPar
DDC  Data Line Used for Flat Panel Detection and Control
\\
\sphinxbottomrule
\end{tabulary}
\sphinxtableafterendhook\par
\sphinxattableend\end{savenotes}


\section{4.3 MIPI Interface}
\label{\detokenize{hardware:mipi-interface}}
\sphinxAtStartPar
The FET\sphinxhyphen{}MX8MP\sphinxhyphen{}SMARC does not support MIPI DSI functionality by default. A selection must be made between MIPI DSI0 and LVDS0.


\subsection{4.3.1 Overview}
\label{\detokenize{hardware:id2}}
\sphinxAtStartPar
The MIPI Display Serial Interface (DSI) is a flexible, high\sphinxhyphen{}performance core that enables communication with peripherals compliant with the MIPI DSI standard.


\subsection{4.3.2 Block Diagram}
\label{\detokenize{hardware:block-diagram}}
\sphinxAtStartPar
\sphinxincludegraphics{{e89c58ae65c9434baa8281dcc0ced4f6}.png}

\sphinxAtStartPar
Figure 4\sphinxhyphen{}1 MIPI DSI Master System Block Diagram


\subsection{4.3.3 Features}
\label{\detokenize{hardware:id3}}\begin{itemize}
\item {} 
\sphinxAtStartPar
Compliant with MIPI DSI standard specification V1.01r11

\item {} 
\sphinxAtStartPar
The maximum resolution supported can reach up to WQHD (2560×1440).

\item {} 
\sphinxAtStartPar
Supports 1, 2, 3, or 4 data lanes

\item {} 
\sphinxAtStartPar
Supported pixel formats: 16bpp, 18bpp packed, 18bpp loosely packed (3 byte format), and 24bpp

\item {} 
\sphinxAtStartPar
Compliant with the PHY Interface Protocol (PPI) in 1.0Gbps/1.5Gbps MIPI DPHY.

\end{itemize}

\sphinxAtStartPar
Refer to i.MX 8M Plus Applications Processor Reference Manual for more details.


\subsection{4.3.4 MIPI DSI External Signal}
\label{\detokenize{hardware:mipi-dsi-external-signal}}
\sphinxAtStartPar
Table 4\sphinxhyphen{}5 MIPI DSI0 Interface Signals


\begin{savenotes}\sphinxattablestart
\sphinxthistablewithglobalstyle
\centering
\begin{tabulary}{\linewidth}[t]{TTT}
\sphinxtoprule
\sphinxstyletheadfamily 
\sphinxAtStartPar
\sphinxstylestrong{Number}
&\sphinxstyletheadfamily 
\sphinxAtStartPar
\sphinxstylestrong{Name}
&\sphinxstyletheadfamily 
\sphinxAtStartPar
\sphinxstylestrong{Description}
\\
\sphinxmidrule
\sphinxtableatstartofbodyhook
\sphinxAtStartPar
\sphinxstylestrong{S125}
&
\sphinxAtStartPar
DSI0\_D0+
&
\sphinxAtStartPar
MIPI  DSI0 Positive Data0 Signal
\\
\sphinxhline
\sphinxAtStartPar
\sphinxstylestrong{S126}
&
\sphinxAtStartPar
DSI0\_D0\sphinxhyphen{}
&
\sphinxAtStartPar
MIPI  DSI0 Negative Data0 Signal
\\
\sphinxhline
\sphinxAtStartPar
\sphinxstylestrong{S128}
&
\sphinxAtStartPar
DSI0\_D1+
&
\sphinxAtStartPar
MIPI  DSI0 Positive Data1 Signal
\\
\sphinxhline
\sphinxAtStartPar
\sphinxstylestrong{S129}
&
\sphinxAtStartPar
DSI0\_D1\sphinxhyphen{}
&
\sphinxAtStartPar
MIPI  DSI0 Negative Data1 Signal
\\
\sphinxhline
\sphinxAtStartPar
\sphinxstylestrong{S131}
&
\sphinxAtStartPar
DSI0\_D2+
&
\sphinxAtStartPar
MIPI  DSI0 Positive Data2 Signal
\\
\sphinxhline
\sphinxAtStartPar
\sphinxstylestrong{S132}
&
\sphinxAtStartPar
DSI0\_D2\sphinxhyphen{}
&
\sphinxAtStartPar
MIPI  DSI0 Negative Data2 Signal
\\
\sphinxhline
\sphinxAtStartPar
\sphinxstylestrong{S134}
&
\sphinxAtStartPar
DSI0\_CLK+
&
\sphinxAtStartPar
MIPI  DSI0 Positive Clock Signal
\\
\sphinxhline
\sphinxAtStartPar
\sphinxstylestrong{S135}
&
\sphinxAtStartPar
DSI0\_CLK
&
\sphinxAtStartPar
MIPI  DSI0 Negative Clock Signal
\\
\sphinxhline
\sphinxAtStartPar
\sphinxstylestrong{S137}
&
\sphinxAtStartPar
DSI0\_D3+
&
\sphinxAtStartPar
MIPI  DSI0 Positive Data3 Signal
\\
\sphinxhline
\sphinxAtStartPar
\sphinxstylestrong{S138}
&
\sphinxAtStartPar
DSI0\_D3\sphinxhyphen{}
&
\sphinxAtStartPar
MIPI  DSI0 Negative Data3 Signal
\\
\sphinxbottomrule
\end{tabulary}
\sphinxtableafterendhook\par
\sphinxattableend\end{savenotes}


\section{4.4  MIPI CSI}
\label{\detokenize{hardware:mipi-csi}}

\subsection{4.4.1 Overview}
\label{\detokenize{hardware:id4}}
\sphinxAtStartPar
The MIPI Camera Serial Interface (MIPI\_CSI2) is the camera interface of this chip. It works in conjunction with the MIPI DPHY module and connects to the host processor. When used in conjunction with output to the ISI (Image Sensor Interface) or ISP (Image Signal Processor), MIPI\_CSI2 supports RAW, YUV, and RGB image formats.


\subsection{4.4.2 Features}
\label{\detokenize{hardware:id5}}\begin{itemize}
\item {} 
\sphinxAtStartPar
Compliant with the MIPI D\sphinxhyphen{}PHY V1.2 specification;

\item {} 
\sphinxAtStartPar
Compliant with the MIPI CSI2 Specification V1.3, except for C\sphinxhyphen{}PHY functionality;

\item {} 
\sphinxAtStartPar
Supports primary and secondary image formats;
\begin{itemize}
\item {} 
\sphinxAtStartPar
YUV420, YUV420 (Legacy), YUV420 (CSPS), YUV422 of 8\sphinxhyphen{}bits and 10\sphinxhyphen{}bits ;

\item {} 
\sphinxAtStartPar
RGB565, RGB666, RGB888;

\item {} 
\sphinxAtStartPar
RAW6, RAW7, RAW8, RAW10, RAW12, RAW14;

\end{itemize}

\item {} 
\sphinxAtStartPar
Supports up to 4 lanes of D\sphinxhyphen{}PHY;

\item {} 
\sphinxAtStartPar
Compatible with the PPI (Protocol to PHY Interface) defined in the MIPI D\sphinxhyphen{}PHY specification;

\end{itemize}

\sphinxAtStartPar
Refer to i.MX 8M Plus Applications Processor Reference Manual for more details.


\subsection{4.4.3 MIPI CSI External Signal}
\label{\detokenize{hardware:mipi-csi-external-signal}}
\sphinxAtStartPar
Table 4\sphinxhyphen{}6 MIPI CSI0 Interface Signals


\begin{savenotes}\sphinxattablestart
\sphinxthistablewithglobalstyle
\centering
\begin{tabulary}{\linewidth}[t]{TTT}
\sphinxtoprule
\sphinxstyletheadfamily 
\sphinxAtStartPar
\sphinxstylestrong{Number}
&\sphinxstyletheadfamily 
\sphinxAtStartPar
\sphinxstylestrong{Name}
&\sphinxstyletheadfamily 
\sphinxAtStartPar
\sphinxstylestrong{Description}
\\
\sphinxmidrule
\sphinxtableatstartofbodyhook
\sphinxAtStartPar
\sphinxstylestrong{S5}
&
\sphinxAtStartPar
I2C\_CAM0\_CK
&
\sphinxAtStartPar
I2C  clock for serial camera data support
\\
\sphinxhline
\sphinxAtStartPar
\sphinxstylestrong{S7}
&
\sphinxAtStartPar
I2C\_CAM0\_DAT
&
\sphinxAtStartPar
I2C  data for serial camera data support link
\\
\sphinxhline
\sphinxAtStartPar
\sphinxstylestrong{P108}
&
\sphinxAtStartPar
CAM0\_PWR\#
&
\sphinxAtStartPar
Camera  0 Power Enable, active low output
\\
\sphinxhline
\sphinxAtStartPar
\sphinxstylestrong{P110}
&
\sphinxAtStartPar
CAM0\_RST\#
&
\sphinxAtStartPar
Camera  0 reset, active low output
\\
\sphinxhline
\sphinxAtStartPar
\sphinxstylestrong{S8}
&
\sphinxAtStartPar
CSI0\_CK+
&
\sphinxAtStartPar
Positive  MIPI CSI0 Differential Clock
\\
\sphinxhline
\sphinxAtStartPar
\sphinxstylestrong{S9}
&
\sphinxAtStartPar
CSI0\_CK\sphinxhyphen{}
&
\sphinxAtStartPar
Negative MIPI CSI0 Differential Clock
\\
\sphinxhline
\sphinxAtStartPar
\sphinxstylestrong{S11}
&
\sphinxAtStartPar
CSI0\_RX0+
&
\sphinxAtStartPar
Positive  MIPI CSI0 Differential Data0
\\
\sphinxhline
\sphinxAtStartPar
\sphinxstylestrong{S12}
&
\sphinxAtStartPar
CSI0\_RX0\sphinxhyphen{}
&
\sphinxAtStartPar
Negative MIPI CSI0 Differential Data0
\\
\sphinxhline
\sphinxAtStartPar
\sphinxstylestrong{S14}
&
\sphinxAtStartPar
CSI0\_RX1+
&
\sphinxAtStartPar
Positive  MIPI CSI0 Differential Data1
\\
\sphinxhline
\sphinxAtStartPar
\sphinxstylestrong{S15}
&
\sphinxAtStartPar
CSI0\_RX1\sphinxhyphen{}
&
\sphinxAtStartPar
Negative MIPI CSI0 Differential Data1
\\
\sphinxbottomrule
\end{tabulary}
\sphinxtableafterendhook\par
\sphinxattableend\end{savenotes}

\sphinxAtStartPar
Table 4\sphinxhyphen{}7 MIPI CSI1 Interface Signals


\begin{savenotes}\sphinxattablestart
\sphinxthistablewithglobalstyle
\centering
\begin{tabulary}{\linewidth}[t]{TTT}
\sphinxtoprule
\sphinxstyletheadfamily 
\sphinxAtStartPar
\sphinxstylestrong{Number}
&\sphinxstyletheadfamily 
\sphinxAtStartPar
\sphinxstylestrong{Name}
&\sphinxstyletheadfamily 
\sphinxAtStartPar
\sphinxstylestrong{Description}
\\
\sphinxmidrule
\sphinxtableatstartofbodyhook
\sphinxAtStartPar
\sphinxstylestrong{S1}
&
\sphinxAtStartPar
I2C\_CAM1\_CK
&
\sphinxAtStartPar
I2C  clock for serial camera data support
\\
\sphinxhline
\sphinxAtStartPar
\sphinxstylestrong{S2}
&
\sphinxAtStartPar
I2C\_CAM1\_DAT
&
\sphinxAtStartPar
I2C  data for serial camera data support link
\\
\sphinxhline
\sphinxAtStartPar
\sphinxstylestrong{P109}
&
\sphinxAtStartPar
CAM1\_PWR\#
&
\sphinxAtStartPar
Camera  1 Power Enable, active low output
\\
\sphinxhline
\sphinxAtStartPar
\sphinxstylestrong{P111}
&
\sphinxAtStartPar
CAM1\_RST\#
&
\sphinxAtStartPar
Camera  1 reset, active low output
\\
\sphinxhline
\sphinxAtStartPar
\sphinxstylestrong{P3}
&
\sphinxAtStartPar
CSI1\_CK+
&
\sphinxAtStartPar
Positive  MIPI CSI1 Differential Clock
\\
\sphinxhline
\sphinxAtStartPar
\sphinxstylestrong{P4}
&
\sphinxAtStartPar
CSI1\_CK\sphinxhyphen{}
&
\sphinxAtStartPar
Negative MIPI CSI1 Differential Clock
\\
\sphinxhline
\sphinxAtStartPar
\sphinxstylestrong{P7}
&
\sphinxAtStartPar
CSI1\_RX0+
&
\sphinxAtStartPar
Positive  MIPI CSI1 Differential Data0
\\
\sphinxhline
\sphinxAtStartPar
\sphinxstylestrong{P8}
&
\sphinxAtStartPar
CSI1\_RX0\sphinxhyphen{}
&
\sphinxAtStartPar
Negative MIPI CSI1 Differential Data0
\\
\sphinxhline
\sphinxAtStartPar
\sphinxstylestrong{P10}
&
\sphinxAtStartPar
CSI1\_RX1+
&
\sphinxAtStartPar
Positive  MIPI CSI1 Differential Data1
\\
\sphinxhline
\sphinxAtStartPar
\sphinxstylestrong{P11}
&
\sphinxAtStartPar
CSI1\_RX1\sphinxhyphen{}
&
\sphinxAtStartPar
Negative MIPI CSI1 Differential Data1
\\
\sphinxhline
\sphinxAtStartPar
\sphinxstylestrong{P13}
&
\sphinxAtStartPar
CSI1\_RX2+
&
\sphinxAtStartPar
Positive  MIPI CSI1 Differential Data2
\\
\sphinxhline
\sphinxAtStartPar
\sphinxstylestrong{P14}
&
\sphinxAtStartPar
CSI1\_RX2\sphinxhyphen{}
&
\sphinxAtStartPar
Negative MIPI CSI1 Differential Data2
\\
\sphinxhline
\sphinxAtStartPar
\sphinxstylestrong{P16}
&
\sphinxAtStartPar
CSI1\_RX3+
&
\sphinxAtStartPar
Positive  MIPI CSI1 Differential Data3
\\
\sphinxhline
\sphinxAtStartPar
\sphinxstylestrong{P17}
&
\sphinxAtStartPar
CSI1\_RX3\sphinxhyphen{}
&
\sphinxAtStartPar
Negative MIPI CSI1 Differential Data3
\\
\sphinxbottomrule
\end{tabulary}
\sphinxtableafterendhook\par
\sphinxattableend\end{savenotes}


\section{4.5 Audio}
\label{\detokenize{hardware:audio}}

\subsection{4.5.1 SAI Overview}
\label{\detokenize{hardware:sai-overview}}
\sphinxAtStartPar
The Synchronous Audio Interface (SAI) provides an interface that supports a full\sphinxhyphen{}duplex serial connection, featuring frame\sphinxhyphen{}synchronized formats such as I2S, AC97, TDM, as well as codec/DSP interfaces.

\sphinxAtStartPar
The FET\sphinxhyphen{}MX8MP\sphinxhyphen{}SMARC has two audio interfaces, I2S0 and I2S2. The I2S0 interface is managed by the SAI3 signal group of the SoC and the I2S2 interface is managed by the SAI5 signal group of the SoC.


\subsection{4.5.2 Features}
\label{\detokenize{hardware:id6}}\begin{itemize}
\item {} 
\sphinxAtStartPar
The transmitter features independent bit clock and frame synchronization;

\item {} 
\sphinxAtStartPar
The receiver is equipped with independent bit clock and frame synchronization;

\item {} 
\sphinxAtStartPar
Each data line supports a maximum frame size of 32 words;

\item {} 
\sphinxAtStartPar
8\sphinxhyphen{} to 32\sphinxhyphen{}bit word size.

\end{itemize}

\sphinxAtStartPar
Refer to i.MX 8M Plus Applications Processor Reference Manual for more details.


\subsection{4.5.3 I2S External Signal}
\label{\detokenize{hardware:i2s-external-signal}}
\sphinxAtStartPar
Table 4\sphinxhyphen{}7 I2S0 Interface Signals


\begin{savenotes}\sphinxattablestart
\sphinxthistablewithglobalstyle
\centering
\begin{tabulary}{\linewidth}[t]{TTT}
\sphinxtoprule
\sphinxstyletheadfamily 
\sphinxAtStartPar
\sphinxstylestrong{Number}
&\sphinxstyletheadfamily 
\sphinxAtStartPar
\sphinxstylestrong{Name}
&\sphinxstyletheadfamily 
\sphinxAtStartPar
\sphinxstylestrong{Description}
\\
\sphinxmidrule
\sphinxtableatstartofbodyhook
\sphinxAtStartPar
\sphinxstylestrong{S38}
&
\sphinxAtStartPar
AUDIO\_MCK
&
\sphinxAtStartPar
Master  Clock Output to I2S Codec(s)
\\
\sphinxhline
\sphinxAtStartPar
\sphinxstylestrong{S39}
&
\sphinxAtStartPar
I2S0\_LRCK
&
\sphinxAtStartPar
I2S0  Left \& Right Synchronization Clock
\\
\sphinxhline
\sphinxAtStartPar
\sphinxstylestrong{S40}
&
\sphinxAtStartPar
I2S0\_SDOUT
&
\sphinxAtStartPar
I2S0  Digital Audio Output
\\
\sphinxhline
\sphinxAtStartPar
\sphinxstylestrong{S41}
&
\sphinxAtStartPar
I2S0\_SDIN
&
\sphinxAtStartPar
I2S0  Digital Audio Input
\\
\sphinxhline
\sphinxAtStartPar
\sphinxstylestrong{S42}
&
\sphinxAtStartPar
I2S0\_CK
&
\sphinxAtStartPar
I2S0  Digital Audio Clock
\\
\sphinxbottomrule
\end{tabulary}
\sphinxtableafterendhook\par
\sphinxattableend\end{savenotes}

\sphinxAtStartPar
Table 4\sphinxhyphen{}8 I2S2 Interface Signals


\begin{savenotes}\sphinxattablestart
\sphinxthistablewithglobalstyle
\centering
\begin{tabulary}{\linewidth}[t]{TTT}
\sphinxtoprule
\sphinxstyletheadfamily 
\sphinxAtStartPar
\sphinxstylestrong{Number}
&\sphinxstyletheadfamily 
\sphinxAtStartPar
\sphinxstylestrong{Name}
&\sphinxstyletheadfamily 
\sphinxAtStartPar
\sphinxstylestrong{Description}
\\
\sphinxmidrule
\sphinxtableatstartofbodyhook
\sphinxAtStartPar
\sphinxstylestrong{S38}
&
\sphinxAtStartPar
AUDIO\_MCK
&
\sphinxAtStartPar
Master  Clock Output to I2S Codec(s)
\\
\sphinxhline
\sphinxAtStartPar
\sphinxstylestrong{S50}
&
\sphinxAtStartPar
I2S2\_LRCK
&
\sphinxAtStartPar
I2S2  Left \& Right Synchronization Clock
\\
\sphinxhline
\sphinxAtStartPar
\sphinxstylestrong{S51}
&
\sphinxAtStartPar
I2S2\_SDOUT
&
\sphinxAtStartPar
I2S2  Digital Audio Output
\\
\sphinxhline
\sphinxAtStartPar
\sphinxstylestrong{S52}
&
\sphinxAtStartPar
I2S2\_SDIN
&
\sphinxAtStartPar
I2S2  Digital Audio Input
\\
\sphinxhline
\sphinxAtStartPar
\sphinxstylestrong{S53}
&
\sphinxAtStartPar
I2S2\_CK
&
\sphinxAtStartPar
I2S2  Digital Audio Clock
\\
\sphinxbottomrule
\end{tabulary}
\sphinxtableafterendhook\par
\sphinxattableend\end{savenotes}


\section{4.6 PCIe}
\label{\detokenize{hardware:pcie}}

\subsection{4.6.1 Overview}
\label{\detokenize{hardware:id7}}
\sphinxAtStartPar
The PCI Express interface complies with the PCI Express™ Base Specification Revision 4.0, Version 0.7 (available at http://www.pcisig.com). This manual does not cover the complex details of the PCI Express protocol.


\subsection{4.6.2 Features}
\label{\detokenize{hardware:id8}}\begin{itemize}
\item {} 
\sphinxAtStartPar
Supports Root Complex (RC) and Endpoint (EP) configurations;

\item {} 
\sphinxAtStartPar
Maximum link speed up to Gen3 (8 GT/s);

\item {} 
\sphinxAtStartPar
x1 link width.

\end{itemize}

\sphinxAtStartPar
Refer to  i.MX 8M Plus Applications Processor Reference Manual for more details.


\subsection{4.6.3 PCIe x 1 External Signal}
\label{\detokenize{hardware:pcie-x-1-external-signal}}
\sphinxAtStartPar
Table 4\sphinxhyphen{}9 PCIe x1 Interface Signals


\begin{savenotes}\sphinxattablestart
\sphinxthistablewithglobalstyle
\centering
\begin{tabulary}{\linewidth}[t]{TTT}
\sphinxtoprule
\sphinxstyletheadfamily 
\sphinxAtStartPar
\sphinxstylestrong{Number}
&\sphinxstyletheadfamily 
\sphinxAtStartPar
\sphinxstylestrong{Name}
&\sphinxstyletheadfamily 
\sphinxAtStartPar
\sphinxstylestrong{Description}
\\
\sphinxmidrule
\sphinxtableatstartofbodyhook
\sphinxAtStartPar
\sphinxstylestrong{P83}
&
\sphinxAtStartPar
PCIE\_A\_REFCK+
&
\sphinxAtStartPar
Positive PCIe link A Differential clock output
\\
\sphinxhline
\sphinxAtStartPar
\sphinxstylestrong{P84}
&
\sphinxAtStartPar
PCIE\_A\_REFCK\sphinxhyphen{}
&
\sphinxAtStartPar
Negative PCIe link A Differential clock output
\\
\sphinxhline
\sphinxAtStartPar
\sphinxstylestrong{P86}
&
\sphinxAtStartPar
PCIE\_A\_RX+
&
\sphinxAtStartPar
Positive PCIe link A Differential receive data
\\
\sphinxhline
\sphinxAtStartPar
\sphinxstylestrong{P87}
&
\sphinxAtStartPar
PCIE\_A\_RX\sphinxhyphen{}
&
\sphinxAtStartPar
Negative PCIe link A Differential receive data
\\
\sphinxhline
\sphinxAtStartPar
\sphinxstylestrong{P89}
&
\sphinxAtStartPar
PCIE\_A\_TX+
&
\sphinxAtStartPar
Positive PCIe link A Differential transmit data
\\
\sphinxhline
\sphinxAtStartPar
\sphinxstylestrong{P90}
&
\sphinxAtStartPar
PCIE\_A\_TX\sphinxhyphen{}
&
\sphinxAtStartPar
Negative PCIe link A Differential transmit data
\\
\sphinxhline
\sphinxAtStartPar
\sphinxstylestrong{P75}
&
\sphinxAtStartPar
PCIE\_A\_RST\#
&
\sphinxAtStartPar
PCIe  Port A reset output
\\
\sphinxhline
\sphinxAtStartPar
\sphinxstylestrong{P78}
&
\sphinxAtStartPar
PCIE\_A\_CKREQ\#
&
\sphinxAtStartPar
PCIe  Port A clock request
\\
\sphinxhline
\sphinxAtStartPar
\sphinxstylestrong{S146}
&
\sphinxAtStartPar
PCIE\_WAKE\#
&
\sphinxAtStartPar
PCIe  wake up interrupt to host – common to PCIe links A
\\
\sphinxbottomrule
\end{tabulary}
\sphinxtableafterendhook\par
\sphinxattableend\end{savenotes}


\section{4.7 Ethernet}
\label{\detokenize{hardware:ethernet}}

\subsection{4.7.1 Overview}
\label{\detokenize{hardware:id9}}
\sphinxAtStartPar
The iMX 8M Plus implements two Ethernet controllers, both capable of running simultaneously. It features a tri\sphinxhyphen{}speed 10/100/1000\sphinxhyphen{}Mbit/s Ethernet MAC compliant with the IEEE 802.3\sphinxhyphen{}2002 standard.

\sphinxAtStartPar
The FET\sphinxhyphen{}MX8MP\sphinxhyphen{}SMARC module includes 2 x on\sphinxhyphen{}board gigabit Ethernet transceivers. Highly integrated Ethernet transceivers comply with 10BASE\sphinxhyphen{}Te, 100BASE\sphinxhyphen{}TX, and 1000BASE\sphinxhyphen{}T IEEE 802.3 standards. It provides all the physical layer functions necessary for sending and receiving Ethernet packets over CAT.5E unshielded twisted\sphinxhyphen{}pair cables.


\subsection{4.7.2 Ethernet External Signal}
\label{\detokenize{hardware:ethernet-external-signal}}
\sphinxAtStartPar
Table 4\sphinxhyphen{}10 GBE0 Interface Signals


\begin{savenotes}\sphinxattablestart
\sphinxthistablewithglobalstyle
\centering
\begin{tabulary}{\linewidth}[t]{TTT}
\sphinxtoprule
\sphinxstyletheadfamily 
\sphinxAtStartPar
\sphinxstylestrong{Number}
&\sphinxstyletheadfamily 
\sphinxAtStartPar
\sphinxstylestrong{Name}
&\sphinxstyletheadfamily 
\sphinxAtStartPar
\sphinxstylestrong{Description}
\\
\sphinxmidrule
\sphinxtableatstartofbodyhook
\sphinxAtStartPar
\sphinxstylestrong{P29}
&
\sphinxAtStartPar
GBE0\_MDI0\sphinxhyphen{}
&
\sphinxAtStartPar
Negative GEB0 Differential Media\sphinxhyphen{}dependent interface 0
\\
\sphinxhline
\sphinxAtStartPar
\sphinxstylestrong{P30}
&
\sphinxAtStartPar
GBE0\_MDI0+
&
\sphinxAtStartPar
Positive GEB0 Differential Media\sphinxhyphen{}dependent interface 0
\\
\sphinxhline
\sphinxAtStartPar
\sphinxstylestrong{P26}
&
\sphinxAtStartPar
GBE0\_MDI1\sphinxhyphen{}
&
\sphinxAtStartPar
Negative GEB0 Differential Media\sphinxhyphen{}dependent interface 1
\\
\sphinxhline
\sphinxAtStartPar
\sphinxstylestrong{P27}
&
\sphinxAtStartPar
GBE0\_MDI1+
&
\sphinxAtStartPar
Positive GEB0 Differential Media\sphinxhyphen{}dependent interface 1
\\
\sphinxhline
\sphinxAtStartPar
\sphinxstylestrong{P23}
&
\sphinxAtStartPar
GBE0\_MDI2\sphinxhyphen{}
&
\sphinxAtStartPar
Negative GEB0 Differential Media\sphinxhyphen{}dependent interface 2
\\
\sphinxhline
\sphinxAtStartPar
\sphinxstylestrong{P24}
&
\sphinxAtStartPar
GBE0\_MDI2+
&
\sphinxAtStartPar
Positive GEB0 Differential Media\sphinxhyphen{}dependent interface 2
\\
\sphinxhline
\sphinxAtStartPar
\sphinxstylestrong{P19}
&
\sphinxAtStartPar
GBE0\_MDI3\sphinxhyphen{}
&
\sphinxAtStartPar
Negative GEB0 Differential Media\sphinxhyphen{}dependent interface 3
\\
\sphinxhline
\sphinxAtStartPar
\sphinxstylestrong{P20}
&
\sphinxAtStartPar
GBE0\_MDI3+
&
\sphinxAtStartPar
Positive GEB0 Differential Media\sphinxhyphen{}dependent interface 3
\\
\sphinxhline
\sphinxAtStartPar
\sphinxstylestrong{P21}
&
\sphinxAtStartPar
GBE0\_LINK100\#
&
\sphinxAtStartPar
Link  Speed Indication LED for GBE0 100Mbps
\\
\sphinxhline
\sphinxAtStartPar
\sphinxstylestrong{P22}
&
\sphinxAtStartPar
GBE0\_LINK1000\#
&
\sphinxAtStartPar
Link  Speed Indication LED for GBE0 1000Mbps
\\
\sphinxhline
\sphinxAtStartPar
\sphinxstylestrong{P25}
&
\sphinxAtStartPar
GBE0\_LINK\_ACT\#
&
\sphinxAtStartPar
Link  / Activity Indication LED Driven Low on Link (10, 100 or 1000 Mbps) Blinks on  Activity
\\
\sphinxhline
\sphinxAtStartPar
\sphinxstylestrong{P6}
&
\sphinxAtStartPar
GBE0\_SDP
&
\sphinxAtStartPar
IEEE  1588 Trigger Signal for Hardware Implementation of PTP (Precision Time  Protocol)
\\
\sphinxbottomrule
\end{tabulary}
\sphinxtableafterendhook\par
\sphinxattableend\end{savenotes}

\sphinxAtStartPar
Table 4\sphinxhyphen{}11 GBE1 Interface Signals


\begin{savenotes}\sphinxattablestart
\sphinxthistablewithglobalstyle
\centering
\begin{tabulary}{\linewidth}[t]{TTT}
\sphinxtoprule
\sphinxstyletheadfamily 
\sphinxAtStartPar
\sphinxstylestrong{Number}
&\sphinxstyletheadfamily 
\sphinxAtStartPar
\sphinxstylestrong{Name}
&\sphinxstyletheadfamily 
\sphinxAtStartPar
\sphinxstylestrong{Description}
\\
\sphinxmidrule
\sphinxtableatstartofbodyhook
\sphinxAtStartPar
\sphinxstylestrong{S18}
&
\sphinxAtStartPar
GBE1\_MDI0\sphinxhyphen{}
&
\sphinxAtStartPar
Negative GEB0 Differential Media\sphinxhyphen{}dependent interface 0
\\
\sphinxhline
\sphinxAtStartPar
\sphinxstylestrong{S17}
&
\sphinxAtStartPar
GBE1\_MDI0+
&
\sphinxAtStartPar
Positive GEB0 Differential Media\sphinxhyphen{}dependent interface 0
\\
\sphinxhline
\sphinxAtStartPar
\sphinxstylestrong{S21}
&
\sphinxAtStartPar
GBE1\_MDI1\sphinxhyphen{}
&
\sphinxAtStartPar
Negative GEB0 Differential Media\sphinxhyphen{}dependent interface 1
\\
\sphinxhline
\sphinxAtStartPar
\sphinxstylestrong{S20}
&
\sphinxAtStartPar
GBE1\_MDI1+
&
\sphinxAtStartPar
Positive GEB0 Differential Media\sphinxhyphen{}dependent interface 1
\\
\sphinxhline
\sphinxAtStartPar
\sphinxstylestrong{S24}
&
\sphinxAtStartPar
GBE1\_MDI2\sphinxhyphen{}
&
\sphinxAtStartPar
Negative GEB0 Differential Media\sphinxhyphen{}dependent interface 2
\\
\sphinxhline
\sphinxAtStartPar
\sphinxstylestrong{S23}
&
\sphinxAtStartPar
GBE1\_MDI2+
&
\sphinxAtStartPar
Positive GEB0 Differential Media\sphinxhyphen{}dependent interface 2
\\
\sphinxhline
\sphinxAtStartPar
\sphinxstylestrong{S27}
&
\sphinxAtStartPar
GBE1\_MDI3\sphinxhyphen{}
&
\sphinxAtStartPar
Negative GEB0 Differential Media\sphinxhyphen{}dependent interface 3
\\
\sphinxhline
\sphinxAtStartPar
\sphinxstylestrong{S26}
&
\sphinxAtStartPar
GBE1\_MDI3+
&
\sphinxAtStartPar
Positive GEB0 Differential Media\sphinxhyphen{}dependent interface 3
\\
\sphinxhline
\sphinxAtStartPar
\sphinxstylestrong{S19}
&
\sphinxAtStartPar
GBE1\_LINK100\#
&
\sphinxAtStartPar
Link  Speed Indication LED for GBE0 100Mbps
\\
\sphinxhline
\sphinxAtStartPar
\sphinxstylestrong{S22}
&
\sphinxAtStartPar
GBE1\_LINK1000\#
&
\sphinxAtStartPar
Link  Speed Indication LED for GBE0 1000Mbps
\\
\sphinxhline
\sphinxAtStartPar
\sphinxstylestrong{S31}
&
\sphinxAtStartPar
GBE1\_LINK\_ACT\#
&
\sphinxAtStartPar
Link  / Activity Indication LED Driven Low on Link (10, 100 or 1000 Mbps) Blinks on  Activity
\\
\sphinxhline
\sphinxAtStartPar
\sphinxstylestrong{P5}
&
\sphinxAtStartPar
GBE1\_SDP
&
\sphinxAtStartPar
IEEE  1588 Trigger Signal for Hardware Implementation of PTP (Precision Time  Protocol)
\\
\sphinxbottomrule
\end{tabulary}
\sphinxtableafterendhook\par
\sphinxattableend\end{savenotes}


\section{4.8 USB}
\label{\detokenize{hardware:usb}}

\subsection{4.8.1 Overview}
\label{\detokenize{hardware:id10}}
\sphinxAtStartPar
The USB interfaces supported by FET\sphinxhyphen{}MX8MP\sphinxhyphen{}SMARC are as follows:
\begin{itemize}
\item {} 
\sphinxAtStartPar
1 x USB 2.0 OTG

\item {} 
\sphinxAtStartPar
3 x USB 2.0 HOST

\item {} 
\sphinxAtStartPar
2 x USB 3.0 HOST

\end{itemize}


\subsection{4.8.2 Features}
\label{\detokenize{hardware:id11}}
\sphinxAtStartPar
The USB 3.0 module includes the following features:
\begin{itemize}
\item {} 
\sphinxAtStartPar
Comply with USB specification rev 3.0;

\item {} 
\sphinxAtStartPar
Super\sphinxhyphen{}speed (5 Gbit/s), high\sphinxhyphen{}speed (480 Mbit/s), full\sphinxhyphen{}speed (12 Mbit/s), and low speed (1.5 Mbit/s) operations.

\end{itemize}

\sphinxAtStartPar
The USB 2.0 module includes the following features:
\begin{itemize}
\item {} 
\sphinxAtStartPar
Comply with USB specification rev 2.0;

\item {} 
\sphinxAtStartPar
High\sphinxhyphen{}speed (480 Mbit/s), full\sphinxhyphen{}speed (12 Mbit/s), and low speed (1.5 Mbit/s) operations;

\item {} 
\sphinxAtStartPar
USB 2.0 OTG supports dual\sphinxhyphen{}role operation and can be configured as a host or device.

\end{itemize}

\sphinxAtStartPar
USB {[}0:5{]} \_ EN \_ OC \# is the muxing function pin, which is pulled up to the 3.3 V power rail on the SoM. The carrier board can realize the OC \# (over\sphinxhyphen{}current) overcurrent detection function through the open\sphinxhyphen{}drain driver.


\subsection{4.8.3 USB External Signal}
\label{\detokenize{hardware:usb-external-signal}}
\sphinxAtStartPar
Table 4\sphinxhyphen{}12 USB0 Port Signals


\begin{savenotes}\sphinxattablestart
\sphinxthistablewithglobalstyle
\centering
\begin{tabulary}{\linewidth}[t]{TTT}
\sphinxtoprule
\sphinxstyletheadfamily 
\sphinxAtStartPar
\sphinxstylestrong{Number}
&\sphinxstyletheadfamily 
\sphinxAtStartPar
\sphinxstylestrong{Name}
&\sphinxstyletheadfamily 
\sphinxAtStartPar
\sphinxstylestrong{Description}
\\
\sphinxmidrule
\sphinxtableatstartofbodyhook
\sphinxAtStartPar
\sphinxstylestrong{P60}
&
\sphinxAtStartPar
USB0+
&
\sphinxAtStartPar
USB PHY Data Plus for  Port 0
\\
\sphinxhline
\sphinxAtStartPar
\sphinxstylestrong{P61}
&
\sphinxAtStartPar
USB0\sphinxhyphen{}
&
\sphinxAtStartPar
USB PHY Data Minus for Port 0
\\
\sphinxhline
\sphinxAtStartPar
\sphinxstylestrong{P62}
&
\sphinxAtStartPar
USB0\_EN\_OC\#
&
\sphinxAtStartPar
USB  Over\sphinxhyphen{}Current Sense for Port 0
\\
\sphinxhline
\sphinxAtStartPar
\sphinxstylestrong{P63}
&
\sphinxAtStartPar
USB0\_VBUS\_DET
&
\sphinxAtStartPar
USB  Port 0 Host Power Detection
\\
\sphinxhline
\sphinxAtStartPar
\sphinxstylestrong{P64}
&
\sphinxAtStartPar
USB0\_OTG\_ID
&
\sphinxAtStartPar
Input  Pin to Announce OTG Device Insertion on USB 2.0 Port
\\
\sphinxbottomrule
\end{tabulary}
\sphinxtableafterendhook\par
\sphinxattableend\end{savenotes}

\sphinxAtStartPar
Table 4\sphinxhyphen{}13 USB1 Port Signals


\begin{savenotes}\sphinxattablestart
\sphinxthistablewithglobalstyle
\centering
\begin{tabulary}{\linewidth}[t]{TTT}
\sphinxtoprule
\sphinxstyletheadfamily 
\sphinxAtStartPar
\sphinxstylestrong{Number}
&\sphinxstyletheadfamily 
\sphinxAtStartPar
\sphinxstylestrong{Name}
&\sphinxstyletheadfamily 
\sphinxAtStartPar
\sphinxstylestrong{Description}
\\
\sphinxmidrule
\sphinxtableatstartofbodyhook
\sphinxAtStartPar
\sphinxstylestrong{P65}
&
\sphinxAtStartPar
USB1+
&
\sphinxAtStartPar
USB PHY Data Plus for  Port 1
\\
\sphinxhline
\sphinxAtStartPar
\sphinxstylestrong{P66}
&
\sphinxAtStartPar
USB1\sphinxhyphen{}
&
\sphinxAtStartPar
USB PHY Data Minus for Port 1
\\
\sphinxhline
\sphinxAtStartPar
\sphinxstylestrong{P67}
&
\sphinxAtStartPar
USB1\_EN\_OC\#
&
\sphinxAtStartPar
USB  Over\sphinxhyphen{}Current Sense for Port 1
\\
\sphinxbottomrule
\end{tabulary}
\sphinxtableafterendhook\par
\sphinxattableend\end{savenotes}

\sphinxAtStartPar
Table 4\sphinxhyphen{}14 USB2 Port Signals


\begin{savenotes}\sphinxattablestart
\sphinxthistablewithglobalstyle
\centering
\begin{tabulary}{\linewidth}[t]{TTT}
\sphinxtoprule
\sphinxstyletheadfamily 
\sphinxAtStartPar
\sphinxstylestrong{Number}
&\sphinxstyletheadfamily 
\sphinxAtStartPar
\sphinxstylestrong{Name}
&\sphinxstyletheadfamily 
\sphinxAtStartPar
\sphinxstylestrong{Description}
\\
\sphinxmidrule
\sphinxtableatstartofbodyhook
\sphinxAtStartPar
\sphinxstylestrong{P69}
&
\sphinxAtStartPar
USB2+
&
\sphinxAtStartPar
USB PHY Data Plus for  Port 2
\\
\sphinxhline
\sphinxAtStartPar
\sphinxstylestrong{P70}
&
\sphinxAtStartPar
USB2\sphinxhyphen{}
&
\sphinxAtStartPar
USB PHY Data Minus for Port 2
\\
\sphinxhline
\sphinxAtStartPar
\sphinxstylestrong{S71}
&
\sphinxAtStartPar
USB2\_SSTX+
&
\sphinxAtStartPar
USB  PHY 3.0 Transmit Data (positive)
\\
\sphinxhline
\sphinxAtStartPar
\sphinxstylestrong{S72}
&
\sphinxAtStartPar
USB2\_SSTX\sphinxhyphen{}
&
\sphinxAtStartPar
USB  PHY 3.0 Transmit Data (negative)
\\
\sphinxhline
\sphinxAtStartPar
\sphinxstylestrong{S74}
&
\sphinxAtStartPar
USB2\_SSRX+
&
\sphinxAtStartPar
USB  PHY 3.0 Receive Data (positive)
\\
\sphinxhline
\sphinxAtStartPar
\sphinxstylestrong{S75}
&
\sphinxAtStartPar
USB2\_SSRX\sphinxhyphen{}
&
\sphinxAtStartPar
USB  PHY 3.0 Receive Data (negative)
\\
\sphinxhline
\sphinxAtStartPar
\sphinxstylestrong{P71}
&
\sphinxAtStartPar
USB2\_EN\_OC\#
&
\sphinxAtStartPar
USB  Over\sphinxhyphen{}Current Sense for Port 2
\\
\sphinxbottomrule
\end{tabulary}
\sphinxtableafterendhook\par
\sphinxattableend\end{savenotes}

\sphinxAtStartPar
Table 4\sphinxhyphen{}15 USB3 OTG Port Interface Signal


\begin{savenotes}\sphinxattablestart
\sphinxthistablewithglobalstyle
\centering
\begin{tabulary}{\linewidth}[t]{TTT}
\sphinxtoprule
\sphinxstyletheadfamily 
\sphinxAtStartPar
\sphinxstylestrong{Number}
&\sphinxstyletheadfamily 
\sphinxAtStartPar
\sphinxstylestrong{Name}
&\sphinxstyletheadfamily 
\sphinxAtStartPar
\sphinxstylestrong{Description}
\\
\sphinxmidrule
\sphinxtableatstartofbodyhook
\sphinxAtStartPar
\sphinxstylestrong{S68}
&
\sphinxAtStartPar
USB3+
&
\sphinxAtStartPar
USB PHY Data Plus for  Port 3
\\
\sphinxhline
\sphinxAtStartPar
\sphinxstylestrong{S69}
&
\sphinxAtStartPar
USB3\sphinxhyphen{}
&
\sphinxAtStartPar
USB PHY Data Minus for Port 3
\\
\sphinxhline
\sphinxAtStartPar
\sphinxstylestrong{S62}
&
\sphinxAtStartPar
USB3\_SSTX+
&
\sphinxAtStartPar
USB  PHY 3.0 Transmit Data (positive)
\\
\sphinxhline
\sphinxAtStartPar
\sphinxstylestrong{S63}
&
\sphinxAtStartPar
USB3\_SSTX\sphinxhyphen{}
&
\sphinxAtStartPar
USB  PHY 3.0 Transmit Data (negative)
\\
\sphinxhline
\sphinxAtStartPar
\sphinxstylestrong{S65}
&
\sphinxAtStartPar
USB3\_SSRX+
&
\sphinxAtStartPar
USB  PHY 3.0 Receive Data (positive)
\\
\sphinxhline
\sphinxAtStartPar
\sphinxstylestrong{S66}
&
\sphinxAtStartPar
USB3\_SSRX\sphinxhyphen{}
&
\sphinxAtStartPar
USB  PHY 3.0 Receive Data (negative)
\\
\sphinxhline
\sphinxAtStartPar
\sphinxstylestrong{P74}
&
\sphinxAtStartPar
USB3\_EN\_OC\#
&
\sphinxAtStartPar
USB  Over\sphinxhyphen{}Current Sense for Port 3
\\
\sphinxbottomrule
\end{tabulary}
\sphinxtableafterendhook\par
\sphinxattableend\end{savenotes}

\sphinxAtStartPar
Table 4\sphinxhyphen{}16 USB4 Port Signals


\begin{savenotes}\sphinxattablestart
\sphinxthistablewithglobalstyle
\centering
\begin{tabulary}{\linewidth}[t]{TTT}
\sphinxtoprule
\sphinxstyletheadfamily 
\sphinxAtStartPar
\sphinxstylestrong{Number}
&\sphinxstyletheadfamily 
\sphinxAtStartPar
\sphinxstylestrong{Name}
&\sphinxstyletheadfamily 
\sphinxAtStartPar
\sphinxstylestrong{Description}
\\
\sphinxmidrule
\sphinxtableatstartofbodyhook
\sphinxAtStartPar
\sphinxstylestrong{S35}
&
\sphinxAtStartPar
USB4+
&
\sphinxAtStartPar
USB PHY Data Plus for  Port 4
\\
\sphinxhline
\sphinxAtStartPar
\sphinxstylestrong{S36}
&
\sphinxAtStartPar
USB4\sphinxhyphen{}
&
\sphinxAtStartPar
USB PHY Data Minus for Port 4
\\
\sphinxhline
\sphinxAtStartPar
\sphinxstylestrong{P76}
&
\sphinxAtStartPar
USB4\_EN\_OC\#
&
\sphinxAtStartPar
USB  Over\sphinxhyphen{}Current Sense for Port 4
\\
\sphinxbottomrule
\end{tabulary}
\sphinxtableafterendhook\par
\sphinxattableend\end{savenotes}

\sphinxAtStartPar
Table 4\sphinxhyphen{}17 USB5 Port Signals


\begin{savenotes}\sphinxattablestart
\sphinxthistablewithglobalstyle
\centering
\begin{tabulary}{\linewidth}[t]{TTT}
\sphinxtoprule
\sphinxstyletheadfamily 
\sphinxAtStartPar
\sphinxstylestrong{Number}
&\sphinxstyletheadfamily 
\sphinxAtStartPar
\sphinxstylestrong{Name}
&\sphinxstyletheadfamily 
\sphinxAtStartPar
\sphinxstylestrong{Description}
\\
\sphinxmidrule
\sphinxtableatstartofbodyhook
\sphinxAtStartPar
\sphinxstylestrong{S59}
&
\sphinxAtStartPar
USB5+
&
\sphinxAtStartPar
USB PHY Data Plus for  Port 5
\\
\sphinxhline
\sphinxAtStartPar
\sphinxstylestrong{S60}
&
\sphinxAtStartPar
USB5\sphinxhyphen{}
&
\sphinxAtStartPar
USB PHY Data Minus for Port 5
\\
\sphinxhline
\sphinxAtStartPar
\sphinxstylestrong{S55}
&
\sphinxAtStartPar
USB5\_EN\_OC\#
&
\sphinxAtStartPar
USB  Over\sphinxhyphen{}Current Sense for Port 5
\\
\sphinxbottomrule
\end{tabulary}
\sphinxtableafterendhook\par
\sphinxattableend\end{savenotes}


\section{4.9 UART}
\label{\detokenize{hardware:uart}}

\subsection{4.9.1 Overview}
\label{\detokenize{hardware:id12}}
\sphinxAtStartPar
The Universal Asynchronous Receiver/Transmitter (UART) provides the serial communication capability with external devices. It offers low\sphinxhyphen{}speed IrDA compatibility through a level shifter and an RS \sphinxhyphen{} 232 cable, or by using an external circuit that converts infrared signals into electrical signals (for receiving) or converts electrical signals into signals to drive an infrared LED (for sending).

\sphinxAtStartPar
UART supports NRZ encoding format, RS485 compatible 9\sphinxhyphen{}bit data format, and IrDA\sphinxhyphen{}compatible infrared Slow Infrared (SIR) data rates.


\subsection{4.9.2 Features}
\label{\detokenize{hardware:id13}}\begin{itemize}
\item {} 
\sphinxAtStartPar
High\sphinxhyphen{}speed TIA/EIA\sphinxhyphen{}232\sphinxhyphen{}F compatible;

\item {} 
\sphinxAtStartPar
Serial IR interface at low speed, IrDA compatible (up to 115.2 Kbit/s);

\item {} 
\sphinxAtStartPar
Supports 9\sphinxhyphen{}bit or multi\sphinxhyphen{}point mode (RS\sphinxhyphen{}485) with automatic slave address detection;

\item {} 
\sphinxAtStartPar
RS\sphinxhyphen{}232 characters support 7 or 8\sphinxhyphen{}bit data, or RS\sphinxhyphen{}485 format supports 9\sphinxhyphen{}bit data;

\item {} 
\sphinxAtStartPar
1 or 2 stop bits.

\end{itemize}

\sphinxAtStartPar
Refer to  i.MX 8M Plus Applications Processor Reference Manual for more details.


\subsection{4.9.3 UART External Signal}
\label{\detokenize{hardware:uart-external-signal}}
\sphinxAtStartPar
Table 4\sphinxhyphen{}18 UART Interface Signal


\begin{savenotes}\sphinxattablestart
\sphinxthistablewithglobalstyle
\centering
\begin{tabulary}{\linewidth}[t]{TTT}
\sphinxtoprule
\sphinxstyletheadfamily 
\sphinxAtStartPar
\sphinxstylestrong{Number}
&\sphinxstyletheadfamily 
\sphinxAtStartPar
\sphinxstylestrong{Name}
&\sphinxstyletheadfamily 
\sphinxAtStartPar
\sphinxstylestrong{Description}
\\
\sphinxmidrule
\sphinxtableatstartofbodyhook
\sphinxAtStartPar
\sphinxstylestrong{P129}
&
\sphinxAtStartPar
SER0\_TX
&
\sphinxAtStartPar
Asynchronous Serial Data Output Port 0
\\
\sphinxhline
\sphinxAtStartPar
\sphinxstylestrong{P130}
&
\sphinxAtStartPar
SER0\_RX
&
\sphinxAtStartPar
Asynchronous  Serial Data Input Port 0
\\
\sphinxhline
\sphinxAtStartPar
\sphinxstylestrong{P131}
&
\sphinxAtStartPar
SER0\_RTS\#
&
\sphinxAtStartPar
Request  to Send Handshake Line for Port 0
\\
\sphinxhline
\sphinxAtStartPar
\sphinxstylestrong{P132}
&
\sphinxAtStartPar
SER0\_CTS\#
&
\sphinxAtStartPar
Clear  to Send Handshake Line for Port 0
\\
\sphinxhline
\sphinxAtStartPar
\sphinxstylestrong{P134}
&
\sphinxAtStartPar
SER1\_TX
&
\sphinxAtStartPar
Asynchronous Serial Data Output Port 1
\\
\sphinxhline
\sphinxAtStartPar
\sphinxstylestrong{P135}
&
\sphinxAtStartPar
SER1\_RX
&
\sphinxAtStartPar
Asynchronous  Serial Data Input Port 1
\\
\sphinxhline
\sphinxAtStartPar
\sphinxstylestrong{P136}
&
\sphinxAtStartPar
SER2\_TX
&
\sphinxAtStartPar
Asynchronous Serial Data Output Port 2
\\
\sphinxhline
\sphinxAtStartPar
\sphinxstylestrong{P137}
&
\sphinxAtStartPar
SER2\_RX
&
\sphinxAtStartPar
Asynchronous  Serial Data Input Port 2
\\
\sphinxhline
\sphinxAtStartPar
\sphinxstylestrong{P138}
&
\sphinxAtStartPar
SER2\_RTS\#
&
\sphinxAtStartPar
Request  to Send Handshake Line for Port 2
\\
\sphinxhline
\sphinxAtStartPar
\sphinxstylestrong{P139}
&
\sphinxAtStartPar
SER2\_CTS\#
&
\sphinxAtStartPar
Clear  to Send Handshake Line for Port 2
\\
\sphinxhline
\sphinxAtStartPar
\sphinxstylestrong{P140}
&
\sphinxAtStartPar
SER3\_TX
&
\sphinxAtStartPar
Asynchronous Serial Data Output Port 3
\\
\sphinxhline
\sphinxAtStartPar
\sphinxstylestrong{P141}
&
\sphinxAtStartPar
SER3\_RX
&
\sphinxAtStartPar
Asynchronous  Serial Data Input Port 3
\\
\sphinxbottomrule
\end{tabulary}
\sphinxtableafterendhook\par
\sphinxattableend\end{savenotes}


\section{4.10 FlexCAN}
\label{\detokenize{hardware:flexcan}}

\subsection{4.10.1 Overview}
\label{\detokenize{hardware:id14}}
\sphinxAtStartPar
The FlexCAN module is a communication controller implementing the CAN protocol according to the ISO 11898\sphinxhyphen{}1 standard and the CAN 2.0 B protocol specification.

\sphinxAtStartPar
The CAN protocol is primarily designed as a serial data bus for vehicles, meeting specific real\sphinxhyphen{}time processing and reliable operation requirements in the vehicle’s electromagnetic interference environment. The FlexCAN module fully implements the CAN protocol specification, Flexible Data\sphinxhyphen{}rate CAN (CAN FD) protocol, and CAN 2.0 protocol, supporting both standard and extended message frames as well as long data payloads.


\subsection{4.10.2 Features}
\label{\detokenize{hardware:id15}}\begin{itemize}
\item {} 
\sphinxAtStartPar
Fully implements the Flexible Data\sphinxhyphen{}Rate CAN (CAN FD) protocol specification and the CAN protocol specification version 2.0 B.
\begin{itemize}
\item {} 
\sphinxAtStartPar
Standard Data Frames

\item {} 
\sphinxAtStartPar
Extended Data Frames

\item {} 
\sphinxAtStartPar
Data length from 0 to 64 bytes.

\item {} 
\sphinxAtStartPar
Content\sphinxhyphen{}related addressing

\end{itemize}

\item {} 
\sphinxAtStartPar
Compliant with the ISO 11898\sphinxhyphen{}1 standard.

\end{itemize}

\sphinxAtStartPar
Refer to i.MX 8M Plus Applications Processor Reference Manual for more details.


\subsection{4.10.3 FlexCAN External Signals}
\label{\detokenize{hardware:flexcan-external-signals}}
\sphinxAtStartPar
Table 4\sphinxhyphen{}19 CAN\sphinxhyphen{} FD Port Signal


\begin{savenotes}\sphinxattablestart
\sphinxthistablewithglobalstyle
\centering
\begin{tabulary}{\linewidth}[t]{TTT}
\sphinxtoprule
\sphinxstyletheadfamily 
\sphinxAtStartPar
\sphinxstylestrong{Number}
&\sphinxstyletheadfamily 
\sphinxAtStartPar
\sphinxstylestrong{Name}
&\sphinxstyletheadfamily 
\sphinxAtStartPar
\sphinxstylestrong{Description}
\\
\sphinxmidrule
\sphinxtableatstartofbodyhook
\sphinxAtStartPar
\sphinxstylestrong{P143}
&
\sphinxAtStartPar
CAN0\_TX
&
\sphinxAtStartPar
CAN  Port 0 Transmit Output
\\
\sphinxhline
\sphinxAtStartPar
\sphinxstylestrong{P144}
&
\sphinxAtStartPar
CAN0\_RX
&
\sphinxAtStartPar
CAN  Port 0 Receive Input
\\
\sphinxhline
\sphinxAtStartPar
\sphinxstylestrong{P145}
&
\sphinxAtStartPar
CAN1\_TX
&
\sphinxAtStartPar
CAN  Port 1 Transmit Output
\\
\sphinxhline
\sphinxAtStartPar
\sphinxstylestrong{P146}
&
\sphinxAtStartPar
CAN1\_RX
&
\sphinxAtStartPar
CAN  Port1 Receive Input
\\
\sphinxbottomrule
\end{tabulary}
\sphinxtableafterendhook\par
\sphinxattableend\end{savenotes}


\section{4.11 uSDHC}
\label{\detokenize{hardware:usdhc}}

\subsection{4.11.1 Overview}
\label{\detokenize{hardware:id16}}
\sphinxAtStartPar
The FET\sphinxhyphen{}MX8MP\sphinxhyphen{}SMARC exposes a 4\sphinxhyphen{}bit interface of the uSDHC2 controller to support communication between the host system and SD/SDIO/MMC cards.

\sphinxAtStartPar
Key Features of uSDHC2:
\begin{itemize}
\item {} 
\sphinxAtStartPar
Compliant with SD/SDIO standard, up to version 3.0.

\item {} 
\sphinxAtStartPar
Compliant with MMC standard, up to version 5.1.

\item {} 
\sphinxAtStartPar
Supports 1.8 V and 3.3 V operation modes.

\item {} 
\sphinxAtStartPar
1\sphinxhyphen{}bit/4\sphinxhyphen{}bit SD and SDIO modes, as well as 1\sphinxhyphen{}bit/4\sphinxhyphen{}bit MMC mode.

\item {} 
\sphinxAtStartPar
Supports up to SDR104 baud rate.

\end{itemize}


\subsection{4.11.2 Features}
\label{\detokenize{hardware:id17}}\begin{itemize}
\item {} 
\sphinxAtStartPar
Compliant with SD Host Controller Standard Specification version 2.0/3.0.

\item {} 
\sphinxAtStartPar
Compatible with MMC System Specification versions 4.2/4.3/4.4/4.41/4.5/5.0/5.1.

\item {} 
\sphinxAtStartPar
Compatible with SD memory card specification version 3.0 and supports extended\sphinxhyphen{}capacity SD memory cards.

\item {} 
\sphinxAtStartPar
Compatible with SDIO card specification version 2.0/3.

\item {} 
\sphinxAtStartPar
Designed to work with SD memory cards, miniSD memory cards, SDIO, miniSDIO, SD combo cards, MMC, MMC plus, and MMC RS cards

\item {} 
\sphinxAtStartPar
Card bus clock frequency up to 208 MHz.

\item {} 
\sphinxAtStartPar
Supports 1\sphinxhyphen{}bit/4\sphinxhyphen{}bit SD and SDIO modes, as well as 1\sphinxhyphen{}bit/4\sphinxhyphen{}bit MMC mode

\end{itemize}


\section{4.12 uSDHC External Signal}
\label{\detokenize{hardware:usdhc-external-signal}}
\sphinxAtStartPar
Table 4\sphinxhyphen{}20 TF Interface Signal


\begin{savenotes}\sphinxattablestart
\sphinxthistablewithglobalstyle
\centering
\begin{tabulary}{\linewidth}[t]{TTT}
\sphinxtoprule
\sphinxstyletheadfamily 
\sphinxAtStartPar
\sphinxstylestrong{Number}
&\sphinxstyletheadfamily 
\sphinxAtStartPar
\sphinxstylestrong{Name}
&\sphinxstyletheadfamily 
\sphinxAtStartPar
\sphinxstylestrong{Description}
\\
\sphinxmidrule
\sphinxtableatstartofbodyhook
\sphinxAtStartPar
\sphinxstylestrong{P39}
&
\sphinxAtStartPar
SDIO\_D0
&
\sphinxAtStartPar
SDIO  Data0 lines
\\
\sphinxhline
\sphinxAtStartPar
\sphinxstylestrong{P40}
&
\sphinxAtStartPar
SDIO\_D1
&
\sphinxAtStartPar
SDIO  Data1 lines
\\
\sphinxhline
\sphinxAtStartPar
\sphinxstylestrong{P41}
&
\sphinxAtStartPar
SDIO\_D2
&
\sphinxAtStartPar
SDIO  Data2 lines
\\
\sphinxhline
\sphinxAtStartPar
\sphinxstylestrong{P42}
&
\sphinxAtStartPar
SDIO\_D3
&
\sphinxAtStartPar
SDIO  Data3 lines
\\
\sphinxhline
\sphinxAtStartPar
\sphinxstylestrong{P33}
&
\sphinxAtStartPar
SDIO\_WP
&
\sphinxAtStartPar
SDIO  Write Protect
\\
\sphinxhline
\sphinxAtStartPar
\sphinxstylestrong{P34}
&
\sphinxAtStartPar
SDIO\_CMD
&
\sphinxAtStartPar
SDIO  Command/Response
\\
\sphinxhline
\sphinxAtStartPar
\sphinxstylestrong{P35}
&
\sphinxAtStartPar
SDIO\_CD\#
&
\sphinxAtStartPar
SDIO  Card Detect
\\
\sphinxhline
\sphinxAtStartPar
\sphinxstylestrong{P36}
&
\sphinxAtStartPar
SDIO\_CK
&
\sphinxAtStartPar
SDIO  Clock
\\
\sphinxhline
\sphinxAtStartPar
\sphinxstylestrong{P37}
&
\sphinxAtStartPar
SDIO\_PWR\_EN
&
\sphinxAtStartPar
SDIO  Power Enable
\\
\sphinxbottomrule
\end{tabulary}
\sphinxtableafterendhook\par
\sphinxattableend\end{savenotes}


\section{4.13 I2C}
\label{\detokenize{hardware:i2c}}

\subsection{4.13.1 Overview}
\label{\detokenize{hardware:id18}}
\sphinxAtStartPar
I2C is a two\sphinxhyphen{}wire bidirectional serial bus that provides a simple, efficient method of data exchange that minimizes interconnections between devices. The bus is suitable for applications that require occasional communication over short distances between multiple devices. The flexible I2C standard allows additional devices to be connected to the bus for expansion and system development.


\subsection{4.13.2 Features}
\label{\detokenize{hardware:id19}}\begin{itemize}
\item {} 
\sphinxAtStartPar
Compatible with I2C bus standard.

\item {} 
\sphinxAtStartPar
Multi\sphinxhyphen{}host operation.

\item {} 
\sphinxAtStartPar
Start and stop signal generation/detection

\item {} 
\sphinxAtStartPar
Repeated start signal generation

\item {} 
\sphinxAtStartPar
Response bit generation/detection

\item {} 
\sphinxAtStartPar
Bus busy detection

\end{itemize}

\sphinxAtStartPar
Refer to i.MX 8M Plus Applications Processor Reference Manual for more details.


\subsection{4.13.3 I2C External Signal}
\label{\detokenize{hardware:i2c-external-signal}}
\sphinxAtStartPar
Table 4\sphinxhyphen{}21 I2C\_PM Interface Signals


\begin{savenotes}\sphinxattablestart
\sphinxthistablewithglobalstyle
\centering
\begin{tabulary}{\linewidth}[t]{TTT}
\sphinxtoprule
\sphinxstyletheadfamily 
\sphinxAtStartPar
\sphinxstylestrong{Number}
&\sphinxstyletheadfamily 
\sphinxAtStartPar
\sphinxstylestrong{Name}
&\sphinxstyletheadfamily 
\sphinxAtStartPar
\sphinxstylestrong{Description}
\\
\sphinxmidrule
\sphinxtableatstartofbodyhook
\sphinxAtStartPar
\sphinxstylestrong{P121}
&
\sphinxAtStartPar
I2C\_PM\_CK
&
\sphinxAtStartPar
Power  management I2C bus CLK
\\
\sphinxhline
\sphinxAtStartPar
\sphinxstylestrong{P122}
&
\sphinxAtStartPar
I2C\_PM\_DAT
&
\sphinxAtStartPar
Power  management I2C bus DATA
\\
\sphinxbottomrule
\end{tabulary}
\sphinxtableafterendhook\par
\sphinxattableend\end{savenotes}

\sphinxAtStartPar
Table 4\sphinxhyphen{}22 I2C\_CAM0 Interface Signals


\begin{savenotes}\sphinxattablestart
\sphinxthistablewithglobalstyle
\centering
\begin{tabulary}{\linewidth}[t]{TTT}
\sphinxtoprule
\sphinxstyletheadfamily 
\sphinxAtStartPar
\sphinxstylestrong{Number}
&\sphinxstyletheadfamily 
\sphinxAtStartPar
\sphinxstylestrong{Name}
&\sphinxstyletheadfamily 
\sphinxAtStartPar
\sphinxstylestrong{Description}
\\
\sphinxmidrule
\sphinxtableatstartofbodyhook
\sphinxAtStartPar
\sphinxstylestrong{S5}
&
\sphinxAtStartPar
I2C\_CAM0\_CK
&
\sphinxAtStartPar
I2C  clock for serial camera data support link
\\
\sphinxhline
\sphinxAtStartPar
\sphinxstylestrong{S7}
&
\sphinxAtStartPar
I2C\_CAM0\_DAT
&
\sphinxAtStartPar
I2C  data for serial camera data support link
\\
\sphinxbottomrule
\end{tabulary}
\sphinxtableafterendhook\par
\sphinxattableend\end{savenotes}

\sphinxAtStartPar
Table 4\sphinxhyphen{}23 I2C\_CAM1 Interface Signals


\begin{savenotes}\sphinxattablestart
\sphinxthistablewithglobalstyle
\centering
\begin{tabulary}{\linewidth}[t]{TTT}
\sphinxtoprule
\sphinxstyletheadfamily 
\sphinxAtStartPar
\sphinxstylestrong{Number}
&\sphinxstyletheadfamily 
\sphinxAtStartPar
\sphinxstylestrong{Name}
&\sphinxstyletheadfamily 
\sphinxAtStartPar
\sphinxstylestrong{Description}
\\
\sphinxmidrule
\sphinxtableatstartofbodyhook
\sphinxAtStartPar
\sphinxstylestrong{S1}
&
\sphinxAtStartPar
I2C\_CAM1\_CK
&
\sphinxAtStartPar
I2C  clock for serial camera data support link
\\
\sphinxhline
\sphinxAtStartPar
\sphinxstylestrong{S2}
&
\sphinxAtStartPar
I2C\_CAM1\_DAT
&
\sphinxAtStartPar
I2C  data for serial camera data support link
\\
\sphinxbottomrule
\end{tabulary}
\sphinxtableafterendhook\par
\sphinxattableend\end{savenotes}

\sphinxAtStartPar
Table 4\sphinxhyphen{}24 I2C\_GP Interface Signals


\begin{savenotes}\sphinxattablestart
\sphinxthistablewithglobalstyle
\centering
\begin{tabulary}{\linewidth}[t]{TTT}
\sphinxtoprule
\sphinxstyletheadfamily 
\sphinxAtStartPar
\sphinxstylestrong{Number}
&\sphinxstyletheadfamily 
\sphinxAtStartPar
\sphinxstylestrong{Name}
&\sphinxstyletheadfamily 
\sphinxAtStartPar
\sphinxstylestrong{Description}
\\
\sphinxmidrule
\sphinxtableatstartofbodyhook
\sphinxAtStartPar
\sphinxstylestrong{S48}
&
\sphinxAtStartPar
I2C\_GP\_CK
&
\sphinxAtStartPar
General  Purpose I2C Clock Signal
\\
\sphinxhline
\sphinxAtStartPar
\sphinxstylestrong{S49}
&
\sphinxAtStartPar
I2C\_GP\_DAT
&
\sphinxAtStartPar
General  Purpose I2C Data Signal
\\
\sphinxbottomrule
\end{tabulary}
\sphinxtableafterendhook\par
\sphinxattableend\end{savenotes}

\sphinxAtStartPar
Table 4\sphinxhyphen{}25 I2C\_LCD Interface Signals


\begin{savenotes}\sphinxattablestart
\sphinxthistablewithglobalstyle
\centering
\begin{tabulary}{\linewidth}[t]{TTT}
\sphinxtoprule
\sphinxstyletheadfamily 
\sphinxAtStartPar
\sphinxstylestrong{Number}
&\sphinxstyletheadfamily 
\sphinxAtStartPar
\sphinxstylestrong{Name}
&\sphinxstyletheadfamily 
\sphinxAtStartPar
\sphinxstylestrong{Description}
\\
\sphinxmidrule
\sphinxtableatstartofbodyhook
\sphinxAtStartPar
\sphinxstylestrong{S139}
&
\sphinxAtStartPar
I2C\_LCD\_CK
&
\sphinxAtStartPar
DDC  Clock Line Used for Flat Panel Detection and Control
\\
\sphinxhline
\sphinxAtStartPar
\sphinxstylestrong{S140}
&
\sphinxAtStartPar
I2C\_LCD\_DAT
&
\sphinxAtStartPar
DDC  Data Line Used for Flat Panel Detection and Control
\\
\sphinxbottomrule
\end{tabulary}
\sphinxtableafterendhook\par
\sphinxattableend\end{savenotes}


\section{4.14 ECSPI \& FlexSPI}
\label{\detokenize{hardware:ecspi-flexspi}}

\subsection{4.14.1 Overview}
\label{\detokenize{hardware:id20}}
\sphinxAtStartPar
The Enhanced Configurable Serial Peripheral Interface (ECSPI) is a full\sphinxhyphen{}duplex, synchronous, four\sphinxhyphen{}wire serial communication module.

\sphinxAtStartPar
ECSPI includes a 64×32 receive buffer (RXFIFO) and a 64×32 transmit buffer (TXFIFO). By leveraging a data FIFO (First\sphinxhyphen{}In\sphinxhyphen{}First\sphinxhyphen{}Out) buffer, the Enhanced Configurable Serial Peripheral Interface (ECSPI) can achieve high\sphinxhyphen{}speed data communication while significantly reducing software interrupt frequency.


\subsection{4.14.2 ECSPI Features}
\label{\detokenize{hardware:ecspi-features}}\begin{itemize}
\item {} 
\sphinxAtStartPar
Full\sphinxhyphen{}duplex synchronous serial interface

\item {} 
\sphinxAtStartPar
Can be configured as master/slave

\item {} 
\sphinxAtStartPar
A chip select (SS) signal

\item {} 
\sphinxAtStartPar
The continuous transfer feature allows data transmission of unlimited length.

\item {} 
\sphinxAtStartPar
Both transmit and receive data use 32\sphinxhyphen{}bit wide, 64\sphinxhyphen{}entry FIFO.

\item {} 
\sphinxAtStartPar
The polarity and phase of both the Chip Select (SS) and SPI Clock (SCLK) signals can be configured into 4 different modes.

\item {} 
\sphinxAtStartPar
Supports Direct Memory Access (DMA)

\item {} 
\sphinxAtStartPar
Data rates up to 52 Mbit/s.

\end{itemize}


\subsection{4.14.3 FlexSPI Overview}
\label{\detokenize{hardware:flexspi-overview}}\begin{itemize}
\item {} 
\sphinxAtStartPar
FlexSPI lanes support single/dual/quad mode data transfer (1/2/4 bidirectional data lines)

\item {} 
\sphinxAtStartPar
FlexSPI supports communication with serial flash and serial RAM devices

\end{itemize}


\subsection{4.14.4 FlexSPI Features}
\label{\detokenize{hardware:flexspi-features}}\begin{itemize}
\item {} 
\sphinxAtStartPar
Flexible timing (LUT table) supports various vendor devices:
\begin{itemize}
\item {} 
\sphinxAtStartPar
Serial NOR Flash or other devices with SPI protocol similar to Serial NOR Flash

\item {} 
\sphinxAtStartPar
Serial NAND Flash

\item {} 
\sphinxAtStartPar
FPGA Device

\end{itemize}

\item {} 
\sphinxAtStartPar
Flash Access Modes
\begin{itemize}
\item {} 
\sphinxAtStartPar
Single/Dual/Quad Mode

\item {} 
\sphinxAtStartPar
SDR/DDR mode

\item {} 
\sphinxAtStartPar
Independent/Parallel Mode

\end{itemize}

\end{itemize}

\sphinxAtStartPar
Refer to i.MX 8M Plus Applications Processor Reference Manual for more details.


\subsection{4.14.5 FlexSPI External Signals}
\label{\detokenize{hardware:flexspi-external-signals}}
\sphinxAtStartPar
Table 4\sphinxhyphen{}26 SPI0 Interface Signals


\begin{savenotes}\sphinxattablestart
\sphinxthistablewithglobalstyle
\centering
\begin{tabulary}{\linewidth}[t]{TTT}
\sphinxtoprule
\sphinxstyletheadfamily 
\sphinxAtStartPar
\sphinxstylestrong{Number}
&\sphinxstyletheadfamily 
\sphinxAtStartPar
\sphinxstylestrong{Name}
&\sphinxstyletheadfamily 
\sphinxAtStartPar
\sphinxstylestrong{Description}
\\
\sphinxmidrule
\sphinxtableatstartofbodyhook
\sphinxAtStartPar
\sphinxstylestrong{P31}
&
\sphinxAtStartPar
SPI0\_CS1\#
&
\sphinxAtStartPar
SPI0  Master Chip Select 1
\\
\sphinxhline
\sphinxAtStartPar
\sphinxstylestrong{P43}
&
\sphinxAtStartPar
SPI0\_CS0\#
&
\sphinxAtStartPar
SPI0  Master Chip Select 0
\\
\sphinxhline
\sphinxAtStartPar
\sphinxstylestrong{P44}
&
\sphinxAtStartPar
SPI0\_CK
&
\sphinxAtStartPar
SPI0  Clock
\\
\sphinxhline
\sphinxAtStartPar
\sphinxstylestrong{P45}
&
\sphinxAtStartPar
SPI0\_DIN
&
\sphinxAtStartPar
SPI0  Master input / Slave output
\\
\sphinxhline
\sphinxAtStartPar
\sphinxstylestrong{P46}
&
\sphinxAtStartPar
SPI0\_DO
&
\sphinxAtStartPar
SPI0  Master output / Slave input
\\
\sphinxbottomrule
\end{tabulary}
\sphinxtableafterendhook\par
\sphinxattableend\end{savenotes}

\sphinxAtStartPar
Table 4\sphinxhyphen{}27 QSPI Interface Signal


\begin{savenotes}\sphinxattablestart
\sphinxthistablewithglobalstyle
\centering
\begin{tabulary}{\linewidth}[t]{TTT}
\sphinxtoprule
\sphinxstyletheadfamily 
\sphinxAtStartPar
\sphinxstylestrong{Number}
&\sphinxstyletheadfamily 
\sphinxAtStartPar
\sphinxstylestrong{Name}
&\sphinxstyletheadfamily 
\sphinxAtStartPar
\sphinxstylestrong{Description}
\\
\sphinxmidrule
\sphinxtableatstartofbodyhook
\sphinxAtStartPar
\sphinxstylestrong{P54}
&
\sphinxAtStartPar
QSPI\_CS0\#
&
\sphinxAtStartPar
QSPI  Master Chip Select 0
\\
\sphinxhline
\sphinxAtStartPar
\sphinxstylestrong{P55}
&
\sphinxAtStartPar
QSPI\_CS1\#
&
\sphinxAtStartPar
QSPI  Master Chip Select 1
\\
\sphinxhline
\sphinxAtStartPar
\sphinxstylestrong{P56}
&
\sphinxAtStartPar
QSPI\_CK
&
\sphinxAtStartPar
QSPI  Clock
\\
\sphinxhline
\sphinxAtStartPar
\sphinxstylestrong{P58}
&
\sphinxAtStartPar
QSPI\_IO\_0
&
\sphinxAtStartPar
QSPI  Data2 input / output
\\
\sphinxhline
\sphinxAtStartPar
\sphinxstylestrong{P57}
&
\sphinxAtStartPar
QSPI\_IO\_1
&
\sphinxAtStartPar
QSPI  Data1 input / output
\\
\sphinxhline
\sphinxAtStartPar
\sphinxstylestrong{S56}
&
\sphinxAtStartPar
QSPI\_IO\_2
&
\sphinxAtStartPar
QSPI  Data2 input / output
\\
\sphinxhline
\sphinxAtStartPar
\sphinxstylestrong{S57}
&
\sphinxAtStartPar
QSPI\_IO\_3
&
\sphinxAtStartPar
QSPI  Data3 input / output
\\
\sphinxbottomrule
\end{tabulary}
\sphinxtableafterendhook\par
\sphinxattableend\end{savenotes}


\section{4.15 PWM}
\label{\detokenize{hardware:pwm}}

\subsection{4.15.1 Overview}
\label{\detokenize{hardware:id21}}
\sphinxAtStartPar
The Pulse Width Modulation (PWM) has a 16\sphinxhyphen{}bit counter and is optimized to generate sound from stored sample audio images or to produce tones. It uses 16\sphinxhyphen{}bit resolution and a 4 × 16 data FIFO.


\subsection{4.15.2 Features}
\label{\detokenize{hardware:id22}}\begin{itemize}
\item {} 
\sphinxAtStartPar
16\sphinxhyphen{}bit up counter with selectable clock source

\item {} 
\sphinxAtStartPar
4×16 FIFO to minimize interrupt overhead

\item {} 
\sphinxAtStartPar
12\sphinxhyphen{}bit prescaler for clock division

\item {} 
\sphinxAtStartPar
Sound and melody generation

\item {} 
\sphinxAtStartPar
Configurable for active\sphinxhyphen{}high or active\sphinxhyphen{}low output

\item {} 
\sphinxAtStartPar
Programmable to operate in low\sphinxhyphen{}power mode

\end{itemize}

\sphinxAtStartPar
Refer to i.MX 8M Plus Applications Processor Reference Manual for more details.


\subsection{4.15.3 PWM External Signal}
\label{\detokenize{hardware:pwm-external-signal}}
\sphinxAtStartPar
Table 4\sphinxhyphen{}28 PWM Interface Signal


\begin{savenotes}\sphinxattablestart
\sphinxthistablewithglobalstyle
\centering
\begin{tabulary}{\linewidth}[t]{TTT}
\sphinxtoprule
\sphinxstyletheadfamily 
\sphinxAtStartPar
\sphinxstylestrong{Number}
&\sphinxstyletheadfamily 
\sphinxAtStartPar
\sphinxstylestrong{Name}
&\sphinxstyletheadfamily 
\sphinxAtStartPar
\sphinxstylestrong{Description}
\\
\sphinxmidrule
\sphinxtableatstartofbodyhook
\sphinxAtStartPar
\sphinxstylestrong{P113}
&
\sphinxAtStartPar
PWM\_OUT
&
\sphinxAtStartPar
Fan Speed Control
\\
\sphinxhline
\sphinxAtStartPar
\sphinxstylestrong{S122}
&
\sphinxAtStartPar
LCD1\_BKLT\_PWM
&
\sphinxAtStartPar
Secondary  LVDS Channel Brightness Control
\\
\sphinxhline
\sphinxAtStartPar
\sphinxstylestrong{S141}
&
\sphinxAtStartPar
LCD0\_BKLT\_PWM
&
\sphinxAtStartPar
Primary  LVDS Channel Brightness Control
\\
\sphinxbottomrule
\end{tabulary}
\sphinxtableafterendhook\par
\sphinxattableend\end{savenotes}


\section{4.16 GPIO}
\label{\detokenize{hardware:gpio}}
\sphinxAtStartPar
The FET\sphinxhyphen{}MX8MP\sphinxhyphen{}SMARC provides IO pins that can be used as GPIO.


\subsection{4.16.1 GPIO Signals}
\label{\detokenize{hardware:gpio-signals}}
\sphinxAtStartPar
Table 4\sphinxhyphen{}29 GPIO Interface Signal


\begin{savenotes}\sphinxattablestart
\sphinxthistablewithglobalstyle
\centering
\begin{tabulary}{\linewidth}[t]{TTT}
\sphinxtoprule
\sphinxstyletheadfamily 
\sphinxAtStartPar
\sphinxstylestrong{Number}
&\sphinxstyletheadfamily 
\sphinxAtStartPar
\sphinxstylestrong{Name}
&\sphinxstyletheadfamily 
\sphinxAtStartPar
\sphinxstylestrong{Description}
\\
\sphinxmidrule
\sphinxtableatstartofbodyhook
\sphinxAtStartPar
\sphinxstylestrong{P108}
&
\sphinxAtStartPar
GPIO0
&
\sphinxAtStartPar
GPIO Pin 0 Preferred Output
\\
\sphinxhline
\sphinxAtStartPar
\sphinxstylestrong{P109}
&
\sphinxAtStartPar
GPIO1
&
\sphinxAtStartPar
GPIO Pin 1 Preferred Output
\\
\sphinxhline
\sphinxAtStartPar
\sphinxstylestrong{P110}
&
\sphinxAtStartPar
GPIO2
&
\sphinxAtStartPar
GPIO Pin 2 Preferred Output
\\
\sphinxhline
\sphinxAtStartPar
\sphinxstylestrong{P111}
&
\sphinxAtStartPar
GPIO3
&
\sphinxAtStartPar
GPIO Pin 3 Preferred Output
\\
\sphinxhline
\sphinxAtStartPar
\sphinxstylestrong{P112}
&
\sphinxAtStartPar
GPIO4
&
\sphinxAtStartPar
GPIO Pin 4 Preferred Output
\\
\sphinxhline
\sphinxAtStartPar
\sphinxstylestrong{P113}
&
\sphinxAtStartPar
GPIO5
&
\sphinxAtStartPar
GPIO Pin 5 Preferred Output
\\
\sphinxhline
\sphinxAtStartPar
\sphinxstylestrong{P114}
&
\sphinxAtStartPar
GPIO6
&
\sphinxAtStartPar
GPIO Pin 6 Preferred Output
\\
\sphinxhline
\sphinxAtStartPar
\sphinxstylestrong{P115}
&
\sphinxAtStartPar
GPIO7
&
\sphinxAtStartPar
GPIO Pin 7 Preferred Output
\\
\sphinxhline
\sphinxAtStartPar
\sphinxstylestrong{P116}
&
\sphinxAtStartPar
GPIO8
&
\sphinxAtStartPar
GPIO Pin 8 Preferred Output
\\
\sphinxhline
\sphinxAtStartPar
\sphinxstylestrong{P117}
&
\sphinxAtStartPar
GPIO9
&
\sphinxAtStartPar
GPIO Pin 9 Preferred Output
\\
\sphinxhline
\sphinxAtStartPar
\sphinxstylestrong{P118}
&
\sphinxAtStartPar
GPIO10
&
\sphinxAtStartPar
GPIO Pin 10 Preferred Output
\\
\sphinxhline
\sphinxAtStartPar
\sphinxstylestrong{P119}
&
\sphinxAtStartPar
GPIO11
&
\sphinxAtStartPar
GPIO Pin 11 Preferred Output
\\
\sphinxhline
\sphinxAtStartPar
\sphinxstylestrong{S142}
&
\sphinxAtStartPar
GPIO12
&
\sphinxAtStartPar
GPIO Pin 12 Preferred Output
\\
\sphinxhline
\sphinxAtStartPar
\sphinxstylestrong{S123}
&
\sphinxAtStartPar
GPIO13
&
\sphinxAtStartPar
GPIO Pin 13 Preferred Output
\\
\sphinxbottomrule
\end{tabulary}
\sphinxtableafterendhook\par
\sphinxattableend\end{savenotes}


\section{4.17 Management IO}
\label{\detokenize{hardware:management-io}}
\sphinxAtStartPar
Management IO complies with the SMARC specification and is used for power management and other functions on the carrier board.


\subsection{4.17.1 IO External Signal Management}
\label{\detokenize{hardware:io-external-signal-management}}
\sphinxAtStartPar
Table 4\sphinxhyphen{}30 GPIO Interface Signal


\begin{savenotes}\sphinxattablestart
\sphinxthistablewithglobalstyle
\centering
\begin{tabulary}{\linewidth}[t]{TTT}
\sphinxtoprule
\sphinxstyletheadfamily 
\sphinxAtStartPar
\sphinxstylestrong{Number}
&\sphinxstyletheadfamily 
\sphinxAtStartPar
\sphinxstylestrong{Name}
&\sphinxstyletheadfamily 
\sphinxAtStartPar
\sphinxstylestrong{Description}
\\
\sphinxmidrule
\sphinxtableatstartofbodyhook
\sphinxAtStartPar
\sphinxstylestrong{P1}
&
\sphinxAtStartPar
SMB\_ALERT\#
&
\sphinxAtStartPar
SMBus Alert \# (interrupt) signal.
\\
\sphinxhline
\sphinxAtStartPar
\sphinxstylestrong{S145}
&
\sphinxAtStartPar
WDT\_TIME\_OUT\#
&
\sphinxAtStartPar
Watchdog timer output, active low.
\\
\sphinxhline
\sphinxAtStartPar
\sphinxstylestrong{S148}
&
\sphinxAtStartPar
LID\#
&
\sphinxAtStartPar
Module cover open/close indicator.
\\
\sphinxhline
\sphinxAtStartPar
\sphinxstylestrong{S149}
&
\sphinxAtStartPar
SLEEP\#
&
\sphinxAtStartPar
On\sphinxhyphen{}board sleep indicator
\\
\sphinxhline
\sphinxAtStartPar
\sphinxstylestrong{S151}
&
\sphinxAtStartPar
CHARGING\#
&
\sphinxAtStartPar
When the battery is charging, the carrier board pulls this signal low.
\\
\sphinxhline
\sphinxAtStartPar
\sphinxstylestrong{S152}
&
\sphinxAtStartPar
CHARGER\_PRSNT\#
&
\sphinxAtStartPar
If the DC input of the battery charger is present, the carrier board pulls this signal low.
\\
\sphinxhline
\sphinxAtStartPar
\sphinxstylestrong{S153}
&
\sphinxAtStartPar
CARRIER\_STBY\#
&
\sphinxAtStartPar
The module should pull this signal low when the system is in standby power state
\\
\sphinxhline
\sphinxAtStartPar
\sphinxstylestrong{S156}
&
\sphinxAtStartPar
BATLOW\#
&
\sphinxAtStartPar
Low battery indication for the module
\\
\sphinxhline
\sphinxAtStartPar
\sphinxstylestrong{S157}
&
\sphinxAtStartPar
TEST\#
&
\sphinxAtStartPar
The carrier board pulls this signal low to enter the test mode (no such mode by default, which can be customized according to the requirements).
\\
\sphinxbottomrule
\end{tabulary}
\sphinxtableafterendhook\par
\sphinxattableend\end{savenotes}


\section{4.18 JTAG}
\label{\detokenize{hardware:jtag}}

\subsection{4.18.1 Overview}
\label{\detokenize{hardware:id23}}
\sphinxAtStartPar
The SJC provides a JTAG interface to the internal logic (designed to be compatible with the JTAG TAP standard). The i.MX 8M Plus series processors use the JTAG port for production, testing, and system debugging. In addition, the SJC provides standard support for Boundary Scan Registers (BSR), with a design compatible with the IEEE 1149.1 and IEEE 1149.6 standards.

\sphinxAtStartPar
During initial platform lab debugging, manufacturing testing, troubleshooting, and software debugging by authorized entities, access to the JTAG port must be available. The i.MX 8M Plus SJC integrates three security modes to prevent unauthorized access. These modes are selected through eFUSE configuration.


\subsection{4.18.2 JTAG External Signals}
\label{\detokenize{hardware:jtag-external-signals}}
\sphinxAtStartPar
Table 4\sphinxhyphen{}31 JTAG Interface Signal


\begin{savenotes}\sphinxattablestart
\sphinxthistablewithglobalstyle
\centering
\begin{tabulary}{\linewidth}[t]{TTT}
\sphinxtoprule
\sphinxstyletheadfamily 
\sphinxAtStartPar
\sphinxstylestrong{Number}
&\sphinxstyletheadfamily 
\sphinxAtStartPar
\sphinxstylestrong{Name}
&\sphinxstyletheadfamily 
\sphinxAtStartPar
\sphinxstylestrong{Description}
\\
\sphinxmidrule
\sphinxtableatstartofbodyhook
\sphinxAtStartPar
\sphinxstylestrong{U22\sphinxhyphen{}1}
&
\sphinxAtStartPar
VDD\_1V8
&
\sphinxAtStartPar
1.8V  Power
\\
\sphinxhline
\sphinxAtStartPar
\sphinxstylestrong{U22\sphinxhyphen{}2}
&
\sphinxAtStartPar
JTAG\_TMS
&
\sphinxAtStartPar
JTAG  mode select
\\
\sphinxhline
\sphinxAtStartPar
\sphinxstylestrong{U22\sphinxhyphen{}3}
&
\sphinxAtStartPar
JTAG\_TCK
&
\sphinxAtStartPar
JTAG  clock
\\
\sphinxhline
\sphinxAtStartPar
\sphinxstylestrong{U22\sphinxhyphen{}4}
&
\sphinxAtStartPar
JTAG\_TDO
&
\sphinxAtStartPar
JTAG  data out
\\
\sphinxhline
\sphinxAtStartPar
\sphinxstylestrong{U22\sphinxhyphen{}5}
&
\sphinxAtStartPar
JTAG\_TDI
&
\sphinxAtStartPar
JTAG  data in
\\
\sphinxhline
\sphinxAtStartPar
\sphinxstylestrong{U22\sphinxhyphen{}6}
&
\sphinxAtStartPar
JTAG\_TRST\#
&
\sphinxAtStartPar
JTAG  reset, active low
\\
\sphinxhline
\sphinxAtStartPar
\sphinxstylestrong{U22\sphinxhyphen{}7}
&
\sphinxAtStartPar
GND
&
\sphinxAtStartPar
GND
\\
\sphinxbottomrule
\end{tabulary}
\sphinxtableafterendhook\par
\sphinxattableend\end{savenotes}


\section{4.19 RTC}
\label{\detokenize{hardware:rtc}}

\subsection{4.19.1 Overview}
\label{\detokenize{hardware:id24}}
\sphinxAtStartPar
The FET\sphinxhyphen{}MX8MPQ\sphinxhyphen{}SMARC uses a low\sphinxhyphen{}power real\sphinxhyphen{}time clock chip that supports programmable clock outputs, interrupt outputs, and low voltage detection. All addresses and data are transferred serially via a two\sphinxhyphen{}wire bidirectional I2C bus at a maximum speed of 400 kbps. After each data byte is read or written, the register address is automatically incremented.


\subsection{4.19.2 Features}
\label{\detokenize{hardware:id25}}\begin{itemize}
\item {} 
\sphinxAtStartPar
Based on a 32.768 kHz crystal oscillator, it provides year, month, day, weekday, hour, minute, and second timekeeping.

\item {} 
\sphinxAtStartPar
Century Logo

\item {} 
\sphinxAtStartPar
Clock operating voltage: 1.0\sphinxhyphen{}5.5V (room temperature)

\item {} 
\sphinxAtStartPar
Low standby current; typical 0.25 μA(VDD = 3.0 V,Tamb = 25 °C)

\item {} 
\sphinxAtStartPar
Alarm and timer functions.

\end{itemize}


\subsection{4.19.3 RTC Power}
\label{\detokenize{hardware:rtc-power}}
\sphinxAtStartPar
Table 4\sphinxhyphen{} 32 RTC Power


\begin{savenotes}\sphinxattablestart
\sphinxthistablewithglobalstyle
\centering
\begin{tabulary}{\linewidth}[t]{TTT}
\sphinxtoprule
\sphinxstyletheadfamily 
\sphinxAtStartPar
\sphinxstylestrong{Number}
&\sphinxstyletheadfamily 
\sphinxAtStartPar
\sphinxstylestrong{Name}
&\sphinxstyletheadfamily 
\sphinxAtStartPar
\sphinxstylestrong{Description}
\\
\sphinxmidrule
\sphinxtableatstartofbodyhook
\sphinxAtStartPar
\sphinxstylestrong{S147}
&
\sphinxAtStartPar
VDD\_RTC
&
\sphinxAtStartPar
Low  current RTC circuit backup power – 3.0V nominal
\\
\sphinxbottomrule
\end{tabulary}
\sphinxtableafterendhook\par
\sphinxattableend\end{savenotes}


\section{4.20 Wi\sphinxhyphen{}Fi \& BT}
\label{\detokenize{hardware:wi-fi-bt}}
\sphinxAtStartPar
Table 4\sphinxhyphen{}33 General Specifications


\begin{savenotes}\sphinxattablestart
\sphinxthistablewithglobalstyle
\centering
\begin{tabulary}{\linewidth}[t]{TT}
\sphinxtoprule
\sphinxstyletheadfamily 
\sphinxAtStartPar
\sphinxstylestrong{Features}
&\sphinxstyletheadfamily 
\sphinxAtStartPar
\sphinxstylestrong{Description}
\\
\sphinxmidrule
\sphinxtableatstartofbodyhook
\sphinxAtStartPar
Product Description
&
\sphinxAtStartPar
IEEE  802.11 2X2 WiFi 5 MIMO Wireless LAN + Bluetooth 5.3 Combo LGA Module
\\
\sphinxhline
\sphinxAtStartPar
Major Chipset
&
\sphinxAtStartPar
NXP  88W8997
\\
\sphinxhline
\sphinxAtStartPar
Host Interface
&
\sphinxAtStartPar
WiFi  + BT  • SDIO3.0 + UART
\\
\sphinxhline
\sphinxAtStartPar
Dimension
&
\sphinxAtStartPar
12  mm X 16 mm x 1.85 mm(Max)
\\
\sphinxhline
\sphinxAtStartPar
Antenna
&
\sphinxAtStartPar
I\sphinxhyphen{}PEX  MHF4 Connector Receptacle (20449)  Main: WiFi \sphinxhyphen{}> TX/RX  Aux: WiFi/Bluetooth \sphinxhyphen{}> TX/RX
\\
\sphinxbottomrule
\end{tabulary}
\sphinxtableafterendhook\par
\sphinxattableend\end{savenotes}

\sphinxAtStartPar
Table 4\sphinxhyphen{}34 WLAN Specifications


\begin{savenotes}\sphinxattablestart
\sphinxthistablewithglobalstyle
\centering
\begin{tabulary}{\linewidth}[t]{TT}
\sphinxtoprule
\sphinxstyletheadfamily 
\sphinxAtStartPar
\sphinxstylestrong{Features}
&\sphinxstyletheadfamily 
\sphinxAtStartPar
\sphinxstylestrong{Description}
\\
\sphinxmidrule
\sphinxtableatstartofbodyhook
\sphinxAtStartPar
WLAN Standard
&
\sphinxAtStartPar
IEEE  802.11 a/b/g/n/ac
\\
\sphinxhline
\sphinxAtStartPar
WLAN VID/PID
&
\sphinxAtStartPar
1B4B/2B42
\\
\sphinxhline
\sphinxAtStartPar
WLAN SVID/SPID
&
\sphinxAtStartPar
N/A
\\
\sphinxhline
\sphinxAtStartPar
Frequency Range
&
\sphinxAtStartPar
2.4  GHz: 2.412 \textasciitilde{}  2.484 GHz  5  GHz: 5.18  \textasciitilde{}5.825GHz
\\
\sphinxhline
\sphinxAtStartPar
Modulation
&
\sphinxAtStartPar
DSSS,  OFDM, DBPSK, DQPSK, CCK, 16\sphinxhyphen{}QAM, 64\sphinxhyphen{}QAM, 256\sphinxhyphen{}QAM
\\
\sphinxhline
\sphinxAtStartPar
Number of Channels
&
\sphinxAtStartPar
2.4GHz  •  USA, NORTH AMERICA, Canada and Taiwan – 1 \textasciitilde{} 11  •  China, Australia, Most European Countries, Japan – 1 \textasciitilde{} 13  5GHz  • USA, EUROPE  –36,40,44,48,52,56,60,64,100,104,108,112,116,120, 124,128,132,136,140,149,153,157,161,165
\\
\sphinxhline
\sphinxAtStartPar
Data Rate
&
\sphinxAtStartPar
• 802.11b: 1, 2, 5.5, 11Mbps  • 802.11a/g: 6, 9, 12, 18, 24, 36, 48,  54Mbps  • 802.11n: up to 150Mbps\sphinxhyphen{}single
\\
\sphinxhline
\sphinxAtStartPar
\sphinxstylestrong{Security}
&
\sphinxAtStartPar
•  WAPI  • WEP 64\sphinxhyphen{}bit and 128\sphinxhyphen{}bit encryption with  H/W TKIP processing  • WPA/WPA2/WPA3 (Wi\sphinxhyphen{}Fi Protected Access)  AES\sphinxhyphen{}CCMP hardware implementation as part of 802.11i security standard
\\
\sphinxbottomrule
\end{tabulary}
\sphinxtableafterendhook\par
\sphinxattableend\end{savenotes}

\sphinxAtStartPar
Table 4\sphinxhyphen{}35 Bluetooth Specification


\begin{savenotes}\sphinxattablestart
\sphinxthistablewithglobalstyle
\centering
\begin{tabulary}{\linewidth}[t]{TT}
\sphinxtoprule
\sphinxstyletheadfamily 
\sphinxAtStartPar
\sphinxstylestrong{Features}
&\sphinxstyletheadfamily 
\sphinxAtStartPar
\sphinxstylestrong{Description}
\\
\sphinxmidrule
\sphinxtableatstartofbodyhook
\sphinxAtStartPar
Bluetooth Standard
&
\sphinxAtStartPar
Bluetooth  2.1 and 3.0+Enhanced Data Rate (EDR) + BT 5.3
\\
\sphinxhline
\sphinxAtStartPar
Bluetooth VID/PID
&
\sphinxAtStartPar
1286/204E
\\
\sphinxhline
\sphinxAtStartPar
Frequency Rage
&
\sphinxAtStartPar
2402MHz\textasciitilde{}2480MHz
\\
\sphinxhline
\sphinxAtStartPar
Modulation
&
\sphinxAtStartPar
Header  GFSK  Payload  2M: π/4\sphinxhyphen{}DQPSK  Payload  3M: 8DPSK
\\
\sphinxhline
\sphinxAtStartPar
Output Power
&
\sphinxAtStartPar
2  dBm
\\
\sphinxhline
\sphinxAtStartPar
Receiver Sensitivity
&
\sphinxAtStartPar
\sphinxhyphen{}83  dBm
\\
\sphinxbottomrule
\end{tabulary}
\sphinxtableafterendhook\par
\sphinxattableend\end{savenotes}

\sphinxAtStartPar
Table 4\sphinxhyphen{}36 Operating Conditions


\begin{savenotes}\sphinxattablestart
\sphinxthistablewithglobalstyle
\centering
\begin{tabulary}{\linewidth}[t]{TT}
\sphinxtoprule
\sphinxstyletheadfamily 
\sphinxAtStartPar
\sphinxstylestrong{Features}
&\sphinxstyletheadfamily 
\sphinxAtStartPar
\sphinxstylestrong{Description}
\\
\sphinxmidrule
\sphinxtableatstartofbodyhook
\sphinxAtStartPar
\sphinxstylestrong{Operating Conditions}
&
\sphinxAtStartPar

\\
\sphinxhline
\sphinxAtStartPar
Voltage
&
\sphinxAtStartPar
3.3V+\sphinxhyphen{}5\%
\\
\sphinxhline
\sphinxAtStartPar
Operating Temperature
&
\sphinxAtStartPar
\sphinxhyphen{}30  ℃\textasciitilde{} 85℃
\\
\sphinxhline
\sphinxAtStartPar
Operating Humidity
&
\sphinxAtStartPar
less  than 85\% R.H.
\\
\sphinxhline
\sphinxAtStartPar
Storage Temperature
&
\sphinxAtStartPar
\sphinxhyphen{}40  ℃\textasciitilde{} 125℃
\\
\sphinxhline
\sphinxAtStartPar
Storage Humidity
&
\sphinxAtStartPar
less  than 60\% R.H.
\\
\sphinxhline
\sphinxAtStartPar
\sphinxstylestrong{ESD Protection}
&
\sphinxAtStartPar

\\
\sphinxhline
\sphinxAtStartPar
Human Body Model
&
\sphinxAtStartPar
+\sphinxhyphen{}2kV
\\
\sphinxhline
\sphinxAtStartPar
Changed Device Model
&
\sphinxAtStartPar
+\sphinxhyphen{}500V
\\
\sphinxbottomrule
\end{tabulary}
\sphinxtableafterendhook\par
\sphinxattableend\end{savenotes}


\section{4.21 Power Supply}
\label{\detokenize{hardware:power-supply}}

\subsection{4.21.1 Power Signals}
\label{\detokenize{hardware:power-signals}}
\sphinxAtStartPar
Table 4\sphinxhyphen{} 37 Power


\begin{savenotes}\sphinxattablestart
\sphinxthistablewithglobalstyle
\centering
\begin{tabulary}{\linewidth}[t]{TTT}
\sphinxtoprule
\sphinxstyletheadfamily 
\sphinxAtStartPar
\sphinxstylestrong{Number}
&\sphinxstyletheadfamily 
\sphinxAtStartPar
\sphinxstylestrong{Name}
&\sphinxstyletheadfamily 
\sphinxAtStartPar
\sphinxstylestrong{Description}
\\
\sphinxmidrule
\sphinxtableatstartofbodyhook
\sphinxAtStartPar
\sphinxstylestrong{P147}  \sphinxstylestrong{P148}  \sphinxstylestrong{P149}  \sphinxstylestrong{P150}  \sphinxstylestrong{P151}  \sphinxstylestrong{P152}  \sphinxstylestrong{P153}  \sphinxstylestrong{P154}  \sphinxstylestrong{P155}  \sphinxstylestrong{P156}
&
\sphinxAtStartPar
VDD\_IN
&
\sphinxAtStartPar

\\
\sphinxhline
\sphinxAtStartPar
\sphinxstylestrong{S147}
&
\sphinxAtStartPar
VDD\_RTC
&
\sphinxAtStartPar

\\
\sphinxhline
\sphinxAtStartPar
\sphinxstylestrong{P2}  \sphinxstylestrong{P9}  \sphinxstylestrong{P12}  \sphinxstylestrong{P15}  \sphinxstylestrong{P18}  \sphinxstylestrong{P32}  \sphinxstylestrong{P38}  \sphinxstylestrong{P47}  \sphinxstylestrong{P50}  \sphinxstylestrong{P53}  \sphinxstylestrong{P59}  \sphinxstylestrong{P68}  \sphinxstylestrong{P79}  \sphinxstylestrong{P82}  \sphinxstylestrong{P85}  \sphinxstylestrong{P88}  \sphinxstylestrong{P91}  \sphinxstylestrong{P94}  \sphinxstylestrong{P97}  \sphinxstylestrong{P100}  \sphinxstylestrong{P103}  \sphinxstylestrong{P120}  \sphinxstylestrong{P133}  \sphinxstylestrong{P142}  \sphinxstylestrong{S3}  \sphinxstylestrong{S10}  \sphinxstylestrong{S13}  \sphinxstylestrong{S16}  \sphinxstylestrong{S25}  \sphinxstylestrong{S34}  \sphinxstylestrong{S47}  \sphinxstylestrong{S61}  \sphinxstylestrong{S64}  \sphinxstylestrong{S67}  \sphinxstylestrong{S70}  \sphinxstylestrong{S73}  \sphinxstylestrong{S80}  \sphinxstylestrong{S83}  \sphinxstylestrong{S86}  \sphinxstylestrong{S89}  \sphinxstylestrong{S92}  \sphinxstylestrong{S101}  \sphinxstylestrong{S110}  \sphinxstylestrong{S119}  \sphinxstylestrong{S124}  \sphinxstylestrong{S130}  \sphinxstylestrong{S136}  \sphinxstylestrong{S143}  \sphinxstylestrong{S158}
&
\sphinxAtStartPar
GND
&
\sphinxAtStartPar

\\
\sphinxbottomrule
\end{tabulary}
\sphinxtableafterendhook\par
\sphinxattableend\end{savenotes}


\section{4.22 General System Control}
\label{\detokenize{hardware:general-system-control}}

\subsection{4.22. 1 General System Control Signals}
\label{\detokenize{hardware:general-system-control-signals}}
\sphinxAtStartPar
Table 4\sphinxhyphen{}38 General System Control Signals


\begin{savenotes}\sphinxattablestart
\sphinxthistablewithglobalstyle
\centering
\begin{tabulary}{\linewidth}[t]{TTT}
\sphinxtoprule
\sphinxstyletheadfamily 
\sphinxAtStartPar
\sphinxstylestrong{Number}
&\sphinxstyletheadfamily 
\sphinxAtStartPar
\sphinxstylestrong{Name}
&\sphinxstyletheadfamily 
\sphinxAtStartPar
\sphinxstylestrong{Description}
\\
\sphinxmidrule
\sphinxtableatstartofbodyhook
\sphinxAtStartPar
\sphinxstylestrong{S154}
&
\sphinxAtStartPar
CARRIER\_PWR\_ON
&
\sphinxAtStartPar
Carrier  Board circuits (apart from power management and power path circuits) should  not be powered up until the Module asserts the CARRIER\_PWR\_ON signal
\\
\sphinxhline
\sphinxAtStartPar
\sphinxstylestrong{P126}
&
\sphinxAtStartPar
RESET\_OUT\#
&
\sphinxAtStartPar
General  purpose reset output to Carrier Board
\\
\sphinxhline
\sphinxAtStartPar
\sphinxstylestrong{P127}
&
\sphinxAtStartPar
RESET\_IN\#
&
\sphinxAtStartPar
Reset  input from Carrier Board
\\
\sphinxhline
\sphinxAtStartPar
\sphinxstylestrong{P128}
&
\sphinxAtStartPar
POWER\_BTN\#
&
\sphinxAtStartPar
Power\sphinxhyphen{}button  input from Carrier Board
\\
\sphinxhline
\sphinxAtStartPar
\sphinxstylestrong{S150}
&
\sphinxAtStartPar
VIN\_PWR\_BAD\#
&
\sphinxAtStartPar
Power  bad indication from Carrier Board
\\
\sphinxbottomrule
\end{tabulary}
\sphinxtableafterendhook\par
\sphinxattableend\end{savenotes}


\subsection{4.22.2 Boot Configuration}
\label{\detokenize{hardware:boot-configuration}}
\sphinxAtStartPar
The SMARC hardware specification defines three SMARC pins, named BOOT\textbackslash{}\_SEL0\# to BOOT\textbackslash{}\_SEL2\#, which are used to indicate from which physical device the module should boot. The SMARC BOOT\textbackslash{}\_SELx\# pins are used to abstract SoC\sphinxhyphen{}specific definitions into a universal SMARC standard. The following table is taken from the SMARC Hardware Specification document.

\sphinxAtStartPar
The FET\sphinxhyphen{}MX8MP\sphinxhyphen{}SMARC module supports the following device boot methods:
\begin{itemize}
\item {} 
\sphinxAtStartPar
On\sphinxhyphen{}board SD Card Boot

\item {} 
\sphinxAtStartPar
Onboard SPI Flash Start

\item {} 
\sphinxAtStartPar
Module integration eMMC startup

\item {} 
\sphinxAtStartPar
QSPI Flash Start

\item {} 
\sphinxAtStartPar
USB Serial Download

\end{itemize}

\sphinxAtStartPar
Table 4\sphinxhyphen{}39 Boot Pin


\begin{savenotes}\sphinxattablestart
\sphinxthistablewithglobalstyle
\centering
\begin{tabulary}{\linewidth}[t]{TTT}
\sphinxtoprule
\sphinxstyletheadfamily 
\sphinxAtStartPar
\sphinxstylestrong{Number}
&\sphinxstyletheadfamily 
\sphinxAtStartPar
\sphinxstylestrong{Name}
&\sphinxstyletheadfamily 
\sphinxAtStartPar
\sphinxstylestrong{Description}
\\
\sphinxmidrule
\sphinxtableatstartofbodyhook
\sphinxAtStartPar
\sphinxstylestrong{P123}
&
\sphinxAtStartPar
BOOT\_SEL0\#
&
\sphinxAtStartPar
Input  straps determine the Module boot device
\\
\sphinxhline
\sphinxAtStartPar
\sphinxstylestrong{P124}
&
\sphinxAtStartPar
BOOT\_SEL1\#
&
\sphinxAtStartPar

\\
\sphinxhline
\sphinxAtStartPar
\sphinxstylestrong{P125}
&
\sphinxAtStartPar
BOOT\_SEL2\#
&
\sphinxAtStartPar

\\
\sphinxhline
\sphinxAtStartPar
\sphinxstylestrong{S155}
&
\sphinxAtStartPar
FORCE\_RECOV\#
&
\sphinxAtStartPar
Low  on this pin allows non\sphinxhyphen{}protected segments of Module boot device to be  rewritten
\\
\sphinxbottomrule
\end{tabulary}
\sphinxtableafterendhook\par
\sphinxattableend\end{savenotes}

\sphinxAtStartPar
Table 4\sphinxhyphen{} 40 Boot Configuration


\begin{savenotes}\sphinxattablestart
\sphinxthistablewithglobalstyle
\centering
\begin{tabulary}{\linewidth}[t]{TTTTT}
\sphinxtoprule
\sphinxstyletheadfamily 
\sphinxAtStartPar
\sphinxstylestrong{BOOT\_SEL{[}2:0{]}}
&\sphinxstyletheadfamily 
\sphinxAtStartPar
\sphinxstylestrong{MODE2}
&\sphinxstyletheadfamily 
\sphinxAtStartPar
\sphinxstylestrong{MODE1}
&\sphinxstyletheadfamily 
\sphinxAtStartPar
\sphinxstylestrong{MODE0}
&\sphinxstyletheadfamily 
\sphinxAtStartPar
\sphinxstylestrong{FORCE\_RECOV\#}
\\
\sphinxmidrule
\sphinxtableatstartofbodyhook
\sphinxAtStartPar
\sphinxstylestrong{Carrier SD Card}
&
\sphinxAtStartPar
L
&
\sphinxAtStartPar
L
&
\sphinxAtStartPar
H
&
\sphinxAtStartPar
H
\\
\sphinxhline
\sphinxAtStartPar
\sphinxstylestrong{Carrier SPI (CS0\#)}
&
\sphinxAtStartPar
L
&
\sphinxAtStartPar
H
&
\sphinxAtStartPar
H
&
\sphinxAtStartPar
H
\\
\sphinxhline
\sphinxAtStartPar
\sphinxstylestrong{Module eMMC Flash}
&
\sphinxAtStartPar
H
&
\sphinxAtStartPar
H
&
\sphinxAtStartPar
L
&
\sphinxAtStartPar
H
\\
\sphinxhline
\sphinxAtStartPar
\sphinxstylestrong{QSPI}
&
\sphinxAtStartPar
H
&
\sphinxAtStartPar
H
&
\sphinxAtStartPar
H
&
\sphinxAtStartPar
H
\\
\sphinxhline
\sphinxAtStartPar
\sphinxstylestrong{USB Serial  Download}
&
\sphinxAtStartPar
X
&
\sphinxAtStartPar
X
&
\sphinxAtStartPar
X
&
\sphinxAtStartPar
L
\\
\sphinxbottomrule
\end{tabulary}
\sphinxtableafterendhook\par
\sphinxattableend\end{savenotes}


\chapter{5. SoC to Connector Pin Fan\sphinxhyphen{}out}
\label{\detokenize{hardware:soc-to-connector-pin-fan-out}}
\sphinxAtStartPar
Based on the pin multiplexing capabilities of the i.MX 8M, some signals available on the SMARC edge connector can be reprogrammed to support different functions.

\sphinxAtStartPar
This table displays a list of connector signals connected to the SoC, including the corresponding SoC pads and their names. For the multiplexing capabilities of the listed pads, please refer to the i.MX 8M documentation.

\sphinxAtStartPar
Table 5\sphinxhyphen{}1 SMARC P\sphinxhyphen{}PIN Connector Pin Output


\begin{savenotes}
\sphinxatlongtablestart
\sphinxthistablewithglobalstyle
\makeatletter
  \LTleft \@totalleftmargin plus1fill
  \LTright\dimexpr\columnwidth-\@totalleftmargin-\linewidth\relax plus1fill
\makeatother
\begin{longtable}{llll}
\sphinxtoprule
\sphinxstyletheadfamily 
\sphinxAtStartPar
\sphinxstylestrong{PIN nr.}
&\sphinxstyletheadfamily 
\sphinxAtStartPar
\sphinxstylestrong{FET\sphinxhyphen{}MX8MP\sphinxhyphen{}SMARC name}
&\sphinxstyletheadfamily 
\sphinxAtStartPar
\sphinxstylestrong{I}. \sphinxstylestrong{MX8M Plus Ball mane}
&\sphinxstyletheadfamily 
\sphinxAtStartPar
\sphinxstylestrong{SoC pad}
\\
\sphinxmidrule
\endfirsthead

\multicolumn{4}{c}{\sphinxnorowcolor
    \makebox[0pt]{\sphinxtablecontinued{\tablename\ \thetable{} \textendash{} continued from previous page}}%
}\\
\sphinxtoprule
\sphinxstyletheadfamily 
\sphinxAtStartPar
\sphinxstylestrong{PIN nr.}
&\sphinxstyletheadfamily 
\sphinxAtStartPar
\sphinxstylestrong{FET\sphinxhyphen{}MX8MP\sphinxhyphen{}SMARC name}
&\sphinxstyletheadfamily 
\sphinxAtStartPar
\sphinxstylestrong{I}. \sphinxstylestrong{MX8M Plus Ball mane}
&\sphinxstyletheadfamily 
\sphinxAtStartPar
\sphinxstylestrong{SoC pad}
\\
\sphinxmidrule
\endhead

\sphinxbottomrule
\multicolumn{4}{r}{\sphinxnorowcolor
    \makebox[0pt][r]{\sphinxtablecontinued{continues on next page}}%
}\\
\endfoot

\endlastfoot
\sphinxtableatstartofbodyhook

\sphinxAtStartPar
P1
&
\sphinxAtStartPar
\sphinxhyphen{}
&
\sphinxAtStartPar
\sphinxhyphen{}
&
\sphinxAtStartPar
\sphinxhyphen{}
\\
\sphinxhline
\sphinxAtStartPar
P2
&
\sphinxAtStartPar
\sphinxhyphen{}
&
\sphinxAtStartPar
\sphinxhyphen{}
&
\sphinxAtStartPar
\sphinxhyphen{}
\\
\sphinxhline
\sphinxAtStartPar
P3
&
\sphinxAtStartPar
CSI1\_CK+
&
\sphinxAtStartPar
MIPI\_CSI2\_CLK\_P
&
\sphinxAtStartPar
A23
\\
\sphinxhline
\sphinxAtStartPar
P4
&
\sphinxAtStartPar
CSI1\_CK\sphinxhyphen{}
&
\sphinxAtStartPar
MIPI\_CSI2\_CLK\_P
&
\sphinxAtStartPar
B23
\\
\sphinxhline
\sphinxAtStartPar
P5
&
\sphinxAtStartPar
GBE1\_SDP
&
\sphinxAtStartPar

&
\sphinxAtStartPar
AH8
\\
\sphinxhline
\sphinxAtStartPar
P6
&
\sphinxAtStartPar
GBE0\_SDP
&
\sphinxAtStartPar

&
\sphinxAtStartPar
B8
\\
\sphinxhline
\sphinxAtStartPar
P7
&
\sphinxAtStartPar
CSI1\_RX0+
&
\sphinxAtStartPar

&
\sphinxAtStartPar
A25
\\
\sphinxhline
\sphinxAtStartPar
P8
&
\sphinxAtStartPar
CSI1\_RX0\sphinxhyphen{}
&
\sphinxAtStartPar

&
\sphinxAtStartPar
B25
\\
\sphinxhline
\sphinxAtStartPar
P9
&
\sphinxAtStartPar
\sphinxhyphen{}
&
\sphinxAtStartPar
\sphinxhyphen{}
&
\sphinxAtStartPar
\sphinxhyphen{}
\\
\sphinxhline
\sphinxAtStartPar
P10
&
\sphinxAtStartPar
CSI1\_RX1+
&
\sphinxAtStartPar

&
\sphinxAtStartPar
A24
\\
\sphinxhline
\sphinxAtStartPar
P11
&
\sphinxAtStartPar
CSI1\_RX1\sphinxhyphen{}
&
\sphinxAtStartPar

&
\sphinxAtStartPar
B24
\\
\sphinxhline
\sphinxAtStartPar
P12
&
\sphinxAtStartPar
\sphinxhyphen{}
&
\sphinxAtStartPar
\sphinxhyphen{}
&
\sphinxAtStartPar
\sphinxhyphen{}
\\
\sphinxhline
\sphinxAtStartPar
P13
&
\sphinxAtStartPar
CSI1\_RX2+
&
\sphinxAtStartPar

&
\sphinxAtStartPar
A22
\\
\sphinxhline
\sphinxAtStartPar
P14
&
\sphinxAtStartPar
CSI1\_RX2\sphinxhyphen{}
&
\sphinxAtStartPar

&
\sphinxAtStartPar
B22
\\
\sphinxhline
\sphinxAtStartPar
P15
&
\sphinxAtStartPar
\sphinxhyphen{}
&
\sphinxAtStartPar
\sphinxhyphen{}
&
\sphinxAtStartPar
\sphinxhyphen{}
\\
\sphinxhline
\sphinxAtStartPar
P16
&
\sphinxAtStartPar
CSI1\_RX3+
&
\sphinxAtStartPar

&
\sphinxAtStartPar
A21
\\
\sphinxhline
\sphinxAtStartPar
P17
&
\sphinxAtStartPar
CSI1\_RX3\sphinxhyphen{}
&
\sphinxAtStartPar

&
\sphinxAtStartPar
B21
\\
\sphinxhline
\sphinxAtStartPar
P18
&
\sphinxAtStartPar
\sphinxhyphen{}
&
\sphinxAtStartPar
\sphinxhyphen{}
&
\sphinxAtStartPar
\sphinxhyphen{}
\\
\sphinxhline
\sphinxAtStartPar
P19
&
\sphinxAtStartPar
\sphinxhyphen{}
&
\sphinxAtStartPar
\sphinxhyphen{}
&
\sphinxAtStartPar
\sphinxhyphen{}
\\
\sphinxhline
\sphinxAtStartPar
P20
&
\sphinxAtStartPar
\sphinxhyphen{}
&
\sphinxAtStartPar
\sphinxhyphen{}
&
\sphinxAtStartPar
\sphinxhyphen{}
\\
\sphinxhline
\sphinxAtStartPar
P21
&
\sphinxAtStartPar
\sphinxhyphen{}
&
\sphinxAtStartPar
\sphinxhyphen{}
&
\sphinxAtStartPar
\sphinxhyphen{}
\\
\sphinxhline
\sphinxAtStartPar
P22
&
\sphinxAtStartPar
\sphinxhyphen{}
&
\sphinxAtStartPar
\sphinxhyphen{}
&
\sphinxAtStartPar
\sphinxhyphen{}
\\
\sphinxhline
\sphinxAtStartPar
P23
&
\sphinxAtStartPar
\sphinxhyphen{}
&
\sphinxAtStartPar
\sphinxhyphen{}
&
\sphinxAtStartPar
\sphinxhyphen{}
\\
\sphinxhline
\sphinxAtStartPar
P24
&
\sphinxAtStartPar
\sphinxhyphen{}
&
\sphinxAtStartPar
\sphinxhyphen{}
&
\sphinxAtStartPar
\sphinxhyphen{}
\\
\sphinxhline
\sphinxAtStartPar
P25
&
\sphinxAtStartPar
\sphinxhyphen{}
&
\sphinxAtStartPar
\sphinxhyphen{}
&
\sphinxAtStartPar
\sphinxhyphen{}
\\
\sphinxhline
\sphinxAtStartPar
P26
&
\sphinxAtStartPar
\sphinxhyphen{}
&
\sphinxAtStartPar
\sphinxhyphen{}
&
\sphinxAtStartPar
\sphinxhyphen{}
\\
\sphinxhline
\sphinxAtStartPar
P27
&
\sphinxAtStartPar
\sphinxhyphen{}
&
\sphinxAtStartPar
\sphinxhyphen{}
&
\sphinxAtStartPar
\sphinxhyphen{}
\\
\sphinxhline
\sphinxAtStartPar
P28
&
\sphinxAtStartPar
\sphinxhyphen{}
&
\sphinxAtStartPar
\sphinxhyphen{}
&
\sphinxAtStartPar
\sphinxhyphen{}
\\
\sphinxhline
\sphinxAtStartPar
P29
&
\sphinxAtStartPar
\sphinxhyphen{}
&
\sphinxAtStartPar
\sphinxhyphen{}
&
\sphinxAtStartPar
\sphinxhyphen{}
\\
\sphinxhline
\sphinxAtStartPar
P30
&
\sphinxAtStartPar
\sphinxhyphen{}
&
\sphinxAtStartPar
\sphinxhyphen{}
&
\sphinxAtStartPar
\sphinxhyphen{}
\\
\sphinxhline
\sphinxAtStartPar
P31
&
\sphinxAtStartPar
SPI0\_CS1\#
&
\sphinxAtStartPar

&
\sphinxAtStartPar
F6
\\
\sphinxhline
\sphinxAtStartPar
P32
&
\sphinxAtStartPar
\sphinxhyphen{}
&
\sphinxAtStartPar
\sphinxhyphen{}
&
\sphinxAtStartPar
\sphinxhyphen{}
\\
\sphinxhline
\sphinxAtStartPar
P33
&
\sphinxAtStartPar
SDIO\_WP
&
\sphinxAtStartPar

&
\sphinxAtStartPar
AC26
\\
\sphinxhline
\sphinxAtStartPar
P34
&
\sphinxAtStartPar
SDIO\_CMD
&
\sphinxAtStartPar

&
\sphinxAtStartPar
AB28
\\
\sphinxhline
\sphinxAtStartPar
P35
&
\sphinxAtStartPar
SDIO\_CD\#
&
\sphinxAtStartPar

&
\sphinxAtStartPar
AD29
\\
\sphinxhline
\sphinxAtStartPar
P36
&
\sphinxAtStartPar
SDIO\_CK
&
\sphinxAtStartPar

&
\sphinxAtStartPar
AB29
\\
\sphinxhline
\sphinxAtStartPar
P37
&
\sphinxAtStartPar
SDIO\_PWR\_EN
&
\sphinxAtStartPar

&
\sphinxAtStartPar
AD28
\\
\sphinxhline
\sphinxAtStartPar
P38
&
\sphinxAtStartPar
\sphinxhyphen{}
&
\sphinxAtStartPar
\sphinxhyphen{}
&
\sphinxAtStartPar
\sphinxhyphen{}
\\
\sphinxhline
\sphinxAtStartPar
P39
&
\sphinxAtStartPar
SDIO\_D0
&
\sphinxAtStartPar

&
\sphinxAtStartPar
AC28
\\
\sphinxhline
\sphinxAtStartPar
P40
&
\sphinxAtStartPar
SDIO\_D1
&
\sphinxAtStartPar

&
\sphinxAtStartPar
AC29
\\
\sphinxhline
\sphinxAtStartPar
P41
&
\sphinxAtStartPar
SDIO\_D2
&
\sphinxAtStartPar

&
\sphinxAtStartPar
AA26
\\
\sphinxhline
\sphinxAtStartPar
P42
&
\sphinxAtStartPar
SDIO\_D3
&
\sphinxAtStartPar

&
\sphinxAtStartPar
AA25
\\
\sphinxhline
\sphinxAtStartPar
P43
&
\sphinxAtStartPar
SPI0\_CS0\#
&
\sphinxAtStartPar

&
\sphinxAtStartPar
AE20
\\
\sphinxhline
\sphinxAtStartPar
P44
&
\sphinxAtStartPar
SPI0\_CK
&
\sphinxAtStartPar

&
\sphinxAtStartPar
AF20
\\
\sphinxhline
\sphinxAtStartPar
P45
&
\sphinxAtStartPar
SPI0\_DIN
&
\sphinxAtStartPar

&
\sphinxAtStartPar
AD20
\\
\sphinxhline
\sphinxAtStartPar
P46
&
\sphinxAtStartPar
SPI0\_DO
&
\sphinxAtStartPar

&
\sphinxAtStartPar
AC20
\\
\sphinxhline
\sphinxAtStartPar
P47
&
\sphinxAtStartPar
\sphinxhyphen{}
&
\sphinxAtStartPar
\sphinxhyphen{}
&
\sphinxAtStartPar
\sphinxhyphen{}
\\
\sphinxhline
\sphinxAtStartPar
P48
&
\sphinxAtStartPar
\sphinxhyphen{}
&
\sphinxAtStartPar
\sphinxhyphen{}
&
\sphinxAtStartPar
\sphinxhyphen{}
\\
\sphinxhline
\sphinxAtStartPar
P49
&
\sphinxAtStartPar
\sphinxhyphen{}
&
\sphinxAtStartPar
\sphinxhyphen{}
&
\sphinxAtStartPar
\sphinxhyphen{}
\\
\sphinxhline
\sphinxAtStartPar
P50
&
\sphinxAtStartPar
\sphinxhyphen{}
&
\sphinxAtStartPar
\sphinxhyphen{}
&
\sphinxAtStartPar
\sphinxhyphen{}
\\
\sphinxhline
\sphinxAtStartPar
P51
&
\sphinxAtStartPar
\sphinxhyphen{}
&
\sphinxAtStartPar
\sphinxhyphen{}
&
\sphinxAtStartPar
\sphinxhyphen{}
\\
\sphinxhline
\sphinxAtStartPar
P52
&
\sphinxAtStartPar
\sphinxhyphen{}
&
\sphinxAtStartPar
\sphinxhyphen{}
&
\sphinxAtStartPar
\sphinxhyphen{}
\\
\sphinxhline
\sphinxAtStartPar
P53
&
\sphinxAtStartPar
\sphinxhyphen{}
&
\sphinxAtStartPar
\sphinxhyphen{}
&
\sphinxAtStartPar
\sphinxhyphen{}
\\
\sphinxhline
\sphinxAtStartPar
P54
&
\sphinxAtStartPar
QSPI\_CS0\#
&
\sphinxAtStartPar

&
\sphinxAtStartPar
L26
\\
\sphinxhline
\sphinxAtStartPar
P55
&
\sphinxAtStartPar
QSPI\_CS1\#
&
\sphinxAtStartPar

&
\sphinxAtStartPar
A5
\\
\sphinxhline
\sphinxAtStartPar
P56
&
\sphinxAtStartPar
QSPI\_CK
&
\sphinxAtStartPar

&
\sphinxAtStartPar
N25
\\
\sphinxhline
\sphinxAtStartPar
P57
&
\sphinxAtStartPar
QSPI\_IO\_1
&
\sphinxAtStartPar

&
\sphinxAtStartPar
L25
\\
\sphinxhline
\sphinxAtStartPar
P58
&
\sphinxAtStartPar
QSPI\_IO\_0
&
\sphinxAtStartPar

&
\sphinxAtStartPar
R25
\\
\sphinxhline
\sphinxAtStartPar
P59
&
\sphinxAtStartPar
\sphinxhyphen{}
&
\sphinxAtStartPar
\sphinxhyphen{}
&
\sphinxAtStartPar
\sphinxhyphen{}
\\
\sphinxhline
\sphinxAtStartPar
P60
&
\sphinxAtStartPar
USB0+
&
\sphinxAtStartPar

&
\sphinxAtStartPar
D10
\\
\sphinxhline
\sphinxAtStartPar
P61
&
\sphinxAtStartPar
USB0\sphinxhyphen{}
&
\sphinxAtStartPar

&
\sphinxAtStartPar
E10
\\
\sphinxhline
\sphinxAtStartPar
P62
&
\sphinxAtStartPar
USB0\_EN\_OC\#
&
\sphinxAtStartPar

&
\sphinxAtStartPar
A6
\\
\sphinxhline
\sphinxAtStartPar
P63
&
\sphinxAtStartPar
USB0\_VBUS\_DET
&
\sphinxAtStartPar

&
\sphinxAtStartPar
A11
\\
\sphinxhline
\sphinxAtStartPar
P64
&
\sphinxAtStartPar
USB0\_OTG\_ID
&
\sphinxAtStartPar

&
\sphinxAtStartPar
B7
\\
\sphinxhline
\sphinxAtStartPar
P65
&
\sphinxAtStartPar
\sphinxhyphen{}
&
\sphinxAtStartPar
\sphinxhyphen{}
&
\sphinxAtStartPar
\sphinxhyphen{}
\\
\sphinxhline
\sphinxAtStartPar
P66
&
\sphinxAtStartPar
\sphinxhyphen{}
&
\sphinxAtStartPar
\sphinxhyphen{}
&
\sphinxAtStartPar
\sphinxhyphen{}
\\
\sphinxhline
\sphinxAtStartPar
P67
&
\sphinxAtStartPar
\sphinxhyphen{}
&
\sphinxAtStartPar
\sphinxhyphen{}
&
\sphinxAtStartPar
\sphinxhyphen{}
\\
\sphinxhline
\sphinxAtStartPar
P68
&
\sphinxAtStartPar
\sphinxhyphen{}
&
\sphinxAtStartPar
\sphinxhyphen{}
&
\sphinxAtStartPar
\sphinxhyphen{}
\\
\sphinxhline
\sphinxAtStartPar
P69
&
\sphinxAtStartPar
\sphinxhyphen{}
&
\sphinxAtStartPar
\sphinxhyphen{}
&
\sphinxAtStartPar
\sphinxhyphen{}
\\
\sphinxhline
\sphinxAtStartPar
P70
&
\sphinxAtStartPar
\sphinxhyphen{}
&
\sphinxAtStartPar
\sphinxhyphen{}
&
\sphinxAtStartPar
\sphinxhyphen{}
\\
\sphinxhline
\sphinxAtStartPar
P71
&
\sphinxAtStartPar
\sphinxhyphen{}
&
\sphinxAtStartPar
\sphinxhyphen{}
&
\sphinxAtStartPar
\sphinxhyphen{}
\\
\sphinxhline
\sphinxAtStartPar
P72
&
\sphinxAtStartPar
\sphinxhyphen{}
&
\sphinxAtStartPar
\sphinxhyphen{}
&
\sphinxAtStartPar
\sphinxhyphen{}
\\
\sphinxhline
\sphinxAtStartPar
P73
&
\sphinxAtStartPar
\sphinxhyphen{}
&
\sphinxAtStartPar
\sphinxhyphen{}
&
\sphinxAtStartPar
\sphinxhyphen{}
\\
\sphinxhline
\sphinxAtStartPar
P74
&
\sphinxAtStartPar
\sphinxhyphen{}
&
\sphinxAtStartPar
\sphinxhyphen{}
&
\sphinxAtStartPar
\sphinxhyphen{}
\\
\sphinxhline
\sphinxAtStartPar
\sphinxstylestrong{Key}
&
\sphinxAtStartPar

&
\sphinxAtStartPar

&
\sphinxAtStartPar

\\
\sphinxhline
\sphinxAtStartPar
\sphinxstylestrong{Key}
&
\sphinxAtStartPar

&
\sphinxAtStartPar

&
\sphinxAtStartPar

\\
\sphinxhline
\sphinxAtStartPar
\sphinxstylestrong{Key}
&
\sphinxAtStartPar

&
\sphinxAtStartPar

&
\sphinxAtStartPar

\\
\sphinxhline
\sphinxAtStartPar
P75
&
\sphinxAtStartPar
PCIE\_A\_RST\#
&
\sphinxAtStartPar

&
\sphinxAtStartPar
A8
\\
\sphinxhline
\sphinxAtStartPar
P76
&
\sphinxAtStartPar
\sphinxhyphen{}
&
\sphinxAtStartPar
\sphinxhyphen{}
&
\sphinxAtStartPar
\sphinxhyphen{}
\\
\sphinxhline
\sphinxAtStartPar
P77
&
\sphinxAtStartPar
\sphinxhyphen{}
&
\sphinxAtStartPar
\sphinxhyphen{}
&
\sphinxAtStartPar
\sphinxhyphen{}
\\
\sphinxhline
\sphinxAtStartPar
P78
&
\sphinxAtStartPar
PCIE\_A\_CKREQ\#
&
\sphinxAtStartPar

&
\sphinxAtStartPar
AJ5
\\
\sphinxhline
\sphinxAtStartPar
P79
&
\sphinxAtStartPar
\sphinxhyphen{}
&
\sphinxAtStartPar
\sphinxhyphen{}
&
\sphinxAtStartPar
\sphinxhyphen{}
\\
\sphinxhline
\sphinxAtStartPar
P80
&
\sphinxAtStartPar
\sphinxhyphen{}
&
\sphinxAtStartPar
\sphinxhyphen{}
&
\sphinxAtStartPar
\sphinxhyphen{}
\\
\sphinxhline
\sphinxAtStartPar
P81
&
\sphinxAtStartPar
\sphinxhyphen{}
&
\sphinxAtStartPar
\sphinxhyphen{}
&
\sphinxAtStartPar
\sphinxhyphen{}
\\
\sphinxhline
\sphinxAtStartPar
P82
&
\sphinxAtStartPar
\sphinxhyphen{}
&
\sphinxAtStartPar
\sphinxhyphen{}
&
\sphinxAtStartPar
\sphinxhyphen{}
\\
\sphinxhline
\sphinxAtStartPar
P83
&
\sphinxAtStartPar
\sphinxhyphen{}
&
\sphinxAtStartPar
\sphinxhyphen{}
&
\sphinxAtStartPar
\sphinxhyphen{}
\\
\sphinxhline
\sphinxAtStartPar
P84
&
\sphinxAtStartPar
\sphinxhyphen{}
&
\sphinxAtStartPar
\sphinxhyphen{}
&
\sphinxAtStartPar
\sphinxhyphen{}
\\
\sphinxhline
\sphinxAtStartPar
P85
&
\sphinxAtStartPar
\sphinxhyphen{}
&
\sphinxAtStartPar
\sphinxhyphen{}
&
\sphinxAtStartPar
\sphinxhyphen{}
\\
\sphinxhline
\sphinxAtStartPar
P86
&
\sphinxAtStartPar
PCIE\_A\_RX+
&
\sphinxAtStartPar

&
\sphinxAtStartPar
A14
\\
\sphinxhline
\sphinxAtStartPar
P87
&
\sphinxAtStartPar
PCIE\_A\_RX\sphinxhyphen{}
&
\sphinxAtStartPar

&
\sphinxAtStartPar
B14
\\
\sphinxhline
\sphinxAtStartPar
P88
&
\sphinxAtStartPar
\sphinxhyphen{}
&
\sphinxAtStartPar
\sphinxhyphen{}
&
\sphinxAtStartPar
\sphinxhyphen{}
\\
\sphinxhline
\sphinxAtStartPar
P89
&
\sphinxAtStartPar
PCIE\_A\_TX+
&
\sphinxAtStartPar

&
\sphinxAtStartPar
A15
\\
\sphinxhline
\sphinxAtStartPar
P90
&
\sphinxAtStartPar
PCIE\_A\_TX\sphinxhyphen{}
&
\sphinxAtStartPar

&
\sphinxAtStartPar
B15
\\
\sphinxhline
\sphinxAtStartPar
P91
&
\sphinxAtStartPar
\sphinxhyphen{}
&
\sphinxAtStartPar
\sphinxhyphen{}
&
\sphinxAtStartPar
\sphinxhyphen{}
\\
\sphinxhline
\sphinxAtStartPar
P92
&
\sphinxAtStartPar
HDMI\_D2+
&
\sphinxAtStartPar

&
\sphinxAtStartPar
AH27
\\
\sphinxhline
\sphinxAtStartPar
P93
&
\sphinxAtStartPar
HDMI\_D2\sphinxhyphen{}
&
\sphinxAtStartPar

&
\sphinxAtStartPar
AJ27
\\
\sphinxhline
\sphinxAtStartPar
P94
&
\sphinxAtStartPar
\sphinxhyphen{}
&
\sphinxAtStartPar
\sphinxhyphen{}
&
\sphinxAtStartPar
\sphinxhyphen{}
\\
\sphinxhline
\sphinxAtStartPar
P95
&
\sphinxAtStartPar
HDMI\_D1+
&
\sphinxAtStartPar

&
\sphinxAtStartPar
AH26
\\
\sphinxhline
\sphinxAtStartPar
P96
&
\sphinxAtStartPar
HDMI\_D1\sphinxhyphen{}
&
\sphinxAtStartPar

&
\sphinxAtStartPar
AJ26
\\
\sphinxhline
\sphinxAtStartPar
P97
&
\sphinxAtStartPar
\sphinxhyphen{}
&
\sphinxAtStartPar
\sphinxhyphen{}
&
\sphinxAtStartPar
\sphinxhyphen{}
\\
\sphinxhline
\sphinxAtStartPar
P98
&
\sphinxAtStartPar
HDMI\_D0+
&
\sphinxAtStartPar

&
\sphinxAtStartPar
AH25
\\
\sphinxhline
\sphinxAtStartPar
P99
&
\sphinxAtStartPar
HDMI\_D0\sphinxhyphen{}
&
\sphinxAtStartPar

&
\sphinxAtStartPar
AJ25
\\
\sphinxhline
\sphinxAtStartPar
P100
&
\sphinxAtStartPar
\sphinxhyphen{}
&
\sphinxAtStartPar
\sphinxhyphen{}
&
\sphinxAtStartPar
\sphinxhyphen{}
\\
\sphinxhline
\sphinxAtStartPar
P101
&
\sphinxAtStartPar
HDMI\_CK+
&
\sphinxAtStartPar

&
\sphinxAtStartPar
AH24
\\
\sphinxhline
\sphinxAtStartPar
P102
&
\sphinxAtStartPar
HDMI\_CK\sphinxhyphen{}
&
\sphinxAtStartPar

&
\sphinxAtStartPar
AJ24
\\
\sphinxhline
\sphinxAtStartPar
P103
&
\sphinxAtStartPar
\sphinxhyphen{}
&
\sphinxAtStartPar
\sphinxhyphen{}
&
\sphinxAtStartPar
\sphinxhyphen{}
\\
\sphinxhline
\sphinxAtStartPar
P104
&
\sphinxAtStartPar
HDMI\_HPD
&
\sphinxAtStartPar

&
\sphinxAtStartPar
AE22
\\
\sphinxhline
\sphinxAtStartPar
P105
&
\sphinxAtStartPar
HDMI\_CTRL\_CK
&
\sphinxAtStartPar

&
\sphinxAtStartPar
AC22
\\
\sphinxhline
\sphinxAtStartPar
P106
&
\sphinxAtStartPar
HDMI\_CTRL\_DAT
&
\sphinxAtStartPar

&
\sphinxAtStartPar
AF22
\\
\sphinxhline
\sphinxAtStartPar
P107
&
\sphinxAtStartPar
\sphinxhyphen{}
&
\sphinxAtStartPar
\sphinxhyphen{}
&
\sphinxAtStartPar
\sphinxhyphen{}
\\
\sphinxhline
\sphinxAtStartPar
P108
&
\sphinxAtStartPar
\sphinxhyphen{}
&
\sphinxAtStartPar
\sphinxhyphen{}
&
\sphinxAtStartPar
\sphinxhyphen{}
\\
\sphinxhline
\sphinxAtStartPar
P109
&
\sphinxAtStartPar
\sphinxhyphen{}
&
\sphinxAtStartPar
\sphinxhyphen{}
&
\sphinxAtStartPar
\sphinxhyphen{}
\\
\sphinxhline
\sphinxAtStartPar
P110
&
\sphinxAtStartPar
\sphinxhyphen{}
&
\sphinxAtStartPar
\sphinxhyphen{}
&
\sphinxAtStartPar
\sphinxhyphen{}
\\
\sphinxhline
\sphinxAtStartPar
P111
&
\sphinxAtStartPar
\sphinxhyphen{}
&
\sphinxAtStartPar
\sphinxhyphen{}
&
\sphinxAtStartPar
\sphinxhyphen{}
\\
\sphinxhline
\sphinxAtStartPar
P112
&
\sphinxAtStartPar
GPIO4  / HDA\_RST\#
&
\sphinxAtStartPar

&
\sphinxAtStartPar
AF14
\\
\sphinxhline
\sphinxAtStartPar
P113
&
\sphinxAtStartPar
GPIO5  / PWM\_OUT
&
\sphinxAtStartPar

&
\sphinxAtStartPar
A4
\\
\sphinxhline
\sphinxAtStartPar
P114
&
\sphinxAtStartPar
GPIO6  / TACHIN
&
\sphinxAtStartPar

&
\sphinxAtStartPar
U26
\\
\sphinxhline
\sphinxAtStartPar
P115
&
\sphinxAtStartPar
\sphinxhyphen{}
&
\sphinxAtStartPar
\sphinxhyphen{}
&
\sphinxAtStartPar
\sphinxhyphen{}
\\
\sphinxhline
\sphinxAtStartPar
P116
&
\sphinxAtStartPar
\sphinxhyphen{}
&
\sphinxAtStartPar
\sphinxhyphen{}
&
\sphinxAtStartPar
\sphinxhyphen{}
\\
\sphinxhline
\sphinxAtStartPar
P117
&
\sphinxAtStartPar
\sphinxhyphen{}
&
\sphinxAtStartPar
\sphinxhyphen{}
&
\sphinxAtStartPar
\sphinxhyphen{}
\\
\sphinxhline
\sphinxAtStartPar
P118
&
\sphinxAtStartPar
\sphinxhyphen{}
&
\sphinxAtStartPar
\sphinxhyphen{}
&
\sphinxAtStartPar
\sphinxhyphen{}
\\
\sphinxhline
\sphinxAtStartPar
P119
&
\sphinxAtStartPar
\sphinxhyphen{}
&
\sphinxAtStartPar
\sphinxhyphen{}
&
\sphinxAtStartPar
\sphinxhyphen{}
\\
\sphinxhline
\sphinxAtStartPar
P120
&
\sphinxAtStartPar
\sphinxhyphen{}
&
\sphinxAtStartPar
\sphinxhyphen{}
&
\sphinxAtStartPar
\sphinxhyphen{}
\\
\sphinxhline
\sphinxAtStartPar
P121
&
\sphinxAtStartPar
I2C\_PM\_CK
&
\sphinxAtStartPar

&
\sphinxAtStartPar
AC14
\\
\sphinxhline
\sphinxAtStartPar
P122
&
\sphinxAtStartPar
I2C\_PM\_DAT
&
\sphinxAtStartPar

&
\sphinxAtStartPar
AD14
\\
\sphinxhline
\sphinxAtStartPar
P123
&
\sphinxAtStartPar
\sphinxhyphen{}
&
\sphinxAtStartPar
\sphinxhyphen{}
&
\sphinxAtStartPar
\sphinxhyphen{}
\\
\sphinxhline
\sphinxAtStartPar
P124
&
\sphinxAtStartPar
\sphinxhyphen{}
&
\sphinxAtStartPar
\sphinxhyphen{}
&
\sphinxAtStartPar
\sphinxhyphen{}
\\
\sphinxhline
\sphinxAtStartPar
P125
&
\sphinxAtStartPar
\sphinxhyphen{}
&
\sphinxAtStartPar
\sphinxhyphen{}
&
\sphinxAtStartPar
\sphinxhyphen{}
\\
\sphinxhline
\sphinxAtStartPar
P126
&
\sphinxAtStartPar
RESET\_OUT\#
&
\sphinxAtStartPar

&
\sphinxAtStartPar
AJ4
\\
\sphinxhline
\sphinxAtStartPar
P127
&
\sphinxAtStartPar
\sphinxhyphen{}
&
\sphinxAtStartPar
\sphinxhyphen{}
&
\sphinxAtStartPar
\sphinxhyphen{}
\\
\sphinxhline
\sphinxAtStartPar
P128
&
\sphinxAtStartPar
\sphinxhyphen{}
&
\sphinxAtStartPar
\sphinxhyphen{}
&
\sphinxAtStartPar
\sphinxhyphen{}
\\
\sphinxhline
\sphinxAtStartPar
P129
&
\sphinxAtStartPar
SER0\_TX
&
\sphinxAtStartPar

&
\sphinxAtStartPar
AA28
\\
\sphinxhline
\sphinxAtStartPar
P130
&
\sphinxAtStartPar
SER0\_RX
&
\sphinxAtStartPar

&
\sphinxAtStartPar
U25
\\
\sphinxhline
\sphinxAtStartPar
P131
&
\sphinxAtStartPar
SER0\_RTS\#
&
\sphinxAtStartPar

&
\sphinxAtStartPar
W26
\\
\sphinxhline
\sphinxAtStartPar
P132
&
\sphinxAtStartPar
SER0\_CTS\#
&
\sphinxAtStartPar

&
\sphinxAtStartPar
W25
\\
\sphinxhline
\sphinxAtStartPar
P133
&
\sphinxAtStartPar
\sphinxhyphen{}
&
\sphinxAtStartPar
\sphinxhyphen{}
&
\sphinxAtStartPar
\sphinxhyphen{}
\\
\sphinxhline
\sphinxAtStartPar
P134
&
\sphinxAtStartPar
SER1\_TX
&
\sphinxAtStartPar

&
\sphinxAtStartPar
AH4
\\
\sphinxhline
\sphinxAtStartPar
P135
&
\sphinxAtStartPar
SER1\_RX
&
\sphinxAtStartPar

&
\sphinxAtStartPar
AF6
\\
\sphinxhline
\sphinxAtStartPar
P136
&
\sphinxAtStartPar
SER2\_TX
&
\sphinxAtStartPar

&
\sphinxAtStartPar
AJ21
\\
\sphinxhline
\sphinxAtStartPar
P137
&
\sphinxAtStartPar
SER2\_RX
&
\sphinxAtStartPar

&
\sphinxAtStartPar
AH21
\\
\sphinxhline
\sphinxAtStartPar
P138
&
\sphinxAtStartPar
SER2\_RTS\#
&
\sphinxAtStartPar

&
\sphinxAtStartPar
AH20
\\
\sphinxhline
\sphinxAtStartPar
P139
&
\sphinxAtStartPar
SER2\_CTS\#
&
\sphinxAtStartPar

&
\sphinxAtStartPar
AJ22
\\
\sphinxhline
\sphinxAtStartPar
P140
&
\sphinxAtStartPar
SER3\_TX
&
\sphinxAtStartPar

&
\sphinxAtStartPar
AJ3
\\
\sphinxhline
\sphinxAtStartPar
P141
&
\sphinxAtStartPar
SER3\_RX
&
\sphinxAtStartPar

&
\sphinxAtStartPar
AD6
\\
\sphinxhline
\sphinxAtStartPar
P142
&
\sphinxAtStartPar
\sphinxhyphen{}
&
\sphinxAtStartPar
\sphinxhyphen{}
&
\sphinxAtStartPar
\sphinxhyphen{}
\\
\sphinxhline
\sphinxAtStartPar
P143
&
\sphinxAtStartPar
CAN0\_TX
&
\sphinxAtStartPar

&
\sphinxAtStartPar
AJ16
\\
\sphinxhline
\sphinxAtStartPar
P144
&
\sphinxAtStartPar
CAN0\_RX
&
\sphinxAtStartPar

&
\sphinxAtStartPar
AH15
\\
\sphinxhline
\sphinxAtStartPar
P145
&
\sphinxAtStartPar
CAN1\_TX
&
\sphinxAtStartPar

&
\sphinxAtStartPar
AH16
\\
\sphinxhline
\sphinxAtStartPar
P146
&
\sphinxAtStartPar
CAN1\_RX
&
\sphinxAtStartPar

&
\sphinxAtStartPar
AJ15
\\
\sphinxhline
\sphinxAtStartPar
P147
&
\sphinxAtStartPar
\sphinxhyphen{}
&
\sphinxAtStartPar
\sphinxhyphen{}
&
\sphinxAtStartPar
\sphinxhyphen{}
\\
\sphinxhline
\sphinxAtStartPar
P148
&
\sphinxAtStartPar
\sphinxhyphen{}
&
\sphinxAtStartPar
\sphinxhyphen{}
&
\sphinxAtStartPar
\sphinxhyphen{}
\\
\sphinxhline
\sphinxAtStartPar
P149
&
\sphinxAtStartPar
\sphinxhyphen{}
&
\sphinxAtStartPar
\sphinxhyphen{}
&
\sphinxAtStartPar
\sphinxhyphen{}
\\
\sphinxhline
\sphinxAtStartPar
P150
&
\sphinxAtStartPar
\sphinxhyphen{}
&
\sphinxAtStartPar
\sphinxhyphen{}
&
\sphinxAtStartPar
\sphinxhyphen{}
\\
\sphinxhline
\sphinxAtStartPar
P151
&
\sphinxAtStartPar
\sphinxhyphen{}
&
\sphinxAtStartPar
\sphinxhyphen{}
&
\sphinxAtStartPar
\sphinxhyphen{}
\\
\sphinxhline
\sphinxAtStartPar
P152
&
\sphinxAtStartPar
\sphinxhyphen{}
&
\sphinxAtStartPar
\sphinxhyphen{}
&
\sphinxAtStartPar
\sphinxhyphen{}
\\
\sphinxhline
\sphinxAtStartPar
P153
&
\sphinxAtStartPar
\sphinxhyphen{}
&
\sphinxAtStartPar
\sphinxhyphen{}
&
\sphinxAtStartPar
\sphinxhyphen{}
\\
\sphinxhline
\sphinxAtStartPar
P154
&
\sphinxAtStartPar
\sphinxhyphen{}
&
\sphinxAtStartPar
\sphinxhyphen{}
&
\sphinxAtStartPar
\sphinxhyphen{}
\\
\sphinxhline
\sphinxAtStartPar
P155
&
\sphinxAtStartPar
\sphinxhyphen{}
&
\sphinxAtStartPar
\sphinxhyphen{}
&
\sphinxAtStartPar
\sphinxhyphen{}
\\
\sphinxhline
\sphinxAtStartPar
P156
&
\sphinxAtStartPar
\sphinxhyphen{}
&
\sphinxAtStartPar
\sphinxhyphen{}
&
\sphinxAtStartPar
\sphinxhyphen{}
\\
\sphinxbottomrule
\end{longtable}
\sphinxtableafterendhook
\sphinxatlongtableend
\end{savenotes}

\sphinxAtStartPar
Table 5\sphinxhyphen{}2 SMARC S\sphinxhyphen{}PIN Connector Pin Fan\sphinxhyphen{}out


\begin{savenotes}
\sphinxatlongtablestart
\sphinxthistablewithglobalstyle
\makeatletter
  \LTleft \@totalleftmargin plus1fill
  \LTright\dimexpr\columnwidth-\@totalleftmargin-\linewidth\relax plus1fill
\makeatother
\begin{longtable}{llll}
\sphinxtoprule
\sphinxstyletheadfamily 
\sphinxAtStartPar
\sphinxstylestrong{PIN nr.}
&\sphinxstyletheadfamily 
\sphinxAtStartPar
\sphinxstylestrong{FET\sphinxhyphen{}MX8MP\sphinxhyphen{}SMARC name}
&\sphinxstyletheadfamily 
\sphinxAtStartPar
\sphinxstylestrong{I}. \sphinxstylestrong{MX8M Plus pin name}
&\sphinxstyletheadfamily 
\sphinxAtStartPar
\sphinxstylestrong{SoC pad}
\\
\sphinxmidrule
\endfirsthead

\multicolumn{4}{c}{\sphinxnorowcolor
    \makebox[0pt]{\sphinxtablecontinued{\tablename\ \thetable{} \textendash{} continued from previous page}}%
}\\
\sphinxtoprule
\sphinxstyletheadfamily 
\sphinxAtStartPar
\sphinxstylestrong{PIN nr.}
&\sphinxstyletheadfamily 
\sphinxAtStartPar
\sphinxstylestrong{FET\sphinxhyphen{}MX8MP\sphinxhyphen{}SMARC name}
&\sphinxstyletheadfamily 
\sphinxAtStartPar
\sphinxstylestrong{I}. \sphinxstylestrong{MX8M Plus pin name}
&\sphinxstyletheadfamily 
\sphinxAtStartPar
\sphinxstylestrong{SoC pad}
\\
\sphinxmidrule
\endhead

\sphinxbottomrule
\multicolumn{4}{r}{\sphinxnorowcolor
    \makebox[0pt][r]{\sphinxtablecontinued{continues on next page}}%
}\\
\endfoot

\endlastfoot
\sphinxtableatstartofbodyhook

\sphinxAtStartPar
S1
&
\sphinxAtStartPar
I2C\_CAM1\_CK
&
\sphinxAtStartPar

&
\sphinxAtStartPar
AH6
\\
\sphinxhline
\sphinxAtStartPar
S2
&
\sphinxAtStartPar
I2C\_CAM1\_DAT
&
\sphinxAtStartPar

&
\sphinxAtStartPar
AE8
\\
\sphinxhline
\sphinxAtStartPar
S3
&
\sphinxAtStartPar
\sphinxhyphen{}
&
\sphinxAtStartPar
\sphinxhyphen{}
&
\sphinxAtStartPar
\sphinxhyphen{}
\\
\sphinxhline
\sphinxAtStartPar
S4
&
\sphinxAtStartPar
\sphinxhyphen{}
&
\sphinxAtStartPar
\sphinxhyphen{}
&
\sphinxAtStartPar
\sphinxhyphen{}
\\
\sphinxhline
\sphinxAtStartPar
S5
&
\sphinxAtStartPar
I2C\_CAM0\_CK
&
\sphinxAtStartPar

&
\sphinxAtStartPar
AJ7
\\
\sphinxhline
\sphinxAtStartPar
S6
&
\sphinxAtStartPar
CAM\_MCK
&
\sphinxAtStartPar

&
\sphinxAtStartPar
B5
\\
\sphinxhline
\sphinxAtStartPar
S7
&
\sphinxAtStartPar
I2C\_CAM0\_DAT
&
\sphinxAtStartPar

&
\sphinxAtStartPar
AJ6
\\
\sphinxhline
\sphinxAtStartPar
S8
&
\sphinxAtStartPar
CSI0\_CK+
&
\sphinxAtStartPar

&
\sphinxAtStartPar
D22
\\
\sphinxhline
\sphinxAtStartPar
S9
&
\sphinxAtStartPar
CSI0\_CK\sphinxhyphen{}
&
\sphinxAtStartPar

&
\sphinxAtStartPar
E22
\\
\sphinxhline
\sphinxAtStartPar
S10
&
\sphinxAtStartPar
\sphinxhyphen{}
&
\sphinxAtStartPar
\sphinxhyphen{}
&
\sphinxAtStartPar
\sphinxhyphen{}
\\
\sphinxhline
\sphinxAtStartPar
S11
&
\sphinxAtStartPar
CSI0\_RX0+
&
\sphinxAtStartPar

&
\sphinxAtStartPar
D18
\\
\sphinxhline
\sphinxAtStartPar
S12
&
\sphinxAtStartPar
CSI0\_RX0\sphinxhyphen{}
&
\sphinxAtStartPar

&
\sphinxAtStartPar
E18
\\
\sphinxhline
\sphinxAtStartPar
S13
&
\sphinxAtStartPar
\sphinxhyphen{}
&
\sphinxAtStartPar
\sphinxhyphen{}
&
\sphinxAtStartPar
\sphinxhyphen{}
\\
\sphinxhline
\sphinxAtStartPar
S14
&
\sphinxAtStartPar
CSI0\_RX1+
&
\sphinxAtStartPar

&
\sphinxAtStartPar
D20
\\
\sphinxhline
\sphinxAtStartPar
S15
&
\sphinxAtStartPar
CSI0\_RX1\sphinxhyphen{}
&
\sphinxAtStartPar

&
\sphinxAtStartPar
E20
\\
\sphinxhline
\sphinxAtStartPar
S16
&
\sphinxAtStartPar
\sphinxhyphen{}
&
\sphinxAtStartPar
\sphinxhyphen{}
&
\sphinxAtStartPar
\sphinxhyphen{}
\\
\sphinxhline
\sphinxAtStartPar
S17
&
\sphinxAtStartPar
\sphinxhyphen{}
&
\sphinxAtStartPar
\sphinxhyphen{}
&
\sphinxAtStartPar
\sphinxhyphen{}
\\
\sphinxhline
\sphinxAtStartPar
S18
&
\sphinxAtStartPar
\sphinxhyphen{}
&
\sphinxAtStartPar
\sphinxhyphen{}
&
\sphinxAtStartPar
\sphinxhyphen{}
\\
\sphinxhline
\sphinxAtStartPar
S19
&
\sphinxAtStartPar
\sphinxhyphen{}
&
\sphinxAtStartPar
\sphinxhyphen{}
&
\sphinxAtStartPar
\sphinxhyphen{}
\\
\sphinxhline
\sphinxAtStartPar
S20
&
\sphinxAtStartPar
\sphinxhyphen{}
&
\sphinxAtStartPar
\sphinxhyphen{}
&
\sphinxAtStartPar
\sphinxhyphen{}
\\
\sphinxhline
\sphinxAtStartPar
S21
&
\sphinxAtStartPar
\sphinxhyphen{}
&
\sphinxAtStartPar
\sphinxhyphen{}
&
\sphinxAtStartPar
\sphinxhyphen{}
\\
\sphinxhline
\sphinxAtStartPar
S22
&
\sphinxAtStartPar
\sphinxhyphen{}
&
\sphinxAtStartPar
\sphinxhyphen{}
&
\sphinxAtStartPar
\sphinxhyphen{}
\\
\sphinxhline
\sphinxAtStartPar
S23
&
\sphinxAtStartPar
\sphinxhyphen{}
&
\sphinxAtStartPar
\sphinxhyphen{}
&
\sphinxAtStartPar
\sphinxhyphen{}
\\
\sphinxhline
\sphinxAtStartPar
S24
&
\sphinxAtStartPar
\sphinxhyphen{}
&
\sphinxAtStartPar
\sphinxhyphen{}
&
\sphinxAtStartPar
\sphinxhyphen{}
\\
\sphinxhline
\sphinxAtStartPar
S25
&
\sphinxAtStartPar
\sphinxhyphen{}
&
\sphinxAtStartPar
\sphinxhyphen{}
&
\sphinxAtStartPar
\sphinxhyphen{}
\\
\sphinxhline
\sphinxAtStartPar
S26
&
\sphinxAtStartPar
\sphinxhyphen{}
&
\sphinxAtStartPar
\sphinxhyphen{}
&
\sphinxAtStartPar
\sphinxhyphen{}
\\
\sphinxhline
\sphinxAtStartPar
S27
&
\sphinxAtStartPar
\sphinxhyphen{}
&
\sphinxAtStartPar
\sphinxhyphen{}
&
\sphinxAtStartPar
\sphinxhyphen{}
\\
\sphinxhline
\sphinxAtStartPar
S28
&
\sphinxAtStartPar
\sphinxhyphen{}
&
\sphinxAtStartPar
\sphinxhyphen{}
&
\sphinxAtStartPar
\sphinxhyphen{}
\\
\sphinxhline
\sphinxAtStartPar
S29
&
\sphinxAtStartPar
\sphinxhyphen{}
&
\sphinxAtStartPar
\sphinxhyphen{}
&
\sphinxAtStartPar
\sphinxhyphen{}
\\
\sphinxhline
\sphinxAtStartPar
S30
&
\sphinxAtStartPar
\sphinxhyphen{}
&
\sphinxAtStartPar
\sphinxhyphen{}
&
\sphinxAtStartPar
\sphinxhyphen{}
\\
\sphinxhline
\sphinxAtStartPar
S31
&
\sphinxAtStartPar
\sphinxhyphen{}
&
\sphinxAtStartPar
\sphinxhyphen{}
&
\sphinxAtStartPar
\sphinxhyphen{}
\\
\sphinxhline
\sphinxAtStartPar
S32
&
\sphinxAtStartPar
\sphinxhyphen{}
&
\sphinxAtStartPar
\sphinxhyphen{}
&
\sphinxAtStartPar
\sphinxhyphen{}
\\
\sphinxhline
\sphinxAtStartPar
S33
&
\sphinxAtStartPar
\sphinxhyphen{}
&
\sphinxAtStartPar
\sphinxhyphen{}
&
\sphinxAtStartPar
\sphinxhyphen{}
\\
\sphinxhline
\sphinxAtStartPar
S34
&
\sphinxAtStartPar
\sphinxhyphen{}
&
\sphinxAtStartPar
\sphinxhyphen{}
&
\sphinxAtStartPar
\sphinxhyphen{}
\\
\sphinxhline
\sphinxAtStartPar
S35
&
\sphinxAtStartPar
\sphinxhyphen{}
&
\sphinxAtStartPar
\sphinxhyphen{}
&
\sphinxAtStartPar
\sphinxhyphen{}
\\
\sphinxhline
\sphinxAtStartPar
S36
&
\sphinxAtStartPar
\sphinxhyphen{}
&
\sphinxAtStartPar
\sphinxhyphen{}
&
\sphinxAtStartPar
\sphinxhyphen{}
\\
\sphinxhline
\sphinxAtStartPar
S37
&
\sphinxAtStartPar
\sphinxhyphen{}
&
\sphinxAtStartPar
\sphinxhyphen{}
&
\sphinxAtStartPar
\sphinxhyphen{}
\\
\sphinxhline
\sphinxAtStartPar
S38
&
\sphinxAtStartPar
AUDIO\_MCK
&
\sphinxAtStartPar

&
\sphinxAtStartPar
AJ20
\\
\sphinxhline
\sphinxAtStartPar
S39
&
\sphinxAtStartPar
I2S0\_LRCK
&
\sphinxAtStartPar

&
\sphinxAtStartPar
AC16
\\
\sphinxhline
\sphinxAtStartPar
S40
&
\sphinxAtStartPar
I2S0\_SDOUT
&
\sphinxAtStartPar

&
\sphinxAtStartPar
AH18
\\
\sphinxhline
\sphinxAtStartPar
S41
&
\sphinxAtStartPar
I2S0\_SDIN
&
\sphinxAtStartPar

&
\sphinxAtStartPar
AF18
\\
\sphinxhline
\sphinxAtStartPar
S42
&
\sphinxAtStartPar
I2S0\_CK
&
\sphinxAtStartPar

&
\sphinxAtStartPar
AH19
\\
\sphinxhline
\sphinxAtStartPar
S43
&
\sphinxAtStartPar
\sphinxhyphen{}
&
\sphinxAtStartPar
\sphinxhyphen{}
&
\sphinxAtStartPar
\sphinxhyphen{}
\\
\sphinxhline
\sphinxAtStartPar
S44
&
\sphinxAtStartPar
\sphinxhyphen{}
&
\sphinxAtStartPar
\sphinxhyphen{}
&
\sphinxAtStartPar
\sphinxhyphen{}
\\
\sphinxhline
\sphinxAtStartPar
S45
&
\sphinxAtStartPar
\sphinxhyphen{}
&
\sphinxAtStartPar
\sphinxhyphen{}
&
\sphinxAtStartPar
\sphinxhyphen{}
\\
\sphinxhline
\sphinxAtStartPar
S46
&
\sphinxAtStartPar
\sphinxhyphen{}
&
\sphinxAtStartPar
\sphinxhyphen{}
&
\sphinxAtStartPar
\sphinxhyphen{}
\\
\sphinxhline
\sphinxAtStartPar
S47
&
\sphinxAtStartPar
\sphinxhyphen{}
&
\sphinxAtStartPar
\sphinxhyphen{}
&
\sphinxAtStartPar
\sphinxhyphen{}
\\
\sphinxhline
\sphinxAtStartPar
S48
&
\sphinxAtStartPar
I2C\_GP\_CK
&
\sphinxAtStartPar

&
\sphinxAtStartPar
AE18
\\
\sphinxhline
\sphinxAtStartPar
S49
&
\sphinxAtStartPar
I2C\_GP\_DAT
&
\sphinxAtStartPar

&
\sphinxAtStartPar
AD18
\\
\sphinxhline
\sphinxAtStartPar
S50
&
\sphinxAtStartPar
I2S2\_LRCK
&
\sphinxAtStartPar

&
\sphinxAtStartPar
AD16
\\
\sphinxhline
\sphinxAtStartPar
S51
&
\sphinxAtStartPar
I2S2\_SDOUT
&
\sphinxAtStartPar

&
\sphinxAtStartPar
AE14
\\
\sphinxhline
\sphinxAtStartPar
S52
&
\sphinxAtStartPar
I2S2\_SDIN
&
\sphinxAtStartPar

&
\sphinxAtStartPar
AE16
\\
\sphinxhline
\sphinxAtStartPar
S53
&
\sphinxAtStartPar
I2S2\_CK
&
\sphinxAtStartPar

&
\sphinxAtStartPar
AF16
\\
\sphinxhline
\sphinxAtStartPar
S54
&
\sphinxAtStartPar
\sphinxhyphen{}
&
\sphinxAtStartPar
\sphinxhyphen{}
&
\sphinxAtStartPar
\sphinxhyphen{}
\\
\sphinxhline
\sphinxAtStartPar
S55
&
\sphinxAtStartPar
\sphinxhyphen{}
&
\sphinxAtStartPar
\sphinxhyphen{}
&
\sphinxAtStartPar
\sphinxhyphen{}
\\
\sphinxhline
\sphinxAtStartPar
S56
&
\sphinxAtStartPar
QSPI\_IO\_2
&
\sphinxAtStartPar

&
\sphinxAtStartPar
L24
\\
\sphinxhline
\sphinxAtStartPar
S57
&
\sphinxAtStartPar
QSPI\_IO\_3
&
\sphinxAtStartPar

&
\sphinxAtStartPar
N24
\\
\sphinxhline
\sphinxAtStartPar
S58
&
\sphinxAtStartPar
\sphinxhyphen{}
&
\sphinxAtStartPar
\sphinxhyphen{}
&
\sphinxAtStartPar
\sphinxhyphen{}
\\
\sphinxhline
\sphinxAtStartPar
S59
&
\sphinxAtStartPar
\sphinxhyphen{}
&
\sphinxAtStartPar
\sphinxhyphen{}
&
\sphinxAtStartPar
\sphinxhyphen{}
\\
\sphinxhline
\sphinxAtStartPar
S60
&
\sphinxAtStartPar
\sphinxhyphen{}
&
\sphinxAtStartPar
\sphinxhyphen{}
&
\sphinxAtStartPar
\sphinxhyphen{}
\\
\sphinxhline
\sphinxAtStartPar
S61
&
\sphinxAtStartPar
\sphinxhyphen{}
&
\sphinxAtStartPar
\sphinxhyphen{}
&
\sphinxAtStartPar
\sphinxhyphen{}
\\
\sphinxhline
\sphinxAtStartPar
S62
&
\sphinxAtStartPar
USB3\_SSTX+
&
\sphinxAtStartPar

&
\sphinxAtStartPar
A13
\\
\sphinxhline
\sphinxAtStartPar
S63
&
\sphinxAtStartPar
USB3\_SSTX\sphinxhyphen{}
&
\sphinxAtStartPar

&
\sphinxAtStartPar
B13
\\
\sphinxhline
\sphinxAtStartPar
S64
&
\sphinxAtStartPar
\sphinxhyphen{}
&
\sphinxAtStartPar

&
\sphinxAtStartPar

\\
\sphinxhline
\sphinxAtStartPar
S65
&
\sphinxAtStartPar
USB3\_SSRX+
&
\sphinxAtStartPar

&
\sphinxAtStartPar
A12
\\
\sphinxhline
\sphinxAtStartPar
S66
&
\sphinxAtStartPar
USB3\_SSRX\sphinxhyphen{}
&
\sphinxAtStartPar

&
\sphinxAtStartPar
B12
\\
\sphinxhline
\sphinxAtStartPar
S67
&
\sphinxAtStartPar
\sphinxhyphen{}
&
\sphinxAtStartPar

&
\sphinxAtStartPar
–
\\
\sphinxhline
\sphinxAtStartPar
S68
&
\sphinxAtStartPar
\sphinxhyphen{}
&
\sphinxAtStartPar
\sphinxhyphen{}
&
\sphinxAtStartPar
\sphinxhyphen{}
\\
\sphinxhline
\sphinxAtStartPar
S69
&
\sphinxAtStartPar
\sphinxhyphen{}
&
\sphinxAtStartPar
\sphinxhyphen{}
&
\sphinxAtStartPar
\sphinxhyphen{}
\\
\sphinxhline
\sphinxAtStartPar
S70
&
\sphinxAtStartPar
\sphinxhyphen{}
&
\sphinxAtStartPar

&
\sphinxAtStartPar

\\
\sphinxhline
\sphinxAtStartPar
S71
&
\sphinxAtStartPar
USB2\_SSTX+
&
\sphinxAtStartPar

&
\sphinxAtStartPar
A10
\\
\sphinxhline
\sphinxAtStartPar
S72
&
\sphinxAtStartPar
USB2\_SSTX\sphinxhyphen{}
&
\sphinxAtStartPar

&
\sphinxAtStartPar
B10
\\
\sphinxhline
\sphinxAtStartPar
S73
&
\sphinxAtStartPar
\sphinxhyphen{}
&
\sphinxAtStartPar
\sphinxhyphen{}
&
\sphinxAtStartPar
\sphinxhyphen{}
\\
\sphinxhline
\sphinxAtStartPar
S74
&
\sphinxAtStartPar
USB2\_SSRX+
&
\sphinxAtStartPar

&
\sphinxAtStartPar
A9
\\
\sphinxhline
\sphinxAtStartPar
S75
&
\sphinxAtStartPar
USB2\_SSRX\sphinxhyphen{}
&
\sphinxAtStartPar

&
\sphinxAtStartPar
B9
\\
\sphinxhline
\sphinxAtStartPar
\sphinxstylestrong{Key}
&
\sphinxAtStartPar

&
\sphinxAtStartPar

&
\sphinxAtStartPar

\\
\sphinxhline
\sphinxAtStartPar
\sphinxstylestrong{Key}
&
\sphinxAtStartPar

&
\sphinxAtStartPar

&
\sphinxAtStartPar

\\
\sphinxhline
\sphinxAtStartPar
\sphinxstylestrong{Key}
&
\sphinxAtStartPar

&
\sphinxAtStartPar

&
\sphinxAtStartPar

\\
\sphinxhline
\sphinxAtStartPar
S76
&
\sphinxAtStartPar
\sphinxhyphen{}
&
\sphinxAtStartPar
\sphinxhyphen{}
&
\sphinxAtStartPar
\sphinxhyphen{}
\\
\sphinxhline
\sphinxAtStartPar
S77
&
\sphinxAtStartPar
\sphinxhyphen{}
&
\sphinxAtStartPar
\sphinxhyphen{}
&
\sphinxAtStartPar
\sphinxhyphen{}
\\
\sphinxhline
\sphinxAtStartPar
S78
&
\sphinxAtStartPar
\sphinxhyphen{}
&
\sphinxAtStartPar
\sphinxhyphen{}
&
\sphinxAtStartPar
\sphinxhyphen{}
\\
\sphinxhline
\sphinxAtStartPar
S79
&
\sphinxAtStartPar
\sphinxhyphen{}
&
\sphinxAtStartPar
\sphinxhyphen{}
&
\sphinxAtStartPar
\sphinxhyphen{}
\\
\sphinxhline
\sphinxAtStartPar
S80
&
\sphinxAtStartPar
\sphinxhyphen{}
&
\sphinxAtStartPar
\sphinxhyphen{}
&
\sphinxAtStartPar
\sphinxhyphen{}
\\
\sphinxhline
\sphinxAtStartPar
S81
&
\sphinxAtStartPar
\sphinxhyphen{}
&
\sphinxAtStartPar
\sphinxhyphen{}
&
\sphinxAtStartPar
\sphinxhyphen{}
\\
\sphinxhline
\sphinxAtStartPar
S82
&
\sphinxAtStartPar
\sphinxhyphen{}
&
\sphinxAtStartPar
\sphinxhyphen{}
&
\sphinxAtStartPar
\sphinxhyphen{}
\\
\sphinxhline
\sphinxAtStartPar
S83
&
\sphinxAtStartPar
\sphinxhyphen{}
&
\sphinxAtStartPar
\sphinxhyphen{}
&
\sphinxAtStartPar
\sphinxhyphen{}
\\
\sphinxhline
\sphinxAtStartPar
S84
&
\sphinxAtStartPar
\sphinxhyphen{}
&
\sphinxAtStartPar
\sphinxhyphen{}
&
\sphinxAtStartPar
\sphinxhyphen{}
\\
\sphinxhline
\sphinxAtStartPar
S85
&
\sphinxAtStartPar
\sphinxhyphen{}
&
\sphinxAtStartPar
\sphinxhyphen{}
&
\sphinxAtStartPar
\sphinxhyphen{}
\\
\sphinxhline
\sphinxAtStartPar
S86
&
\sphinxAtStartPar
\sphinxhyphen{}
&
\sphinxAtStartPar
\sphinxhyphen{}
&
\sphinxAtStartPar
\sphinxhyphen{}
\\
\sphinxhline
\sphinxAtStartPar
S87
&
\sphinxAtStartPar
\sphinxhyphen{}
&
\sphinxAtStartPar
\sphinxhyphen{}
&
\sphinxAtStartPar
\sphinxhyphen{}
\\
\sphinxhline
\sphinxAtStartPar
S88
&
\sphinxAtStartPar
\sphinxhyphen{}
&
\sphinxAtStartPar
\sphinxhyphen{}
&
\sphinxAtStartPar
\sphinxhyphen{}
\\
\sphinxhline
\sphinxAtStartPar
S89
&
\sphinxAtStartPar
\sphinxhyphen{}
&
\sphinxAtStartPar
\sphinxhyphen{}
&
\sphinxAtStartPar
\sphinxhyphen{}
\\
\sphinxhline
\sphinxAtStartPar
S90
&
\sphinxAtStartPar
\sphinxhyphen{}
&
\sphinxAtStartPar
\sphinxhyphen{}
&
\sphinxAtStartPar
\sphinxhyphen{}
\\
\sphinxhline
\sphinxAtStartPar
S91
&
\sphinxAtStartPar
\sphinxhyphen{}
&
\sphinxAtStartPar
\sphinxhyphen{}
&
\sphinxAtStartPar
\sphinxhyphen{}
\\
\sphinxhline
\sphinxAtStartPar
S92
&
\sphinxAtStartPar
\sphinxhyphen{}
&
\sphinxAtStartPar
\sphinxhyphen{}
&
\sphinxAtStartPar
\sphinxhyphen{}
\\
\sphinxhline
\sphinxAtStartPar
S93
&
\sphinxAtStartPar
\sphinxhyphen{}
&
\sphinxAtStartPar
\sphinxhyphen{}
&
\sphinxAtStartPar
\sphinxhyphen{}
\\
\sphinxhline
\sphinxAtStartPar
S94
&
\sphinxAtStartPar
\sphinxhyphen{}
&
\sphinxAtStartPar
\sphinxhyphen{}
&
\sphinxAtStartPar
\sphinxhyphen{}
\\
\sphinxhline
\sphinxAtStartPar
S95
&
\sphinxAtStartPar
\sphinxhyphen{}
&
\sphinxAtStartPar
\sphinxhyphen{}
&
\sphinxAtStartPar
\sphinxhyphen{}
\\
\sphinxhline
\sphinxAtStartPar
S96
&
\sphinxAtStartPar
\sphinxhyphen{}
&
\sphinxAtStartPar
\sphinxhyphen{}
&
\sphinxAtStartPar
\sphinxhyphen{}
\\
\sphinxhline
\sphinxAtStartPar
S97
&
\sphinxAtStartPar
\sphinxhyphen{}
&
\sphinxAtStartPar
\sphinxhyphen{}
&
\sphinxAtStartPar
\sphinxhyphen{}
\\
\sphinxhline
\sphinxAtStartPar
S98
&
\sphinxAtStartPar
\sphinxhyphen{}
&
\sphinxAtStartPar
\sphinxhyphen{}
&
\sphinxAtStartPar
\sphinxhyphen{}
\\
\sphinxhline
\sphinxAtStartPar
S99
&
\sphinxAtStartPar
\sphinxhyphen{}
&
\sphinxAtStartPar
\sphinxhyphen{}
&
\sphinxAtStartPar
\sphinxhyphen{}
\\
\sphinxhline
\sphinxAtStartPar
S100
&
\sphinxAtStartPar
\sphinxhyphen{}
&
\sphinxAtStartPar
\sphinxhyphen{}
&
\sphinxAtStartPar
\sphinxhyphen{}
\\
\sphinxhline
\sphinxAtStartPar
S101
&
\sphinxAtStartPar
\sphinxhyphen{}
&
\sphinxAtStartPar
\sphinxhyphen{}
&
\sphinxAtStartPar
\sphinxhyphen{}
\\
\sphinxhline
\sphinxAtStartPar
S102
&
\sphinxAtStartPar
\sphinxhyphen{}
&
\sphinxAtStartPar
\sphinxhyphen{}
&
\sphinxAtStartPar
\sphinxhyphen{}
\\
\sphinxhline
\sphinxAtStartPar
S103
&
\sphinxAtStartPar
\sphinxhyphen{}
&
\sphinxAtStartPar
\sphinxhyphen{}
&
\sphinxAtStartPar
\sphinxhyphen{}
\\
\sphinxhline
\sphinxAtStartPar
S104
&
\sphinxAtStartPar
\sphinxhyphen{}
&
\sphinxAtStartPar
\sphinxhyphen{}
&
\sphinxAtStartPar
\sphinxhyphen{}
\\
\sphinxhline
\sphinxAtStartPar
S105
&
\sphinxAtStartPar
\sphinxhyphen{}
&
\sphinxAtStartPar
\sphinxhyphen{}
&
\sphinxAtStartPar
\sphinxhyphen{}
\\
\sphinxhline
\sphinxAtStartPar
S106
&
\sphinxAtStartPar
\sphinxhyphen{}
&
\sphinxAtStartPar
\sphinxhyphen{}
&
\sphinxAtStartPar
\sphinxhyphen{}
\\
\sphinxhline
\sphinxAtStartPar
S107
&
\sphinxAtStartPar
LCD1\_BKLT\_EN
&
\sphinxAtStartPar

&
\sphinxAtStartPar
A7
\\
\sphinxhline
\sphinxAtStartPar
S108
&
\sphinxAtStartPar
LVDS1\_CK+
&
\sphinxAtStartPar

&
\sphinxAtStartPar
A28
\\
\sphinxhline
\sphinxAtStartPar
S109
&
\sphinxAtStartPar
LVDS1\_CK\sphinxhyphen{}
&
\sphinxAtStartPar

&
\sphinxAtStartPar
B28
\\
\sphinxhline
\sphinxAtStartPar
S110
&
\sphinxAtStartPar
\sphinxhyphen{}
&
\sphinxAtStartPar
\sphinxhyphen{}
&
\sphinxAtStartPar
\sphinxhyphen{}
\\
\sphinxhline
\sphinxAtStartPar
S111
&
\sphinxAtStartPar
LVDS1\_0+
&
\sphinxAtStartPar

&
\sphinxAtStartPar
A26
\\
\sphinxhline
\sphinxAtStartPar
S112
&
\sphinxAtStartPar
LVDS1\_0\sphinxhyphen{}
&
\sphinxAtStartPar

&
\sphinxAtStartPar
B26
\\
\sphinxhline
\sphinxAtStartPar
S113
&
\sphinxAtStartPar
\sphinxhyphen{}
&
\sphinxAtStartPar
\sphinxhyphen{}
&
\sphinxAtStartPar
\sphinxhyphen{}
\\
\sphinxhline
\sphinxAtStartPar
S114
&
\sphinxAtStartPar
LVDS1\_1+
&
\sphinxAtStartPar

&
\sphinxAtStartPar
A27
\\
\sphinxhline
\sphinxAtStartPar
S115
&
\sphinxAtStartPar
LVDS1\_1\sphinxhyphen{}
&
\sphinxAtStartPar

&
\sphinxAtStartPar
B27
\\
\sphinxhline
\sphinxAtStartPar
S116
&
\sphinxAtStartPar
LCD1\_VDD\_EN
&
\sphinxAtStartPar

&
\sphinxAtStartPar
AH5
\\
\sphinxhline
\sphinxAtStartPar
S117
&
\sphinxAtStartPar
LVDS1\_2+
&
\sphinxAtStartPar

&
\sphinxAtStartPar
B29
\\
\sphinxhline
\sphinxAtStartPar
S118
&
\sphinxAtStartPar
LVDS1\_2\sphinxhyphen{}
&
\sphinxAtStartPar

&
\sphinxAtStartPar
C28
\\
\sphinxhline
\sphinxAtStartPar
S119
&
\sphinxAtStartPar
\sphinxhyphen{}
&
\sphinxAtStartPar
\sphinxhyphen{}
&
\sphinxAtStartPar
\sphinxhyphen{}
\\
\sphinxhline
\sphinxAtStartPar
S120
&
\sphinxAtStartPar
LVDS1\_3+
&
\sphinxAtStartPar

&
\sphinxAtStartPar
C29
\\
\sphinxhline
\sphinxAtStartPar
S121
&
\sphinxAtStartPar
LVDS1\_3\sphinxhyphen{}
&
\sphinxAtStartPar

&
\sphinxAtStartPar
D28
\\
\sphinxhline
\sphinxAtStartPar
S122
&
\sphinxAtStartPar
LCD1\_BKLT\_PWM
&
\sphinxAtStartPar

&
\sphinxAtStartPar
D8
\\
\sphinxhline
\sphinxAtStartPar
S123
&
\sphinxAtStartPar
\sphinxhyphen{}
&
\sphinxAtStartPar
\sphinxhyphen{}
&
\sphinxAtStartPar
\sphinxhyphen{}
\\
\sphinxhline
\sphinxAtStartPar
S124
&
\sphinxAtStartPar
\sphinxhyphen{}
&
\sphinxAtStartPar
\sphinxhyphen{}
&
\sphinxAtStartPar
\sphinxhyphen{}
\\
\sphinxhline
\sphinxAtStartPar
S125
&
\sphinxAtStartPar
LVDS0\_0+ / DSI0\_D0+
&
\sphinxAtStartPar

&
\sphinxAtStartPar
D29 /  A16
\\
\sphinxhline
\sphinxAtStartPar
S126
&
\sphinxAtStartPar
LVDS0\_0\sphinxhyphen{}  / DSI0\_D0\sphinxhyphen{}
&
\sphinxAtStartPar

&
\sphinxAtStartPar
E28 / B16
\\
\sphinxhline
\sphinxAtStartPar
S127
&
\sphinxAtStartPar
LCD0\_BKLT\_EN
&
\sphinxAtStartPar

&
\sphinxAtStartPar
A3
\\
\sphinxhline
\sphinxAtStartPar
S128
&
\sphinxAtStartPar
LVDS0\_1+  / DSI0\_D1+
&
\sphinxAtStartPar

&
\sphinxAtStartPar
E29 / A17
\\
\sphinxhline
\sphinxAtStartPar
S129
&
\sphinxAtStartPar
LVDS0\_1\sphinxhyphen{} / DSI0\_D1\sphinxhyphen{}
&
\sphinxAtStartPar

&
\sphinxAtStartPar
F28 /  B17
\\
\sphinxhline
\sphinxAtStartPar
S130
&
\sphinxAtStartPar
\sphinxhyphen{}
&
\sphinxAtStartPar
\sphinxhyphen{}
&
\sphinxAtStartPar
\sphinxhyphen{}
\\
\sphinxhline
\sphinxAtStartPar
S131
&
\sphinxAtStartPar
LVDS0\_2+ / DSI0\_D2+
&
\sphinxAtStartPar

&
\sphinxAtStartPar
G29 /  A19
\\
\sphinxhline
\sphinxAtStartPar
S132
&
\sphinxAtStartPar
LVDS0\_2\sphinxhyphen{}  / DSI0\_D2\sphinxhyphen{}
&
\sphinxAtStartPar

&
\sphinxAtStartPar
H28 / B19
\\
\sphinxhline
\sphinxAtStartPar
S133
&
\sphinxAtStartPar
LCD0\_VDD\_EN
&
\sphinxAtStartPar

&
\sphinxAtStartPar
B4
\\
\sphinxhline
\sphinxAtStartPar
S134
&
\sphinxAtStartPar
LVDS0\_CK+  / DSI0\_CLK+
&
\sphinxAtStartPar

&
\sphinxAtStartPar
D29 / A18
\\
\sphinxhline
\sphinxAtStartPar
S135
&
\sphinxAtStartPar
LVDS0\_CK\sphinxhyphen{} / DSI0\_CLK
&
\sphinxAtStartPar

&
\sphinxAtStartPar
G28 /  B18
\\
\sphinxhline
\sphinxAtStartPar
S136
&
\sphinxAtStartPar
\sphinxhyphen{}
&
\sphinxAtStartPar
\sphinxhyphen{}
&
\sphinxAtStartPar
\sphinxhyphen{}
\\
\sphinxhline
\sphinxAtStartPar
S137
&
\sphinxAtStartPar
LVDS0\_3+ / DSI0\_D3+
&
\sphinxAtStartPar

&
\sphinxAtStartPar
H29 /  A20
\\
\sphinxhline
\sphinxAtStartPar
S138
&
\sphinxAtStartPar
LVDS0\_3\sphinxhyphen{}  / DSI0\_D3\sphinxhyphen{}
&
\sphinxAtStartPar

&
\sphinxAtStartPar
J28 / B20
\\
\sphinxhline
\sphinxAtStartPar
S139
&
\sphinxAtStartPar
I2C\_LCD\_CK
&
\sphinxAtStartPar

&
\sphinxAtStartPar
AF8
\\
\sphinxhline
\sphinxAtStartPar
S140
&
\sphinxAtStartPar
I2C\_LCD\_DAT
&
\sphinxAtStartPar

&
\sphinxAtStartPar
AD8
\\
\sphinxhline
\sphinxAtStartPar
S141
&
\sphinxAtStartPar
LCD0\_BKLT\_PWM
&
\sphinxAtStartPar

&
\sphinxAtStartPar
E8
\\
\sphinxhline
\sphinxAtStartPar
S142
&
\sphinxAtStartPar
\sphinxhyphen{}
&
\sphinxAtStartPar
\sphinxhyphen{}
&
\sphinxAtStartPar
\sphinxhyphen{}
\\
\sphinxhline
\sphinxAtStartPar
S143
&
\sphinxAtStartPar
\sphinxhyphen{}
&
\sphinxAtStartPar
\sphinxhyphen{}
&
\sphinxAtStartPar
\sphinxhyphen{}
\\
\sphinxhline
\sphinxAtStartPar
S144
&
\sphinxAtStartPar
\sphinxhyphen{}
&
\sphinxAtStartPar
\sphinxhyphen{}
&
\sphinxAtStartPar
\sphinxhyphen{}
\\
\sphinxhline
\sphinxAtStartPar
S145
&
\sphinxAtStartPar
WDT\_TIME\_OUT\#
&
\sphinxAtStartPar

&
\sphinxAtStartPar
B6
\\
\sphinxhline
\sphinxAtStartPar
S146
&
\sphinxAtStartPar
PCIE\_WAKE\#
&
\sphinxAtStartPar

&
\sphinxAtStartPar
AA29
\\
\sphinxhline
\sphinxAtStartPar
S147
&
\sphinxAtStartPar
\sphinxhyphen{}
&
\sphinxAtStartPar
\sphinxhyphen{}
&
\sphinxAtStartPar
\sphinxhyphen{}
\\
\sphinxhline
\sphinxAtStartPar
S148
&
\sphinxAtStartPar
\sphinxhyphen{}
&
\sphinxAtStartPar
\sphinxhyphen{}
&
\sphinxAtStartPar
\sphinxhyphen{}
\\
\sphinxhline
\sphinxAtStartPar
S149
&
\sphinxAtStartPar
\sphinxhyphen{}
&
\sphinxAtStartPar
\sphinxhyphen{}
&
\sphinxAtStartPar
\sphinxhyphen{}
\\
\sphinxhline
\sphinxAtStartPar
S150
&
\sphinxAtStartPar
\sphinxhyphen{}
&
\sphinxAtStartPar
\sphinxhyphen{}
&
\sphinxAtStartPar
\sphinxhyphen{}
\\
\sphinxhline
\sphinxAtStartPar
S151
&
\sphinxAtStartPar
\sphinxhyphen{}
&
\sphinxAtStartPar
\sphinxhyphen{}
&
\sphinxAtStartPar
\sphinxhyphen{}
\\
\sphinxhline
\sphinxAtStartPar
S152
&
\sphinxAtStartPar
\sphinxhyphen{}
&
\sphinxAtStartPar
\sphinxhyphen{}
&
\sphinxAtStartPar
\sphinxhyphen{}
\\
\sphinxhline
\sphinxAtStartPar
S153
&
\sphinxAtStartPar
CARRIER\_STBY\#
&
\sphinxAtStartPar

&
\sphinxAtStartPar
AJ17
\\
\sphinxhline
\sphinxAtStartPar
S154
&
\sphinxAtStartPar
\sphinxhyphen{}
&
\sphinxAtStartPar
\sphinxhyphen{}
&
\sphinxAtStartPar
\sphinxhyphen{}
\\
\sphinxhline
\sphinxAtStartPar
S155
&
\sphinxAtStartPar
\sphinxhyphen{}
&
\sphinxAtStartPar
\sphinxhyphen{}
&
\sphinxAtStartPar
\sphinxhyphen{}
\\
\sphinxhline
\sphinxAtStartPar
S156
&
\sphinxAtStartPar
\sphinxhyphen{}
&
\sphinxAtStartPar
\sphinxhyphen{}
&
\sphinxAtStartPar
\sphinxhyphen{}
\\
\sphinxhline
\sphinxAtStartPar
S157
&
\sphinxAtStartPar
\sphinxhyphen{}
&
\sphinxAtStartPar
\sphinxhyphen{}
&
\sphinxAtStartPar
\sphinxhyphen{}
\\
\sphinxhline
\sphinxAtStartPar
S158
&
\sphinxAtStartPar
\sphinxhyphen{}
&
\sphinxAtStartPar
\sphinxhyphen{}
&
\sphinxAtStartPar
\sphinxhyphen{}
\\
\sphinxbottomrule
\end{longtable}
\sphinxtableafterendhook
\sphinxatlongtableend
\end{savenotes}


\chapter{6. Power Consumption of the Whole Development Board}
\label{\detokenize{hardware:power-consumption-of-the-whole-development-board}}
\sphinxAtStartPar
| Status:| Configuration|
|———\sphinxhyphen{}|———\sphinxhyphen{}|———\sphinxhyphen{}
| | i.MX 8M Plus Quad 2GB LPDDR4 16GB eMMC| i.MX 8M Plus Quad 4GB LPDDR4 32GB eMMC
| Idle (USB HUB Operating)| 3W| 3.3W
| Same as above (VPU at full load when GPU is active)| 4.7W| 5.2W
| The same as above, and with the CPU and LPDDR4 operating at full load.| 5.7W| 6W
| Same as above, with 2 x GbE, Wi\sphinxhyphen{}Fi, and 2 x USB 3.0e.| 7.4W| 7.7W
| RTC power consumption on VDD\_RTC (VDD\_IN is off)| 0.25 μ A|


\chapter{7. Environmental Specification}
\label{\detokenize{hardware:environmental-specification}}
\sphinxAtStartPar
Table 7\sphinxhyphen{} 1 Environment Specification


\begin{savenotes}\sphinxattablestart
\sphinxthistablewithglobalstyle
\centering
\begin{tabulary}{\linewidth}[t]{TTT}
\sphinxtoprule
\sphinxstyletheadfamily 
\sphinxAtStartPar
\sphinxstylestrong{Parameter}
&\sphinxstyletheadfamily 
\sphinxAtStartPar
\sphinxstylestrong{Min.}
&\sphinxstyletheadfamily 
\sphinxAtStartPar
\sphinxstylestrong{Max.}
\\
\sphinxmidrule
\sphinxtableatstartofbodyhook
\sphinxAtStartPar
Industrial  Operating Temperature Range
&
\sphinxAtStartPar
\sphinxhyphen{}40 ℃ *\sphinxstylestrong{1}
&
\sphinxAtStartPar
85℃
\\
\sphinxhline
\sphinxAtStartPar
Storage temperature
&
\sphinxAtStartPar
\sphinxhyphen{}40℃
&
\sphinxAtStartPar
125℃
\\
\sphinxhline
\sphinxAtStartPar
Relative humidity (operation)
&
\sphinxAtStartPar
10\%
&
\sphinxAtStartPar
90\%
\\
\sphinxhline
\sphinxAtStartPar
Relative humidity (storage)
&
\sphinxAtStartPar
5\%
&
\sphinxAtStartPar
95\%
\\
\sphinxbottomrule
\end{tabulary}
\sphinxtableafterendhook\par
\sphinxattableend\end{savenotes}

\sphinxAtStartPar
The operating temperature range of the WIFI and BT modules is \sphinxhyphen{}30 ℃ to 85 ℃.

\sphinxAtStartPar
The FET\sphinxhyphen{}MX8MP\sphinxhyphen{}SMARC module integrates a high\sphinxhyphen{}performance processor, memory, storage, PMIC, USB HUB, ENET PHY, and other functional modules in a small footprint.

\sphinxAtStartPar
It offers very high performance and a wide range of interfaces, while also generating significant heat.

\sphinxAtStartPar
To ensure the CPU operates within the allowable temperature range, it is necessary to dissipate this heat.

\sphinxAtStartPar
Customers are required to evaluate processor workload, device enclosure, airflow, and thermal analysis.

\sphinxAtStartPar
Perform accurate evaluation and develop suitable thermal solutions accordingly.

\sphinxAtStartPar
Forlinx can provide a heatsink for the FET\sphinxhyphen{}MX8MP\sphinxhyphen{}SMARC module, but please keep in mind that its usage must be accurately evaluated within the final system and that it should only be considered as part of a more comprehensive cooling solution.


\chapter{8. OK\sphinxhyphen{}MX8MPQ\sphinxhyphen{}SMARC Development Board Description}
\label{\detokenize{hardware:ok-mx8mpq-smarc-development-board-description}}

\section{8.1 Development Board Interface Diagram}
\label{\detokenize{hardware:development-board-interface-diagram}}
\sphinxAtStartPar
\sphinxincludegraphics{{8d46fa22b58145f0b488f8c226c771e2}.png}

\sphinxAtStartPar
Figure 8\sphinxhyphen{}1  OK\sphinxhyphen{}MX8MPQ\sphinxhyphen{}SMARC Embedded Development Platform Interface Diagram


\section{8.2 Development Board Dimension}
\label{\detokenize{hardware:development-board-dimension}}
\sphinxAtStartPar
\sphinxincludegraphics{{c7588ee02b324c128a13d38e6542d629}.png}

\sphinxAtStartPar
Figure 8\sphinxhyphen{}2 OK\sphinxhyphen{}MX8MPQ\sphinxhyphen{}SMARC Development Board Dimension Diagram

\sphinxAtStartPar
Development board PCB dimensions: 130 mm × 190 mm. For more detailed dimensions, please refer to the user documentation DXF file.\\
Mounting hole dimensions: spacing 120 mm × 180 mm, hole diameter 3.2 mm. Plate making process: thickness of 1.6mm, 4\sphinxhyphen{}layer PCB;

\sphinxAtStartPar
Supply voltage: DC 12 V;


\section{8.3 Development Board Naming Rules}
\label{\detokenize{hardware:development-board-naming-rules}}

\begin{savenotes}\sphinxattablestart
\sphinxthistablewithglobalstyle
\centering
\begin{tabulary}{\linewidth}[t]{TTTT}
\sphinxtoprule
\sphinxstyletheadfamily 
\sphinxAtStartPar
Field
&\sphinxstyletheadfamily 
\sphinxAtStartPar
Field Description
&\sphinxstyletheadfamily 
\sphinxAtStartPar
Value
&\sphinxstyletheadfamily 
\sphinxAtStartPar
Description
\\
\sphinxmidrule
\sphinxtableatstartofbodyhook
\sphinxAtStartPar
A
&
\sphinxAtStartPar
Acceptance Level
&
\sphinxAtStartPar
PC
&
\sphinxAtStartPar
Prototype Sample
\\
\sphinxhline
\sphinxAtStartPar
Empty
&
\sphinxAtStartPar
Mass Production
&
\sphinxAtStartPar

&
\sphinxAtStartPar

\\
\sphinxhline
\sphinxAtStartPar
B
&
\sphinxAtStartPar
Product line identification
&
\sphinxAtStartPar
OK
&
\sphinxAtStartPar
Forlinx Embedded development board
\\
\sphinxhline
\sphinxAtStartPar
C
&
\sphinxAtStartPar
CPU Type
&
\sphinxAtStartPar
MX6UL
&
\sphinxAtStartPar
i.MX6UL
\\
\sphinxhline
\sphinxAtStartPar
\sphinxhyphen{}
&
\sphinxAtStartPar
Segment Identification
&
\sphinxAtStartPar
\sphinxhyphen{}
&
\sphinxAtStartPar

\\
\sphinxhline
\sphinxAtStartPar
D
&
\sphinxAtStartPar
Connection
&
\sphinxAtStartPar
Cx
&
\sphinxAtStartPar
Board\sphinxhyphen{}to\sphinxhyphen{}board Connector
\\
\sphinxhline
\sphinxAtStartPar
+
&
\sphinxAtStartPar
Segment Identification
&
\sphinxAtStartPar
+
&
\sphinxAtStartPar
This identifier is followed by the configuration parameter.
\\
\sphinxhline
\sphinxAtStartPar
I
&
\sphinxAtStartPar
Operating Temperature
&
\sphinxAtStartPar
C
&
\sphinxAtStartPar
0 to 70°C Commercial Grade
\\
\sphinxhline
\sphinxAtStartPar
I
&
\sphinxAtStartPar
\sphinxhyphen{}40 to 85°C Industrial Grade
&
\sphinxAtStartPar

&
\sphinxAtStartPar

\\
\sphinxhline
\sphinxAtStartPar
K
&
\sphinxAtStartPar
PCB Version
&
\sphinxAtStartPar
11
&
\sphinxAtStartPar
V1.1
\\
\sphinxhline
\sphinxAtStartPar
xx
&
\sphinxAtStartPar
Vx.x
&
\sphinxAtStartPar

&
\sphinxAtStartPar

\\
\sphinxhline
\sphinxAtStartPar
:M
&
\sphinxAtStartPar
Manufacturer’s Internal Logo
&
\sphinxAtStartPar
:X
&
\sphinxAtStartPar
It is manufacturer’s internal logo without influence on use.
\\
\sphinxbottomrule
\end{tabulary}
\sphinxtableafterendhook\par
\sphinxattableend\end{savenotes}


\section{8.4 Development Board  Resources}
\label{\detokenize{hardware:development-board-resources}}
\sphinxAtStartPar
The interface functions and quantities used on the OK\sphinxhyphen{}MX8MPQ\sphinxhyphen{}SMARC development board are determined based on a combination of the SMARC specification requirements and the resources provided by the processor.


\begin{savenotes}\sphinxattablestart
\sphinxthistablewithglobalstyle
\centering
\begin{tabulary}{\linewidth}[t]{TTT}
\sphinxtoprule
\sphinxstyletheadfamily 
\sphinxAtStartPar
Function
&\sphinxstyletheadfamily 
\sphinxAtStartPar
Quantity:
&\sphinxstyletheadfamily 
\sphinxAtStartPar
Parameter
\\
\sphinxmidrule
\sphinxtableatstartofbodyhook
\sphinxAtStartPar
USB 3.0
&
\sphinxAtStartPar
1
&
\sphinxAtStartPar
USB Type A connector: Serves only as HOST. Load switch with over \sphinxhyphen{} voltage and over \sphinxhyphen{} current protection.
\\
\sphinxhline
\sphinxAtStartPar
USB 2.0
&
\sphinxAtStartPar
3
&
\sphinxAtStartPar
USB Type A connector: Serves only as HOST. Load switch with over \sphinxhyphen{} voltage and over \sphinxhyphen{} current protection.
\\
\sphinxhline
\sphinxAtStartPar
USB 2.0 OTG
&
\sphinxAtStartPar
1
&
\sphinxAtStartPar
The USB Type C connector is led out. The HOST/SLAVE function is switched through the DIP switch, and the load switch is provided with overvoltage and overcurrent protection; it can be used for USB burning;
\\
\sphinxhline
\sphinxAtStartPar
MIPI CSI
&
\sphinxAtStartPar
2
&
\sphinxAtStartPar
CSI1: Supports daA3840\sphinxhyphen{}30mc\sphinxhyphen{}IMX8MP\sphinxhyphen{}EVK camera module set, resolution 3840X2160; CSI0: uses double data channels, led out through a 26Pin FPC row; support OV5645 module set;
\\
\sphinxhline
\sphinxAtStartPar
MIPI DSI
&
\sphinxAtStartPar
1
&
\sphinxAtStartPar
According to the SMARC specification, the development board uses a switch chip to toggle between DSI0 and LVDS0 functions; the 4\sphinxhyphen{}lane MIPI DSI interface is routed out via an FPC connector; it is compatible with the Forlinx 7\sphinxhyphen{}inch MIPI display, with a resolution of 1024 x 600 @ 30fps.
\\
\sphinxhline
\sphinxAtStartPar
LVDS
&
\sphinxAtStartPar
2
&
\sphinxAtStartPar
According to the SMARC specification, the development board uses a switch chip to toggle between DSI0 and LVDS0 functions. It supports 2 x sets of 4\sphinxhyphen{}lane LVDS 1080P displays, with LVDS0 and DSI0 sharing the same data channel. It is compatible with Forlinx’s 10.1\sphinxhyphen{}inch LVDS display.
\\
\sphinxhline
\sphinxAtStartPar
HDMI
&
\sphinxAtStartPar
1
&
\sphinxAtStartPar
Support HDMI 2.0a with resolution up to 3840 x 2160@30fps;
\\
\sphinxhline
\sphinxAtStartPar
Ethernet
&
\sphinxAtStartPar
2
&
\sphinxAtStartPar
Support 10/100/1000Mbps self\sphinxhyphen{}adaption, which is led out through RJ45 interface, and 1 x supports TSN;
\\
\sphinxhline
\sphinxAtStartPar
PCIE x1
&
\sphinxAtStartPar
1
&
\sphinxAtStartPar
The development board uses standard PCIE x1 card interface and supports PCI Express Gen3;
\\
\sphinxhline
\sphinxAtStartPar
TF Card
&
\sphinxAtStartPar
1
&
\sphinxAtStartPar
Dev board supports 1 x SDIO for UHS \sphinxhyphen{} I TF cards, up to 104MB/s.
\\
\sphinxhline
\sphinxAtStartPar
4G/5G
&
\sphinxAtStartPar
1
&
\sphinxAtStartPar
The M.2 B\sphinxhyphen{}KEY slot is reserved for the development board, and one of 4G and 5G functions can be used; 4G supports EC20 by default; 5G supports RM500Q by default; the SIM card is inserted into the onboard MicroSIM card slot;
\\
\sphinxhline
\sphinxAtStartPar
I2S
&
\sphinxAtStartPar
2
&
\sphinxAtStartPar
The development board utilizes one set of I2S interfaces connected to the CODEC chip for the following audio functions, while providing another set of I2S interfaces through pin headers for expansion.
\\
\sphinxhline
\sphinxAtStartPar
Audio
&
\sphinxAtStartPar
1
&
\sphinxAtStartPar
Default on\sphinxhyphen{}board NAU88C22YG chip, I2S interface; support headphone output and MIC input, integrated in a 3.5mm headphone interface; support 2 x 1 W 8Ω speaker output, led out through XH2.54 white terminal;
\\
\sphinxhline
\sphinxAtStartPar
CANFD
&
\sphinxAtStartPar
2
&
\sphinxAtStartPar
Industrial\sphinxhyphen{}grade isolated CANFD chip; compliant with CAN protocol version 2.0B specification, with DG128 green terminal led out.
\\
\sphinxhline
\sphinxAtStartPar
RS485
&
\sphinxAtStartPar
2
&
\sphinxAtStartPar
Industrial\sphinxhyphen{}grade isolated RS485 chip, supporting speeds up to 4Mbps, with DG128 green terminal led out.
\\
\sphinxhline
\sphinxAtStartPar
QSPI
&
\sphinxAtStartPar
1
&
\sphinxAtStartPar
Dev board features 2 pcs 16MB FLASH chips, 1 using QSPI for communication.
\\
\sphinxhline
\sphinxAtStartPar
SPI
&
\sphinxAtStartPar
1
&
\sphinxAtStartPar
The development board is equipped with 2 x 16MB FLASH storage chips, one of which uses SPI communication; can be configured as SPI startup;
\\
\sphinxhline
\sphinxAtStartPar
RTC
&
\sphinxAtStartPar
1
&
\sphinxAtStartPar
The development board is equipped with a CR2032 coin cell battery to supply RTC power for the SoM. After the development board is powered off, the coin cell battery can be used to record time.
\\
\sphinxhline
\sphinxAtStartPar
I2C
&
\sphinxAtStartPar
4
&
\sphinxAtStartPar
It is used to mount devices such as audio, cameras, and touchscreens on the development board.
\\
\sphinxhline
\sphinxAtStartPar
Debug UART
&
\sphinxAtStartPar
2
&
\sphinxAtStartPar
Convert 2 x serial ports to 1 x USB for device debugging; the development board uses UART1 and UART2 as debug functions;
\\
\sphinxhline
\sphinxAtStartPar
UART
&
\sphinxAtStartPar
2
&
\sphinxAtStartPar
The development board uses UART0 and UART3 for RS485 functionality.
\\
\sphinxhline
\sphinxAtStartPar
PWM
&
\sphinxAtStartPar
5
&
\sphinxAtStartPar
Used to adjust the backlight brightness of the display screen and LED breathing lights;
\\
\sphinxhline
\sphinxAtStartPar
GPIO
&
\sphinxAtStartPar
| Multiple GPIO pins on the pin headers, including special \sphinxhyphen{} function pins specified by the SMARC protocol.
&
\sphinxAtStartPar

\\
\sphinxbottomrule
\end{tabulary}
\sphinxtableafterendhook\par
\sphinxattableend\end{savenotes}

\sphinxAtStartPar
Note: The parameters in the table are the theoretical values of hardware design or CPU;


\section{8.5 Development Board  Resources Block Diagram}
\label{\detokenize{hardware:development-board-resources-block-diagram}}
\sphinxAtStartPar
\sphinxincludegraphics{{5b6f3c29e63a4196a856db02410d7ac1}.png}

\sphinxAtStartPar
Figure 8\sphinxhyphen{}3 OK\sphinxhyphen{}MX8MPQ\sphinxhyphen{}SMARC Development Board Resources Block Diagram


\chapter{9. OK\sphinxhyphen{}MX8MPQ\sphinxhyphen{}SMARC Schematic Diagram}
\label{\detokenize{hardware:ok-mx8mpq-smarc-schematic-diagram}}
\sphinxAtStartPar
This chapter presents the schematic diagrams of the module interfaces and the connector pin assignments for the OK\sphinxhyphen{}MX8MPQ\sphinxhyphen{}SMARC development board, along with explanatory notes and important information that requires attention.\\
\sphinxstylestrong{Note: The component UID with “\_DNP” mark in the diagram below represents it is not soldered by\\
default}


\section{9.1 Development Board Power}
\label{\detokenize{hardware:development-board-power}}
\sphinxAtStartPar
DC 12V Power Supply: A 12V adapter is connected via P5, and after being switched on by S1, power passes through a TVS diode, resettable fuse, reverse protection diode, and filter capacitors to supply power to the subsequent circuits, as shown in the figure.

\sphinxAtStartPar
\sphinxincludegraphics{{fcbde94f07e04595929516e7fb34e463}.png}

\sphinxAtStartPar
Figure 9\sphinxhyphen{}1 12V Adapter Power Input

\sphinxAtStartPar
The VCC12V\_DCIN voltage is converted to VSYS\_5V by the switching power supply U2, providing direct power to the SoM, as shown in the figure.

\sphinxAtStartPar
\sphinxincludegraphics{{13df717fda2346cbbb94b1fa00eecf9c}.png}

\sphinxAtStartPar
Figure 9\sphinxhyphen{}2 Switching Power Supply Output VSYS\_5V to SoM

\sphinxAtStartPar
Once the SoM is successfully powered on, the CARRIER\_PWR\_ON\_1.8V power control signal from the SoM goes high, sequentially enabling transistor Q1 and PMOS chip U8. The PMOS chip subsequently outputs VCC\_5V to power the peripherals of the development board. If the VCC\_5V power supply is functioning properly, LED D3 will light up, indicating that the development board is powered normally. As shown in the figure below:

\sphinxAtStartPar
\sphinxincludegraphics{{b7b4ffbd4f2945ffa5f9813916763cd5}.png}

\sphinxAtStartPar
Figure 9\sphinxhyphen{}3: MOS Switch Output VCC\_5V to Power Carrier Board Peripherals

\sphinxAtStartPar
VCC\_5V is converted by the switching regulator U3 to produce VCC\_3V3, which powers certain peripherals on the development board and also provides input power for downstream linear regulators. No additional enable control is needed for the U3 chip. As shown in the figure below:

\sphinxAtStartPar
\sphinxincludegraphics{{0e91bb45812042b297d790f942af0ea8}.png}

\sphinxAtStartPar
Figure 9\sphinxhyphen{}4: Switching Power Supply Output VCC\_3V3 to Power Carrier Board Peripherals

\sphinxAtStartPar
The VCC\_3V3 voltage passes through the low\sphinxhyphen{}dropout linear regulator U1 to generate the VCC\_1V8 voltage. The driving capability of VCC\_1V8 is relatively weak, and it is mainly used to provide a 1.8V pull\sphinxhyphen{}up voltage for the signal lines on the development board. As shown in the figure below:

\sphinxAtStartPar
\sphinxincludegraphics{{c8cb5c9705b94fc0935d9b6e17317d22}.png}

\sphinxAtStartPar
Figure 9\sphinxhyphen{}5: LDO Switch Output VCC\_1V8 to Provide Pull\sphinxhyphen{}up Power for Carrier Board Peripherals


\subsection{9.1.1 Power Connector Pin Description}
\label{\detokenize{hardware:power-connector-pin-description}}
\sphinxAtStartPar
Table 9\sphinxhyphen{}1 P5 Interface Pin Function Description


\begin{savenotes}\sphinxattablestart
\sphinxthistablewithglobalstyle
\centering
\begin{tabulary}{\linewidth}[t]{TTTT}
\sphinxtoprule
\sphinxstyletheadfamily 
\sphinxAtStartPar
\sphinxstylestrong{Pin}
&\sphinxstyletheadfamily 
\sphinxAtStartPar
\sphinxstylestrong{Development Board Signal}
&\sphinxstyletheadfamily 
\sphinxAtStartPar
\sphinxstylestrong{Type}
&\sphinxstyletheadfamily 
\sphinxAtStartPar
\sphinxstylestrong{Description}
\\
\sphinxmidrule
\sphinxtableatstartofbodyhook
\sphinxAtStartPar
\sphinxstylestrong{1}
&
\sphinxAtStartPar
VCC\_12V
&
\sphinxAtStartPar
P
&
\sphinxAtStartPar
12V power input
\\
\sphinxhline
\sphinxAtStartPar
\sphinxstylestrong{2}
&
\sphinxAtStartPar
GND
&
\sphinxAtStartPar
P
&
\sphinxAtStartPar
Ground
\\
\sphinxhline
\sphinxAtStartPar
\sphinxstylestrong{3}
&
\sphinxAtStartPar
| | \textbackslash{}
&
\sphinxAtStartPar

&
\sphinxAtStartPar

\\
\sphinxbottomrule
\end{tabulary}
\sphinxtableafterendhook\par
\sphinxattableend\end{savenotes}

\sphinxAtStartPar
\sphinxstylestrong{Note: The 12 V supplied by the adapter supplies 12 V to the PCIe device through the PMOS power switch, so try to use the power adapter provided with the development board.}


\section{9.2 Control Key}
\label{\detokenize{hardware:control-key}}
\sphinxAtStartPar
The OK\sphinxhyphen{}MX8MPQ\sphinxhyphen{}SMARC development board is configured with a reset button and a power control button. Press the reset button for a short time to reset the whole machine after power failure.

\sphinxAtStartPar
Press the power control button for more than 6 seconds to realize the startup and shutdown function (this function is optional).

\sphinxAtStartPar
The reset and power\sphinxhyphen{}on/off signals have been reserved for 1.8 V pull\sphinxhyphen{}up in the SoM version, so there is no need to add a pull\sphinxhyphen{}up level on the development board, and only need to consider anti\sphinxhyphen{}jitter and anti\sphinxhyphen{}static treatment. As shown in the figure below:

\sphinxAtStartPar
\sphinxincludegraphics{{f0ec22a093084f5fa52eb60ffbc525c5}.png}

\sphinxAtStartPar
Figure 9.2\sphinxhyphen{}1 Reset Button

\sphinxAtStartPar
\sphinxincludegraphics{{78f2084a054644d8b9c986104c598c1c}.png}

\sphinxAtStartPar
Figure 9.2\sphinxhyphen{}2 On/off Button

\sphinxAtStartPar
\sphinxincludegraphics{{00d1f8e71d8f486384793740df83d9f2}.png}


\section{9.3 Boot Configuration}
\label{\detokenize{hardware:id26}}
\sphinxAtStartPar
\sphinxincludegraphics{{8b4feb6bb7c74a8f8bf49c46b90b94ed}.png}

\sphinxAtStartPar
Figure 9.3\sphinxhyphen{}1 BOOT DIP Switch Mode Selection

\sphinxAtStartPar
The OK\sphinxhyphen{}MX8MPQ\sphinxhyphen{}SMARC development board supports the following device boot modes:
\begin{itemize}
\item {} 
\sphinxAtStartPar
Onboard SD Card Startup

\item {} 
\sphinxAtStartPar
Onboard SPI Flash Start

\item {} 
\sphinxAtStartPar
SoM eMMC startup (the system default startup configuration of the development board is eMMC startup)

\item {} 
\sphinxAtStartPar
USB Flashing

\end{itemize}

\sphinxAtStartPar
\sphinxstylestrong{The startup method of the OK\sphinxhyphen{}MX8MPQ\sphinxhyphen{}SMARC development board needs to comply with the SMARC protocol requirements.}\\
\sphinxstylestrong{The BOOT pin on the SoM is default pulled up, so the signal pin of the S2 dip switch on the development board does not need to be pulled up.}


\begin{savenotes}\sphinxattablestart
\sphinxthistablewithglobalstyle
\centering
\begin{tabulary}{\linewidth}[t]{TTTTT}
\sphinxtoprule
\sphinxstyletheadfamily 
\sphinxAtStartPar
\sphinxstylestrong{BOOT\_SEL{[}2:0{]}}
&\sphinxstyletheadfamily 
\sphinxAtStartPar
\sphinxstylestrong{MODE2}
&\sphinxstyletheadfamily 
\sphinxAtStartPar
\sphinxstylestrong{MODE1}
&\sphinxstyletheadfamily 
\sphinxAtStartPar
\sphinxstylestrong{MODE0}
&\sphinxstyletheadfamily 
\sphinxAtStartPar
\sphinxstylestrong{FORCE\_RECOV\#}
\\
\sphinxmidrule
\sphinxtableatstartofbodyhook
\sphinxAtStartPar
Carrier SD Card
&
\sphinxAtStartPar
GND
&
\sphinxAtStartPar
GND
&
\sphinxAtStartPar
Float
&
\sphinxAtStartPar
Float
\\
\sphinxhline
\sphinxAtStartPar
Carrier SPI (CS0\#)
&
\sphinxAtStartPar
GND
&
\sphinxAtStartPar
Float
&
\sphinxAtStartPar
Float
&
\sphinxAtStartPar
Float
\\
\sphinxhline
\sphinxAtStartPar
Module eMMC Flash
&
\sphinxAtStartPar
Float
&
\sphinxAtStartPar
Float
&
\sphinxAtStartPar
GND
&
\sphinxAtStartPar
Float
\\
\sphinxhline
\sphinxAtStartPar
Module SPI
&
\sphinxAtStartPar
Float
&
\sphinxAtStartPar
Float
&
\sphinxAtStartPar
Float
&
\sphinxAtStartPar
Float
\\
\sphinxhline
\sphinxAtStartPar
USB Serial Download
&
\sphinxAtStartPar
X
&
\sphinxAtStartPar
X
&
\sphinxAtStartPar
X
&
\sphinxAtStartPar
GND
\\
\sphinxbottomrule
\end{tabulary}
\sphinxtableafterendhook\par
\sphinxattableend\end{savenotes}


\section{9.4 Debugging Serial Port}
\label{\detokenize{hardware:debugging-serial-port}}
\sphinxAtStartPar
The i.MX8M Plus has multiple Cortex\sphinxhyphen{}A53 cores and 1 x Cortex\sphinxhyphen{}M7 core, so each core uses its own serial port for debugging.

\sphinxAtStartPar
The UART1 of SMARC interface is from the UART2 of I.MX8MP, which is mainly used for debugging Cortex\sphinxhyphen{}A53 core (Mainly used);

\sphinxAtStartPar
The UART2 of SMARC interface is from UART4 of I.MX8MP, which is for debugging Cortex\sphinxhyphen{}M7 core;

\sphinxAtStartPar
The digital level converter U10 plays a role in equal level conversion and isolation protection, preventing external power from entering the SoM through debugging when the development board is powered off. U9 is a linear regulator that provides isolated power supply for level converters, and the DEBUG\_5V power supply comes from external devices. This part of the circuit is shown in the figure.

\sphinxAtStartPar
\sphinxincludegraphics{{7a326207df71420bbde19a972ed60759}.png}

\sphinxAtStartPar
Figure 9.4\sphinxhyphen{}1 Debug External Power Supply and Level Conversion

\sphinxAtStartPar
The development board converts two common debug serial ports into USB signals through U11, and the USB signals are led out from the Type\sphinxhyphen{}C socket (P2). In this way, only 1 x Type\sphinxhyphen{}C interface can be used to debug 2 x debug ports. Positive and negative sensing is not required, so two CC pins can be pulled down to ground. The circuit for converting two serial ports to USB is shown in the figure.

\sphinxAtStartPar
\sphinxincludegraphics{{7d7a166ba37441b4b6a32cac4f9e221c}.png}

\sphinxAtStartPar
Figure 9.4\sphinxhyphen{}2 Dual TTL Serial Port to USB


\subsection{9.4.1 Debug Connector Pin Description}
\label{\detokenize{hardware:debug-connector-pin-description}}
\sphinxAtStartPar
Table 9.4\sphinxhyphen{}1 P2 Interface Pin Function Description


\begin{savenotes}\sphinxattablestart
\sphinxthistablewithglobalstyle
\centering
\begin{tabulary}{\linewidth}[t]{TTTT}
\sphinxtoprule
\sphinxstyletheadfamily 
\sphinxAtStartPar
\sphinxstylestrong{Pin}
&\sphinxstyletheadfamily 
\sphinxAtStartPar
\sphinxstylestrong{Development Board Signal}
&\sphinxstyletheadfamily 
\sphinxAtStartPar
\sphinxstylestrong{Type}
&\sphinxstyletheadfamily 
\sphinxAtStartPar
\sphinxstylestrong{Description}
\\
\sphinxmidrule
\sphinxtableatstartofbodyhook
\sphinxAtStartPar
\sphinxstylestrong{A1/B12}
&
\sphinxAtStartPar
GND
&
\sphinxAtStartPar
P
&
\sphinxAtStartPar
Ground
\\
\sphinxhline
\sphinxAtStartPar
\sphinxstylestrong{B1/A12}
&
\sphinxAtStartPar
GND
&
\sphinxAtStartPar
P
&
\sphinxAtStartPar
Ground
\\
\sphinxhline
\sphinxAtStartPar
\sphinxstylestrong{A4/B9}
&
\sphinxAtStartPar
VBUS
&
\sphinxAtStartPar
P
&
\sphinxAtStartPar
5V power input
\\
\sphinxhline
\sphinxAtStartPar
\sphinxstylestrong{B4/A9}
&
\sphinxAtStartPar
VBUS
&
\sphinxAtStartPar
P
&
\sphinxAtStartPar
5V power input
\\
\sphinxhline
\sphinxAtStartPar
\sphinxstylestrong{A5}
&
\sphinxAtStartPar
CC1
&
\sphinxAtStartPar
| \textbackslash{}
&
\sphinxAtStartPar

\\
\sphinxhline
\sphinxAtStartPar
\sphinxstylestrong{B5}
&
\sphinxAtStartPar
CC2
&
\sphinxAtStartPar
| \textbackslash{}
&
\sphinxAtStartPar

\\
\sphinxhline
\sphinxAtStartPar
\sphinxstylestrong{A6}
&
\sphinxAtStartPar
DP1
&
\sphinxAtStartPar
DSIO
&
\sphinxAtStartPar
USB Data Positive
\\
\sphinxhline
\sphinxAtStartPar
\sphinxstylestrong{B6}
&
\sphinxAtStartPar
DP2
&
\sphinxAtStartPar
DSIO
&
\sphinxAtStartPar
USB Data Positive
\\
\sphinxhline
\sphinxAtStartPar
\sphinxstylestrong{A7}
&
\sphinxAtStartPar
DM1
&
\sphinxAtStartPar
DSIO
&
\sphinxAtStartPar
USB Data Negative
\\
\sphinxhline
\sphinxAtStartPar
\sphinxstylestrong{B7}
&
\sphinxAtStartPar
DM2
&
\sphinxAtStartPar
DSIO
&
\sphinxAtStartPar
USB Data Negative
\\
\sphinxhline
\sphinxAtStartPar
\sphinxstylestrong{E1}
&
\sphinxAtStartPar
GND
&
\sphinxAtStartPar
P
&
\sphinxAtStartPar
SHIELD  pin reference
\\
\sphinxhline
\sphinxAtStartPar
\sphinxstylestrong{E2}
&
\sphinxAtStartPar
GND
&
\sphinxAtStartPar
P
&
\sphinxAtStartPar
SHIELD  pin reference
\\
\sphinxhline
\sphinxAtStartPar
\sphinxstylestrong{E3}
&
\sphinxAtStartPar
GND
&
\sphinxAtStartPar
P
&
\sphinxAtStartPar
SHIELD  pin reference
\\
\sphinxhline
\sphinxAtStartPar
\sphinxstylestrong{E4}
&
\sphinxAtStartPar
GND
&
\sphinxAtStartPar
P
&
\sphinxAtStartPar
SHIELD  pin reference
\\
\sphinxbottomrule
\end{tabulary}
\sphinxtableafterendhook\par
\sphinxattableend\end{savenotes}


\section{9.5 USB 3.0}
\label{\detokenize{hardware:usb-3-0}}
\sphinxAtStartPar
The OK\sphinxhyphen{}MX8MPQ\sphinxhyphen{}SMARC development board supports 1 x USB 3.0 Type\sphinxhyphen{}A interface P18, as shown in Figure 6.5\sphinxhyphen{}1. Only used for HOST function. Add TVS devices to the USB signal cable and USB power supply for electrostatic protection and plug and unplug protection, as shown in the figure.

\sphinxAtStartPar
\sphinxincludegraphics{{cd710ed6964340b29d0813ba29098e09}.png}

\sphinxAtStartPar
Figure 9.5\sphinxhyphen{}1 USB 3.0 Type\sphinxhyphen{}A Interface

\sphinxAtStartPar
\sphinxincludegraphics{{a6db99da4fb54dd5a54b14e257d491b5}.png}

\sphinxAtStartPar
Figure 9.5\sphinxhyphen{}2 Static Protection of USB 3.0 Interface

\sphinxAtStartPar
The external power supply voltage of the USB3.0 Type\sphinxhyphen{}A interface on the development board is from VCC \_ 5V, and the VCC \_ 5V performs 1A overcurrent protection on the USB output power supply voltage USB3 \_ PWR \_ 5V through the load switch U19. USB3 \_ EN \_ OC \# is normally output to the CPU at high level. When the overcurrent event occurs, the signal is pulled low, and the CPU will control the GPIO \_ P10 \_ USB3 \_ EN signal to cut off the external output of the USB3.0 Type\sphinxhyphen{}A interface, as shown in the figure.

\sphinxAtStartPar
\sphinxincludegraphics{{a0de7d31b12149eb83d2fbf640ab70a4}.png}

\sphinxAtStartPar
Figure 9.5.3 USB Load Switch


\subsection{9.5.1 USB3.0 Connector Pin Description}
\label{\detokenize{hardware:usb3-0-connector-pin-description}}
\sphinxAtStartPar
Table 9.5\sphinxhyphen{}1 P18 Interface Pin Function Description


\begin{savenotes}\sphinxattablestart
\sphinxthistablewithglobalstyle
\centering
\begin{tabulary}{\linewidth}[t]{TTTT}
\sphinxtoprule
\sphinxstyletheadfamily 
\sphinxAtStartPar
\sphinxstylestrong{Pin}
&\sphinxstyletheadfamily 
\sphinxAtStartPar
\sphinxstylestrong{Development Board Signal}
&\sphinxstyletheadfamily 
\sphinxAtStartPar
\sphinxstylestrong{Type}
&\sphinxstyletheadfamily 
\sphinxAtStartPar
\sphinxstylestrong{Description}
\\
\sphinxmidrule
\sphinxtableatstartofbodyhook
\sphinxAtStartPar
\sphinxstylestrong{1}
&
\sphinxAtStartPar
USB3\_PWR\_5V
&
\sphinxAtStartPar
P
&
\sphinxAtStartPar
VBUS  power
\\
\sphinxhline
\sphinxAtStartPar
\sphinxstylestrong{2}
&
\sphinxAtStartPar
USB3\_D\_N
&
\sphinxAtStartPar
DSIO
&
\sphinxAtStartPar
Non\sphinxhyphen{}SuperSpeed  diff. pair, negative
\\
\sphinxhline
\sphinxAtStartPar
\sphinxstylestrong{3}
&
\sphinxAtStartPar
USB3\_D\_P
&
\sphinxAtStartPar
DSIO
&
\sphinxAtStartPar
Non\sphinxhyphen{}SuperSpeed  diff. pair, positive
\\
\sphinxhline
\sphinxAtStartPar
\sphinxstylestrong{4}
&
\sphinxAtStartPar
GND
&
\sphinxAtStartPar
P
&
\sphinxAtStartPar
Digital  Ground
\\
\sphinxhline
\sphinxAtStartPar
\sphinxstylestrong{5}
&
\sphinxAtStartPar
USB3\_SSRX\_R\_N
&
\sphinxAtStartPar
DSI
&
\sphinxAtStartPar
SuperSpeed  diff. pair, RX, negative
\\
\sphinxhline
\sphinxAtStartPar
\sphinxstylestrong{6}
&
\sphinxAtStartPar
USB3\_SSRX\_R\_P
&
\sphinxAtStartPar
DSI
&
\sphinxAtStartPar
SuperSpeed  diff. pair, RX, positive
\\
\sphinxhline
\sphinxAtStartPar
\sphinxstylestrong{7}
&
\sphinxAtStartPar
GND
&
\sphinxAtStartPar
P
&
\sphinxAtStartPar
Digital  Ground
\\
\sphinxhline
\sphinxAtStartPar
\sphinxstylestrong{8}
&
\sphinxAtStartPar
USB3\_SSTX\_T\_N
&
\sphinxAtStartPar
DSO
&
\sphinxAtStartPar
SuperSpeed  diff. pair, TX, negative
\\
\sphinxhline
\sphinxAtStartPar
\sphinxstylestrong{9}
&
\sphinxAtStartPar
USB3\_SSTX\_T\_P
&
\sphinxAtStartPar
DSO
&
\sphinxAtStartPar
SuperSpeed  diff. pair, TX, positive
\\
\sphinxhline
\sphinxAtStartPar
\sphinxstylestrong{SH}
&
\sphinxAtStartPar
GND
&
\sphinxAtStartPar
P
&
\sphinxAtStartPar
SHIELD  pin reference
\\
\sphinxhline
\sphinxAtStartPar
\sphinxstylestrong{SH}
&
\sphinxAtStartPar
GND
&
\sphinxAtStartPar
P
&
\sphinxAtStartPar
SHIELD  pin reference
\\
\sphinxbottomrule
\end{tabulary}
\sphinxtableafterendhook\par
\sphinxattableend\end{savenotes}


\section{9.6 USB2.0}
\label{\detokenize{hardware:usb2-0}}
\sphinxAtStartPar
The OK\sphinxhyphen{}MX8 MPQ\sphinxhyphen{}SMARC development board supports 3 x USB2.0 Type\sphinxhyphen{}A interfaces P20, P21, and P22, and is only used for the HOST function. Add TVS devices to USB signal cables and USB power supplies for electrostatic protection and plug and unplug protection.

\sphinxAtStartPar
The power supply voltage of each USB2.0 Type\sphinxhyphen{}A interface is from VCC \_ 5V, and the overvoltage and overcurrent load switch provides 0.5A current limiting protection for the USB2.0 Type\sphinxhyphen{}A interface. If an overcurrent situation occurs, the normally high OC pin will pull low to alert the CPU, and the CPU will perform power\sphinxhyphen{}off protection on the load switch. As shown in the figure, there are three sets of USB 2.0 Type\sphinxhyphen{}A interfaces and load switch protection circuits that are externally connected.

\sphinxAtStartPar
\sphinxincludegraphics{{d78905baa8084597a5186b81a73e2552}.png}

\sphinxAtStartPar
Figure 9.6\sphinxhyphen{}1 USB 2.0\sphinxhyphen{}1 Type\sphinxhyphen{}A Interface

\sphinxAtStartPar
\sphinxincludegraphics{{47e7f706b0c341e6acb8c4f464d2f86a}.png}

\sphinxAtStartPar
Figure 9.6\sphinxhyphen{}2 USB 2.0\sphinxhyphen{}1 USB Load Switch

\sphinxAtStartPar
\sphinxincludegraphics{{14fcc79150414c009b8d4287f09f43ad}.png}

\sphinxAtStartPar
Figure 9.6\sphinxhyphen{}3 USB 2.0\sphinxhyphen{}4 Type\sphinxhyphen{}A Interface

\sphinxAtStartPar
\sphinxincludegraphics{{1b80a867026a42a28293c7560bf2e0f1}.png}

\sphinxAtStartPar
Figure 9.6\sphinxhyphen{}4 USB 2.0\sphinxhyphen{}4 USB Load Switch

\sphinxAtStartPar
\sphinxincludegraphics{{e9081054db8e4c3285afaee092b54705}.png}

\sphinxAtStartPar
Figure 9.6\sphinxhyphen{}5 USB 2.0\sphinxhyphen{}5 Type\sphinxhyphen{}A Interface

\sphinxAtStartPar
\sphinxincludegraphics{{f9adb4d2c6554d398017a0543325c279}.png}

\sphinxAtStartPar
Figure 9.6\sphinxhyphen{}6 USB 2.0\sphinxhyphen{}5 USB Load Switch


\subsection{9.6.1 USB2.0 Connector Pin Description}
\label{\detokenize{hardware:usb2-0-connector-pin-description}}
\sphinxAtStartPar
Table 9.6\sphinxhyphen{}1 P20 Interface Pin Function Description


\begin{savenotes}\sphinxattablestart
\sphinxthistablewithglobalstyle
\centering
\begin{tabulary}{\linewidth}[t]{TTTT}
\sphinxtoprule
\sphinxstyletheadfamily 
\sphinxAtStartPar
\sphinxstylestrong{Pin}
&\sphinxstyletheadfamily 
\sphinxAtStartPar
\sphinxstylestrong{Development Board Signal}
&\sphinxstyletheadfamily 
\sphinxAtStartPar
\sphinxstylestrong{Type}
&\sphinxstyletheadfamily 
\sphinxAtStartPar
\sphinxstylestrong{Description}
\\
\sphinxmidrule
\sphinxtableatstartofbodyhook
\sphinxAtStartPar
\sphinxstylestrong{1}
&
\sphinxAtStartPar
USB1\_PWR\_5V
&
\sphinxAtStartPar
P
&
\sphinxAtStartPar
VBUS  power
\\
\sphinxhline
\sphinxAtStartPar
\sphinxstylestrong{2}
&
\sphinxAtStartPar
USB1\_D\_N
&
\sphinxAtStartPar
DSIO
&
\sphinxAtStartPar
Non\sphinxhyphen{}SuperSpeed  diff. pair, negative
\\
\sphinxhline
\sphinxAtStartPar
\sphinxstylestrong{3}
&
\sphinxAtStartPar
USB1\_D\_P
&
\sphinxAtStartPar
DSIO
&
\sphinxAtStartPar
Non\sphinxhyphen{}SuperSpeed  diff. pair, positive
\\
\sphinxhline
\sphinxAtStartPar
\sphinxstylestrong{4}
&
\sphinxAtStartPar
GND
&
\sphinxAtStartPar
P
&
\sphinxAtStartPar
Digital  Ground
\\
\sphinxhline
\sphinxAtStartPar
\sphinxstylestrong{5}
&
\sphinxAtStartPar
GND
&
\sphinxAtStartPar
P
&
\sphinxAtStartPar
SHIELD  pin reference
\\
\sphinxhline
\sphinxAtStartPar
\sphinxstylestrong{6}
&
\sphinxAtStartPar
GND
&
\sphinxAtStartPar
P
&
\sphinxAtStartPar
SHIELD  pin reference
\\
\sphinxhline
\sphinxAtStartPar
\sphinxstylestrong{7}
&
\sphinxAtStartPar
GND
&
\sphinxAtStartPar
P
&
\sphinxAtStartPar
SHIELD  pin reference
\\
\sphinxhline
\sphinxAtStartPar
\sphinxstylestrong{8}
&
\sphinxAtStartPar
GND
&
\sphinxAtStartPar
P
&
\sphinxAtStartPar
SHIELD  pin reference
\\
\sphinxbottomrule
\end{tabulary}
\sphinxtableafterendhook\par
\sphinxattableend\end{savenotes}

\sphinxAtStartPar
Table 9.6\sphinxhyphen{}2 P21 Interface Pin Function Description


\begin{savenotes}\sphinxattablestart
\sphinxthistablewithglobalstyle
\centering
\begin{tabulary}{\linewidth}[t]{TTTT}
\sphinxtoprule
\sphinxstyletheadfamily 
\sphinxAtStartPar
\sphinxstylestrong{Pin}
&\sphinxstyletheadfamily 
\sphinxAtStartPar
\sphinxstylestrong{Development Board Signal}
&\sphinxstyletheadfamily 
\sphinxAtStartPar
\sphinxstylestrong{Type}
&\sphinxstyletheadfamily 
\sphinxAtStartPar
\sphinxstylestrong{Description}
\\
\sphinxmidrule
\sphinxtableatstartofbodyhook
\sphinxAtStartPar
\sphinxstylestrong{1}
&
\sphinxAtStartPar
USB4\_PWR\_5V
&
\sphinxAtStartPar
P
&
\sphinxAtStartPar
VBUS  power
\\
\sphinxhline
\sphinxAtStartPar
\sphinxstylestrong{2}
&
\sphinxAtStartPar
USB4\_D\_N
&
\sphinxAtStartPar
DSIO
&
\sphinxAtStartPar
Non\sphinxhyphen{}SuperSpeed  diff. pair, negative
\\
\sphinxhline
\sphinxAtStartPar
\sphinxstylestrong{3}
&
\sphinxAtStartPar
USB4\_D\_P
&
\sphinxAtStartPar
DSIO
&
\sphinxAtStartPar
Non\sphinxhyphen{}SuperSpeed  diff. pair, positive
\\
\sphinxhline
\sphinxAtStartPar
\sphinxstylestrong{4}
&
\sphinxAtStartPar
GND
&
\sphinxAtStartPar
P
&
\sphinxAtStartPar
Digital  Ground
\\
\sphinxhline
\sphinxAtStartPar
\sphinxstylestrong{5}
&
\sphinxAtStartPar
GND
&
\sphinxAtStartPar
P
&
\sphinxAtStartPar
SHIELD  pin reference
\\
\sphinxhline
\sphinxAtStartPar
\sphinxstylestrong{6}
&
\sphinxAtStartPar
GND
&
\sphinxAtStartPar
P
&
\sphinxAtStartPar
SHIELD  pin reference
\\
\sphinxhline
\sphinxAtStartPar
\sphinxstylestrong{7}
&
\sphinxAtStartPar
GND
&
\sphinxAtStartPar
P
&
\sphinxAtStartPar
SHIELD  pin reference
\\
\sphinxhline
\sphinxAtStartPar
\sphinxstylestrong{8}
&
\sphinxAtStartPar
GND
&
\sphinxAtStartPar
P
&
\sphinxAtStartPar
SHIELD  pin reference
\\
\sphinxbottomrule
\end{tabulary}
\sphinxtableafterendhook\par
\sphinxattableend\end{savenotes}

\sphinxAtStartPar
Table 9.6\sphinxhyphen{}3 P22 Interface Pin Function Description


\begin{savenotes}\sphinxattablestart
\sphinxthistablewithglobalstyle
\centering
\begin{tabulary}{\linewidth}[t]{TTTT}
\sphinxtoprule
\sphinxstyletheadfamily 
\sphinxAtStartPar
\sphinxstylestrong{Pin}
&\sphinxstyletheadfamily 
\sphinxAtStartPar
\sphinxstylestrong{Development Board Signal}
&\sphinxstyletheadfamily 
\sphinxAtStartPar
\sphinxstylestrong{Type}
&\sphinxstyletheadfamily 
\sphinxAtStartPar
\sphinxstylestrong{Description}
\\
\sphinxmidrule
\sphinxtableatstartofbodyhook
\sphinxAtStartPar
\sphinxstylestrong{1}
&
\sphinxAtStartPar
USB5\_PWR\_5V
&
\sphinxAtStartPar
P
&
\sphinxAtStartPar
VBUS  power
\\
\sphinxhline
\sphinxAtStartPar
\sphinxstylestrong{2}
&
\sphinxAtStartPar
USB5\_D\_N
&
\sphinxAtStartPar
DSIO
&
\sphinxAtStartPar
Non\sphinxhyphen{}SuperSpeed  diff. pair, negative
\\
\sphinxhline
\sphinxAtStartPar
\sphinxstylestrong{3}
&
\sphinxAtStartPar
USB5\_D\_P
&
\sphinxAtStartPar
DSIO
&
\sphinxAtStartPar
Non\sphinxhyphen{}SuperSpeed  diff. pair, positive
\\
\sphinxhline
\sphinxAtStartPar
\sphinxstylestrong{4}
&
\sphinxAtStartPar
GND
&
\sphinxAtStartPar
P
&
\sphinxAtStartPar
Digital  Ground
\\
\sphinxhline
\sphinxAtStartPar
\sphinxstylestrong{5}
&
\sphinxAtStartPar
GND
&
\sphinxAtStartPar
P
&
\sphinxAtStartPar
SHIELD  pin reference
\\
\sphinxhline
\sphinxAtStartPar
\sphinxstylestrong{6}
&
\sphinxAtStartPar
GND
&
\sphinxAtStartPar
P
&
\sphinxAtStartPar
SHIELD  pin reference
\\
\sphinxhline
\sphinxAtStartPar
\sphinxstylestrong{7}
&
\sphinxAtStartPar
GND
&
\sphinxAtStartPar
P
&
\sphinxAtStartPar
SHIELD  pin reference
\\
\sphinxhline
\sphinxAtStartPar
\sphinxstylestrong{8}
&
\sphinxAtStartPar
GND
&
\sphinxAtStartPar
P
&
\sphinxAtStartPar
SHIELD  pin reference
\\
\sphinxbottomrule
\end{tabulary}
\sphinxtableafterendhook\par
\sphinxattableend\end{savenotes}


\section{9.7 USB2.0 OTG}
\label{\detokenize{hardware:usb2-0-otg}}
\sphinxAtStartPar
The OK\sphinxhyphen{}MX8MPQ\sphinxhyphen{}SMARC development board supports 1 x USB 2.0 OTG connection to a micro USB interface. Add TVS devices to USB signal cables and USB power supplies for electrostatic protection and plug and unplug protection. As shown in the figure below:

\sphinxAtStartPar
\sphinxincludegraphics{{331ce5d58161480d856612ce4881ec18}.png}

\sphinxAtStartPar
Figure 9.7\sphinxhyphen{}1 USB 2.0 Type\sphinxhyphen{}C Interface

\sphinxAtStartPar
\sphinxincludegraphics{{39352191133f4c59baa980fe6aa2dfa5}.png}

\sphinxAtStartPar
Figure 9.7\sphinxhyphen{}2 Selection of DIP Switch Mode

\sphinxAtStartPar
\sphinxincludegraphics{{b407697caa9d44ca806a7832efec78d1}.png}

\sphinxAtStartPar
Figure 9.7\sphinxhyphen{}3 VBUS External Power Supply

\sphinxAtStartPar
\sphinxincludegraphics{{1dac101484d345baa86242ee871dd88a}.png}

\sphinxAtStartPar
Figure 9.7\sphinxhyphen{}4 VBUS Voltage Detection

\sphinxAtStartPar
CARRIER\_PWR\_ON\_1.8V is the power enable pin for the development board. When the board is powered normally, both switch transistors Q18 and Q19 are turned on. The USB0\_VBUS\_DET pin of the SoM reads the VBUS\_TYPEC voltage to detect the current power status and supply voltage of the USB 2.0 Type\sphinxhyphen{}C port


\section{9.8 4G/ 5G Module}
\label{\detokenize{hardware:g-5g-module}}
\sphinxAtStartPar
The OK\sphinxhyphen{}MX8MPQ\sphinxhyphen{}SMARC development board provides 1 x M.2 B\sphinxhyphen{}KEY slot for optional 4G/5G modules. The board uses DIP switch S4 and NMOS switch Q26 to adjust the feedback resistance ratio of U13, thereby changing the power output voltage

\sphinxAtStartPar
Optional 4G module Quectel’EM05, transmitting data through USB 2.0 signal, typically powered by 3.3V; When the DIP switch S4 is on, the gate of the NMOS is pulled low, the drain and source are disconnected, and R141 does not participate in the feedback loop of the switching power supply. At this time, the output voltage is 3.3V;

\sphinxAtStartPar
Optional 5G module Quectel\sphinxhyphen{}RM500Q\_5G, transmitting data through USB 3.0 signal, with a conventional 3.7V power supply; When the DIP switch S4 is Off, the gate of the NMOS is pulled high, and the drain and source are turned on. R141 participates in the feedback loop of the switching power supply, and outputs a voltage of 3.7V at this time;

\sphinxAtStartPar
The switch power supply controlled by dip switch S4 is shown in the diagram.

\sphinxAtStartPar
\sphinxincludegraphics{{2454a4bbe21e4a28932a2d0a71aaab76}.png}

\sphinxAtStartPar
Figure 9.8\sphinxhyphen{}1 Switching Power Supply Output VCC \_ 4G/5G

\sphinxAtStartPar
\sphinxincludegraphics{{fcf4b0eec80f4129bc1835fcea494956}.png}

\sphinxAtStartPar
Figure 9.8\sphinxhyphen{}2 4G/5G Module M.2 \_ B\sphinxhyphen{}KEY Slot

\sphinxAtStartPar
\sphinxincludegraphics{{b41b6fc4e09e4488a142bef822dac311}.png}

\sphinxAtStartPar
Figure 9.8\sphinxhyphen{}3 Micro SIM Card Slot


\section{9.9 Gigabit Network Port}
\label{\detokenize{hardware:gigabit-network-port}}
\sphinxAtStartPar
The OK\sphinxhyphen{}MX8MPQ\sphinxhyphen{}SMARC development board supports dual 1000M/100M/10M network ports, and the PHY of the Gigabit network port is integrated on the SoM. Therefore, only the differential pair of the MDI interface needs to be led out through the RJ45 interface with its own isolation transformer, and TVS devices need to be added for electrostatic protection.

\sphinxAtStartPar
\sphinxincludegraphics{{f909b5b6ea4b45f5b548dab42cb71d35}.png}

\sphinxAtStartPar
Figure 9.9\sphinxhyphen{}1 GBE0 Interface Connection Diagram

\sphinxAtStartPar
\sphinxincludegraphics{{50b8f7a3846149b8897b409b859fb70a}.png}

\sphinxAtStartPar
Figure 9.9\sphinxhyphen{}2 GBE1 Interface Connection Diagram

\sphinxAtStartPar
According to the SMARC protocol, the 2 x LED on the RJ45 connector are the link activity indicator and the 1 x Gbps/100 Mbps link speed indicator. Series resistors are reserved on the GBE\_LINK100\# and GBE\_LINK1000\# signals to provide current\sphinxhyphen{}limiting protection, though they also reduce the brightness of the Ethernet port LEDs. This design is based on the SMARC design guidelines, as shown in the figure.

\sphinxAtStartPar
\sphinxincludegraphics{{a9bfd0894e424b6cba5c2e4318aadc40}.png}

\sphinxAtStartPar
Figure 9.9\sphinxhyphen{}3: Ethernet Port Connection in the SMARC Design Guide


\subsection{9.9.1 Gigabit Ethernet RJ45 Connector}
\label{\detokenize{hardware:gigabit-ethernet-rj45-connector}}
\sphinxAtStartPar
Table 9.9\sphinxhyphen{}1 P4 Interface Pin Function Description


\begin{savenotes}\sphinxattablestart
\sphinxthistablewithglobalstyle
\centering
\begin{tabulary}{\linewidth}[t]{TTTT}
\sphinxtoprule
\sphinxstyletheadfamily 
\sphinxAtStartPar
\sphinxstylestrong{Pin}
&\sphinxstyletheadfamily 
\sphinxAtStartPar
\sphinxstylestrong{Development Board Signal}
&\sphinxstyletheadfamily 
\sphinxAtStartPar
\sphinxstylestrong{Type}
&\sphinxstyletheadfamily 
\sphinxAtStartPar
\sphinxstylestrong{Description}
\\
\sphinxmidrule
\sphinxtableatstartofbodyhook
\sphinxAtStartPar
\sphinxstylestrong{1}
&
\sphinxAtStartPar
GND
&
\sphinxAtStartPar
P
&
\sphinxAtStartPar
Ground
\\
\sphinxhline
\sphinxAtStartPar
\sphinxstylestrong{2}
&
\sphinxAtStartPar
NC
&
\sphinxAtStartPar
| \textbackslash{}
&
\sphinxAtStartPar

\\
\sphinxhline
\sphinxAtStartPar
\sphinxstylestrong{3}
&
\sphinxAtStartPar
GBE0\_MDI3\_P
&
\sphinxAtStartPar
DSIO
&
\sphinxAtStartPar
Bi\sphinxhyphen{}directional diff. positive
\\
\sphinxhline
\sphinxAtStartPar
\sphinxstylestrong{4}
&
\sphinxAtStartPar
GBE0\_MDI3\_N
&
\sphinxAtStartPar
DSIO
&
\sphinxAtStartPar
Bi\sphinxhyphen{}directional  diff. negative
\\
\sphinxhline
\sphinxAtStartPar
\sphinxstylestrong{5}
&
\sphinxAtStartPar
GBE0\_MDI2\_P
&
\sphinxAtStartPar
DSIO
&
\sphinxAtStartPar
Bi\sphinxhyphen{}directional diff. positive
\\
\sphinxhline
\sphinxAtStartPar
\sphinxstylestrong{6}
&
\sphinxAtStartPar
GBE0\_MDI2\_N
&
\sphinxAtStartPar
DSIO
&
\sphinxAtStartPar
Bi\sphinxhyphen{}directional  diff. negative
\\
\sphinxhline
\sphinxAtStartPar
\sphinxstylestrong{7}
&
\sphinxAtStartPar
GBE0\_MDI1\_P
&
\sphinxAtStartPar
DSIO
&
\sphinxAtStartPar
Bi\sphinxhyphen{}directional diff. positive
\\
\sphinxhline
\sphinxAtStartPar
\sphinxstylestrong{8}
&
\sphinxAtStartPar
GBE0\_MDI1\_N
&
\sphinxAtStartPar
DSIO
&
\sphinxAtStartPar
Bi\sphinxhyphen{}directional  diff. negative
\\
\sphinxhline
\sphinxAtStartPar
\sphinxstylestrong{9}
&
\sphinxAtStartPar
GBE0\_MDI0\_P
&
\sphinxAtStartPar
DSIO
&
\sphinxAtStartPar
Bi\sphinxhyphen{}directional diff. positive
\\
\sphinxhline
\sphinxAtStartPar
\sphinxstylestrong{10}
&
\sphinxAtStartPar
GBE0\_MDI0\_N
&
\sphinxAtStartPar
DSIO
&
\sphinxAtStartPar
Bi\sphinxhyphen{}directional  diff. negative
\\
\sphinxhline
\sphinxAtStartPar
\sphinxstylestrong{11}
&
\sphinxAtStartPar
GBE0\_LINK\_ACT\#
&
\sphinxAtStartPar
IO
&
\sphinxAtStartPar
Activity  LED Cathode
\\
\sphinxhline
\sphinxAtStartPar
\sphinxstylestrong{12}
&
\sphinxAtStartPar
GBE0\_3V3
&
\sphinxAtStartPar
I
&
\sphinxAtStartPar
Activity  LED Anode
\\
\sphinxhline
\sphinxAtStartPar
\sphinxstylestrong{13}
&
\sphinxAtStartPar
GBE0\_LINK100\#
&
\sphinxAtStartPar
IO
&
\sphinxAtStartPar
Link  100 LED
\\
\sphinxhline
\sphinxAtStartPar
\sphinxstylestrong{14}
&
\sphinxAtStartPar
GBE0\_LINK1000\#
&
\sphinxAtStartPar
IO
&
\sphinxAtStartPar
Link  1000 LED
\\
\sphinxhline
\sphinxAtStartPar
\sphinxstylestrong{SH1}
&
\sphinxAtStartPar
PE
&
\sphinxAtStartPar
P
&
\sphinxAtStartPar
SHIELD  pin reference
\\
\sphinxhline
\sphinxAtStartPar
\sphinxstylestrong{SH2}
&
\sphinxAtStartPar
PE
&
\sphinxAtStartPar
P
&
\sphinxAtStartPar
SHIELD  pin reference
\\
\sphinxbottomrule
\end{tabulary}
\sphinxtableafterendhook\par
\sphinxattableend\end{savenotes}

\sphinxAtStartPar
Table 9.9\sphinxhyphen{}2 P10 Interface Pin Function Description


\begin{savenotes}\sphinxattablestart
\sphinxthistablewithglobalstyle
\centering
\begin{tabulary}{\linewidth}[t]{TTTT}
\sphinxtoprule
\sphinxstyletheadfamily 
\sphinxAtStartPar
\sphinxstylestrong{Pin}
&\sphinxstyletheadfamily 
\sphinxAtStartPar
\sphinxstylestrong{Development Board Signal}
&\sphinxstyletheadfamily 
\sphinxAtStartPar
\sphinxstylestrong{Type}
&\sphinxstyletheadfamily 
\sphinxAtStartPar
\sphinxstylestrong{Description}
\\
\sphinxmidrule
\sphinxtableatstartofbodyhook
\sphinxAtStartPar
\sphinxstylestrong{1}
&
\sphinxAtStartPar
GND
&
\sphinxAtStartPar
P
&
\sphinxAtStartPar
Ground
\\
\sphinxhline
\sphinxAtStartPar
\sphinxstylestrong{2}
&
\sphinxAtStartPar
NC
&
\sphinxAtStartPar
| \textbackslash{}
&
\sphinxAtStartPar

\\
\sphinxhline
\sphinxAtStartPar
\sphinxstylestrong{3}
&
\sphinxAtStartPar
GBE1\_MDI3\_P
&
\sphinxAtStartPar
DSIO
&
\sphinxAtStartPar
Bi\sphinxhyphen{}directional diff. positive
\\
\sphinxhline
\sphinxAtStartPar
\sphinxstylestrong{4}
&
\sphinxAtStartPar
GBE1\_MDI3\_N
&
\sphinxAtStartPar
DSIO
&
\sphinxAtStartPar
Bi\sphinxhyphen{}directional  diff. negative
\\
\sphinxhline
\sphinxAtStartPar
\sphinxstylestrong{5}
&
\sphinxAtStartPar
GBE1\_MDI2\_P
&
\sphinxAtStartPar
DSIO
&
\sphinxAtStartPar
Bi\sphinxhyphen{}directional diff. positive
\\
\sphinxhline
\sphinxAtStartPar
\sphinxstylestrong{6}
&
\sphinxAtStartPar
GBE1\_MDI2\_N
&
\sphinxAtStartPar
DSIO
&
\sphinxAtStartPar
Bi\sphinxhyphen{}directional  diff. negative
\\
\sphinxhline
\sphinxAtStartPar
\sphinxstylestrong{7}
&
\sphinxAtStartPar
GBE1\_MDI1\_P
&
\sphinxAtStartPar
DSIO
&
\sphinxAtStartPar
Bi\sphinxhyphen{}directional diff. positive
\\
\sphinxhline
\sphinxAtStartPar
\sphinxstylestrong{8}
&
\sphinxAtStartPar
GBE1\_MDI1\_N
&
\sphinxAtStartPar
DSIO
&
\sphinxAtStartPar
Bi\sphinxhyphen{}directional  diff. negative
\\
\sphinxhline
\sphinxAtStartPar
\sphinxstylestrong{9}
&
\sphinxAtStartPar
GBE1\_MDI0\_P
&
\sphinxAtStartPar
DSIO
&
\sphinxAtStartPar
Bi\sphinxhyphen{}directional diff. positive
\\
\sphinxhline
\sphinxAtStartPar
\sphinxstylestrong{10}
&
\sphinxAtStartPar
GBE1\_MDI0\_N
&
\sphinxAtStartPar
DSIO
&
\sphinxAtStartPar
Bi\sphinxhyphen{}directional  diff. negative
\\
\sphinxhline
\sphinxAtStartPar
\sphinxstylestrong{11}
&
\sphinxAtStartPar
GBE1\_LINK\_ACT\#
&
\sphinxAtStartPar
IO
&
\sphinxAtStartPar
Activity  LED Cathode
\\
\sphinxhline
\sphinxAtStartPar
\sphinxstylestrong{12}
&
\sphinxAtStartPar
GBE1\_3V3
&
\sphinxAtStartPar
I
&
\sphinxAtStartPar
Activity  LED Anode
\\
\sphinxhline
\sphinxAtStartPar
\sphinxstylestrong{13}
&
\sphinxAtStartPar
GBE1\_LINK100\#
&
\sphinxAtStartPar
IO
&
\sphinxAtStartPar
Link  100 LED
\\
\sphinxhline
\sphinxAtStartPar
\sphinxstylestrong{14}
&
\sphinxAtStartPar
GBE1\_LINK1000\#
&
\sphinxAtStartPar
IO
&
\sphinxAtStartPar
Link  1000 LED
\\
\sphinxhline
\sphinxAtStartPar
\sphinxstylestrong{SH1}
&
\sphinxAtStartPar
PE
&
\sphinxAtStartPar
P
&
\sphinxAtStartPar
SHIELD  pin reference
\\
\sphinxhline
\sphinxAtStartPar
\sphinxstylestrong{SH2}
&
\sphinxAtStartPar
PE
&
\sphinxAtStartPar
P
&
\sphinxAtStartPar
SHIELD  pin reference
\\
\sphinxbottomrule
\end{tabulary}
\sphinxtableafterendhook\par
\sphinxattableend\end{savenotes}


\section{9.10 Audio}
\label{\detokenize{hardware:id27}}
\sphinxAtStartPar
The OK \sphinxhyphen{} MX8MPQ \sphinxhyphen{} SMARC development board is equipped with the NAU88C22 24 \sphinxhyphen{} bit stereo audio codec U29. It externally provides a stereo headphone with a built \sphinxhyphen{} in microphone, a microphone, and 2 x 1W 8Ω speaker outputs. These are led out through the XH2.54 white terminal. The corresponding circuit is shown in the figure.

\sphinxAtStartPar
\sphinxincludegraphics{{37fc76fc8fae4686a316924874ff52f5}.png}

\sphinxAtStartPar
Figure 9.10\sphinxhyphen{}1 Audio Circuit Diagram


\subsection{9.10.1 Audio Connector Pin Description}
\label{\detokenize{hardware:audio-connector-pin-description}}
\sphinxAtStartPar
Table 9.10\sphinxhyphen{}1 P31 Interface Pin Function Description


\begin{savenotes}\sphinxattablestart
\sphinxthistablewithglobalstyle
\centering
\begin{tabulary}{\linewidth}[t]{TTTT}
\sphinxtoprule
\sphinxstyletheadfamily 
\sphinxAtStartPar
\sphinxstylestrong{Pin}
&\sphinxstyletheadfamily 
\sphinxAtStartPar
\sphinxstylestrong{Development Board Signal}
&\sphinxstyletheadfamily 
\sphinxAtStartPar
\sphinxstylestrong{Type}
&\sphinxstyletheadfamily 
\sphinxAtStartPar
\sphinxstylestrong{Description}
\\
\sphinxmidrule
\sphinxtableatstartofbodyhook
\sphinxAtStartPar
\sphinxstylestrong{1}
&
\sphinxAtStartPar
LMIC\_P
&
\sphinxAtStartPar
I
&
\sphinxAtStartPar
Microphone  Data
\\
\sphinxhline
\sphinxAtStartPar
\sphinxstylestrong{2}
&
\sphinxAtStartPar
GND
&
\sphinxAtStartPar
P
&
\sphinxAtStartPar
Digital  Ground
\\
\sphinxhline
\sphinxAtStartPar
\sphinxstylestrong{3}
&
\sphinxAtStartPar
HP\_R
&
\sphinxAtStartPar
O
&
\sphinxAtStartPar
Headphone  out Right
\\
\sphinxhline
\sphinxAtStartPar
\sphinxstylestrong{4}
&
\sphinxAtStartPar
| | \textbackslash{}
&
\sphinxAtStartPar

&
\sphinxAtStartPar

\\
\sphinxhline
\sphinxAtStartPar
\sphinxstylestrong{5}
&
\sphinxAtStartPar
HP\_L
&
\sphinxAtStartPar
O
&
\sphinxAtStartPar
Headphone  out Left
\\
\sphinxhline
\sphinxAtStartPar
\sphinxstylestrong{6}
&
\sphinxAtStartPar
| | \textbackslash{}
&
\sphinxAtStartPar

&
\sphinxAtStartPar

\\
\sphinxhline
\sphinxAtStartPar
\sphinxstylestrong{7}
&
\sphinxAtStartPar
GPIO\_P03\_CODEC\_DET
&
\sphinxAtStartPar
I
&
\sphinxAtStartPar
Hot  swap detection
\\
\sphinxhline
\sphinxAtStartPar
\sphinxstylestrong{8}
&
\sphinxAtStartPar
| | \textbackslash{}
&
\sphinxAtStartPar

&
\sphinxAtStartPar

\\
\sphinxbottomrule
\end{tabulary}
\sphinxtableafterendhook\par
\sphinxattableend\end{savenotes}

\sphinxAtStartPar
Table 9.10\sphinxhyphen{}2 P28 Interface Pin Function Description


\begin{savenotes}\sphinxattablestart
\sphinxthistablewithglobalstyle
\centering
\begin{tabulary}{\linewidth}[t]{TTTT}
\sphinxtoprule
\sphinxstyletheadfamily 
\sphinxAtStartPar
\sphinxstylestrong{Pin}
&\sphinxstyletheadfamily 
\sphinxAtStartPar
\sphinxstylestrong{Development Board Signal}
&\sphinxstyletheadfamily 
\sphinxAtStartPar
\sphinxstylestrong{Type}
&\sphinxstyletheadfamily 
\sphinxAtStartPar
\sphinxstylestrong{Description}
\\
\sphinxmidrule
\sphinxtableatstartofbodyhook
\sphinxAtStartPar
\sphinxstylestrong{1}
&
\sphinxAtStartPar
LSPKOUT
&
\sphinxAtStartPar
O
&
\sphinxAtStartPar
Speaker  left channel output
\\
\sphinxhline
\sphinxAtStartPar
\sphinxstylestrong{2}
&
\sphinxAtStartPar
RSPKOUT
&
\sphinxAtStartPar
O
&
\sphinxAtStartPar
Speaker right channel output
\\
\sphinxbottomrule
\end{tabulary}
\sphinxtableafterendhook\par
\sphinxattableend\end{savenotes}

\sphinxAtStartPar
Table 9.10\sphinxhyphen{}3 P29 Interface Pin Function Description


\begin{savenotes}\sphinxattablestart
\sphinxthistablewithglobalstyle
\centering
\begin{tabulary}{\linewidth}[t]{TTTT}
\sphinxtoprule
\sphinxstyletheadfamily 
\sphinxAtStartPar
\sphinxstylestrong{Pin}
&\sphinxstyletheadfamily 
\sphinxAtStartPar
\sphinxstylestrong{Development Board Signal}
&\sphinxstyletheadfamily 
\sphinxAtStartPar
\sphinxstylestrong{Type}
&\sphinxstyletheadfamily 
\sphinxAtStartPar
\sphinxstylestrong{Description}
\\
\sphinxmidrule
\sphinxtableatstartofbodyhook
\sphinxAtStartPar
\sphinxstylestrong{1}
&
\sphinxAtStartPar
LSPKOUT
&
\sphinxAtStartPar
O
&
\sphinxAtStartPar
Speaker  left channel output
\\
\sphinxhline
\sphinxAtStartPar
\sphinxstylestrong{2}
&
\sphinxAtStartPar
GND
&
\sphinxAtStartPar
P
&
\sphinxAtStartPar
Digital  Ground
\\
\sphinxhline
\sphinxAtStartPar
\sphinxstylestrong{3}
&
\sphinxAtStartPar
RSPKOUT
&
\sphinxAtStartPar
O
&
\sphinxAtStartPar
Speaker right channel output
\\
\sphinxhline
\sphinxAtStartPar
\sphinxstylestrong{4}
&
\sphinxAtStartPar
GND
&
\sphinxAtStartPar
P
&
\sphinxAtStartPar
Digital  Ground
\\
\sphinxbottomrule
\end{tabulary}
\sphinxtableafterendhook\par
\sphinxattableend\end{savenotes}


\section{9.11 RTC}
\label{\detokenize{hardware:id28}}
\sphinxAtStartPar
In accordance with the SMARC specification, the development board is only required to provide the RTC backup power supply, VDD\_RTC. Both the RTC chip and the crystal oscillator are installed on the SoM.

\sphinxAtStartPar
\sphinxincludegraphics{{125d08b104444be686a16c15315b5c9b}.png}

\sphinxAtStartPar
Figure 9.11\sphinxhyphen{}1: CR2032 Battery Holder


\section{9.12 LVDS}
\label{\detokenize{hardware:lvds}}
\sphinxAtStartPar
The OK\sphinxhyphen{}MX8MPQ\sphinxhyphen{}SMARC development board supports dual LVDS outputs routed to 2.0mm pitch dual\sphinxhyphen{}row headers, compatible with the Forlinx 10.1\sphinxhyphen{}inch LVDS display, and supports screen brightness adjustment and capacitive touch.

\sphinxAtStartPar
1 x is a dedicated 4\sphinxhyphen{}lane LVDS1 interface, as shown in the figure.

\sphinxAtStartPar
The other 4\sphinxhyphen{}lane LVDS0 channel shares the data path with DSI0 and switches between them via a Switch chip (see Section 11.13).

\sphinxAtStartPar
\sphinxincludegraphics{{a0756013fcc54cc58f24aceee1e08c27}.png}

\sphinxAtStartPar
Figure 9.12\sphinxhyphen{}1 Independent 4\sphinxhyphen{}lane LVDS Connector


\subsection{9.12.1 LVDS Connector Pin Description}
\label{\detokenize{hardware:lvds-connector-pin-description}}
\sphinxAtStartPar
Table 9.12\sphinxhyphen{}1 P15 Interface Pin Function Description


\begin{savenotes}
\sphinxatlongtablestart
\sphinxthistablewithglobalstyle
\makeatletter
  \LTleft \@totalleftmargin plus1fill
  \LTright\dimexpr\columnwidth-\@totalleftmargin-\linewidth\relax plus1fill
\makeatother
\begin{longtable}{llll}
\sphinxtoprule
\sphinxstyletheadfamily 
\sphinxAtStartPar
\sphinxstylestrong{Pin}
&\sphinxstyletheadfamily 
\sphinxAtStartPar
\sphinxstylestrong{Development Board Signal}
&\sphinxstyletheadfamily 
\sphinxAtStartPar
\sphinxstylestrong{Type}
&\sphinxstyletheadfamily 
\sphinxAtStartPar
\sphinxstylestrong{Description}
\\
\sphinxmidrule
\endfirsthead

\multicolumn{4}{c}{\sphinxnorowcolor
    \makebox[0pt]{\sphinxtablecontinued{\tablename\ \thetable{} \textendash{} continued from previous page}}%
}\\
\sphinxtoprule
\sphinxstyletheadfamily 
\sphinxAtStartPar
\sphinxstylestrong{Pin}
&\sphinxstyletheadfamily 
\sphinxAtStartPar
\sphinxstylestrong{Development Board Signal}
&\sphinxstyletheadfamily 
\sphinxAtStartPar
\sphinxstylestrong{Type}
&\sphinxstyletheadfamily 
\sphinxAtStartPar
\sphinxstylestrong{Description}
\\
\sphinxmidrule
\endhead

\sphinxbottomrule
\multicolumn{4}{r}{\sphinxnorowcolor
    \makebox[0pt][r]{\sphinxtablecontinued{continues on next page}}%
}\\
\endfoot

\endlastfoot
\sphinxtableatstartofbodyhook

\sphinxAtStartPar
\sphinxstylestrong{1}
&
\sphinxAtStartPar
LVDS1\_5V
&
\sphinxAtStartPar
P
&
\sphinxAtStartPar
Display  power 5V
\\
\sphinxhline
\sphinxAtStartPar
\sphinxstylestrong{2}
&
\sphinxAtStartPar
LVDS1\_5V
&
\sphinxAtStartPar
P
&
\sphinxAtStartPar
Display  power 5V
\\
\sphinxhline
\sphinxAtStartPar
\sphinxstylestrong{3}
&
\sphinxAtStartPar
LVDS1\_5V
&
\sphinxAtStartPar
P
&
\sphinxAtStartPar
Display  power 5V
\\
\sphinxhline
\sphinxAtStartPar
\sphinxstylestrong{4}
&
\sphinxAtStartPar
GND
&
\sphinxAtStartPar
P
&
\sphinxAtStartPar
Digital  Ground
\\
\sphinxhline
\sphinxAtStartPar
\sphinxstylestrong{5}
&
\sphinxAtStartPar
GND
&
\sphinxAtStartPar
P
&
\sphinxAtStartPar
Digital  Ground
\\
\sphinxhline
\sphinxAtStartPar
\sphinxstylestrong{6}
&
\sphinxAtStartPar
GND
&
\sphinxAtStartPar
P
&
\sphinxAtStartPar
Digital  Ground
\\
\sphinxhline
\sphinxAtStartPar
\sphinxstylestrong{7}
&
\sphinxAtStartPar
LVDS1\_TX0\_N
&
\sphinxAtStartPar
DSO
&
\sphinxAtStartPar
LVDS1  Data0 Diff. Negative
\\
\sphinxhline
\sphinxAtStartPar
\sphinxstylestrong{8}
&
\sphinxAtStartPar
LVDS1\_TX0\_P
&
\sphinxAtStartPar
DSO
&
\sphinxAtStartPar
LVDS1  Data0 Diff. Positive
\\
\sphinxhline
\sphinxAtStartPar
\sphinxstylestrong{9}
&
\sphinxAtStartPar
LVDS1\_TX1\_N
&
\sphinxAtStartPar
DSO
&
\sphinxAtStartPar
LVDS1  Data1 Diff. Negative
\\
\sphinxhline
\sphinxAtStartPar
\sphinxstylestrong{10}
&
\sphinxAtStartPar
LVDS1\_TX1\_P
&
\sphinxAtStartPar
DSO
&
\sphinxAtStartPar
LVDS1  Data1 Diff. Positive
\\
\sphinxhline
\sphinxAtStartPar
\sphinxstylestrong{11}
&
\sphinxAtStartPar
LVDS1\_TX2\_N
&
\sphinxAtStartPar
DSO
&
\sphinxAtStartPar
LVDS1  Data2 Diff. Negative
\\
\sphinxhline
\sphinxAtStartPar
\sphinxstylestrong{12}
&
\sphinxAtStartPar
LVDS1\_TX2\_P
&
\sphinxAtStartPar
DSO
&
\sphinxAtStartPar
LVDS1  Data2 Diff. Positive
\\
\sphinxhline
\sphinxAtStartPar
\sphinxstylestrong{13}
&
\sphinxAtStartPar
GND
&
\sphinxAtStartPar
P
&
\sphinxAtStartPar
Digital  Ground
\\
\sphinxhline
\sphinxAtStartPar
\sphinxstylestrong{14}
&
\sphinxAtStartPar
GND
&
\sphinxAtStartPar
P
&
\sphinxAtStartPar
Digital  Ground
\\
\sphinxhline
\sphinxAtStartPar
\sphinxstylestrong{15}
&
\sphinxAtStartPar
LVDS1\_CLK\_N
&
\sphinxAtStartPar
DSO
&
\sphinxAtStartPar
LVDS1  CLK Diff. Negative
\\
\sphinxhline
\sphinxAtStartPar
\sphinxstylestrong{16}
&
\sphinxAtStartPar
LVDS1\_CLK\_P
&
\sphinxAtStartPar
DSO
&
\sphinxAtStartPar
LVDS1  CLK Diff. Positive
\\
\sphinxhline
\sphinxAtStartPar
\sphinxstylestrong{17}
&
\sphinxAtStartPar
LVDS1\_TX3\_N
&
\sphinxAtStartPar
DSO
&
\sphinxAtStartPar
LVDS1  Data3 Diff. Negative
\\
\sphinxhline
\sphinxAtStartPar
\sphinxstylestrong{18}
&
\sphinxAtStartPar
LVDS1\_TX3\_P
&
\sphinxAtStartPar
DSO
&
\sphinxAtStartPar
LVDS1  Data3 Diff. Positive
\\
\sphinxhline
\sphinxAtStartPar
\sphinxstylestrong{19}
&
\sphinxAtStartPar
| | \textbackslash{}
&
\sphinxAtStartPar

&
\sphinxAtStartPar

\\
\sphinxhline
\sphinxAtStartPar
\sphinxstylestrong{20}
&
\sphinxAtStartPar
| | \textbackslash{}
&
\sphinxAtStartPar

&
\sphinxAtStartPar

\\
\sphinxhline
\sphinxAtStartPar
\sphinxstylestrong{21}
&
\sphinxAtStartPar
| | \textbackslash{}
&
\sphinxAtStartPar

&
\sphinxAtStartPar

\\
\sphinxhline
\sphinxAtStartPar
\sphinxstylestrong{22}
&
\sphinxAtStartPar
| | \textbackslash{}
&
\sphinxAtStartPar

&
\sphinxAtStartPar

\\
\sphinxhline
\sphinxAtStartPar
\sphinxstylestrong{23}
&
\sphinxAtStartPar
| | \textbackslash{}
&
\sphinxAtStartPar

&
\sphinxAtStartPar

\\
\sphinxhline
\sphinxAtStartPar
\sphinxstylestrong{24}
&
\sphinxAtStartPar
| | \textbackslash{}
&
\sphinxAtStartPar

&
\sphinxAtStartPar

\\
\sphinxhline
\sphinxAtStartPar
\sphinxstylestrong{25}
&
\sphinxAtStartPar
GND
&
\sphinxAtStartPar
P
&
\sphinxAtStartPar
Digital  Ground
\\
\sphinxhline
\sphinxAtStartPar
\sphinxstylestrong{26}
&
\sphinxAtStartPar
GND
&
\sphinxAtStartPar
P
&
\sphinxAtStartPar
Digital  Ground
\\
\sphinxhline
\sphinxAtStartPar
\sphinxstylestrong{27}
&
\sphinxAtStartPar
| | \textbackslash{}
&
\sphinxAtStartPar

&
\sphinxAtStartPar

\\
\sphinxhline
\sphinxAtStartPar
\sphinxstylestrong{28}
&
\sphinxAtStartPar
| | \textbackslash{}
&
\sphinxAtStartPar

&
\sphinxAtStartPar

\\
\sphinxhline
\sphinxAtStartPar
\sphinxstylestrong{29}
&
\sphinxAtStartPar
| | \textbackslash{}
&
\sphinxAtStartPar

&
\sphinxAtStartPar

\\
\sphinxhline
\sphinxAtStartPar
\sphinxstylestrong{30}
&
\sphinxAtStartPar
| | \textbackslash{}
&
\sphinxAtStartPar

&
\sphinxAtStartPar

\\
\sphinxhline
\sphinxAtStartPar
\sphinxstylestrong{31}
&
\sphinxAtStartPar
| | \textbackslash{}
&
\sphinxAtStartPar

&
\sphinxAtStartPar

\\
\sphinxhline
\sphinxAtStartPar
\sphinxstylestrong{32}
&
\sphinxAtStartPar
| | \textbackslash{}
&
\sphinxAtStartPar

&
\sphinxAtStartPar

\\
\sphinxhline
\sphinxAtStartPar
\sphinxstylestrong{33}
&
\sphinxAtStartPar
LCD1\_BKLT\_PWM\_3.3V
&
\sphinxAtStartPar
O
&
\sphinxAtStartPar
Backlight  Brightness Control
\\
\sphinxhline
\sphinxAtStartPar
\sphinxstylestrong{34}
&
\sphinxAtStartPar
GPIO\_P05\_LVDS1\_INT
&
\sphinxAtStartPar
I
&
\sphinxAtStartPar
LVDS  Touch Interrupt, active low
\\
\sphinxhline
\sphinxAtStartPar
\sphinxstylestrong{35}
&
\sphinxAtStartPar
GPIO\_P12\_LVDS1\_RST
&
\sphinxAtStartPar
O
&
\sphinxAtStartPar
LVDS Reset,  active low
\\
\sphinxhline
\sphinxAtStartPar
\sphinxstylestrong{36}
&
\sphinxAtStartPar
| | \textbackslash{}
&
\sphinxAtStartPar

&
\sphinxAtStartPar

\\
\sphinxhline
\sphinxAtStartPar
\sphinxstylestrong{37}
&
\sphinxAtStartPar
I2C\_GP\_DAT\_3.3V
&
\sphinxAtStartPar
IO
&
\sphinxAtStartPar
I2C  Data
\\
\sphinxhline
\sphinxAtStartPar
\sphinxstylestrong{38}
&
\sphinxAtStartPar
I2C\_GP\_CK\_3.3V
&
\sphinxAtStartPar
O
&
\sphinxAtStartPar
I2C  Clock
\\
\sphinxbottomrule
\end{longtable}
\sphinxtableafterendhook
\sphinxatlongtableend
\end{savenotes}


\section{9.13 MIPI DSI \& LVDS}
\label{\detokenize{hardware:mipi-dsi-lvds}}
\sphinxAtStartPar
According to the SMARC protocol specification, DSI0 and LVDS0 share a set of channels and can only use one peripheral at a time. In order to fully verify the peripheral performance of the CPU, the OK\sphinxhyphen{}MX8 MPQ\sphinxhyphen{}SMARC development board separates the DSI0 signal from the LVDS0 signal through a Switch chip U18, which can be used to verify their respective peripherals.

\sphinxAtStartPar
\sphinxstylestrong{Note: For more information on the SMARC protocol, please refer to”Smart  Mobility  ARChitecture  Hardware  Specification”}

\sphinxAtStartPar
\sphinxincludegraphics{{95cbf5f349ac43ef86fa7f39010b6029}.png}

\sphinxAtStartPar
Figure 9.13\sphinxhyphen{}1 SMARC Hardware Specification Interface Description

\sphinxAtStartPar
\sphinxincludegraphics{{6161e818e17e42398fa114f94d304d6c}.png}

\sphinxAtStartPar
Figure 9.13\sphinxhyphen{}2 LVDS0 and DSI0 Data Channel Switching Chip

\sphinxAtStartPar
The separated LVDS0 differential signal group is connected to a 2.0mm spacing dual pin, compatible with Forlinx 10.1\sphinxhyphen{}inch LVDS screen, as shown in the figure.

\sphinxAtStartPar
\sphinxincludegraphics{{86fe2a1aae874941af906eae7e49815c}.png}

\sphinxAtStartPar
Figure 9.13\sphinxhyphen{}3 Separated 4\sphinxhyphen{}lane LVDS0 Connector

\sphinxAtStartPar
The separated DSI0 differential signal group is connected to the FPC seat to adapt to the 7\sphinxhyphen{}inch MIPI screen of Forlinx, as shown in the figure.

\sphinxAtStartPar
\sphinxincludegraphics{{c3705b22bab74340a3022319dca4eb1d}.png}

\sphinxAtStartPar
Figure 9.13\sphinxhyphen{}4 Separated 4\sphinxhyphen{}lane DS0 Connector


\subsection{9.13.1 MIPI DSI \& LVDS Connector Pin Description}
\label{\detokenize{hardware:mipi-dsi-lvds-connector-pin-description}}
\sphinxAtStartPar
Table 9.13\sphinxhyphen{}1 P16 LVDS0 Interface Pin Function Description


\begin{savenotes}
\sphinxatlongtablestart
\sphinxthistablewithglobalstyle
\makeatletter
  \LTleft \@totalleftmargin plus1fill
  \LTright\dimexpr\columnwidth-\@totalleftmargin-\linewidth\relax plus1fill
\makeatother
\begin{longtable}{llll}
\sphinxtoprule
\sphinxstyletheadfamily 
\sphinxAtStartPar
\sphinxstylestrong{Pin}
&\sphinxstyletheadfamily 
\sphinxAtStartPar
\sphinxstylestrong{Development Board Signal}
&\sphinxstyletheadfamily 
\sphinxAtStartPar
\sphinxstylestrong{Type}
&\sphinxstyletheadfamily 
\sphinxAtStartPar
\sphinxstylestrong{Description}
\\
\sphinxmidrule
\endfirsthead

\multicolumn{4}{c}{\sphinxnorowcolor
    \makebox[0pt]{\sphinxtablecontinued{\tablename\ \thetable{} \textendash{} continued from previous page}}%
}\\
\sphinxtoprule
\sphinxstyletheadfamily 
\sphinxAtStartPar
\sphinxstylestrong{Pin}
&\sphinxstyletheadfamily 
\sphinxAtStartPar
\sphinxstylestrong{Development Board Signal}
&\sphinxstyletheadfamily 
\sphinxAtStartPar
\sphinxstylestrong{Type}
&\sphinxstyletheadfamily 
\sphinxAtStartPar
\sphinxstylestrong{Description}
\\
\sphinxmidrule
\endhead

\sphinxbottomrule
\multicolumn{4}{r}{\sphinxnorowcolor
    \makebox[0pt][r]{\sphinxtablecontinued{continues on next page}}%
}\\
\endfoot

\endlastfoot
\sphinxtableatstartofbodyhook

\sphinxAtStartPar
\sphinxstylestrong{1}
&
\sphinxAtStartPar
LVDS0\_5V
&
\sphinxAtStartPar
P
&
\sphinxAtStartPar
Display  power 5V
\\
\sphinxhline
\sphinxAtStartPar
\sphinxstylestrong{2}
&
\sphinxAtStartPar
LVDS0\_5V
&
\sphinxAtStartPar
P
&
\sphinxAtStartPar
Display  power 5V
\\
\sphinxhline
\sphinxAtStartPar
\sphinxstylestrong{3}
&
\sphinxAtStartPar
LVDS0\_5V
&
\sphinxAtStartPar
P
&
\sphinxAtStartPar
Display  power 5V
\\
\sphinxhline
\sphinxAtStartPar
\sphinxstylestrong{4}
&
\sphinxAtStartPar
GND
&
\sphinxAtStartPar
P
&
\sphinxAtStartPar
Digital  Ground
\\
\sphinxhline
\sphinxAtStartPar
\sphinxstylestrong{5}
&
\sphinxAtStartPar
GND
&
\sphinxAtStartPar
P
&
\sphinxAtStartPar
Digital  Ground
\\
\sphinxhline
\sphinxAtStartPar
\sphinxstylestrong{6}
&
\sphinxAtStartPar
GND
&
\sphinxAtStartPar
P
&
\sphinxAtStartPar
Digital  Ground
\\
\sphinxhline
\sphinxAtStartPar
\sphinxstylestrong{7}
&
\sphinxAtStartPar
LVDS0\_TX0\_N
&
\sphinxAtStartPar
DSO
&
\sphinxAtStartPar
LVDS0  Data0 Diff. Negative
\\
\sphinxhline
\sphinxAtStartPar
\sphinxstylestrong{8}
&
\sphinxAtStartPar
LVDS0\_TX0\_P
&
\sphinxAtStartPar
DSO
&
\sphinxAtStartPar
LVDS0  Data0 Diff. Positive
\\
\sphinxhline
\sphinxAtStartPar
\sphinxstylestrong{9}
&
\sphinxAtStartPar
LVDS0\_TX1\_N
&
\sphinxAtStartPar
DSO
&
\sphinxAtStartPar
LVDS0  Data1 Diff. Negative
\\
\sphinxhline
\sphinxAtStartPar
\sphinxstylestrong{10}
&
\sphinxAtStartPar
LVDS0\_TX1\_P
&
\sphinxAtStartPar
DSO
&
\sphinxAtStartPar
LVDS0  Data1 Diff. Positive
\\
\sphinxhline
\sphinxAtStartPar
\sphinxstylestrong{11}
&
\sphinxAtStartPar
LVDS0\_TX2\_N
&
\sphinxAtStartPar
DSO
&
\sphinxAtStartPar
LVDS0  Data2 Diff. Negative
\\
\sphinxhline
\sphinxAtStartPar
\sphinxstylestrong{12}
&
\sphinxAtStartPar
LVDS0\_TX2\_P
&
\sphinxAtStartPar
DSO
&
\sphinxAtStartPar
LVDS0  Data2 Diff. Positive
\\
\sphinxhline
\sphinxAtStartPar
\sphinxstylestrong{13}
&
\sphinxAtStartPar
GND
&
\sphinxAtStartPar
P
&
\sphinxAtStartPar
Digital  Ground
\\
\sphinxhline
\sphinxAtStartPar
\sphinxstylestrong{14}
&
\sphinxAtStartPar
GND
&
\sphinxAtStartPar
P
&
\sphinxAtStartPar
Digital  Ground
\\
\sphinxhline
\sphinxAtStartPar
\sphinxstylestrong{15}
&
\sphinxAtStartPar
LVDS0\_CLK\_N
&
\sphinxAtStartPar
DSO
&
\sphinxAtStartPar
LVDS0  CLK Diff. Negative
\\
\sphinxhline
\sphinxAtStartPar
\sphinxstylestrong{16}
&
\sphinxAtStartPar
LVDS0\_CLK\_P
&
\sphinxAtStartPar
DSO
&
\sphinxAtStartPar
LVDS0  CLK Diff. Positive
\\
\sphinxhline
\sphinxAtStartPar
\sphinxstylestrong{17}
&
\sphinxAtStartPar
LVDS0\_TX3\_N
&
\sphinxAtStartPar
DSO
&
\sphinxAtStartPar
LVDS0  Data3 Diff. Negative
\\
\sphinxhline
\sphinxAtStartPar
\sphinxstylestrong{18}
&
\sphinxAtStartPar
LVDS0\_TX3\_P
&
\sphinxAtStartPar
DSO
&
\sphinxAtStartPar
LVDS0  Data3 Diff. Positive
\\
\sphinxhline
\sphinxAtStartPar
\sphinxstylestrong{19}
&
\sphinxAtStartPar
| | \textbackslash{}
&
\sphinxAtStartPar

&
\sphinxAtStartPar

\\
\sphinxhline
\sphinxAtStartPar
\sphinxstylestrong{20}
&
\sphinxAtStartPar
| | \textbackslash{}
&
\sphinxAtStartPar

&
\sphinxAtStartPar

\\
\sphinxhline
\sphinxAtStartPar
\sphinxstylestrong{21}
&
\sphinxAtStartPar
| | \textbackslash{}
&
\sphinxAtStartPar

&
\sphinxAtStartPar

\\
\sphinxhline
\sphinxAtStartPar
\sphinxstylestrong{22}
&
\sphinxAtStartPar
| | \textbackslash{}
&
\sphinxAtStartPar

&
\sphinxAtStartPar

\\
\sphinxhline
\sphinxAtStartPar
\sphinxstylestrong{23}
&
\sphinxAtStartPar
| | \textbackslash{}
&
\sphinxAtStartPar

&
\sphinxAtStartPar

\\
\sphinxhline
\sphinxAtStartPar
\sphinxstylestrong{24}
&
\sphinxAtStartPar
| | \textbackslash{}
&
\sphinxAtStartPar

&
\sphinxAtStartPar

\\
\sphinxhline
\sphinxAtStartPar
\sphinxstylestrong{25}
&
\sphinxAtStartPar
GND
&
\sphinxAtStartPar
P
&
\sphinxAtStartPar
Digital  Ground
\\
\sphinxhline
\sphinxAtStartPar
\sphinxstylestrong{26}
&
\sphinxAtStartPar
GND
&
\sphinxAtStartPar
P
&
\sphinxAtStartPar
Digital  Ground
\\
\sphinxhline
\sphinxAtStartPar
\sphinxstylestrong{27}
&
\sphinxAtStartPar
| | \textbackslash{}
&
\sphinxAtStartPar

&
\sphinxAtStartPar

\\
\sphinxhline
\sphinxAtStartPar
\sphinxstylestrong{28}
&
\sphinxAtStartPar
| | \textbackslash{}
&
\sphinxAtStartPar

&
\sphinxAtStartPar

\\
\sphinxhline
\sphinxAtStartPar
\sphinxstylestrong{29}
&
\sphinxAtStartPar
| | \textbackslash{}
&
\sphinxAtStartPar

&
\sphinxAtStartPar

\\
\sphinxhline
\sphinxAtStartPar
\sphinxstylestrong{30}
&
\sphinxAtStartPar
| | \textbackslash{}
&
\sphinxAtStartPar

&
\sphinxAtStartPar

\\
\sphinxhline
\sphinxAtStartPar
\sphinxstylestrong{31}
&
\sphinxAtStartPar
| | \textbackslash{}
&
\sphinxAtStartPar

&
\sphinxAtStartPar

\\
\sphinxhline
\sphinxAtStartPar
\sphinxstylestrong{32}
&
\sphinxAtStartPar
| | \textbackslash{}
&
\sphinxAtStartPar

&
\sphinxAtStartPar

\\
\sphinxhline
\sphinxAtStartPar
\sphinxstylestrong{33}
&
\sphinxAtStartPar
LCD1\_BKLT\_PWM\_3.3V
&
\sphinxAtStartPar
O
&
\sphinxAtStartPar
Backlight  Brightness Control
\\
\sphinxhline
\sphinxAtStartPar
\sphinxstylestrong{34}
&
\sphinxAtStartPar
GPIO\_P05\_LVDS0\_INT
&
\sphinxAtStartPar
I
&
\sphinxAtStartPar
LVDS  Touch Interrupt, active low
\\
\sphinxhline
\sphinxAtStartPar
\sphinxstylestrong{35}
&
\sphinxAtStartPar
GPIO\_P12\_LVDS0\_RST
&
\sphinxAtStartPar
O
&
\sphinxAtStartPar
LVDS Reset,  active low
\\
\sphinxhline
\sphinxAtStartPar
\sphinxstylestrong{36}
&
\sphinxAtStartPar
| | \textbackslash{}
&
\sphinxAtStartPar

&
\sphinxAtStartPar

\\
\sphinxhline
\sphinxAtStartPar
\sphinxstylestrong{37}
&
\sphinxAtStartPar
I2C\_GP\_DAT\_3.3V
&
\sphinxAtStartPar
IO
&
\sphinxAtStartPar
I2C  Data
\\
\sphinxhline
\sphinxAtStartPar
\sphinxstylestrong{38}
&
\sphinxAtStartPar
I2C\_GP\_CK\_3.3V
&
\sphinxAtStartPar
O
&
\sphinxAtStartPar
I2C  Clock
\\
\sphinxbottomrule
\end{longtable}
\sphinxtableafterendhook
\sphinxatlongtableend
\end{savenotes}

\sphinxAtStartPar
Table 9.13\sphinxhyphen{}2 P17 MIPI DSI0 Interface Pin Function Description


\begin{savenotes}
\sphinxatlongtablestart
\sphinxthistablewithglobalstyle
\makeatletter
  \LTleft \@totalleftmargin plus1fill
  \LTright\dimexpr\columnwidth-\@totalleftmargin-\linewidth\relax plus1fill
\makeatother
\begin{longtable}{llll}
\sphinxtoprule
\sphinxstyletheadfamily 
\sphinxAtStartPar
\sphinxstylestrong{Pin}
&\sphinxstyletheadfamily 
\sphinxAtStartPar
\sphinxstylestrong{Development Board Signal}
&\sphinxstyletheadfamily 
\sphinxAtStartPar
\sphinxstylestrong{Type}
&\sphinxstyletheadfamily 
\sphinxAtStartPar
\sphinxstylestrong{Description}
\\
\sphinxmidrule
\endfirsthead

\multicolumn{4}{c}{\sphinxnorowcolor
    \makebox[0pt]{\sphinxtablecontinued{\tablename\ \thetable{} \textendash{} continued from previous page}}%
}\\
\sphinxtoprule
\sphinxstyletheadfamily 
\sphinxAtStartPar
\sphinxstylestrong{Pin}
&\sphinxstyletheadfamily 
\sphinxAtStartPar
\sphinxstylestrong{Development Board Signal}
&\sphinxstyletheadfamily 
\sphinxAtStartPar
\sphinxstylestrong{Type}
&\sphinxstyletheadfamily 
\sphinxAtStartPar
\sphinxstylestrong{Description}
\\
\sphinxmidrule
\endhead

\sphinxbottomrule
\multicolumn{4}{r}{\sphinxnorowcolor
    \makebox[0pt][r]{\sphinxtablecontinued{continues on next page}}%
}\\
\endfoot

\endlastfoot
\sphinxtableatstartofbodyhook

\sphinxAtStartPar
\sphinxstylestrong{1}
&
\sphinxAtStartPar
VCC\_5V
&
\sphinxAtStartPar
P
&
\sphinxAtStartPar
Display  power 5V
\\
\sphinxhline
\sphinxAtStartPar
\sphinxstylestrong{2}
&
\sphinxAtStartPar
VCC\_5V
&
\sphinxAtStartPar
P
&
\sphinxAtStartPar
Display  power 5V
\\
\sphinxhline
\sphinxAtStartPar
\sphinxstylestrong{3}
&
\sphinxAtStartPar
VCC\_5V
&
\sphinxAtStartPar
P
&
\sphinxAtStartPar
Display  power 5V
\\
\sphinxhline
\sphinxAtStartPar
\sphinxstylestrong{4}
&
\sphinxAtStartPar
VCC\_5V
&
\sphinxAtStartPar
P
&
\sphinxAtStartPar
Display  power 5V
\\
\sphinxhline
\sphinxAtStartPar
\sphinxstylestrong{5}
&
\sphinxAtStartPar
GND
&
\sphinxAtStartPar
P
&
\sphinxAtStartPar
Digital  Ground
\\
\sphinxhline
\sphinxAtStartPar
\sphinxstylestrong{6}
&
\sphinxAtStartPar
GND
&
\sphinxAtStartPar
P
&
\sphinxAtStartPar
Digital  Ground
\\
\sphinxhline
\sphinxAtStartPar
\sphinxstylestrong{7}
&
\sphinxAtStartPar
GND
&
\sphinxAtStartPar
P
&
\sphinxAtStartPar
Digital  Ground
\\
\sphinxhline
\sphinxAtStartPar
\sphinxstylestrong{8}
&
\sphinxAtStartPar
LCD0\_BKLT\_PWM\_3.3V
&
\sphinxAtStartPar
O
&
\sphinxAtStartPar
Backlight  Brightness Control
\\
\sphinxhline
\sphinxAtStartPar
\sphinxstylestrong{9}
&
\sphinxAtStartPar
LCD0\_VDD\_EN\_3.3V
&
\sphinxAtStartPar
O
&
\sphinxAtStartPar
LCD Power  Enable
\\
\sphinxhline
\sphinxAtStartPar
\sphinxstylestrong{10}
&
\sphinxAtStartPar
GND
&
\sphinxAtStartPar
P
&
\sphinxAtStartPar
Digital  Ground
\\
\sphinxhline
\sphinxAtStartPar
\sphinxstylestrong{11}
&
\sphinxAtStartPar
DSI0\_TX3\_P
&
\sphinxAtStartPar
DSO
&
\sphinxAtStartPar
DSI0  Data3 Diff. Positive
\\
\sphinxhline
\sphinxAtStartPar
\sphinxstylestrong{12}
&
\sphinxAtStartPar
DSI0\_TX3\_N
&
\sphinxAtStartPar
DSO
&
\sphinxAtStartPar
DSI0  Data3 Diff. Negative
\\
\sphinxhline
\sphinxAtStartPar
\sphinxstylestrong{13}
&
\sphinxAtStartPar
GND
&
\sphinxAtStartPar
P
&
\sphinxAtStartPar
Digital  Ground
\\
\sphinxhline
\sphinxAtStartPar
\sphinxstylestrong{14}
&
\sphinxAtStartPar
DSI0\_TX2\_P
&
\sphinxAtStartPar
DSO
&
\sphinxAtStartPar
DSI0  Data2 Diff. Positive
\\
\sphinxhline
\sphinxAtStartPar
\sphinxstylestrong{15}
&
\sphinxAtStartPar
DSI0\_TX2\_N
&
\sphinxAtStartPar
DSO
&
\sphinxAtStartPar
DSI0  Data2 Diff. Negative
\\
\sphinxhline
\sphinxAtStartPar
\sphinxstylestrong{16}
&
\sphinxAtStartPar
GND
&
\sphinxAtStartPar
P
&
\sphinxAtStartPar
Digital  Ground
\\
\sphinxhline
\sphinxAtStartPar
\sphinxstylestrong{17}
&
\sphinxAtStartPar
DSI0\_CLK\_P
&
\sphinxAtStartPar
DSO
&
\sphinxAtStartPar
DSI0  CLK Diff. Positive
\\
\sphinxhline
\sphinxAtStartPar
\sphinxstylestrong{18}
&
\sphinxAtStartPar
DSI0\_CLK\_N
&
\sphinxAtStartPar
DSO
&
\sphinxAtStartPar
DSI0  CLK Diff. Negative
\\
\sphinxhline
\sphinxAtStartPar
\sphinxstylestrong{19}
&
\sphinxAtStartPar
GND
&
\sphinxAtStartPar
P
&
\sphinxAtStartPar
Digital  Ground
\\
\sphinxhline
\sphinxAtStartPar
\sphinxstylestrong{20}
&
\sphinxAtStartPar
DSI0\_TX1\_P
&
\sphinxAtStartPar
DSO
&
\sphinxAtStartPar
DSI0  Data1 Diff. Positive
\\
\sphinxhline
\sphinxAtStartPar
\sphinxstylestrong{21}
&
\sphinxAtStartPar
DSI0\_TX1\_N
&
\sphinxAtStartPar
DSO
&
\sphinxAtStartPar
DSI0  Data1 Diff. Negative
\\
\sphinxhline
\sphinxAtStartPar
\sphinxstylestrong{22}
&
\sphinxAtStartPar
GND
&
\sphinxAtStartPar
P
&
\sphinxAtStartPar
Digital  Ground
\\
\sphinxhline
\sphinxAtStartPar
\sphinxstylestrong{23}
&
\sphinxAtStartPar
DSI0\_TX0\_P
&
\sphinxAtStartPar
DSO
&
\sphinxAtStartPar
DSI0  Data0 Diff. Positive
\\
\sphinxhline
\sphinxAtStartPar
\sphinxstylestrong{24}
&
\sphinxAtStartPar
DSI0\_TX0\_N
&
\sphinxAtStartPar
DSO
&
\sphinxAtStartPar
DSI0  Data0 Diff. Negative
\\
\sphinxhline
\sphinxAtStartPar
\sphinxstylestrong{25}
&
\sphinxAtStartPar
GND
&
\sphinxAtStartPar
P
&
\sphinxAtStartPar
Digital  Ground
\\
\sphinxhline
\sphinxAtStartPar
\sphinxstylestrong{26}
&
\sphinxAtStartPar
DSI0\_SCL\_B
&
\sphinxAtStartPar
O
&
\sphinxAtStartPar
I2C  Clock
\\
\sphinxhline
\sphinxAtStartPar
\sphinxstylestrong{27}
&
\sphinxAtStartPar
DSI0\_SDA\_B
&
\sphinxAtStartPar
IO
&
\sphinxAtStartPar
I2C  Data
\\
\sphinxhline
\sphinxAtStartPar
\sphinxstylestrong{28}
&
\sphinxAtStartPar
GND
&
\sphinxAtStartPar
P
&
\sphinxAtStartPar
Digital  Ground
\\
\sphinxhline
\sphinxAtStartPar
\sphinxstylestrong{29}
&
\sphinxAtStartPar
GPIO\_P14\_DSI0\_RST
&
\sphinxAtStartPar
O
&
\sphinxAtStartPar
DSI0 Reset,  active low
\\
\sphinxhline
\sphinxAtStartPar
\sphinxstylestrong{30}
&
\sphinxAtStartPar
GPIO\_P06\_DSI0\_INT
&
\sphinxAtStartPar
O
&
\sphinxAtStartPar
DSI0 Touch Interrupt, active low
\\
\sphinxbottomrule
\end{longtable}
\sphinxtableafterendhook
\sphinxatlongtableend
\end{savenotes}


\section{9.14×MIPI CSI}
\label{\detokenize{hardware:id29}}
\sphinxAtStartPar
The OK\sphinxhyphen{}MX8 MPQ\sphinxhyphen{}SMARC development board complies with the SMARC protocol and has two sets of MIPI CSI interfaces The main control chip supports 2 x ISP. A single ISP can support up to 12MP (4096x3072) @ 30fps. When dual ISPs are used, it can support up to 1080p (1936x1188) @ 80fps;

\sphinxAtStartPar
Use the MIPI CSI0 of 2\sphinxhyphen{}lane to access the 26pin FPC seat, and the OV5645 camera is adapted by default, as shown in the figure;

\sphinxAtStartPar
The MIPI CSI1 with single ISP 4\sphinxhyphen{}lane is connected to the Molex075783 connector, and the daA3840\sphinxhyphen{}30mc\sphinxhyphen{}IMX8MP\sphinxhyphen{}EVK camera module is adapted by default, with a maximum resolution of 4K;

\sphinxAtStartPar
Using dual ISP, the maximum supported is 1080p@80fps. For more detailed description, please refer to Chapter 13 of the ‘i.MX 8M Plus Applications Processor Reference Manual’ on the NXP official website.

\sphinxAtStartPar
\sphinxincludegraphics{{34c6e85679cd4ef6985024f274c3a47c}.png}

\sphinxAtStartPar
Figure 9.14\sphinxhyphen{}1 MIPI CSI0 Connector Schematic

\sphinxAtStartPar
\sphinxincludegraphics{{f35928c17c00475980ea6a9153dec52a}.png}

\sphinxAtStartPar
Figure 9.14\sphinxhyphen{}2 IMX8MP\sphinxhyphen{}EVK Camera Module Connector


\subsection{9.14.1 MIPI CS Connector Pin Description}
\label{\detokenize{hardware:mipi-cs-connector-pin-description}}
\sphinxAtStartPar
Table 9.14\sphinxhyphen{}1 P12 MIPI CSI0 Interface Pin Function Description


\begin{savenotes}\sphinxattablestart
\sphinxthistablewithglobalstyle
\centering
\begin{tabulary}{\linewidth}[t]{TTTT}
\sphinxtoprule
\sphinxstyletheadfamily 
\sphinxAtStartPar
\sphinxstylestrong{Pin}
&\sphinxstyletheadfamily 
\sphinxAtStartPar
\sphinxstylestrong{Development Board Signal}
&\sphinxstyletheadfamily 
\sphinxAtStartPar
\sphinxstylestrong{Type}
&\sphinxstyletheadfamily 
\sphinxAtStartPar
\sphinxstylestrong{Description}
\\
\sphinxmidrule
\sphinxtableatstartofbodyhook
\sphinxAtStartPar
\sphinxstylestrong{1}
&
\sphinxAtStartPar
GND
&
\sphinxAtStartPar
P
&
\sphinxAtStartPar
Digital  Ground
\\
\sphinxhline
\sphinxAtStartPar
\sphinxstylestrong{2}
&
\sphinxAtStartPar
MIPI\_CSI0\_CK\_P
&
\sphinxAtStartPar
DSI
&
\sphinxAtStartPar
CSI0  Port CLK; Positive
\\
\sphinxhline
\sphinxAtStartPar
\sphinxstylestrong{3}
&
\sphinxAtStartPar
MIPI\_CSI0\_CK\_N
&
\sphinxAtStartPar
DSI
&
\sphinxAtStartPar
CSI0  Port CLK; Negative
\\
\sphinxhline
\sphinxAtStartPar
\sphinxstylestrong{4}
&
\sphinxAtStartPar
GND
&
\sphinxAtStartPar
P
&
\sphinxAtStartPar
Digital  Ground
\\
\sphinxhline
\sphinxAtStartPar
\sphinxstylestrong{5}
&
\sphinxAtStartPar
MIPI\_CSI0\_RX0\_P
&
\sphinxAtStartPar
DSI
&
\sphinxAtStartPar
CSI0  Port Lane0; Positive
\\
\sphinxhline
\sphinxAtStartPar
\sphinxstylestrong{6}
&
\sphinxAtStartPar
MIPI\_CSI0\_RX0\_N
&
\sphinxAtStartPar
DSI
&
\sphinxAtStartPar
CSI0  Port Lane0; Negative
\\
\sphinxhline
\sphinxAtStartPar
\sphinxstylestrong{7}
&
\sphinxAtStartPar
GND
&
\sphinxAtStartPar
P
&
\sphinxAtStartPar
Digital  Ground
\\
\sphinxhline
\sphinxAtStartPar
\sphinxstylestrong{8}
&
\sphinxAtStartPar
MIPI\_CSI0\_RX1\_P
&
\sphinxAtStartPar
DSI
&
\sphinxAtStartPar
CSI0  Port Lane1; Positive
\\
\sphinxhline
\sphinxAtStartPar
\sphinxstylestrong{9}
&
\sphinxAtStartPar
MIPI\_CSI0\_RX1\_N
&
\sphinxAtStartPar
DSI
&
\sphinxAtStartPar
CSI0  Port Lane1; Negative
\\
\sphinxhline
\sphinxAtStartPar
\sphinxstylestrong{10}
&
\sphinxAtStartPar
GND
&
\sphinxAtStartPar
P
&
\sphinxAtStartPar
Digital  Ground
\\
\sphinxhline
\sphinxAtStartPar
\sphinxstylestrong{11}
&
\sphinxAtStartPar
| | \textbackslash{}
&
\sphinxAtStartPar

&
\sphinxAtStartPar

\\
\sphinxhline
\sphinxAtStartPar
\sphinxstylestrong{12}
&
\sphinxAtStartPar
| | \textbackslash{}
&
\sphinxAtStartPar

&
\sphinxAtStartPar

\\
\sphinxhline
\sphinxAtStartPar
\sphinxstylestrong{13}
&
\sphinxAtStartPar
GND
&
\sphinxAtStartPar
P
&
\sphinxAtStartPar
Digital  Ground
\\
\sphinxhline
\sphinxAtStartPar
\sphinxstylestrong{14}
&
\sphinxAtStartPar
| | \textbackslash{}
&
\sphinxAtStartPar

&
\sphinxAtStartPar

\\
\sphinxhline
\sphinxAtStartPar
\sphinxstylestrong{15}
&
\sphinxAtStartPar
| | \textbackslash{}
&
\sphinxAtStartPar

&
\sphinxAtStartPar

\\
\sphinxhline
\sphinxAtStartPar
\sphinxstylestrong{16}
&
\sphinxAtStartPar
GND
&
\sphinxAtStartPar
P
&
\sphinxAtStartPar
Digital  Ground
\\
\sphinxhline
\sphinxAtStartPar
\sphinxstylestrong{17}
&
\sphinxAtStartPar
I2C\_CAM0\_DAT\_1.8V
&
\sphinxAtStartPar
IO
&
\sphinxAtStartPar
I2C  Data
\\
\sphinxhline
\sphinxAtStartPar
\sphinxstylestrong{18}
&
\sphinxAtStartPar
I2C\_CAM0\_CK\_1.8V
&
\sphinxAtStartPar
O
&
\sphinxAtStartPar
I2C  CLK
\\
\sphinxhline
\sphinxAtStartPar
\sphinxstylestrong{19}
&
\sphinxAtStartPar
GPIO2\_CAM0\_RST\#\_1.8V
&
\sphinxAtStartPar
O
&
\sphinxAtStartPar
Camera  Reset signal
\\
\sphinxhline
\sphinxAtStartPar
\sphinxstylestrong{20}
&
\sphinxAtStartPar
GPIO0\_CAM0\_PWR\#\_1.8V
&
\sphinxAtStartPar
O
&
\sphinxAtStartPar
Camera  Power down signal
\\
\sphinxhline
\sphinxAtStartPar
\sphinxstylestrong{21}
&
\sphinxAtStartPar
CSI0\_MCLK\_1.8V
&
\sphinxAtStartPar
O
&
\sphinxAtStartPar
Camera  MCLK
\\
\sphinxhline
\sphinxAtStartPar
\sphinxstylestrong{22}
&
\sphinxAtStartPar
VCC1V8\_CSI0
&
\sphinxAtStartPar
P
&
\sphinxAtStartPar
Base  board 1.8V
\\
\sphinxhline
\sphinxAtStartPar
\sphinxstylestrong{23}
&
\sphinxAtStartPar
VCC2V8\_CSI0
&
\sphinxAtStartPar
P
&
\sphinxAtStartPar
Base  board 2.8V
\\
\sphinxhline
\sphinxAtStartPar
\sphinxstylestrong{24}
&
\sphinxAtStartPar
VCC2V8\_CSI0
&
\sphinxAtStartPar
P
&
\sphinxAtStartPar
Base  board 2.8V
\\
\sphinxhline
\sphinxAtStartPar
\sphinxstylestrong{25}
&
\sphinxAtStartPar
VCC1V5\_CSI0
&
\sphinxAtStartPar
P
&
\sphinxAtStartPar
Base  board 1.5V
\\
\sphinxhline
\sphinxAtStartPar
\sphinxstylestrong{26}
&
\sphinxAtStartPar
GND
&
\sphinxAtStartPar
P
&
\sphinxAtStartPar
Digital  Ground
\\
\sphinxbottomrule
\end{tabulary}
\sphinxtableafterendhook\par
\sphinxattableend\end{savenotes}

\sphinxAtStartPar
Table 9.14\sphinxhyphen{}2 P13 MIPI CSI1 Interface Pin Function Description


\begin{savenotes}
\sphinxatlongtablestart
\sphinxthistablewithglobalstyle
\makeatletter
  \LTleft \@totalleftmargin plus1fill
  \LTright\dimexpr\columnwidth-\@totalleftmargin-\linewidth\relax plus1fill
\makeatother
\begin{longtable}{llll}
\sphinxtoprule
\sphinxstyletheadfamily 
\sphinxAtStartPar
\sphinxstylestrong{Pin}
&\sphinxstyletheadfamily 
\sphinxAtStartPar
\sphinxstylestrong{Development Board Signal}
&\sphinxstyletheadfamily 
\sphinxAtStartPar
\sphinxstylestrong{Type}
&\sphinxstyletheadfamily 
\sphinxAtStartPar
\sphinxstylestrong{Description}
\\
\sphinxmidrule
\endfirsthead

\multicolumn{4}{c}{\sphinxnorowcolor
    \makebox[0pt]{\sphinxtablecontinued{\tablename\ \thetable{} \textendash{} continued from previous page}}%
}\\
\sphinxtoprule
\sphinxstyletheadfamily 
\sphinxAtStartPar
\sphinxstylestrong{Pin}
&\sphinxstyletheadfamily 
\sphinxAtStartPar
\sphinxstylestrong{Development Board Signal}
&\sphinxstyletheadfamily 
\sphinxAtStartPar
\sphinxstylestrong{Type}
&\sphinxstyletheadfamily 
\sphinxAtStartPar
\sphinxstylestrong{Description}
\\
\sphinxmidrule
\endhead

\sphinxbottomrule
\multicolumn{4}{r}{\sphinxnorowcolor
    \makebox[0pt][r]{\sphinxtablecontinued{continues on next page}}%
}\\
\endfoot

\endlastfoot
\sphinxtableatstartofbodyhook

\sphinxAtStartPar
\sphinxstylestrong{A1}
&
\sphinxAtStartPar
GND
&
\sphinxAtStartPar
P
&
\sphinxAtStartPar
Digital  Ground
\\
\sphinxhline
\sphinxAtStartPar
\sphinxstylestrong{A2}
&
\sphinxAtStartPar
MIPI\_CSI1\_CLK\_N
&
\sphinxAtStartPar
DSI
&
\sphinxAtStartPar
CSI1  Port CLK; Negative
\\
\sphinxhline
\sphinxAtStartPar
\sphinxstylestrong{A3}
&
\sphinxAtStartPar
MIPI\_CSI1\_CLK\_P
&
\sphinxAtStartPar
DSI
&
\sphinxAtStartPar
CSI1  Port CLK; Positive
\\
\sphinxhline
\sphinxAtStartPar
\sphinxstylestrong{A4}
&
\sphinxAtStartPar
GND
&
\sphinxAtStartPar
P
&
\sphinxAtStartPar
Digital  Ground
\\
\sphinxhline
\sphinxAtStartPar
\sphinxstylestrong{A5}
&
\sphinxAtStartPar
VCC\_1V8
&
\sphinxAtStartPar
P
&
\sphinxAtStartPar
Base  board 1.8V
\\
\sphinxhline
\sphinxAtStartPar
\sphinxstylestrong{A6}
&
\sphinxAtStartPar
VCC\_1V8
&
\sphinxAtStartPar
P
&
\sphinxAtStartPar
Base  board 1.8V
\\
\sphinxhline
\sphinxAtStartPar
\sphinxstylestrong{A7}
&
\sphinxAtStartPar
GND
&
\sphinxAtStartPar
P
&
\sphinxAtStartPar
Digital  Ground
\\
\sphinxhline
\sphinxAtStartPar
\sphinxstylestrong{A8}
&
\sphinxAtStartPar
| | \textbackslash{}
&
\sphinxAtStartPar

&
\sphinxAtStartPar

\\
\sphinxhline
\sphinxAtStartPar
\sphinxstylestrong{A9}
&
\sphinxAtStartPar
| | \textbackslash{}
&
\sphinxAtStartPar

&
\sphinxAtStartPar

\\
\sphinxhline
\sphinxAtStartPar
\sphinxstylestrong{A10}
&
\sphinxAtStartPar
GPIO12\_CSI1\_SYNC\_1.8V
&
\sphinxAtStartPar

&
\sphinxAtStartPar

\\
\sphinxhline
\sphinxAtStartPar
\sphinxstylestrong{A11}
&
\sphinxAtStartPar
GPIO1\_CAM1\_PWR\#\_1.8V
&
\sphinxAtStartPar

&
\sphinxAtStartPar

\\
\sphinxhline
\sphinxAtStartPar
\sphinxstylestrong{A12}
&
\sphinxAtStartPar
GND
&
\sphinxAtStartPar
P
&
\sphinxAtStartPar
Digital  Ground
\\
\sphinxhline
\sphinxAtStartPar
\sphinxstylestrong{A13}
&
\sphinxAtStartPar
VCC\_3V3
&
\sphinxAtStartPar
P
&
\sphinxAtStartPar
Base  board 3.3V
\\
\sphinxhline
\sphinxAtStartPar
\sphinxstylestrong{A14}
&
\sphinxAtStartPar
VCC\_3V3
&
\sphinxAtStartPar
P
&
\sphinxAtStartPar
Base  board 3.3V
\\
\sphinxhline
\sphinxAtStartPar
\sphinxstylestrong{A15}
&
\sphinxAtStartPar
GND
&
\sphinxAtStartPar
P
&
\sphinxAtStartPar
Digital  Ground
\\
\sphinxhline
\sphinxAtStartPar
\sphinxstylestrong{A16}
&
\sphinxAtStartPar
VCC\_5V
&
\sphinxAtStartPar
P
&
\sphinxAtStartPar
Base  board 5V
\\
\sphinxhline
\sphinxAtStartPar
\sphinxstylestrong{A17}
&
\sphinxAtStartPar
VCC\_5V
&
\sphinxAtStartPar
P
&
\sphinxAtStartPar
Base  board 5V
\\
\sphinxhline
\sphinxAtStartPar
\sphinxstylestrong{A18}
&
\sphinxAtStartPar
GND
&
\sphinxAtStartPar
P
&
\sphinxAtStartPar
Digital  Ground
\\
\sphinxhline
\sphinxAtStartPar
\sphinxstylestrong{B1}
&
\sphinxAtStartPar
GND
&
\sphinxAtStartPar
P
&
\sphinxAtStartPar
Digital  Ground
\\
\sphinxhline
\sphinxAtStartPar
\sphinxstylestrong{B2}
&
\sphinxAtStartPar
MIPI\_CSI1\_RX0\_N
&
\sphinxAtStartPar
DSI
&
\sphinxAtStartPar
CSI1  Port Lane0; Negative
\\
\sphinxhline
\sphinxAtStartPar
\sphinxstylestrong{B3}
&
\sphinxAtStartPar
MIPI\_CSI1\_RX0\_P
&
\sphinxAtStartPar
DSI
&
\sphinxAtStartPar
CSI1  Port Lane0; Positive
\\
\sphinxhline
\sphinxAtStartPar
\sphinxstylestrong{B4}
&
\sphinxAtStartPar
GND
&
\sphinxAtStartPar
P
&
\sphinxAtStartPar
Digital  Ground
\\
\sphinxhline
\sphinxAtStartPar
\sphinxstylestrong{B5}
&
\sphinxAtStartPar
MIPI\_CSI1\_RX1\_N
&
\sphinxAtStartPar
DSI
&
\sphinxAtStartPar
CSI1  Port Lane1; Negative
\\
\sphinxhline
\sphinxAtStartPar
\sphinxstylestrong{B6}
&
\sphinxAtStartPar
MIPI\_CSI1\_RX1\_P
&
\sphinxAtStartPar
DSI
&
\sphinxAtStartPar
CSI1  Port Lane1; Positive
\\
\sphinxhline
\sphinxAtStartPar
\sphinxstylestrong{B7}
&
\sphinxAtStartPar
GND
&
\sphinxAtStartPar
P
&
\sphinxAtStartPar
Digital  Ground
\\
\sphinxhline
\sphinxAtStartPar
\sphinxstylestrong{B8}
&
\sphinxAtStartPar
CAM\_MCK\_1.8V
&
\sphinxAtStartPar
O
&
\sphinxAtStartPar
Camera  MCLK
\\
\sphinxhline
\sphinxAtStartPar
\sphinxstylestrong{B9}
&
\sphinxAtStartPar
GPIO3\_CAM1\_RST\#\_1.8V
&
\sphinxAtStartPar
O
&
\sphinxAtStartPar
Camera  Reset signal
\\
\sphinxhline
\sphinxAtStartPar
\sphinxstylestrong{B10}
&
\sphinxAtStartPar
I2C\_CAM1\_DAT\_1.8V
&
\sphinxAtStartPar
IO
&
\sphinxAtStartPar
I2C  Data
\\
\sphinxhline
\sphinxAtStartPar
\sphinxstylestrong{B11}
&
\sphinxAtStartPar
I2C\_CAM1\_CK\_1.8V
&
\sphinxAtStartPar
O
&
\sphinxAtStartPar
I2C  CLK
\\
\sphinxhline
\sphinxAtStartPar
\sphinxstylestrong{B12}
&
\sphinxAtStartPar
GND
&
\sphinxAtStartPar
P
&
\sphinxAtStartPar
Digital  Ground
\\
\sphinxhline
\sphinxAtStartPar
\sphinxstylestrong{B13}
&
\sphinxAtStartPar
MIPI\_CSI1\_RX2\_N
&
\sphinxAtStartPar
DSI
&
\sphinxAtStartPar
CSI1  Port Lane2; Negative
\\
\sphinxhline
\sphinxAtStartPar
\sphinxstylestrong{B14}
&
\sphinxAtStartPar
MIPI\_CSI1\_RX2\_P
&
\sphinxAtStartPar
DSI
&
\sphinxAtStartPar
CSI1  Port Lane2; Positive
\\
\sphinxhline
\sphinxAtStartPar
\sphinxstylestrong{B15}
&
\sphinxAtStartPar
GND
&
\sphinxAtStartPar
P
&
\sphinxAtStartPar
Digital  Ground
\\
\sphinxhline
\sphinxAtStartPar
\sphinxstylestrong{B16}
&
\sphinxAtStartPar
MIPI\_CSI1\_RX3\_N
&
\sphinxAtStartPar
DSI
&
\sphinxAtStartPar
CSI1  Port Lane3; Negative
\\
\sphinxhline
\sphinxAtStartPar
\sphinxstylestrong{B17}
&
\sphinxAtStartPar
MIPI\_CSI1\_RX3\_P
&
\sphinxAtStartPar
DSI
&
\sphinxAtStartPar
CSI1  Port Lane3; Positive
\\
\sphinxhline
\sphinxAtStartPar
\sphinxstylestrong{B18}
&
\sphinxAtStartPar
GND
&
\sphinxAtStartPar
P
&
\sphinxAtStartPar
Digital  Ground
\\
\sphinxhline
\sphinxAtStartPar
\sphinxstylestrong{SH1}
&
\sphinxAtStartPar
PE
&
\sphinxAtStartPar
P
&
\sphinxAtStartPar
EARTH
\\
\sphinxhline
\sphinxAtStartPar
\sphinxstylestrong{SH2}
&
\sphinxAtStartPar
PE
&
\sphinxAtStartPar
P
&
\sphinxAtStartPar
EARTH
\\
\sphinxhline
\sphinxAtStartPar
\sphinxstylestrong{SH3}
&
\sphinxAtStartPar
PE
&
\sphinxAtStartPar
P
&
\sphinxAtStartPar
EARTH
\\
\sphinxhline
\sphinxAtStartPar
\sphinxstylestrong{SH4}
&
\sphinxAtStartPar
PE
&
\sphinxAtStartPar
P
&
\sphinxAtStartPar
EARTH
\\
\sphinxhline
\sphinxAtStartPar
\sphinxstylestrong{SH5}
&
\sphinxAtStartPar
PE
&
\sphinxAtStartPar
P
&
\sphinxAtStartPar
EARTH
\\
\sphinxhline
\sphinxAtStartPar
\sphinxstylestrong{SH6}
&
\sphinxAtStartPar
PE
&
\sphinxAtStartPar
P
&
\sphinxAtStartPar
EARTH
\\
\sphinxbottomrule
\end{longtable}
\sphinxtableafterendhook
\sphinxatlongtableend
\end{savenotes}


\section{9.15 PCIe x 1}
\label{\detokenize{hardware:pcie-x-1}}
\sphinxAtStartPar
The OK\sphinxhyphen{}MX8MPQ\sphinxhyphen{}SMARC development board is equipped with a standard PCIe x1 slot, supporting PCI Express Gen3. It features an independent DC\sphinxhyphen{}DC power supply configuration, making it convenient for users to connect various PCIe devices.

\sphinxAtStartPar
It supports both RC (Root Complex) and EP (Endpoint) modes, with a maximum link speed of Gen3 (8 GT/s).

\sphinxAtStartPar
\sphinxincludegraphics{{f62e997efc0b4d0e812298964eadd926}.png}

\sphinxAtStartPar
Figure 9.15\sphinxhyphen{}1 PCIE x1 Connector

\sphinxAtStartPar
\sphinxincludegraphics{{873c14fa7b2a47b49319b9150dd00723}.png}

\sphinxAtStartPar
Figure 9.15\sphinxhyphen{}2 Switching Power Supply Output PCIE\_3V3 Power for PCIe Module

\sphinxAtStartPar
\sphinxincludegraphics{{301f6a3d9fe942cd9093a4fc4118d160}.png}

\sphinxAtStartPar
Figure 9.15\sphinxhyphen{}3 MOSFET Used as a Switch to Output 12V Power for PCIe Module

\sphinxAtStartPar
Note:
\begin{itemize}
\item {} 
\sphinxAtStartPar
For PCIe 3.0 TX/RX, AC coupling capacitors are required to be between 180nF and 265nF, with 0.22µF commonly selected. The TX coupling capacitors are placed on the SoC side, while the RX coupling capacitors are placed on the device side;

\item {} 
\sphinxAtStartPar
PCIe uses stripline routing with an impedance requirement of 100 Ohms, and the length matching should be controlled within 10 mils. If signal traces change layers, two ground vias should accompany each of the TX\_P/N and RX\_P/N pairs;

\item {} 
\sphinxAtStartPar
If multiple connectors are cascaded in a PCIe connection, special attention must be paid to impedance continuity and signal integrity; otherwise, it will severely affect the PCIe signals. If the PCIe signals pass through multiple connector transitions, it may prevent the signals from achieving the 8Gbps data rate required for PCIe 3.0;

\item {} 
\sphinxAtStartPar
When connecting PCIe interfaces, it is especially important to avoid connecting TX to TX. The correct connection is TX to RX, with clear identification of which side is the host and which is the device.

\end{itemize}


\subsection{9.15.1 PCIe x1 Connector Pin Description}
\label{\detokenize{hardware:pcie-x1-connector-pin-description}}
\sphinxAtStartPar
Table 9.15\sphinxhyphen{}1 P11 PCIe x1 Interface Pin Function Description


\begin{savenotes}
\sphinxatlongtablestart
\sphinxthistablewithglobalstyle
\makeatletter
  \LTleft \@totalleftmargin plus1fill
  \LTright\dimexpr\columnwidth-\@totalleftmargin-\linewidth\relax plus1fill
\makeatother
\begin{longtable}{llll}
\sphinxtoprule
\sphinxstyletheadfamily 
\sphinxAtStartPar
\sphinxstylestrong{Pin}
&\sphinxstyletheadfamily 
\sphinxAtStartPar
\sphinxstylestrong{Development Board Signal}
&\sphinxstyletheadfamily 
\sphinxAtStartPar
\sphinxstylestrong{Type}
&\sphinxstyletheadfamily 
\sphinxAtStartPar
\sphinxstylestrong{Description}
\\
\sphinxmidrule
\endfirsthead

\multicolumn{4}{c}{\sphinxnorowcolor
    \makebox[0pt]{\sphinxtablecontinued{\tablename\ \thetable{} \textendash{} continued from previous page}}%
}\\
\sphinxtoprule
\sphinxstyletheadfamily 
\sphinxAtStartPar
\sphinxstylestrong{Pin}
&\sphinxstyletheadfamily 
\sphinxAtStartPar
\sphinxstylestrong{Development Board Signal}
&\sphinxstyletheadfamily 
\sphinxAtStartPar
\sphinxstylestrong{Type}
&\sphinxstyletheadfamily 
\sphinxAtStartPar
\sphinxstylestrong{Description}
\\
\sphinxmidrule
\endhead

\sphinxbottomrule
\multicolumn{4}{r}{\sphinxnorowcolor
    \makebox[0pt][r]{\sphinxtablecontinued{continues on next page}}%
}\\
\endfoot

\endlastfoot
\sphinxtableatstartofbodyhook

\sphinxAtStartPar
\sphinxstylestrong{A1}
&
\sphinxAtStartPar
PRSNT1\#
&
\sphinxAtStartPar
O
&
\sphinxAtStartPar
Hot\sphinxhyphen{}Plug presence detect
\\
\sphinxhline
\sphinxAtStartPar
\sphinxstylestrong{A2}
&
\sphinxAtStartPar
VCC12V\_PCIE30
&
\sphinxAtStartPar
P
&
\sphinxAtStartPar
Base  board 12V
\\
\sphinxhline
\sphinxAtStartPar
\sphinxstylestrong{A3}
&
\sphinxAtStartPar
VCC12V\_PCIE30
&
\sphinxAtStartPar
P
&
\sphinxAtStartPar
Base  board 12V
\\
\sphinxhline
\sphinxAtStartPar
\sphinxstylestrong{A4}
&
\sphinxAtStartPar
GND
&
\sphinxAtStartPar
P
&
\sphinxAtStartPar
Digital  Ground
\\
\sphinxhline
\sphinxAtStartPar
\sphinxstylestrong{A5}
&
\sphinxAtStartPar
| | \textbackslash{}
&
\sphinxAtStartPar

&
\sphinxAtStartPar

\\
\sphinxhline
\sphinxAtStartPar
\sphinxstylestrong{A6}
&
\sphinxAtStartPar
| | \textbackslash{}
&
\sphinxAtStartPar

&
\sphinxAtStartPar

\\
\sphinxhline
\sphinxAtStartPar
\sphinxstylestrong{A7}
&
\sphinxAtStartPar
| | \textbackslash{}
&
\sphinxAtStartPar

&
\sphinxAtStartPar

\\
\sphinxhline
\sphinxAtStartPar
\sphinxstylestrong{A8}
&
\sphinxAtStartPar
| | \textbackslash{}
&
\sphinxAtStartPar

&
\sphinxAtStartPar

\\
\sphinxhline
\sphinxAtStartPar
\sphinxstylestrong{A9}
&
\sphinxAtStartPar
PCIE\_3V3
&
\sphinxAtStartPar
P
&
\sphinxAtStartPar
Base  board 3.3V
\\
\sphinxhline
\sphinxAtStartPar
\sphinxstylestrong{A10}
&
\sphinxAtStartPar
PCIE\_3V3
&
\sphinxAtStartPar
P
&
\sphinxAtStartPar
Base  board 3.3V
\\
\sphinxhline
\sphinxAtStartPar
\sphinxstylestrong{A11}
&
\sphinxAtStartPar
PCIE\_A\_RST\#
&
\sphinxAtStartPar
O
&
\sphinxAtStartPar
PCIe  Reset signal
\\
\sphinxhline
\sphinxAtStartPar
\sphinxstylestrong{A12}
&
\sphinxAtStartPar
GND
&
\sphinxAtStartPar
P
&
\sphinxAtStartPar
Digital  Ground
\\
\sphinxhline
\sphinxAtStartPar
\sphinxstylestrong{A13}
&
\sphinxAtStartPar
PCIE\_A\_REFCK\_P
&
\sphinxAtStartPar
DSO
&
\sphinxAtStartPar
PCIe  Clock Diff. Positive; 100MHz HCSL
\\
\sphinxhline
\sphinxAtStartPar
\sphinxstylestrong{A14}
&
\sphinxAtStartPar
PCIE\_A\_REFCK\_N
&
\sphinxAtStartPar
DSO
&
\sphinxAtStartPar
PCIe  Clock Diff. Negative; 100MHz HCSL
\\
\sphinxhline
\sphinxAtStartPar
\sphinxstylestrong{A15}
&
\sphinxAtStartPar
GND
&
\sphinxAtStartPar
P
&
\sphinxAtStartPar
Digital  Ground
\\
\sphinxhline
\sphinxAtStartPar
\sphinxstylestrong{A16}
&
\sphinxAtStartPar
PCIE\_A\_RX\_P
&
\sphinxAtStartPar
DSI
&
\sphinxAtStartPar
PCIe  Receive Lane Diff. Positive
\\
\sphinxhline
\sphinxAtStartPar
\sphinxstylestrong{A17}
&
\sphinxAtStartPar
PCIE\_A\_RX\_N
&
\sphinxAtStartPar
DSI
&
\sphinxAtStartPar
PCIe  Receive Lane Diff. Negative
\\
\sphinxhline
\sphinxAtStartPar
\sphinxstylestrong{A18}
&
\sphinxAtStartPar
GND
&
\sphinxAtStartPar
P
&
\sphinxAtStartPar
Digital  Ground
\\
\sphinxhline
\sphinxAtStartPar
\sphinxstylestrong{B1}
&
\sphinxAtStartPar
VCC12V\_PCIE30
&
\sphinxAtStartPar
P
&
\sphinxAtStartPar
Base  board 12V
\\
\sphinxhline
\sphinxAtStartPar
\sphinxstylestrong{B2}
&
\sphinxAtStartPar
VCC12V\_PCIE30
&
\sphinxAtStartPar
P
&
\sphinxAtStartPar
Base  board 12V
\\
\sphinxhline
\sphinxAtStartPar
\sphinxstylestrong{B3}
&
\sphinxAtStartPar
VCC12V\_PCIE30
&
\sphinxAtStartPar
P
&
\sphinxAtStartPar
Base  board 12V
\\
\sphinxhline
\sphinxAtStartPar
\sphinxstylestrong{B4}
&
\sphinxAtStartPar
GND
&
\sphinxAtStartPar
P
&
\sphinxAtStartPar
Digital  Ground
\\
\sphinxhline
\sphinxAtStartPar
\sphinxstylestrong{B5}
&
\sphinxAtStartPar
| | \textbackslash{}
&
\sphinxAtStartPar

&
\sphinxAtStartPar

\\
\sphinxhline
\sphinxAtStartPar
\sphinxstylestrong{B6}
&
\sphinxAtStartPar
| | \textbackslash{}
&
\sphinxAtStartPar

&
\sphinxAtStartPar

\\
\sphinxhline
\sphinxAtStartPar
\sphinxstylestrong{B7}
&
\sphinxAtStartPar
GND
&
\sphinxAtStartPar
P
&
\sphinxAtStartPar
Digital  Ground
\\
\sphinxhline
\sphinxAtStartPar
\sphinxstylestrong{B8}
&
\sphinxAtStartPar
PCIE\_3V3
&
\sphinxAtStartPar
P
&
\sphinxAtStartPar
Base  board 3.3V
\\
\sphinxhline
\sphinxAtStartPar
\sphinxstylestrong{B9}
&
\sphinxAtStartPar
| | \textbackslash{}
&
\sphinxAtStartPar

&
\sphinxAtStartPar

\\
\sphinxhline
\sphinxAtStartPar
\sphinxstylestrong{B10}
&
\sphinxAtStartPar
VCC\_3V3
&
\sphinxAtStartPar
P
&
\sphinxAtStartPar
+3.3V  auxiliary power
\\
\sphinxhline
\sphinxAtStartPar
\sphinxstylestrong{B11}
&
\sphinxAtStartPar
PCIE\_WAKE\#
&
\sphinxAtStartPar
O
&
\sphinxAtStartPar
PCIe  Wake
\\
\sphinxhline
\sphinxAtStartPar
\sphinxstylestrong{B12}
&
\sphinxAtStartPar
PCIE\_A\_CKREQ\#
&
\sphinxAtStartPar
O
&
\sphinxAtStartPar
Clock  Request Signal
\\
\sphinxhline
\sphinxAtStartPar
\sphinxstylestrong{B13}
&
\sphinxAtStartPar
GND
&
\sphinxAtStartPar
P
&
\sphinxAtStartPar
Digital  Ground
\\
\sphinxhline
\sphinxAtStartPar
\sphinxstylestrong{B14}
&
\sphinxAtStartPar
PCIE\_A\_TX\_P
&
\sphinxAtStartPar
DSO
&
\sphinxAtStartPar
PCIe  Transmit Lane Diff. Positive
\\
\sphinxhline
\sphinxAtStartPar
\sphinxstylestrong{B15}
&
\sphinxAtStartPar
PCIE\_A\_TX\_N
&
\sphinxAtStartPar
DSO
&
\sphinxAtStartPar
PCIe  Transmit Lane Diff. Negative
\\
\sphinxhline
\sphinxAtStartPar
\sphinxstylestrong{B16}
&
\sphinxAtStartPar
GND
&
\sphinxAtStartPar
P
&
\sphinxAtStartPar
Digital  Ground
\\
\sphinxhline
\sphinxAtStartPar
\sphinxstylestrong{B17}
&
\sphinxAtStartPar
GPIO\_P02\_PCIE\_PRSNT
&
\sphinxAtStartPar
I
&
\sphinxAtStartPar
Hot\sphinxhyphen{}Plug presence detect
\\
\sphinxhline
\sphinxAtStartPar
\sphinxstylestrong{B18}
&
\sphinxAtStartPar
GND
&
\sphinxAtStartPar
P
&
\sphinxAtStartPar
Digital  Ground
\\
\sphinxbottomrule
\end{longtable}
\sphinxtableafterendhook
\sphinxatlongtableend
\end{savenotes}


\section{9.16 CANFD \& RS485}
\label{\detokenize{hardware:canfd-rs485}}
\sphinxAtStartPar
The OK\sphinxhyphen{}MX8MPQ\sphinxhyphen{}SMARC development board is equipped with 2 x CANFD converters and 2 x RS485 converters with quarantine power supply, which are connected side by side to the 2EDGRC\sphinxhyphen{}3.81 green terminal through 120 ohm matching resistors, common mode inductors, TVS, resettable fuse, and anti\sphinxhyphen{}detonator.

\sphinxAtStartPar
The CANFD circuit complies with or exceeds the technical specifications of the ISO11898\sphinxhyphen{}2 standard. It integrates a 5V isolated power supply and supports logic level conversion between 3.3V and 5V on the logic side, with a signal transmission rate of up to 5 Mbps.

\sphinxAtStartPar
RS485 is an isolated half\sphinxhyphen{}duplex enhanced transceiver that complies with the TIA/EIA\sphinxhyphen{}485A standard. The bus receiver uses a 1/8 unit load design, allowing a bus load capacity of up to 256 nodes, meeting the requirements of multi\sphinxhyphen{}node designs. The bus transmission rate can reach up to 4 Mbps.

\sphinxAtStartPar
\sphinxincludegraphics{{0954710209dd4e6fb89a7eb9c0e0e0ed}.png}

\sphinxAtStartPar
Figure 9.16\sphinxhyphen{}1 CANFD Module Circuit Diagram

\sphinxAtStartPar
\sphinxincludegraphics{{2f86d2bb4a214919b68a7717400b775a}.png}

\sphinxAtStartPar
Figure 9.16\sphinxhyphen{}2 RS485 Module Circuit Diagram

\sphinxAtStartPar
\sphinxincludegraphics{{58ba908fbca64b99806d08682da99ad3}.png}

\sphinxAtStartPar
Figure 9.16\sphinxhyphen{}3 2EDGRC\sphinxhyphen{}3.81 Connector Outputting CANFD and RS485 Signals


\subsection{9.16.1 CANFD \& RS485 Connector Pin Description}
\label{\detokenize{hardware:canfd-rs485-connector-pin-description}}
\sphinxAtStartPar
Table 9.16\sphinxhyphen{}1 P27 CANFD \& RS485 Interface Pin Function Description


\begin{savenotes}\sphinxattablestart
\sphinxthistablewithglobalstyle
\centering
\begin{tabulary}{\linewidth}[t]{TTTT}
\sphinxtoprule
\sphinxstyletheadfamily 
\sphinxAtStartPar
\sphinxstylestrong{Pin}
&\sphinxstyletheadfamily 
\sphinxAtStartPar
\sphinxstylestrong{Development Board Signal}
&\sphinxstyletheadfamily 
\sphinxAtStartPar
\sphinxstylestrong{Type}
&\sphinxstyletheadfamily 
\sphinxAtStartPar
\sphinxstylestrong{Description}
\\
\sphinxmidrule
\sphinxtableatstartofbodyhook
\sphinxAtStartPar
\sphinxstylestrong{1}
&
\sphinxAtStartPar
485\_B1
&
\sphinxAtStartPar
IO
&
\sphinxAtStartPar
RS485  Signal\sphinxhyphen{}B Differential signal
\\
\sphinxhline
\sphinxAtStartPar
\sphinxstylestrong{2}
&
\sphinxAtStartPar
485\_A1
&
\sphinxAtStartPar
IO
&
\sphinxAtStartPar
RS485  Signal\sphinxhyphen{}A Differential signal
\\
\sphinxhline
\sphinxAtStartPar
\sphinxstylestrong{3}
&
\sphinxAtStartPar
GND\_485\_1
&
\sphinxAtStartPar
P
&
\sphinxAtStartPar
GND\_485
\\
\sphinxhline
\sphinxAtStartPar
\sphinxstylestrong{4}
&
\sphinxAtStartPar
485\_B0
&
\sphinxAtStartPar
IO
&
\sphinxAtStartPar
RS485  Signal\sphinxhyphen{}B Differential signal
\\
\sphinxhline
\sphinxAtStartPar
\sphinxstylestrong{5}
&
\sphinxAtStartPar
485\_A0
&
\sphinxAtStartPar
IO
&
\sphinxAtStartPar
RS485  Signal\sphinxhyphen{}A Differential signal
\\
\sphinxhline
\sphinxAtStartPar
\sphinxstylestrong{6}
&
\sphinxAtStartPar
GND\_485\_0
&
\sphinxAtStartPar
P
&
\sphinxAtStartPar
GND\_485
\\
\sphinxhline
\sphinxAtStartPar
\sphinxstylestrong{7}
&
\sphinxAtStartPar
PE
&
\sphinxAtStartPar
P
&
\sphinxAtStartPar
EARTH
\\
\sphinxhline
\sphinxAtStartPar
\sphinxstylestrong{8}
&
\sphinxAtStartPar
CAN1\_L
&
\sphinxAtStartPar
IO
&
\sphinxAtStartPar
CAN  Low Differential signal
\\
\sphinxhline
\sphinxAtStartPar
\sphinxstylestrong{9}
&
\sphinxAtStartPar
CAN1\_H
&
\sphinxAtStartPar
IO
&
\sphinxAtStartPar
CAN  High Differential signal
\\
\sphinxhline
\sphinxAtStartPar
\sphinxstylestrong{10}
&
\sphinxAtStartPar
GND\_CAN\_1
&
\sphinxAtStartPar
P
&
\sphinxAtStartPar
GND\_CAN
\\
\sphinxhline
\sphinxAtStartPar
\sphinxstylestrong{11}
&
\sphinxAtStartPar
CAN0\_L
&
\sphinxAtStartPar
IO
&
\sphinxAtStartPar
CAN  Low Differential signal
\\
\sphinxhline
\sphinxAtStartPar
\sphinxstylestrong{12}
&
\sphinxAtStartPar
CAN0\_H
&
\sphinxAtStartPar
IO
&
\sphinxAtStartPar
CAN  High Differential signal
\\
\sphinxhline
\sphinxAtStartPar
\sphinxstylestrong{13}
&
\sphinxAtStartPar
GND\_CAN\_0
&
\sphinxAtStartPar
P
&
\sphinxAtStartPar
GND\_CAN
\\
\sphinxbottomrule
\end{tabulary}
\sphinxtableafterendhook\par
\sphinxattableend\end{savenotes}


\section{9.17 HDMI}
\label{\detokenize{hardware:id30}}
\sphinxAtStartPar
The OK\sphinxhyphen{}MX8MPQ\sphinxhyphen{}SMARC development board provides 1 x HDMI output via P14, supporting HDMI 2.0a. Compatible with the HDMI v2.0a specification, it supports decoding/encoding of the following video modes. All 2D video formats are described in the CEA\sphinxhyphen{}861\sphinxhyphen{}F specification, with resolutions up to 4K\textbackslash{}@30fps.

\sphinxAtStartPar
Although the MIMX8ML8CVNKZAB processor’s HDMI interface supports the HDMI 2.1 eARC feature, the development board removes the eARC functionality to comply with the SMARC specification requirements.

\sphinxAtStartPar
\sphinxincludegraphics{{4b1892272a964b36a3ef0ede85480f63}.png}

\sphinxAtStartPar
Figure 9.17\sphinxhyphen{}1 HDMI Interface and Protection Circuit


\subsection{9.17.1 HDMI Connector Pin Description}
\label{\detokenize{hardware:hdmi-connector-pin-description}}
\sphinxAtStartPar
Table 9.17\sphinxhyphen{}1 P14 HDMI Interface Pin Function Description


\begin{savenotes}\sphinxattablestart
\sphinxthistablewithglobalstyle
\centering
\begin{tabulary}{\linewidth}[t]{TTTT}
\sphinxtoprule
\sphinxstyletheadfamily 
\sphinxAtStartPar
\sphinxstylestrong{Pin}
&\sphinxstyletheadfamily 
\sphinxAtStartPar
\sphinxstylestrong{Development Board Signal}
&\sphinxstyletheadfamily 
\sphinxAtStartPar
\sphinxstylestrong{Type}
&\sphinxstyletheadfamily 
\sphinxAtStartPar
\sphinxstylestrong{Description}
\\
\sphinxmidrule
\sphinxtableatstartofbodyhook
\sphinxAtStartPar
\sphinxstylestrong{1}
&
\sphinxAtStartPar
HDMI\_D2\_P
&
\sphinxAtStartPar
DSO
&
\sphinxAtStartPar
HDMI  TMDS Diff. Data 2; Positive
\\
\sphinxhline
\sphinxAtStartPar
\sphinxstylestrong{2}
&
\sphinxAtStartPar
D2\_SHIELD
&
\sphinxAtStartPar
P
&
\sphinxAtStartPar
Digital  Ground
\\
\sphinxhline
\sphinxAtStartPar
\sphinxstylestrong{3}
&
\sphinxAtStartPar
HDMI\_D2\_N
&
\sphinxAtStartPar
DSO
&
\sphinxAtStartPar
HDMI  TMDS Diff. Data 2; Negative
\\
\sphinxhline
\sphinxAtStartPar
\sphinxstylestrong{4}
&
\sphinxAtStartPar
HDMI\_D1\_P
&
\sphinxAtStartPar
DSO
&
\sphinxAtStartPar
HDMI  TMDS Diff. Data 1; Positive
\\
\sphinxhline
\sphinxAtStartPar
\sphinxstylestrong{5}
&
\sphinxAtStartPar
D1\_SHIELD
&
\sphinxAtStartPar
P
&
\sphinxAtStartPar
Digital  Ground
\\
\sphinxhline
\sphinxAtStartPar
\sphinxstylestrong{6}
&
\sphinxAtStartPar
HDMI\_D1\_N
&
\sphinxAtStartPar
DSO
&
\sphinxAtStartPar
HDMI  TMDS Diff. Data 1; Negative
\\
\sphinxhline
\sphinxAtStartPar
\sphinxstylestrong{7}
&
\sphinxAtStartPar
HDMI\_D0\_P
&
\sphinxAtStartPar
DSO
&
\sphinxAtStartPar
HDMI  TMDS Diff. Data 0; Positive
\\
\sphinxhline
\sphinxAtStartPar
\sphinxstylestrong{8}
&
\sphinxAtStartPar
D0\_SHIELD
&
\sphinxAtStartPar
P
&
\sphinxAtStartPar
Digital  Ground
\\
\sphinxhline
\sphinxAtStartPar
\sphinxstylestrong{9}
&
\sphinxAtStartPar
HDMI\_D0\_N
&
\sphinxAtStartPar
DSO
&
\sphinxAtStartPar
HDMI  TMDS Diff. Data 0; Negative
\\
\sphinxhline
\sphinxAtStartPar
\sphinxstylestrong{10}
&
\sphinxAtStartPar
HDMI\_CLK\_P
&
\sphinxAtStartPar
DSO
&
\sphinxAtStartPar
HDMI  TMDS Diff. Clock; Positive
\\
\sphinxhline
\sphinxAtStartPar
\sphinxstylestrong{11}
&
\sphinxAtStartPar
CK\_SHIELD
&
\sphinxAtStartPar
P
&
\sphinxAtStartPar
Digital  Ground
\\
\sphinxhline
\sphinxAtStartPar
\sphinxstylestrong{12}
&
\sphinxAtStartPar
HDMI\_CLK\_N
&
\sphinxAtStartPar
DSO
&
\sphinxAtStartPar
HDMI  TMDS Diff. Clock; Negative
\\
\sphinxhline
\sphinxAtStartPar
\sphinxstylestrong{13}
&
\sphinxAtStartPar
HDMI\_CEC\_3.3V
&
\sphinxAtStartPar
IO
&
\sphinxAtStartPar
HDMI  Consumer Electronics Control
\\
\sphinxhline
\sphinxAtStartPar
\sphinxstylestrong{14}
&
\sphinxAtStartPar
HEC
&
\sphinxAtStartPar
| \textbackslash{}
&
\sphinxAtStartPar

\\
\sphinxhline
\sphinxAtStartPar
\sphinxstylestrong{15}
&
\sphinxAtStartPar
HDMI\_SCL\_5V
&
\sphinxAtStartPar
O
&
\sphinxAtStartPar
I2C  Clock for HDMI DDC
\\
\sphinxhline
\sphinxAtStartPar
\sphinxstylestrong{16}
&
\sphinxAtStartPar
HDMI\_SDA\_5V
&
\sphinxAtStartPar
IO
&
\sphinxAtStartPar
I2C  Data for HDMI DDC
\\
\sphinxhline
\sphinxAtStartPar
\sphinxstylestrong{17}
&
\sphinxAtStartPar
GND
&
\sphinxAtStartPar
P
&
\sphinxAtStartPar
Digital  Ground
\\
\sphinxhline
\sphinxAtStartPar
\sphinxstylestrong{18}
&
\sphinxAtStartPar
VCC5V\_HDMI
&
\sphinxAtStartPar
P
&
\sphinxAtStartPar
Base  board 5V
\\
\sphinxhline
\sphinxAtStartPar
\sphinxstylestrong{19}
&
\sphinxAtStartPar
HDMI\_HPD\_5V
&
\sphinxAtStartPar
I
&
\sphinxAtStartPar
HDMI  Hot Plug Detect
\\
\sphinxhline
\sphinxAtStartPar
\sphinxstylestrong{20}
&
\sphinxAtStartPar
PE
&
\sphinxAtStartPar
P
&
\sphinxAtStartPar
SHIELD  pin reference
\\
\sphinxbottomrule
\end{tabulary}
\sphinxtableafterendhook\par
\sphinxattableend\end{savenotes}


\section{9.18 TF CARD}
\label{\detokenize{hardware:tf-card}}
\sphinxAtStartPar
The OK\sphinxhyphen{}MX8MPQ\sphinxhyphen{}SMARC development board reserves a TF card slot, with added electrostatic protection components. Signal lines include reserved 0Ω resistors for debugging purposes.

\sphinxAtStartPar
\sphinxincludegraphics{{3777451b826f44ee98d9906733aa0b66}.png}

\sphinxAtStartPar
Figure 9.18\sphinxhyphen{}1 TF Card Connector Interfaces

\sphinxAtStartPar
The 3.3V power supply for the TF card is controlled by two sets of MOSFET switches, managed by the reset output signal RESET\_OUT\#\_3.3V (active low) and the power enable signal SDIO\_PWR\_EN.

\sphinxAtStartPar
\sphinxincludegraphics{{09e4195f996941dea850a005f4b92568}.png}

\sphinxAtStartPar
Figure 9.18\sphinxhyphen{}2 TF Card Power Supply Circuit


\subsection{9.18.1 TF Card Connector Pin Description}
\label{\detokenize{hardware:tf-card-connector-pin-description}}
\sphinxAtStartPar
Table 9.18\sphinxhyphen{}1 P27 TF Card Interface Pin Function Description


\begin{savenotes}\sphinxattablestart
\sphinxthistablewithglobalstyle
\centering
\begin{tabulary}{\linewidth}[t]{TTTT}
\sphinxtoprule
\sphinxstyletheadfamily 
\sphinxAtStartPar
\sphinxstylestrong{Pin}
&\sphinxstyletheadfamily 
\sphinxAtStartPar
\sphinxstylestrong{Development Board Signal}
&\sphinxstyletheadfamily 
\sphinxAtStartPar
\sphinxstylestrong{Type}
&\sphinxstyletheadfamily 
\sphinxAtStartPar
\sphinxstylestrong{Description}
\\
\sphinxmidrule
\sphinxtableatstartofbodyhook
\sphinxAtStartPar
\sphinxstylestrong{1}
&
\sphinxAtStartPar
SDIO\_D2\_1.8V/3.3V
&
\sphinxAtStartPar
IO
&
\sphinxAtStartPar
SD  Parallel Data2
\\
\sphinxhline
\sphinxAtStartPar
\sphinxstylestrong{2}
&
\sphinxAtStartPar
SDIO\_D3\_1.8V/3.3V
&
\sphinxAtStartPar
IO
&
\sphinxAtStartPar
SD  Parallel Data3
\\
\sphinxhline
\sphinxAtStartPar
\sphinxstylestrong{3}
&
\sphinxAtStartPar
SDIO\_CMD\_1.8V/3.3V
&
\sphinxAtStartPar
IO
&
\sphinxAtStartPar
SD  Command
\\
\sphinxhline
\sphinxAtStartPar
\sphinxstylestrong{4}
&
\sphinxAtStartPar
VCC\_3V3\_SD
&
\sphinxAtStartPar
P
&
\sphinxAtStartPar
SD  card 3.3V supply
\\
\sphinxhline
\sphinxAtStartPar
\sphinxstylestrong{5}
&
\sphinxAtStartPar
SDIO\_CK\_1.8V/3.3V
&
\sphinxAtStartPar
O
&
\sphinxAtStartPar
SD  Clock
\\
\sphinxhline
\sphinxAtStartPar
\sphinxstylestrong{6}
&
\sphinxAtStartPar
GND
&
\sphinxAtStartPar
P
&
\sphinxAtStartPar
Digital  Ground
\\
\sphinxhline
\sphinxAtStartPar
\sphinxstylestrong{7}
&
\sphinxAtStartPar
SDIO\_D0\_1.8V/3.3V
&
\sphinxAtStartPar
IO
&
\sphinxAtStartPar
SD  Parallel Data0
\\
\sphinxhline
\sphinxAtStartPar
\sphinxstylestrong{8}
&
\sphinxAtStartPar
SDIO\_D1\_1.8V/3.3V
&
\sphinxAtStartPar
IO
&
\sphinxAtStartPar
SD  Parallel Data1
\\
\sphinxhline
\sphinxAtStartPar
\sphinxstylestrong{9}
&
\sphinxAtStartPar
SDIO\_CD\#\_1.8V/3.3V
&
\sphinxAtStartPar
I
&
\sphinxAtStartPar
SD  Card Detect
\\
\sphinxhline
\sphinxAtStartPar
\sphinxstylestrong{PAD}
&
\sphinxAtStartPar
GND
&
\sphinxAtStartPar
P
&
\sphinxAtStartPar
SHIELD  pin reference
\\
\sphinxhline
\sphinxAtStartPar
\sphinxstylestrong{PAD}
&
\sphinxAtStartPar
GND
&
\sphinxAtStartPar
P
&
\sphinxAtStartPar
SHIELD  pin reference
\\
\sphinxhline
\sphinxAtStartPar
\sphinxstylestrong{PAD}
&
\sphinxAtStartPar
GND
&
\sphinxAtStartPar
P
&
\sphinxAtStartPar
SHIELD  pin reference
\\
\sphinxhline
\sphinxAtStartPar
\sphinxstylestrong{PAD}
&
\sphinxAtStartPar
GND
&
\sphinxAtStartPar
P
&
\sphinxAtStartPar
SHIELD  pin reference
\\
\sphinxbottomrule
\end{tabulary}
\sphinxtableafterendhook\par
\sphinxattableend\end{savenotes}


\section{9.19 QSPI \& SPI}
\label{\detokenize{hardware:qspi-spi}}
\sphinxAtStartPar
The OK\sphinxhyphen{}MX8MPQ\sphinxhyphen{}SMARC development board is equipped with 2 x 16MB FLASH memory chips, using QSPI and SPI interfaces respectively.

\sphinxAtStartPar
\sphinxincludegraphics{{3f27b10c83034f72a391e371ae647333}.png}

\sphinxAtStartPar
Figure 9.19\sphinxhyphen{}1 QSPI Module Circuit Diagram

\sphinxAtStartPar
\sphinxincludegraphics{{89e4835fd56b4fd6b545939dc42585ef}.png}

\sphinxAtStartPar
Figure 9.19\sphinxhyphen{}2 SPI Module Circuit Diagram


\section{9.20 PWM}
\label{\detokenize{hardware:id31}}
\sphinxAtStartPar
The OK\sphinxhyphen{}MX8MPQ\sphinxhyphen{}SMARC development board outputs a PWM breathing light signal through the GPIO5\_PWM\_OUT\_1.8V pin to control an NPN transistor that lights an LED. The LED brightness can be adjusted by changing the PWM duty cycle.

\sphinxAtStartPar
\sphinxincludegraphics{{e866396d3bc1476fb63c471c5dff7e43}.png}

\sphinxAtStartPar
Figure 9.20\sphinxhyphen{}1 PWM LED Circuit Diagram


\subsection{9.21 I/O expand}
\label{\detokenize{hardware:i-o-expand}}
\sphinxAtStartPar
The soM’s available I/O pins are insufficient to meet the development board’s pin requirements, so a GPIO expansion chip is used on the SoM.

\sphinxAtStartPar
\sphinxincludegraphics{{f13676df21034bfa8941511bfeca8ce1}.png}

\sphinxAtStartPar
Figure 9.21\sphinxhyphen{}1 GPIO Expansion Chip


\section{9.22 FAN}
\label{\detokenize{hardware:fan}}
\sphinxAtStartPar
The FAN speed is controlled using the GPIO5\_PWM\_OUT\_1.8V pin.

\sphinxAtStartPar
\sphinxincludegraphics{{2b196cdf32854aa5a443bd7a246cbf89}.png}

\sphinxAtStartPar
Figure 9.22\sphinxhyphen{}1 FAN Control Circuit


\chapter{10. OK\sphinxhyphen{}MX8MPQ\sphinxhyphen{}SMARC Hardware Design Guide}
\label{\detokenize{hardware:ok-mx8mpq-smarc-hardware-design-guide}}
\sphinxAtStartPar
1. Power Design
\begin{itemize}
\item {} 
\sphinxAtStartPar
The 12V/2A adapter supplies power directly to the SoM after filtering.

\item {} 
\sphinxAtStartPar
For DC/DC switching power supply, the switching circuit shall be as small as possible, and multi\sphinxhyphen{}capacitance capacitors shall be placed for filtering;

\item {} 
\sphinxAtStartPar
In LDO design, it is essential to ensure that the power rating is not excessively high and that the difference between the input and output voltages is also not too large.

\item {} 
\sphinxAtStartPar
For IC designs that are sensitive to voltage or have high instantaneous power consumption, it is necessary to place capacitors with sufficient capacitance near the pins to mitigate voltage transients.

\item {} 
\sphinxAtStartPar
When a MOSFET is used as a power supply switch in a circuit, it is necessary to consider adding a MOSFET soft\sphinxhyphen{}start capacitor or resistor to prevent instantaneous power supply voltage drops.

\end{itemize}

\sphinxAtStartPar
\sphinxstylestrong{2.}   \sphinxstylestrong{I2C Design}
\begin{itemize}
\item {} 
\sphinxAtStartPar
Multiple slave devices can be connected on a single I2C bus, ensuring no address conflicts.

\item {} 
\sphinxAtStartPar
Pull\sphinxhyphen{}up resistors need to be added to the I2C bus, and the bus loading conditions must be taken into account. Avoid using multiple resistors for pull\sphinxhyphen{}up purposes.

\item {} 
\sphinxAtStartPar
Please ensure level matching between the I2C on the SoM side and the I2C of the slave device.

\end{itemize}

\sphinxAtStartPar
\sphinxstylestrong{3. High\sphinxhyphen{}Speed Differential Signal Design}
\begin{itemize}
\item {} 
\sphinxAtStartPar
In order to meet the eye diagram requirements of high\sphinxhyphen{}speed signals, the routing length of differential lines shall be controlled within 6 inches as far as possible (different signals shall be analyzed differently);

\item {} 
\sphinxAtStartPar
High\sphinxhyphen{}speed differential signals require a complete reference ground plane. If signal punching is required, a ground via is added around the hole as a return path;

\item {} 
\sphinxAtStartPar
The characteristic impedance values of different signals are different, generally 100Ω, 90Ω for USB and 85Ω for PCIe. Refer to the SoM design for details;

\item {} 
\sphinxAtStartPar
Consider equal length within, equal length between groups (high\sphinxhyphen{}speed serial buses, such as PCIe, etc., without equal length between groups)

\item {} 
\sphinxAtStartPar
Long\sphinxhyphen{}distance differential wiring needs to consider the glass fiber braiding effect of the PCB, and 10 ° wiring is used;

\item {} 
\sphinxAtStartPar
If the high\sphinxhyphen{}speed signal lines are required to be designed to ground, ground holes shall be drilled at the same spacing, and the signal lines shall be spaced at a sufficient distance;

\item {} 
\sphinxAtStartPar
when the high\sphinxhyphen{}speed differential signal passes through the bonding pad, the reference plane area under the bonding pad may be partially hollowed out to ensure continuous impedance.

\end{itemize}

\sphinxAtStartPar
\sphinxstylestrong{4. SoM Reference Ground Design}
\begin{itemize}
\item {} 
\sphinxAtStartPar
The unused signal pins of the SoM can be left floating, but please make sure to connect all the GND pins.

\item {} 
\sphinxAtStartPar
It is recommended not to use a large\sphinxhyphen{}area copper pour for continuous power pins on the SoM. Instead, route the power from the pads and then apply copper pour along the traces.

\end{itemize}

\sphinxAtStartPar
\sphinxstylestrong{5. Single\sphinxhyphen{}ended High\sphinxhyphen{}speed Signal Design}
\begin{itemize}
\item {} 
\sphinxAtStartPar
For SDIO, QSPI and other single\sphinxhyphen{}ended signals with high speed, pay attention to the reference plane, equal length and signal line spacing;

\item {} 
\sphinxAtStartPar
If CLK is selected for grounding, it is necessary to add ground vias at least within a certain interval.

\end{itemize}


\chapter{11. Connector Dimension Diagram}
\label{\detokenize{hardware:connector-dimension-diagram}}
\sphinxAtStartPar
Carrier Board Connector Model: AS0B821\sphinxhyphen{}S78B, SoM Dimension:

\sphinxAtStartPar
\sphinxincludegraphics{{e8f7cf3d16bf40a690b1e2c2471be0dc}.png}

\sphinxAtStartPar
Figure 11.1\sphinxhyphen{}1 MXM 3.0 Connector Dimension

\sphinxAtStartPar
\sphinxincludegraphics{{627f486ea198413eb29bcb3be2000900}.png}

\sphinxAtStartPar
Figure 11.1\sphinxhyphen{}2 MXM 3.0 Carrier Packaging Connector



\renewcommand{\indexname}{Index}
\printindex
\end{document}